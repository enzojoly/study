\documentclass[11pt,a4paper]{article}
\usepackage[margin=1in]{geometry}
\usepackage{amsmath,amssymb,amsthm}
\usepackage{mathtools}
\usepackage{enumitem}
\usepackage{xcolor}
\usepackage{tcolorbox}

% Custom commands for XYZ methodology
\newcommand{\stage}[1]{\textbf{\textcolor{blue}{#1}}}
\newcommand{\critical}[1]{\textbf{\textcolor{red}{#1}}}

\title{Exercise Sheet 1: Question 1\\
Rewriting Systems as First Order ODEs\\
Methods of Applied Mathematics [SEMT30006]}
\author{Complete Solutions with XYZ Methodology}
\date{}

\begin{document}

\maketitle

\section*{Problem Statement}

Rewrite the following systems as first order ODEs, making it clear what state variables you have chosen and what their state-space is.

\begin{enumerate}[label=(\alph*)]
    \item $\displaystyle\frac{d^3u}{dt^3} - \frac{du}{dt} + \sin(u) = 0$

    \item $\displaystyle\frac{d^2u}{dt^2} + \frac{du}{dt} + u - 2v = 0, \quad \frac{d^2v}{dt^2} + \frac{dv}{dt} + v - 2u = 0$

    \item $\displaystyle\frac{d^2u}{dt^2} + \frac{du}{dt} - u + u^3 - v = 0, \quad \frac{dv}{dt} = u - v$
\end{enumerate}

\vspace{0.5cm}

\critical{CONTEXT FROM COURSE:} In dynamical systems theory (lecture notes pages 6-13), we always work with \textbf{first-order} systems of ODEs. Higher-order equations must be converted to first-order form by introducing new state variables. This canonical form allows us to:
\begin{itemize}
    \item Analyze phase space structure
    \item Find equilibria and study their stability
    \item Construct phase portraits
    \item Apply existence and uniqueness theorems
\end{itemize}

\newpage

\section{Problem 1(a): Third-Order Scalar ODE}

\subsection*{Problem Statement}
Convert to first-order form:
\begin{equation}
\frac{d^3u}{dt^3} - \frac{du}{dt} + \sin(u) = 0
\end{equation}

\subsection*{Step 1: Identify the Order and Required State Variables}

\begin{itemize}[leftmargin=*]
\item \stage{STAGE X (What we have):}
A third-order ODE involving $u$, $\frac{du}{dt}$, and $\frac{d^3u}{dt^3}$. The highest derivative is of order 3.

\item \stage{STAGE Y (Why we need 3 state variables):}
To convert an $n$-th order ODE to first-order form, we need exactly $n$ state variables. Each variable represents one derivative level:
\begin{itemize}
    \item One variable for $u$ itself
    \item One variable for $\frac{du}{dt}$
    \item One variable for $\frac{d^2u}{dt^2}$
\end{itemize}

The third derivative $\frac{d^3u}{dt^3}$ will be expressed in terms of these state variables using the original equation.

\item \stage{STAGE Z (Strategy):}
Introduce new variables for each derivative up to order $n-1 = 2$, then solve the original equation for the highest derivative.
\end{itemize}

\subsection*{Step 2: Define State Variables}

\begin{itemize}[leftmargin=*]
\item \stage{STAGE X (Introducing state variables):}
Define:
\begin{align}
x_1 &= u \\
x_2 &= \frac{du}{dt} = \dot{u} \\
x_3 &= \frac{d^2u}{dt^2} = \ddot{u}
\end{align}

\item \stage{STAGE Y (Why this choice):}
This is the \textbf{standard canonical form}. By defining each variable as the derivative of the previous one, we create a "chain" structure:
\begin{align}
\dot{x}_1 &= x_2 \\
\dot{x}_2 &= x_3 \\
\dot{x}_3 &= ?
\end{align}

The first two equations come directly from our definitions. The third equation comes from the original ODE.

\item \stage{STAGE Z (State space):}
Our state vector is:
\begin{equation}
\mathbf{x} = \begin{pmatrix} x_1 \\ x_2 \\ x_3 \end{pmatrix} = \begin{pmatrix} u \\ \dot{u} \\ \ddot{u} \end{pmatrix}
\end{equation}
\end{itemize}

\subsection*{Step 3: Express Highest Derivative from Original Equation}

\begin{itemize}[leftmargin=*]
\item \stage{STAGE X (Solving for $\frac{d^3u}{dt^3}$):}
From the original equation:
\begin{equation}
\frac{d^3u}{dt^3} - \frac{du}{dt} + \sin(u) = 0
\end{equation}

Rearrange:
\begin{equation}
\frac{d^3u}{dt^3} = \frac{du}{dt} - \sin(u)
\end{equation}

\item \stage{STAGE Y (Converting to state variables):}
Express in terms of $x_1, x_2, x_3$:
\begin{align}
\frac{d^3u}{dt^3} &= \frac{du}{dt} - \sin(u) \\
&= x_2 - \sin(x_1)
\end{align}

Since $\dot{x}_3 = \frac{d}{dt}\left(\frac{d^2u}{dt^2}\right) = \frac{d^3u}{dt^3}$:
\begin{equation}
\dot{x}_3 = x_2 - \sin(x_1)
\end{equation}

\item \stage{STAGE Z (Complete third equation):}
We now have all three first-order equations.
\end{itemize}

\subsection*{Step 4: Write Complete First-Order System}

\begin{itemize}[leftmargin=*]
\item \stage{STAGE X (System of first-order ODEs):}
\begin{align}
\frac{dx_1}{dt} &= x_2 \\
\frac{dx_2}{dt} &= x_3 \\
\frac{dx_3}{dt} &= x_2 - \sin(x_1)
\end{align}

\item \stage{STAGE Y (Vector form):}
More compactly, as a vector ODE:
\begin{equation}
\frac{d}{dt}\begin{pmatrix} x_1 \\ x_2 \\ x_3 \end{pmatrix} = \begin{pmatrix} x_2 \\ x_3 \\ x_2 - \sin(x_1) \end{pmatrix}
\end{equation}

Or using the notation $\dot{\mathbf{x}} = \mathbf{f}(\mathbf{x})$:
\begin{equation}
\dot{\mathbf{x}} = \mathbf{f}(\mathbf{x}) \quad \text{where} \quad \mathbf{f}(\mathbf{x}) = \begin{pmatrix} x_2 \\ x_3 \\ x_2 - \sin(x_1) \end{pmatrix}
\end{equation}

\item \stage{STAGE Z (Key observations):}
\begin{itemize}
    \item This is a \textbf{3-dimensional} dynamical system
    \item The system is \textbf{autonomous} (no explicit dependence on $t$)
    \item The system is \textbf{nonlinear} due to $\sin(x_1)$
\end{itemize}
\end{itemize}

\subsection*{Step 5: Specify State Space}

\begin{itemize}[leftmargin=*]
\item \stage{STAGE X (State space definition):}
The state space is the set of all possible values of $\mathbf{x} = (x_1, x_2, x_3)$.

\item \stage{STAGE Y (Determining constraints):}
Since $u$ can be any real number and its derivatives can also be any real numbers (no physical constraints given), the state space is:
\begin{equation}
\mathcal{S} = \mathbb{R}^3 = \{(x_1, x_2, x_3) \in \mathbb{R}^3\}
\end{equation}

This is \textbf{three-dimensional Euclidean space}.

\item \stage{STAGE Z (Physical interpretation):}
Each point $(x_1, x_2, x_3)$ in this space represents a complete state of the system:
\begin{itemize}
    \item $x_1$: position
    \item $x_2$: velocity
    \item $x_3$: acceleration
\end{itemize}

Trajectories in this 3D space describe how the system evolves over time.
\end{itemize}

\subsection*{Final Answer for Problem 1(a)}

\begin{tcolorbox}[colback=blue!5!white,colframe=blue!75!black,title=\textbf{Complete Solution}]

\textbf{State Variables:}
\begin{align}
x_1 &= u \\
x_2 &= \dot{u} \\
x_3 &= \ddot{u}
\end{align}

\textbf{First-Order System:}
\begin{equation}
\boxed{
\begin{aligned}
\dot{x}_1 &= x_2 \\
\dot{x}_2 &= x_3 \\
\dot{x}_3 &= x_2 - \sin(x_1)
\end{aligned}
}
\end{equation}

\textbf{State Space:}
\begin{equation}
\boxed{\mathcal{S} = \mathbb{R}^3}
\end{equation}

\textbf{Vector Form:}
\begin{equation}
\boxed{
\dot{\mathbf{x}} = \begin{pmatrix} x_2 \\ x_3 \\ x_2 - \sin(x_1) \end{pmatrix}, \quad \mathbf{x} \in \mathbb{R}^3
}
\end{equation}
\end{tcolorbox}

\subsection*{Verification}

\critical{CHECK:} Let's verify by recovering the original equation.

\begin{itemize}[leftmargin=*]
\item From $\dot{x}_1 = x_2$, we have $x_2 = \dot{u}$
\item From $\dot{x}_2 = x_3$, we have $x_3 = \ddot{x}_2 = \ddot{u}$
\item From $\dot{x}_3 = x_2 - \sin(x_1)$, we have:
\begin{equation}
\dddot{u} = \dot{u} - \sin(u)
\end{equation}

Rearranging:
\begin{equation}
\dddot{u} - \dot{u} + \sin(u) = 0 \quad \checkmark
\end{equation}
\end{itemize}

\newpage

\section{Problem 1(b): Coupled Second-Order ODEs}

\subsection*{Problem Statement}
Convert to first-order form:
\begin{align}
\frac{d^2u}{dt^2} + \frac{du}{dt} + u - 2v &= 0 \\
\frac{d^2v}{dt^2} + \frac{dv}{dt} + v - 2u &= 0
\end{align}

\subsection*{Step 1: Count Total Order and Required State Variables}

\begin{itemize}[leftmargin=*]
\item \stage{STAGE X (What we have):}
Two coupled second-order ODEs. Each equation involves a variable and its first and second derivatives.

\item \stage{STAGE Y (Why we need 4 state variables):}
We have:
\begin{itemize}
    \item Variable $u$ appears with derivatives up to order 2
    \item Variable $v$ appears with derivatives up to order 2
\end{itemize}

For each second-order variable, we need 2 state variables:
\begin{itemize}
    \item Total state variables needed: $2 + 2 = 4$
\end{itemize}

\item \stage{STAGE Z (Strategy):}
Introduce state variables for $u$, $\dot{u}$, $v$, and $\dot{v}$. Express $\ddot{u}$ and $\ddot{v}$ from the original equations.
\end{itemize}

\subsection*{Step 2: Define State Variables}

\begin{itemize}[leftmargin=*]
\item \stage{STAGE X (Introducing state variables):}
Define:
\begin{align}
x_1 &= u \\
x_2 &= \frac{du}{dt} = \dot{u} \\
x_3 &= v \\
x_4 &= \frac{dv}{dt} = \dot{v}
\end{align}

\item \stage{STAGE Y (Why this ordering):}
We group variables related to $u$ first $(x_1, x_2)$, then variables related to $v$ $(x_3, x_4)$. This is a natural organization, though any consistent ordering works.

Alternative notation sometimes used: $(u, \dot{u}, v, \dot{v})$ or $(u, v, \dot{u}, \dot{v})$. The key is consistency.

\item \stage{STAGE Z (Initial equations):}
From our definitions, we immediately have:
\begin{align}
\dot{x}_1 &= x_2 \quad \text{(definition of } x_2\text{)} \\
\dot{x}_3 &= x_4 \quad \text{(definition of } x_4\text{)}
\end{align}
\end{itemize}

\subsection*{Step 3: Solve Original Equations for Second Derivatives}

\begin{itemize}[leftmargin=*]
\item \stage{STAGE X (First equation):}
From $\frac{d^2u}{dt^2} + \frac{du}{dt} + u - 2v = 0$:
\begin{equation}
\frac{d^2u}{dt^2} = -\frac{du}{dt} - u + 2v
\end{equation}

\item \stage{STAGE Y (Converting to state variables):}
Express in terms of $x_1, x_2, x_3, x_4$:
\begin{equation}
\frac{d^2u}{dt^2} = -x_2 - x_1 + 2x_3
\end{equation}

Since $\dot{x}_2 = \frac{d}{dt}\left(\frac{du}{dt}\right) = \frac{d^2u}{dt^2}$:
\begin{equation}
\dot{x}_2 = -x_2 - x_1 + 2x_3
\end{equation}

\item \stage{STAGE Z (First half complete):}
We now have $\dot{x}_1$ and $\dot{x}_2$.
\end{itemize}

\subsection*{Step 4: Handle Second Equation}

\begin{itemize}[leftmargin=*]
\item \stage{STAGE X (Second equation):}
From $\frac{d^2v}{dt^2} + \frac{dv}{dt} + v - 2u = 0$:
\begin{equation}
\frac{d^2v}{dt^2} = -\frac{dv}{dt} - v + 2u
\end{equation}

\item \stage{STAGE Y (Converting to state variables):}
Express in terms of $x_1, x_2, x_3, x_4$:
\begin{equation}
\frac{d^2v}{dt^2} = -x_4 - x_3 + 2x_1
\end{equation}

Since $\dot{x}_4 = \frac{d}{dt}\left(\frac{dv}{dt}\right) = \frac{d^2v}{dt^2}$:
\begin{equation}
\dot{x}_4 = -x_4 - x_3 + 2x_1
\end{equation}

\item \stage{STAGE Z (System complete):}
We now have all four equations for $\dot{x}_1, \dot{x}_2, \dot{x}_3, \dot{x}_4$.
\end{itemize}

\subsection*{Step 5: Write Complete First-Order System}

\begin{itemize}[leftmargin=*]
\item \stage{STAGE X (System of first-order ODEs):}
\begin{align}
\frac{dx_1}{dt} &= x_2 \\
\frac{dx_2}{dt} &= -x_1 - x_2 + 2x_3 \\
\frac{dx_3}{dt} &= x_4 \\
\frac{dx_4}{dt} &= 2x_1 - x_3 - x_4
\end{align}

\item \stage{STAGE Y (Vector form):}
\begin{equation}
\frac{d}{dt}\begin{pmatrix} x_1 \\ x_2 \\ x_3 \\ x_4 \end{pmatrix} = \begin{pmatrix} x_2 \\ -x_1 - x_2 + 2x_3 \\ x_4 \\ 2x_1 - x_3 - x_4 \end{pmatrix}
\end{equation}

\item \stage{STAGE Z (Matrix representation):}
Since this system is \textbf{linear}, we can write it as $\dot{\mathbf{x}} = A\mathbf{x}$ where:
\begin{equation}
A = \begin{pmatrix}
0 & 1 & 0 & 0 \\
-1 & -1 & 2 & 0 \\
0 & 0 & 0 & 1 \\
2 & 0 & -1 & -1
\end{pmatrix}
\end{equation}

This is a $4 \times 4$ constant coefficient system.
\end{itemize}

\subsection*{Step 6: Specify State Space}

\begin{itemize}[leftmargin=*]
\item \stage{STAGE X (State space definition):}
The state space consists of all possible values of $\mathbf{x} = (x_1, x_2, x_3, x_4)$.

\item \stage{STAGE Y (Determining constraints):}
The variables $u, \dot{u}, v, \dot{v}$ can all take any real values (no constraints given), so:
\begin{equation}
\mathcal{S} = \mathbb{R}^4 = \{(x_1, x_2, x_3, x_4) \in \mathbb{R}^4\}
\end{equation}

\item \stage{STAGE Z (Interpretation):}
This is a \textbf{four-dimensional} phase space. Each point represents:
\begin{itemize}
    \item $x_1 = u$: position/state of first variable
    \item $x_2 = \dot{u}$: velocity of first variable
    \item $x_3 = v$: position/state of second variable
    \item $x_4 = \dot{v}$: velocity of second variable
\end{itemize}
\end{itemize}

\subsection*{Step 7: Analyze System Structure}

\begin{itemize}[leftmargin=*]
\item \stage{STAGE X (Coupling structure):}
Notice the coupling:
\begin{itemize}
    \item $\dot{x}_2$ depends on $x_1$ (from $u$), $x_2$ (from $\dot{u}$), and $x_3$ (from $v$)
    \item $\dot{x}_4$ depends on $x_1$ (from $u$), $x_3$ (from $v$), and $x_4$ (from $\dot{v}$)
\end{itemize}

\item \stage{STAGE Y (Symmetry observation):}
The original equations have a special symmetry:
\begin{align}
\ddot{u} + \dot{u} + u - 2v &= 0 \\
\ddot{v} + \dot{v} + v - 2u &= 0
\end{align}

If we swap $u \leftrightarrow v$, the system remains unchanged. This is a \textbf{permutation symmetry}.

In the matrix form:
\begin{equation}
A = \begin{pmatrix}
0 & 1 & 0 & 0 \\
-1 & -1 & 2 & 0 \\
0 & 0 & 0 & 1 \\
2 & 0 & -1 & -1
\end{pmatrix}
\end{equation}

The $(2,1)$ and $(4,3)$ entries are both $-1$ (self-coupling), while $(2,3) = 2$ and $(4,1) = 2$ (cross-coupling).

\item \stage{STAGE Z (Physical interpretation):}
This could model two coupled oscillators where:
\begin{itemize}
    \item Each oscillator has damping (the $\dot{u}$ and $\dot{v}$ terms)
    \item Each oscillator has restoring force (the $u$ and $v$ terms)
    \item They are coupled with strength 2 (the $-2v$ and $-2u$ terms)
\end{itemize}
\end{itemize}

\subsection*{Final Answer for Problem 1(b)}

\begin{tcolorbox}[colback=blue!5!white,colframe=blue!75!black,title=\textbf{Complete Solution}]

\textbf{State Variables:}
\begin{align}
x_1 &= u \\
x_2 &= \dot{u} \\
x_3 &= v \\
x_4 &= \dot{v}
\end{align}

\textbf{First-Order System:}
\begin{equation}
\boxed{
\begin{aligned}
\dot{x}_1 &= x_2 \\
\dot{x}_2 &= -x_1 - x_2 + 2x_3 \\
\dot{x}_3 &= x_4 \\
\dot{x}_4 &= 2x_1 - x_3 - x_4
\end{aligned}
}
\end{equation}

\textbf{State Space:}
\begin{equation}
\boxed{\mathcal{S} = \mathbb{R}^4}
\end{equation}

\textbf{Matrix Form:}
\begin{equation}
\boxed{
\dot{\mathbf{x}} = A\mathbf{x}, \quad A = \begin{pmatrix}
0 & 1 & 0 & 0 \\
-1 & -1 & 2 & 0 \\
0 & 0 & 0 & 1 \\
2 & 0 & -1 & -1
\end{pmatrix}, \quad \mathbf{x} \in \mathbb{R}^4
}
\end{equation}
\end{tcolorbox}

\subsection*{Verification}

\critical{CHECK:} Verify by recovering the original equations.

\begin{itemize}[leftmargin=*]
\item From $\dot{x}_1 = x_2$: we have $x_2 = \dot{u}$
\item From $\dot{x}_2 = -x_1 - x_2 + 2x_3$:
\begin{equation}
\ddot{u} = -u - \dot{u} + 2v \quad \Rightarrow \quad \ddot{u} + \dot{u} + u - 2v = 0 \quad \checkmark
\end{equation}

\item From $\dot{x}_3 = x_4$: we have $x_4 = \dot{v}$
\item From $\dot{x}_4 = 2x_1 - x_3 - x_4$:
\begin{equation}
\ddot{v} = 2u - v - \dot{v} \quad \Rightarrow \quad \ddot{v} + \dot{v} + v - 2u = 0 \quad \checkmark
\end{equation}
\end{itemize}

\newpage

\section{Problem 1(c): Mixed-Order Coupled System}

\subsection*{Problem Statement}
Convert to first-order form:
\begin{align}
\frac{d^2u}{dt^2} + \frac{du}{dt} - u + u^3 - v &= 0 \\
\frac{dv}{dt} &= u - v
\end{align}

\subsection*{Step 1: Analyze the System Structure}

\begin{itemize}[leftmargin=*]
\item \stage{STAGE X (What we have):}
A \textbf{mixed-order} system:
\begin{itemize}
    \item First equation: second-order in $u$, involves $u, \dot{u}, \ddot{u}, v$
    \item Second equation: first-order in $v$, involves $u, v, \dot{v}$
\end{itemize}

\item \stage{STAGE Y (Why this is different):}
Unlike problem 1(b) where both equations were second-order, here we have:
\begin{itemize}
    \item Variable $u$: appears up to order 2 $\Rightarrow$ need 2 state variables
    \item Variable $v$: appears up to order 1 $\Rightarrow$ need 1 state variable
\end{itemize}

Total state variables needed: $2 + 1 = 3$

The second equation is \textbf{already first-order}, so it will transfer directly with minimal modification.

\item \stage{STAGE Z (Strategy):}
Introduce state variables for $u$, $\dot{u}$, and $v$. The second equation is already in the right form.
\end{itemize}

\subsection*{Step 2: Define State Variables}

\begin{itemize}[leftmargin=*]
\item \stage{STAGE X (Introducing state variables):}
Define:
\begin{align}
x_1 &= u \\
x_2 &= \frac{du}{dt} = \dot{u} \\
x_3 &= v
\end{align}

\item \stage{STAGE Y (Why only 3 variables):}
Since $v$ only appears as $v$ and $\dot{v}$ (never $\ddot{v}$), we don't need a separate variable for $\dot{v}$. The variable $x_3 = v$ is sufficient.

Note: If $v$ had appeared with higher derivatives, we would need additional state variables.

\item \stage{STAGE Z (First equation from definition):}
From the definition of $x_2$:
\begin{equation}
\dot{x}_1 = x_2
\end{equation}
\end{itemize}

\subsection*{Step 3: Convert First Original Equation}

\begin{itemize}[leftmargin=*]
\item \stage{STAGE X (Solving for $\ddot{u}$):}
From $\frac{d^2u}{dt^2} + \frac{du}{dt} - u + u^3 - v = 0$:
\begin{equation}
\frac{d^2u}{dt^2} = -\frac{du}{dt} + u - u^3 + v
\end{equation}

\item \stage{STAGE Y (Converting to state variables):}
Express in terms of $x_1, x_2, x_3$:
\begin{equation}
\frac{d^2u}{dt^2} = -x_2 + x_1 - x_1^3 + x_3
\end{equation}

Since $\dot{x}_2 = \frac{d^2u}{dt^2}$:
\begin{equation}
\dot{x}_2 = -x_2 + x_1 - x_1^3 + x_3
\end{equation}

\item \stage{STAGE Z (Nonlinearity):}
Note the \textbf{cubic nonlinearity} $x_1^3$. This makes the system \textbf{nonlinear}, which means:
\begin{itemize}
    \item Cannot be written as $\dot{\mathbf{x}} = A\mathbf{x}$ (no constant matrix $A$)
    \item More complex dynamics possible (multiple equilibria, limit cycles, etc.)
    \item Linearization will be needed for stability analysis
\end{itemize}
\end{itemize}

\subsection*{Step 4: Convert Second Original Equation}

\begin{itemize}[leftmargin=*]
\item \stage{STAGE X (Already first-order):}
The equation $\frac{dv}{dt} = u - v$ is already first-order.

\item \stage{STAGE Y (Direct substitution):}
Simply replace $v$ with $x_3$ and $u$ with $x_1$:
\begin{equation}
\frac{dx_3}{dt} = x_1 - x_3
\end{equation}

Or:
\begin{equation}
\dot{x}_3 = x_1 - x_3
\end{equation}

\item \stage{STAGE Z (Linear coupling):}
This equation is linear in the state variables, even though the overall system is nonlinear due to the first equation.
\end{itemize}

\subsection*{Step 5: Write Complete First-Order System}

\begin{itemize}[leftmargin=*]
\item \stage{STAGE X (System of first-order ODEs):}
\begin{align}
\frac{dx_1}{dt} &= x_2 \\
\frac{dx_2}{dt} &= -x_2 + x_1 - x_1^3 + x_3 \\
\frac{dx_3}{dt} &= x_1 - x_3
\end{align}

\item \stage{STAGE Y (Vector form):}
\begin{equation}
\frac{d}{dt}\begin{pmatrix} x_1 \\ x_2 \\ x_3 \end{pmatrix} = \begin{pmatrix} x_2 \\ -x_2 + x_1 - x_1^3 + x_3 \\ x_1 - x_3 \end{pmatrix}
\end{equation}

Or using the notation $\dot{\mathbf{x}} = \mathbf{f}(\mathbf{x})$:
\begin{equation}
\dot{\mathbf{x}} = \mathbf{f}(\mathbf{x}) \quad \text{where} \quad \mathbf{f}(\mathbf{x}) = \begin{pmatrix} x_2 \\ -x_2 + x_1 - x_1^3 + x_3 \\ x_1 - x_3 \end{pmatrix}
\end{equation}

\item \stage{STAGE Z (System classification):}
\begin{itemize}
    \item \textbf{Dimension:} 3 (three state variables)
    \item \textbf{Autonomy:} Autonomous (no explicit $t$ dependence)
    \item \textbf{Linearity:} Nonlinear (due to $x_1^3$ term)
    \item \textbf{Coupling:} All three variables are coupled together
\end{itemize}
\end{itemize}

\subsection*{Step 6: Specify State Space}

\begin{itemize}[leftmargin=*]
\item \stage{STAGE X (State space definition):}
The state space consists of all possible values of $\mathbf{x} = (x_1, x_2, x_3)$.

\item \stage{STAGE Y (Determining constraints):}
No constraints are specified on $u$, $\dot{u}$, or $v$, so:
\begin{equation}
\mathcal{S} = \mathbb{R}^3 = \{(x_1, x_2, x_3) \in \mathbb{R}^3\}
\end{equation}

\item \stage{STAGE Z (Geometric picture):}
This is a \textbf{three-dimensional} phase space where:
\begin{itemize}
    \item $x_1 = u$: first dynamical variable
    \item $x_2 = \dot{u}$: rate of change of first variable
    \item $x_3 = v$: second dynamical variable
\end{itemize}

Trajectories are curves in $\mathbb{R}^3$ that never cross (by uniqueness of solutions).
\end{itemize}

\subsection*{Step 7: Find Equilibria}

\begin{itemize}[leftmargin=*]
\item \stage{STAGE X (Equilibrium definition):}
Equilibria occur where $\dot{\mathbf{x}} = \mathbf{0}$:
\begin{align}
x_2 &= 0 \\
-x_2 + x_1 - x_1^3 + x_3 &= 0 \\
x_1 - x_3 &= 0
\end{align}

\item \stage{STAGE Y (Solving the system):}
From the first equation: $x_2 = 0$

From the third equation: $x_3 = x_1$

Substitute into the second equation:
\begin{align}
-0 + x_1 - x_1^3 + x_1 &= 0 \\
2x_1 - x_1^3 &= 0 \\
x_1(2 - x_1^2) &= 0
\end{align}

This gives:
\begin{equation}
x_1 = 0 \quad \text{or} \quad x_1 = \pm\sqrt{2}
\end{equation}

\item \stage{STAGE Z (Three equilibria):}
The equilibrium points are:
\begin{align}
\mathbf{x}^*_1 &= (0, 0, 0) \\
\mathbf{x}^*_2 &= (\sqrt{2}, 0, \sqrt{2}) \\
\mathbf{x}^*_3 &= (-\sqrt{2}, 0, -\sqrt{2})
\end{align}

In terms of original variables:
\begin{align}
(u, \dot{u}, v) &= (0, 0, 0) \\
(u, \dot{u}, v) &= (\sqrt{2}, 0, \sqrt{2}) \\
(u, \dot{u}, v) &= (-\sqrt{2}, 0, -\sqrt{2})
\end{align}

The nonlinearity creates multiple equilibria!
\end{itemize}

\subsection*{Step 8: Physical Interpretation}

\begin{itemize}[leftmargin=*]
\item \stage{STAGE X (System structure):}
The term $u - u^3$ is a classic \textbf{Duffing-type} nonlinearity, suggesting this could model:
\begin{itemize}
    \item A nonlinear oscillator with cubic restoring force
    \item Damping from the $\dot{u}$ term
    \item Coupling to another variable $v$
\end{itemize}

\item \stage{STAGE Y (The role of $v$):}
The variable $v$ satisfies $\dot{v} = u - v$, which means:
\begin{itemize}
    \item $v$ is "driven" by $u$
    \item $v$ decays exponentially toward $u$ with rate 1
    \item $v$ acts as a "filtered" or "smoothed" version of $u$
\end{itemize}

\item \stage{STAGE Z (Feedback loop):}
There's a feedback structure:
\begin{itemize}
    \item $u$ influences $v$ through $\dot{v} = u - v$
    \item $v$ influences $u$ through the $+v$ term in $\ddot{u}$
    \item This creates a bidirectional coupling
\end{itemize}
\end{itemize}

\subsection*{Final Answer for Problem 1(c)}

\begin{tcolorbox}[colback=blue!5!white,colframe=blue!75!black,title=\textbf{Complete Solution}]

\textbf{State Variables:}
\begin{align}
x_1 &= u \\
x_2 &= \dot{u} \\
x_3 &= v
\end{align}

\textbf{First-Order System:}
\begin{equation}
\boxed{
\begin{aligned}
\dot{x}_1 &= x_2 \\
\dot{x}_2 &= -x_2 + x_1 - x_1^3 + x_3 \\
\dot{x}_3 &= x_1 - x_3
\end{aligned}
}
\end{equation}

\textbf{State Space:}
\begin{equation}
\boxed{\mathcal{S} = \mathbb{R}^3}
\end{equation}

\textbf{Vector Form:}
\begin{equation}
\boxed{
\dot{\mathbf{x}} = \begin{pmatrix} x_2 \\ -x_2 + x_1 - x_1^3 + x_3 \\ x_1 - x_3 \end{pmatrix}, \quad \mathbf{x} \in \mathbb{R}^3
}
\end{equation}

\textbf{Equilibria:}
\begin{equation}
\boxed{
(u, \dot{u}, v) \in \{(0, 0, 0), \; (\sqrt{2}, 0, \sqrt{2}), \; (-\sqrt{2}, 0, -\sqrt{2})\}
}
\end{equation}
\end{tcolorbox}

\subsection*{Verification}

\critical{CHECK:} Verify by recovering the original equations.

\begin{itemize}[leftmargin=*]
\item From $\dot{x}_1 = x_2$: we have $\dot{u} = x_2$ ✓

\item From $\dot{x}_2 = -x_2 + x_1 - x_1^3 + x_3$:
\begin{align}
\ddot{u} &= -\dot{u} + u - u^3 + v \\
\ddot{u} + \dot{u} - u + u^3 - v &= 0 \quad \checkmark
\end{align}

\item From $\dot{x}_3 = x_1 - x_3$:
\begin{equation}
\dot{v} = u - v \quad \checkmark
\end{equation}
\end{itemize}

\newpage

\section*{Summary: Converting Higher-Order ODEs to First-Order Form}

\subsection*{General Procedure (The XYZ Method)}

\begin{enumerate}
\item \stage{STAGE X (Identify):}
\begin{itemize}
    \item Count the variables and their highest derivative orders
    \item Determine total number of state variables needed
    \item Recognize any equations already in first-order form
\end{itemize}

\item \stage{STAGE Y (Define):}
\begin{itemize}
    \item Introduce state variables for each variable up to order $n-1$
    \item Standard choice: $x_1 = u$, $x_2 = \dot{u}$, $x_3 = \ddot{u}$, etc.
    \item Write immediate relations: $\dot{x}_i = x_{i+1}$
\end{itemize}

\item \stage{STAGE Z (Solve):}
\begin{itemize}
    \item Solve original equations for highest derivatives
    \item Express in terms of state variables only
    \item Verify by substituting back into original equations
\end{itemize}
\end{enumerate}

\subsection*{Key Principles from Course Material}

\critical{From lecture notes (pages 6-13):}

\begin{enumerate}
\item \textbf{Phase space dimension:} For $m$ variables with maximum orders $n_1, n_2, \ldots, n_m$, the phase space has dimension $N = n_1 + n_2 + \cdots + n_m$.

\item \textbf{Autonomy:} A system is autonomous if $\dot{\mathbf{x}} = \mathbf{f}(\mathbf{x})$ with no explicit $t$ dependence. All three problems here are autonomous.

\item \textbf{Linearity:} System is linear if $\mathbf{f}(\mathbf{x}) = A\mathbf{x}$ for some matrix $A$. Only problem (b) is linear.

\item \textbf{Trajectories:} Solutions are curves $\mathbf{x}(t)$ in phase space that cannot cross (by uniqueness).

\item \textbf{Equilibria:} Found by solving $\mathbf{f}(\mathbf{x}^*) = \mathbf{0}$.
\end{enumerate}

\subsection*{Comparison Table}

\begin{center}
\begin{tabular}{|c|c|c|c|c|}
\hline
\textbf{Problem} & \textbf{Original Orders} & \textbf{State Dim.} & \textbf{Linear?} & \textbf{Equilibria} \\
\hline
1(a) & $u$: 3rd order & 3 & No & Need analysis \\
\hline
1(b) & $u, v$: 2nd order & 4 & Yes & $(0,0,0,0)$ only \\
\hline
1(c) & $u$: 2nd, $v$: 1st & 3 & No & Three: see above \\
\hline
\end{tabular}
\end{center}

\subsection*{Connection to Future Analysis}

Once in first-order form, we can:
\begin{itemize}
\item Find equilibria by solving $\dot{\mathbf{x}} = \mathbf{0}$
\item Linearize about equilibria using Jacobian matrix
\item Analyze stability using eigenvalues
\item Construct phase portraits
\item Apply existence and uniqueness theorems
\end{itemize}

These techniques will be explored in subsequent exercise problems.

\vfill

\begin{center}
\Large\textbf{END OF QUESTION 1 SOLUTIONS}
\end{center}

\end{document}
