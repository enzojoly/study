\documentclass[11pt,a4paper]{article}
\usepackage[margin=1in]{geometry}
\usepackage{amsmath,amssymb,amsthm}
\usepackage{mathtools}
\usepackage{enumitem}
\usepackage{xcolor}

% Custom commands for XYZ methodology
\newcommand{\stage}[1]{\textbf{\textcolor{blue}{#1}}}
\newcommand{\critical}[1]{\textbf{\textcolor{red}{#1}}}

\title{Exercise Sheet 1, Question 3: Autonomy (Time-Independence)\\
Complete Solution with XYZ Methodology\\
Methods of Applied Mathematics [SEMT30006]}
\author{}
\date{}

\begin{document}

\maketitle

\section*{Problem Statement}

We've seen systems that either depend on time or don't. A system that does not depend explicitly on its independent variable is called \textbf{autonomous}. Which of the following is autonomous? What is the independent variable?

\begin{enumerate}[label=(\alph*)]
    \item The ODE $\ddot{u} = u + \sin(t)$
    \item The ODE $y'' - y - \sin(x) = 0$
    \item The ODE $\ddot{\theta} + a\dot{\theta} + b = 0$
    \item The map $x_{n+1} = ax_n + x_n^2$
    \item The map $x_{n+1} = nx_n + b$
\end{enumerate}

\section{Foundational Concepts}

\subsection*{What is Autonomy?}

\begin{itemize}[leftmargin=*]
\item \stage{STAGE X (Definition):}
A dynamical system is \textbf{autonomous} if it does not depend explicitly on the independent variable.

For ODEs: The right-hand side depends only on the state variables, not on time $t$ explicitly.
\begin{align}
\text{Autonomous:} \quad & \dot{x} = f(x) \\
\text{Non-autonomous:} \quad & \dot{x} = f(x, t)
\end{align}

For maps: The map depends only on the current state $x_n$, not on the index $n$ explicitly.
\begin{align}
\text{Autonomous:} \quad & x_{n+1} = f(x_n) \\
\text{Non-autonomous:} \quad & x_{n+1} = f(x_n, n)
\end{align}

\item \stage{STAGE Y (Why this matters):}
Autonomous systems have special properties:
\begin{enumerate}
    \item \textbf{Time-translation invariance:} If $x(t)$ is a solution, then $x(t + t_0)$ is also a solution for any constant $t_0$. The system "looks the same" at all times.
    \item \textbf{Phase space structure:} Trajectories in phase space never cross (uniqueness). The flow is time-independent.
    \item \textbf{Equilibria are stationary:} Fixed points don't move with time.
    \item \textbf{Simpler analysis:} We can study the system in phase space without tracking time explicitly.
\end{enumerate}

Non-autonomous systems can have time-varying equilibria, crossing trajectories (at different times), and more complex behavior.

\item \stage{STAGE Z (How to identify):}
\begin{enumerate}
    \item Identify the independent variable (usually $t$ for ODEs, $n$ for maps)
    \item Check if this variable appears explicitly on the right-hand side
    \item If it appears explicitly $\Rightarrow$ non-autonomous
    \item If it doesn't appear $\Rightarrow$ autonomous
\end{enumerate}
\end{itemize}

\critical{KEY DISTINCTION:} The independent variable appearing in derivatives (like $\frac{d}{dt}$) doesn't count—it must appear explicitly in the function itself.

\newpage

\section{Part (a): The ODE $\ddot{u} = u + \sin(t)$}

\subsection*{Step 1: Identify the Independent Variable}

\begin{itemize}[leftmargin=*]
\item \stage{STAGE X (What we have):}
The equation is:
\begin{align}
\ddot{u} = u + \sin(t)
\end{align}

where $\ddot{u} = \frac{d^2u}{dt^2}$.

\item \stage{STAGE Y (Finding the independent variable):}
The notation $\ddot{u}$ means differentiation with respect to time $t$. Therefore, the independent variable is:
\begin{align}
\boxed{\text{Independent variable: } t \text{ (time)}}
\end{align}

The dependent (state) variable is $u$.

\item \stage{STAGE Z (Clear identification):}
This is an ODE where we evolve the state $u$ as the independent variable $t$ changes.
\end{itemize}

\subsection*{Step 2: Check for Explicit Dependence on Independent Variable}

\begin{itemize}[leftmargin=*]
\item \stage{STAGE X (Rewrite in standard form):}
Write as a first-order system. Let $u_1 = u$ and $u_2 = \dot{u}$:
\begin{align}
\dot{u}_1 &= u_2 \\
\dot{u}_2 &= u_1 + \sin(t)
\end{align}

Or in vector form with $\mathbf{u} = \begin{pmatrix} u_1 \\ u_2 \end{pmatrix}$:
\begin{align}
\dot{\mathbf{u}} = \begin{pmatrix} u_2 \\ u_1 + \sin(t) \end{pmatrix} = f(\mathbf{u}, t)
\end{align}

\item \stage{STAGE Y (Does $t$ appear explicitly?):}
YES! The term $\sin(t)$ contains $t$ explicitly on the right-hand side. The function $f$ depends on both $\mathbf{u}$ and $t$:
\begin{align}
f(\mathbf{u}, t) = \begin{pmatrix} u_2 \\ u_1 + \sin(t) \end{pmatrix}
\end{align}

We cannot write this as $f(\mathbf{u})$ alone—the time $t$ must be included.

\item \stage{STAGE Z (Conclusion):}
\begin{align}
\boxed{\text{NOT AUTONOMOUS}}
\end{align}

The system is \textbf{non-autonomous} because of the explicit $\sin(t)$ term.
\end{itemize}

\subsection*{Step 3: Physical Interpretation}

\begin{itemize}[leftmargin=*]
\item \stage{STAGE X (What this equation represents):}
This could model a forced oscillator:
\begin{align}
\ddot{u} - u = \sin(t)
\end{align}

The $\sin(t)$ term is an external periodic forcing that varies with time.

\item \stage{STAGE Y (Why non-autonomy matters):}
Because the forcing varies with time:
\begin{enumerate}
    \item The "rules" of the system change with $t$
    \item Different solutions starting at the same state $u$ but different times $t$ will evolve differently
    \item Phase portraits change with time
    \item Equilibria (if any) may move with time
\end{enumerate}

\item \stage{STAGE Z (Example):}
If we evaluate at $t = 0$: $\ddot{u} = u + \sin(0) = u$

If we evaluate at $t = \pi/2$: $\ddot{u} = u + \sin(\pi/2) = u + 1$

The system dynamics differ at different times, confirming non-autonomy.
\end{itemize}

\critical{KEY INSIGHT:} The presence of an explicit time-dependent forcing term $\sin(t)$ makes the system non-autonomous. This is common in physics with external driving forces.

\newpage

\section{Part (b): The ODE $y'' - y - \sin(x) = 0$}

\subsection*{Step 1: Identify the Independent Variable}

\begin{itemize}[leftmargin=*]
\item \stage{STAGE X (What we have):}
The equation is:
\begin{align}
y'' - y - \sin(x) = 0
\end{align}

or equivalently:
\begin{align}
y'' = y + \sin(x)
\end{align}

\item \stage{STAGE Y (Interpreting the notation):}
The prime notation $y''$ means:
\begin{align}
y'' = \frac{d^2y}{dx^2}
\end{align}

We are differentiating $y$ with respect to $x$. Therefore:
\begin{align}
\boxed{\text{Independent variable: } x}
\end{align}

The dependent (state) variable is $y$.

\item \stage{STAGE Z (Important distinction):}
Here, $x$ is the independent variable (like time), not the state variable! The state variable is $y$, which depends on $x$.
\end{itemize}

\subsection*{Step 2: Check for Explicit Dependence on Independent Variable}

\begin{itemize}[leftmargin=*]
\item \stage{STAGE X (Rewrite as first-order system):}
Let $y_1 = y$ and $y_2 = y'$ where primes denote derivatives with respect to $x$:
\begin{align}
y_1' &= y_2 \\
y_2' &= y_1 + \sin(x)
\end{align}

In vector form with $\mathbf{y} = \begin{pmatrix} y_1 \\ y_2 \end{pmatrix}$:
\begin{align}
\mathbf{y}' = \begin{pmatrix} y_2 \\ y_1 + \sin(x) \end{pmatrix} = f(\mathbf{y}, x)
\end{align}

\item \stage{STAGE Y (Does $x$ appear explicitly?):}
YES! The term $\sin(x)$ contains the independent variable $x$ explicitly on the right-hand side. The function depends on both $\mathbf{y}$ and $x$:
\begin{align}
f(\mathbf{y}, x) = \begin{pmatrix} y_2 \\ y_1 + \sin(x) \end{pmatrix}
\end{align}

We cannot write this purely as a function of $\mathbf{y}$.

\item \stage{STAGE Z (Conclusion):}
\begin{align}
\boxed{\text{NOT AUTONOMOUS}}
\end{align}

The system is \textbf{non-autonomous} because the independent variable $x$ appears explicitly in $\sin(x)$.
\end{itemize}

\subsection*{Step 3: Comparison with Part (a)}

\begin{itemize}[leftmargin=*]
\item \stage{STAGE X (Structural similarity):}
Compare parts (a) and (b):
\begin{align}
\text{Part (a):} \quad & \ddot{u} = u + \sin(t) \quad \text{(independent var: } t\text{)} \\
\text{Part (b):} \quad & y'' = y + \sin(x) \quad \text{(independent var: } x\text{)}
\end{align}

These have \textbf{identical mathematical structure}!

\item \stage{STAGE Y (Why both are non-autonomous):}
In both cases:
\begin{enumerate}
    \item The independent variable appears explicitly in a $\sin(\cdot)$ term
    \item The "forcing" changes as we move along the independent variable
    \item Different starting positions along the independent variable lead to different dynamics
\end{enumerate}

\item \stage{STAGE Z (The confusion):}
Students often think (b) might be different because $x$ "looks like" a state variable. But:
\begin{itemize}
    \item In part (a): $t$ is independent, $u$ is dependent (state)
    \item In part (b): $x$ is independent, $y$ is dependent (state)
\end{itemize}

Both have the same non-autonomous structure: independent variable appears explicitly in $\sin(\text{independent var})$.
\end{itemize}

\critical{COMMON MISTAKE:} Don't assume $x$ is always a state variable! Here, $x$ is the independent variable (like time), and $y$ is evolving along $x$.

\newpage

\section{Part (c): The ODE $\ddot{\theta} + a\dot{\theta} + b = 0$}

\subsection*{Step 1: Identify the Independent Variable}

\begin{itemize}[leftmargin=*]
\item \stage{STAGE X (What we have):}
The equation is:
\begin{align}
\ddot{\theta} + a\dot{\theta} + b = 0
\end{align}

\item \stage{STAGE Y (Interpreting the notation):}
The dot notation means differentiation with respect to time:
\begin{align}
\dot{\theta} = \frac{d\theta}{dt}, \quad \ddot{\theta} = \frac{d^2\theta}{dt^2}
\end{align}

Therefore:
\begin{align}
\boxed{\text{Independent variable: } t \text{ (time)}}
\end{align}

The dependent (state) variable is $\theta$.

\item \stage{STAGE Z (Parameters):}
Note that $a$ and $b$ are constants (parameters), not variables.
\end{itemize}

\subsection*{Step 2: Rewrite as First-Order System}

\begin{itemize}[leftmargin=*]
\item \stage{STAGE X (Standard form):}
Rewrite the equation:
\begin{align}
\ddot{\theta} = -a\dot{\theta} - b
\end{align}

Let $\theta_1 = \theta$ and $\theta_2 = \dot{\theta}$:
\begin{align}
\dot{\theta}_1 &= \theta_2 \\
\dot{\theta}_2 &= -a\theta_2 - b
\end{align}

\item \stage{STAGE Y (Vector form):}
With $\boldsymbol{\theta} = \begin{pmatrix} \theta_1 \\ \theta_2 \end{pmatrix}$:
\begin{align}
\dot{\boldsymbol{\theta}} = \begin{pmatrix} \theta_2 \\ -a\theta_2 - b \end{pmatrix} = f(\boldsymbol{\theta})
\end{align}

\item \stage{STAGE Z (Key observation):}
The right-hand side depends only on $\boldsymbol{\theta}$ (the state variables $\theta_1$ and $\theta_2$), not on $t$ explicitly.
\end{itemize}

\subsection*{Step 3: Check for Explicit Time Dependence}

\begin{itemize}[leftmargin=*]
\item \stage{STAGE X (Examining the right-hand side):}
The function is:
\begin{align}
f(\boldsymbol{\theta}) = \begin{pmatrix} \theta_2 \\ -a\theta_2 - b \end{pmatrix}
\end{align}

\item \stage{STAGE Y (Does $t$ appear?):}
NO! The independent variable $t$ does not appear anywhere on the right-hand side:
\begin{itemize}
    \item $\theta_2$ is a state variable (not $t$)
    \item $a$ and $b$ are constants (not $t$)
    \item No terms like $\sin(t)$, $e^t$, $t^2$, etc.
\end{itemize}

The function depends only on the state $\boldsymbol{\theta}$, not on $t$.

\item \stage{STAGE Z (Conclusion):}
\begin{align}
\boxed{\text{AUTONOMOUS}}
\end{align}

The system is \textbf{autonomous}.
\end{itemize}

\subsection*{Step 4: Physical Interpretation}

\begin{itemize}[leftmargin=*]
\item \stage{STAGE X (What this equation models):}
This is a damped system with constant forcing:
\begin{align}
\ddot{\theta} + a\dot{\theta} = -b
\end{align}

Example: A pendulum with friction (damping coefficient $a$) and a constant external torque ($-b$).

\item \stage{STAGE Y (Why it's autonomous):}
The damping coefficient $a$ and forcing $b$ are constant—they don't change with time. The system follows the same rules at all times $t$.

Key properties:
\begin{enumerate}
    \item Time-translation invariant: shifting a solution in time gives another solution
    \item Phase portrait is time-independent
    \item Equilibria are stationary (don't move with $t$)
\end{enumerate}

\item \stage{STAGE Z (Finding equilibrium):}
At equilibrium: $\dot{\theta}_1 = 0$ and $\dot{\theta}_2 = 0$

From the first equation: $\theta_2 = 0$

From the second: $0 = -a(0) - b = -b$

If $b \neq 0$, there's no equilibrium in this system! This represents constant acceleration (if $b \neq 0$).

If $b = 0$: equilibrium at $(\theta_1, \theta_2) = (\theta^*, 0)$ for any $\theta^*$ (line of equilibria).
\end{itemize}

\critical{KEY INSIGHT:} Constant parameters (like $a$ and $b$) don't make a system non-autonomous. Only explicit dependence on the independent variable matters.

\newpage

\section{Part (d): The map $x_{n+1} = ax_n + x_n^2$}

\subsection*{Step 1: Understand Discrete Maps}

\begin{itemize}[leftmargin=*]
\item \stage{STAGE X (What is a map?):}
A discrete dynamical system (or map) evolves in discrete steps rather than continuously:
\begin{align}
x_{n+1} = f(x_n, n)
\end{align}

where:
\begin{itemize}
    \item $n$ is the discrete "time" index ($n = 0, 1, 2, 3, \ldots$)
    \item $x_n$ is the state at step $n$
    \item $x_{n+1}$ is the state at step $n+1$
\end{itemize}

\item \stage{STAGE Y (Independent variable for maps):}
For discrete maps, the independent variable is $n$ (the iteration index), analogous to time $t$ in ODEs.

\item \stage{STAGE Z (Autonomy for maps):}
A map is autonomous if $f$ depends only on $x_n$, not on $n$ explicitly:
\begin{align}
\text{Autonomous:} \quad & x_{n+1} = f(x_n) \\
\text{Non-autonomous:} \quad & x_{n+1} = f(x_n, n)
\end{align}
\end{itemize}

\subsection*{Step 2: Identify the Independent Variable}

\begin{itemize}[leftmargin=*]
\item \stage{STAGE X (Our map):}
\begin{align}
x_{n+1} = ax_n + x_n^2
\end{align}

\item \stage{STAGE Y (Independent variable):}
The subscript $n$ indicates the iteration number. We iterate forward: $n \to n+1$.

\begin{align}
\boxed{\text{Independent variable: } n \text{ (iteration index)}}
\end{align}

\item \stage{STAGE Z (State variable):}
The state variable is $x$ (which takes value $x_n$ at iteration $n$).
\end{itemize}

\subsection*{Step 3: Check for Explicit Dependence on $n$}

\begin{itemize}[leftmargin=*]
\item \stage{STAGE X (Examine the right-hand side):}
\begin{align}
x_{n+1} = ax_n + x_n^2
\end{align}

The right-hand side contains:
\begin{itemize}
    \item $a$: a constant parameter
    \item $x_n$: the state at iteration $n$
    \item $x_n^2$: a function of the state
\end{itemize}

\item \stage{STAGE Y (Does $n$ appear explicitly?):}
NO! The iteration index $n$ only appears as a subscript to indicate "which iteration," not as an explicit variable in the function.

The function is:
\begin{align}
f(x_n) = ax_n + x_n^2
\end{align}

This depends only on the current state $x_n$, not on $n$ itself.

\item \stage{STAGE Z (Conclusion):}
\begin{align}
\boxed{\text{AUTONOMOUS}}
\end{align}

The map is \textbf{autonomous}.
\end{itemize}

\subsection*{Step 4: Understanding Map Dynamics}

\begin{itemize}[leftmargin=*]
\item \stage{STAGE X (Iterative evolution):}
Starting from $x_0$:
\begin{align}
x_1 &= ax_0 + x_0^2 \\
x_2 &= ax_1 + x_1^2 = a(ax_0 + x_0^2) + (ax_0 + x_0^2)^2 \\
x_3 &= ax_2 + x_2^2 \\
&\vdots
\end{align}

\item \stage{STAGE Y (Why it's autonomous):}
The rule for computing $x_{n+1}$ from $x_n$ is the same at every iteration:
\begin{itemize}
    \item The map at $n=0$: $x_1 = ax_0 + x_0^2$
    \item The map at $n=5$: $x_6 = ax_5 + x_5^2$
    \item The map at $n=100$: $x_{101} = ax_{100} + x_{100}^2$
\end{itemize}

All use the same rule! The iteration number $n$ doesn't affect the dynamics.

\item \stage{STAGE Z (Fixed points):}
Fixed points satisfy $x^* = ax^* + (x^*)^2$:
\begin{align}
(x^*)^2 + (a-1)x^* &= 0 \\
x^*(x^* + a - 1) &= 0
\end{align}

Fixed points: $x^* = 0$ and $x^* = 1 - a$.

These don't change with $n$ (time-independent), confirming autonomy.
\end{itemize}

\critical{COMMON CONFUSION:} The subscript $n$ in $x_n$ is just notation for "value at iteration $n$." It doesn't make the system non-autonomous unless $n$ appears explicitly in the function (like $nx_n$ or $\sin(n)$).

\newpage

\section{Part (e): The map $x_{n+1} = nx_n + b$}

\subsection*{Step 1: Identify the Independent Variable}

\begin{itemize}[leftmargin=*]
\item \stage{STAGE X (Our map):}
\begin{align}
x_{n+1} = nx_n + b
\end{align}

\item \stage{STAGE Y (Independent variable):}
As with part (d), the iteration index $n$ is the independent variable:

\begin{align}
\boxed{\text{Independent variable: } n \text{ (iteration index)}}
\end{align}

The state variable is $x$.

\item \stage{STAGE Z (Parameters):}
$b$ is a constant parameter.
\end{itemize}

\subsection*{Step 2: Check for Explicit Dependence on $n$}

\begin{itemize}[leftmargin=*]
\item \stage{STAGE X (Examine the right-hand side):}
\begin{align}
x_{n+1} = nx_n + b
\end{align}

The right-hand side contains:
\begin{itemize}
    \item $n$: the iteration index (independent variable!)
    \item $x_n$: the state at iteration $n$
    \item $b$: a constant
\end{itemize}

\item \stage{STAGE Y (Does $n$ appear explicitly?):}
YES! The iteration number $n$ appears explicitly as a coefficient multiplying $x_n$.

The function is:
\begin{align}
f(x_n, n) = nx_n + b
\end{align}

This depends on \textbf{both} the current state $x_n$ \textbf{and} the iteration number $n$.

\item \stage{STAGE Z (Conclusion):}
\begin{align}
\boxed{\text{NOT AUTONOMOUS}}
\end{align}

The map is \textbf{non-autonomous}.
\end{itemize}

\subsection*{Step 3: Understanding the Non-Autonomous Dynamics}

\begin{itemize}[leftmargin=*]
\item \stage{STAGE X (How the map changes):}
Starting from $x_0$:
\begin{align}
x_1 &= 0 \cdot x_0 + b = b \\
x_2 &= 1 \cdot x_1 + b = 1 \cdot b + b = 2b \\
x_3 &= 2 \cdot x_2 + b = 2(2b) + b = 5b \\
x_4 &= 3 \cdot x_3 + b = 3(5b) + b = 16b \\
&\vdots
\end{align}

\item \stage{STAGE Y (Why it's non-autonomous):}
The "rule" changes at each iteration:
\begin{itemize}
    \item At $n=0$: multiply by $0$ and add $b$
    \item At $n=1$: multiply by $1$ and add $b$
    \item At $n=2$: multiply by $2$ and add $b$
    \item At $n=10$: multiply by $10$ and add $b$
\end{itemize}

The dynamics are \textbf{different at each iteration} because the coefficient $n$ changes. This is the hallmark of non-autonomy.

\item \stage{STAGE Z (Growth behavior):}
As $n$ increases, the coefficient grows, causing increasingly rapid changes in $x_n$. The system "accelerates" over iterations.

For $b > 0$ and typical initial conditions, this map exhibits explosive growth as $n \to \infty$.
\end{itemize}

\subsection*{Step 4: Comparison with Part (d)}

\begin{itemize}[leftmargin=*]
\item \stage{STAGE X (Side-by-side comparison):}
\begin{align}
\text{Part (d):} \quad & x_{n+1} = ax_n + x_n^2 \quad \text{(autonomous)} \\
\text{Part (e):} \quad & x_{n+1} = nx_n + b \quad \text{(non-autonomous)}
\end{align}

\item \stage{STAGE Y (The crucial difference):}
\begin{itemize}
    \item In (d): $a$ is a \textbf{constant}—same at all iterations
    \item In (e): $n$ is the \textbf{iteration index}—changes with each step
\end{itemize}

Part (d) uses a constant parameter $a$, while part (e) uses the variable $n$ that increases with each iteration.

\item \stage{STAGE Z (Physical analogy):}
\begin{itemize}
    \item Part (d): Like a system with constant friction coefficient
    \item Part (e): Like a system where friction increases with each time step
\end{itemize}

The changing "rules" in (e) make it non-autonomous.
\end{itemize}

\critical{KEY DISTINCTION:}
\begin{itemize}
    \item $x_{n+1} = ax_n + b$ with constant $a$ $\Rightarrow$ AUTONOMOUS
    \item $x_{n+1} = nx_n + b$ with variable $n$ $\Rightarrow$ NON-AUTONOMOUS
\end{itemize}

The explicit appearance of $n$ (not just as a subscript) makes all the difference!

\newpage

\section{Summary Table and Final Comparison}

\subsection*{Complete Results}

\begin{center}
\begin{tabular}{|c|l|c|c|c|}
\hline
\textbf{Part} & \textbf{System} & \textbf{Independent Var} & \textbf{Autonomous?} & \textbf{Reason} \\
\hline
(a) & $\ddot{u} = u + \sin(t)$ & $t$ & NO & $t$ in $\sin(t)$ \\
\hline
(b) & $y'' - y - \sin(x) = 0$ & $x$ & NO & $x$ in $\sin(x)$ \\
\hline
(c) & $\ddot{\theta} + a\dot{\theta} + b = 0$ & $t$ & YES & No explicit $t$ \\
\hline
(d) & $x_{n+1} = ax_n + x_n^2$ & $n$ & YES & No explicit $n$ \\
\hline
(e) & $x_{n+1} = nx_n + b$ & $n$ & NO & $n$ multiplies $x_n$ \\
\hline
\end{tabular}
\end{center}

\subsection*{Pattern Recognition}

\begin{itemize}[leftmargin=*]
\item \stage{STAGE X (Common patterns for NON-autonomous):}
\begin{enumerate}
    \item Time-varying forcing: $f(x, t) = \text{function}(x) + g(t)$

    Examples: $\sin(t)$, $e^t$, $t^2$, etc.

    \item Time-varying coefficients: $f(x, t) = h(t) \cdot x$

    Examples: $tx$, $\sin(t) \cdot x$, etc.

    \item For maps: Iteration-dependent terms

    Examples: $nx_n$, $\sin(n)$, $n^2$, etc.
\end{enumerate}

\item \stage{STAGE Y (Common patterns for AUTONOMOUS):}
\begin{enumerate}
    \item Constant coefficients: $f(x) = ax + bx^2 + c$
    \item State-dependent only: $f(x) = \sin(x)$, $x^3$, $e^x$, etc.
    \item No explicit time/iteration dependence
\end{enumerate}

\item \stage{STAGE Z (How to avoid confusion):}
\begin{enumerate}
    \item First, identify the independent variable clearly
    \item Then scan the right-hand side for explicit occurrences
    \item Remember: subscripts are just notation, not explicit dependence
    \item Constants/parameters don't count as variables
\end{enumerate}
\end{itemize}

\subsection*{Connection to Course Material}

\begin{itemize}[leftmargin=*]
\item \stage{STAGE X (Lecture notes context):}
The lecture notes primarily focus on \textbf{autonomous systems} because:
\begin{enumerate}
    \item They are simpler to analyze
    \item Phase portraits are time-independent
    \item Equilibrium analysis is straightforward
    \item They capture essential dynamics of many physical systems
\end{enumerate}

\item \stage{STAGE Y (When non-autonomy arises):}
Non-autonomous systems appear when:
\begin{enumerate}
    \item External forcing varies with time (e.g., seasonal effects, periodic driving)
    \item Parameters change with time (e.g., aging, growth)
    \item Boundary conditions move (e.g., moving walls)
    \item Control inputs are time-dependent
\end{enumerate}

\item \stage{STAGE Z (Converting non-autonomous to autonomous):}
A non-autonomous ODE can sometimes be made autonomous by introducing time as a state variable:
\begin{align}
\dot{x} &= f(x, t) \quad \text{(non-autonomous)}
\end{align}

Introduce $y = t$ with $\dot{y} = 1$:
\begin{align}
\dot{x} &= f(x, y) \\
\dot{y} &= 1 \quad \text{(autonomous in } (x,y) \text{ space!)}
\end{align}

This trick increases the dimension but makes the system autonomous.
\end{itemize}

\critical{EXAM STRATEGY:}
\begin{enumerate}
    \item Always identify the independent variable first
    \item Look for explicit appearances in the function
    \item Constants and parameters don't make systems non-autonomous
    \item Subscripts in maps ($x_n$) are just notation
\end{enumerate}

\vfill

\begin{center}
\Large\textbf{END OF QUESTION 3}
\end{center}

\end{document}
