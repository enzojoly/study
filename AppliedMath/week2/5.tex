\documentclass[12pt,a4paper]{article}

% Packages
\usepackage{amsmath}
\usepackage{amssymb}
\usepackage{amsthm}
\usepackage[margin=1in]{geometry}
\usepackage{enumitem}
\usepackage{xcolor}
\usepackage{mathtools}
\usepackage{tikz}
\usepackage{tikz-3dplot}
\usetikzlibrary{arrows,decorations.markings}

% Custom environments
\newtheorem{explanation}{Explanation}
\theoremstyle{definition}
\newtheorem{solution}{Solution}

% Custom commands
\newcommand{\stage}[1]{\textbf{\textcolor{blue}{#1}}}
\newcommand{\RR}{\mathbb{R}}
\newcommand{\CC}{\mathbb{C}}

% Title information
\title{Methods of Applied Mathematics - Part 1\\
Exercise Sheet 2: Question 5\\
Classification of Equilibria in 3D}
\author{Complete Solution with XYZ Methodology}
\date{}

\begin{document}

\maketitle

\section*{Problem Statement}

Classify all hyperbolic equilibria of a linear vector field in three dimensions, i.e., draw phase portraits for all topologically different cases when the origin is a hyperbolic equilibrium of the vector field.

\textit{Hint: Start from the 2D cases (e.g., attracting node, attracting spiral, saddle, etc.), and bear in mind that a 3D system has 3 eigenvalues; where in the complex plane can they be?}

\section{Step 1: Foundation - Definition and Constraints}

\begin{solution}

\subsection*{Define the System and Hyperbolicity}

\begin{itemize}[leftmargin=*]
\item \stage{STAGE X (What we have):} A linear 3D vector field near the origin:
\begin{equation}
\dot{\mathbf{x}} = A\mathbf{x}, \quad \mathbf{x} \in \RR^3
\end{equation}
where $A$ is a $3 \times 3$ real matrix with equilibrium at $\mathbf{x}^* = \mathbf{0}$.

\item \stage{STAGE Y (Why hyperbolicity matters):} From Lecture Notes (Section 11, page 38), an equilibrium is \textbf{hyperbolic} if none of its eigenvalues lie on the imaginary axis, i.e., $\text{Re}(\lambda_i) \neq 0$ for all $i$.

\textbf{Significance:} Hyperbolic equilibria are structurally stable - small perturbations don't change their topological type. The Hartman-Grobman Theorem (page 38) guarantees that the nonlinear system near a hyperbolic equilibrium is topologically equivalent to its linearization.

\item \stage{STAGE Z (Our approach):} We'll systematically enumerate all possible eigenvalue configurations for a $3 \times 3$ real matrix, excluding non-hyperbolic cases (eigenvalues on imaginary axis).
\end{itemize}

\subsection*{Fundamental Constraints on Eigenvalues}

\begin{explanation}[Eigenvalue Structure for Real Matrices]
For a real matrix $A \in \RR^{3 \times 3}$:

\textbf{Complex Conjugate Pairs:}
\begin{itemize}
\item Complex eigenvalues must occur in conjugate pairs: if $\lambda = a + bi$ is an eigenvalue, then $\bar{\lambda} = a - bi$ is also an eigenvalue
\item This is because the characteristic polynomial has real coefficients
\end{itemize}

\textbf{Parity Constraint:}
\begin{itemize}
\item A $3 \times 3$ matrix has exactly 3 eigenvalues (counting multiplicity)
\item Complex eigenvalues come in pairs (even count)
\item Therefore: Either \textbf{3 real} eigenvalues OR \textbf{1 real + 2 complex conjugate} eigenvalues
\item Cannot have 3 complex eigenvalues (would need 4 or 6 with conjugate pairing)
\end{itemize}
\end{explanation}

\subsection*{Eigenvalue Location in Complex Plane}

From Lecture Notes (Section 8, pages 29-31):
\begin{itemize}
\item Eigenvalues with $\text{Re}(\lambda) < 0$: contribute \textbf{stable} directions (attraction)
\item Eigenvalues with $\text{Re}(\lambda) > 0$: contribute \textbf{unstable} directions (repulsion)
\item Real eigenvalues: exponential approach/departure without rotation
\item Complex eigenvalues: spiral approach/departure with frequency $|\text{Im}(\lambda)|$
\end{itemize}

\end{solution}

\vspace{10pt}
\hrule
\vspace{10pt}

\section{Step 2: Enumeration Strategy}

\begin{solution}

\subsection*{Classification Tree}

\begin{itemize}[leftmargin=*]
\item \stage{STAGE X (Systematic approach):} We classify by eigenvalue structure:

\begin{enumerate}
\item \textbf{Case A}: Three real eigenvalues $\lambda_1, \lambda_2, \lambda_3 \in \RR$
\item \textbf{Case B}: One real + two complex conjugate $\lambda_1 \in \RR$, $\lambda_{2,3} = a \pm bi$ with $b \neq 0$
\end{enumerate}

\item \stage{STAGE Y (Why this suffices):} These two cases exhaust all possibilities for a $3 \times 3$ real matrix. Within each case, we further classify by the signs of the real parts, which determine stability.

\item \stage{STAGE Z (Counting hyperbolic types):}
\begin{itemize}
\item Case A: $2^3 = 8$ sign combinations, but exclude all-zero (non-hyperbolic) $\Rightarrow$ actually we have 4 distinct topological types
\item Case B: $2 \times 2 = 4$ sign combinations for (real eigenvalue sign) $\times$ (complex real part sign)
\item \textbf{Total}: 8 topologically distinct hyperbolic equilibria in 3D
\end{itemize}
\end{itemize}

\end{solution}

\vspace{10pt}
\hrule
\vspace{10pt}

\section{Step 3: Case A - Three Real Eigenvalues}

\begin{solution}

For three real eigenvalues $\lambda_1, \lambda_2, \lambda_3$, hyperbolicity requires all $\lambda_i \neq 0$.

\subsection*{Subcase A1: All Three Negative ($\lambda_1, \lambda_2, \lambda_3 < 0$)}

\begin{itemize}[leftmargin=*]
\item \stage{STAGE X (Eigenvalue configuration):}
\begin{equation}
\lambda_1 < 0, \quad \lambda_2 < 0, \quad \lambda_3 < 0
\end{equation}
Example: $\lambda_1 = -3, \lambda_2 = -2, \lambda_3 = -1$

\item \stage{STAGE Y (Why this gives stability):} All three eigendirections attract. The general solution is:
\begin{equation}
\mathbf{x}(t) = c_1 e^{\lambda_1 t} \mathbf{v}_1 + c_2 e^{\lambda_2 t} \mathbf{v}_2 + c_3 e^{\lambda_3 t} \mathbf{v}_3
\end{equation}
Since $\lambda_i < 0$ for all $i$, we have $e^{\lambda_i t} \to 0$ as $t \to \infty$, so $\mathbf{x}(t) \to \mathbf{0}$.

\item \stage{STAGE Z (Classification):} \textbf{STABLE NODE} (or attracting node)
\end{itemize}

\textbf{Stability Manifolds:}
\begin{itemize}
\item Stable manifold: $W^s = \RR^3$ (entire space)
\item Unstable manifold: $W^u = \{\mathbf{0}\}$ (just the origin)
\item Dimensions: $\dim(W^s) = 3$, $\dim(W^u) = 0$
\end{itemize}

\textbf{Phase Portrait Description:}
\begin{itemize}
\item All trajectories approach the origin
\item Fastest approach along eigenvector with most negative $\lambda$ (largest $|\lambda|$)
\item Slowest approach along eigenvector with least negative $\lambda$ (smallest $|\lambda|$)
\item No rotation - purely exponential decay
\end{itemize}

\begin{center}
\textit{[Phase portrait: 3D stable node - all arrows point toward origin from all directions]}
\end{center}

\vspace{10pt}
\hrule
\vspace{5pt}

\subsection*{Subcase A2: All Three Positive ($\lambda_1, \lambda_2, \lambda_3 > 0$)}

\begin{itemize}[leftmargin=*]
\item \stage{STAGE X (Eigenvalue configuration):}
\begin{equation}
\lambda_1 > 0, \quad \lambda_2 > 0, \quad \lambda_3 > 0
\end{equation}
Example: $\lambda_1 = 1, \lambda_2 = 2, \lambda_3 = 3$

\item \stage{STAGE Y (Why this gives instability):} All three eigendirections repel. The solution:
\begin{equation}
\mathbf{x}(t) = c_1 e^{\lambda_1 t} \mathbf{v}_1 + c_2 e^{\lambda_2 t} \mathbf{v}_2 + c_3 e^{\lambda_3 t} \mathbf{v}_3
\end{equation}
Since $\lambda_i > 0$ for all $i$, we have $e^{\lambda_i t} \to \infty$ as $t \to \infty$, so $\mathbf{x}(t) \to \infty$.

\item \stage{STAGE Z (Classification):} \textbf{UNSTABLE NODE} (or repelling node)
\end{itemize}

\textbf{Stability Manifolds:}
\begin{itemize}
\item Stable manifold: $W^s = \{\mathbf{0}\}$ (just the origin)
\item Unstable manifold: $W^u = \RR^3$ (entire space)
\item Dimensions: $\dim(W^s) = 0$, $\dim(W^u) = 3$
\end{itemize}

\textbf{Phase Portrait Description:}
\begin{itemize}
\item All trajectories repel from the origin (except $\mathbf{x} = \mathbf{0}$)
\item Fastest escape along eigenvector with most positive $\lambda$
\item Slowest escape along eigenvector with least positive $\lambda$
\item Time-reversal of stable node
\end{itemize}

\begin{center}
\textit{[Phase portrait: 3D unstable node - all arrows point away from origin in all directions]}
\end{center}

\vspace{10pt}
\hrule
\vspace{5pt}

\subsection*{Subcase A3: Two Negative, One Positive}

\begin{itemize}[leftmargin=*]
\item \stage{STAGE X (Eigenvalue configuration):}
\begin{equation}
\lambda_1 < 0, \quad \lambda_2 < 0, \quad \lambda_3 > 0
\end{equation}
Example: $\lambda_1 = -2, \lambda_2 = -1, \lambda_3 = 3$

\item \stage{STAGE Y (Why this gives saddle):} Two directions attract, one repels. The solution:
\begin{equation}
\mathbf{x}(t) = c_1 e^{\lambda_1 t} \mathbf{v}_1 + c_2 e^{\lambda_2 t} \mathbf{v}_2 + c_3 e^{\lambda_3 t} \mathbf{v}_3
\end{equation}
\begin{itemize}
\item If $c_3 = 0$: trajectory stays in span$\{\mathbf{v}_1, \mathbf{v}_2\}$ and decays to origin (2D stable)
\item If $c_3 \neq 0$: $e^{\lambda_3 t}$ term dominates for large $t$, trajectory escapes along $\mathbf{v}_3$
\end{itemize}

\item \stage{STAGE Z (Classification):} \textbf{SADDLE} with 2D stable manifold, 1D unstable manifold
\end{itemize}

\textbf{Stability Manifolds:}
\begin{itemize}
\item Stable manifold: $W^s = \text{span}\{\mathbf{v}_1, \mathbf{v}_2\}$ (2D plane)
\item Unstable manifold: $W^u = \text{span}\{\mathbf{v}_3\}$ (1D line)
\item Dimensions: $\dim(W^s) = 2$, $\dim(W^u) = 1$
\end{itemize}

\textbf{Phase Portrait Description:}
\begin{itemize}
\item Trajectories starting in $W^s$ approach origin
\item Trajectories starting on $W^u$ (except origin) escape along the line
\item Generic trajectories: approach the 2D stable manifold, then follow it toward origin, but get deflected and escape along unstable direction
\item Creates characteristic "saddle surface"
\end{itemize}

\begin{center}
\textit{[Phase portrait: 3D saddle - 2D stable plane, 1D unstable line perpendicular]}
\end{center}

\textbf{Notation:} Sometimes denoted as $(2,1)$-saddle (2 stable dimensions, 1 unstable dimension).

\vspace{10pt}
\hrule
\vspace{5pt}

\subsection*{Subcase A4: One Negative, Two Positive}

\begin{itemize}[leftmargin=*]
\item \stage{STAGE X (Eigenvalue configuration):}
\begin{equation}
\lambda_1 < 0, \quad \lambda_2 > 0, \quad \lambda_3 > 0
\end{equation}
Example: $\lambda_1 = -3, \lambda_2 = 1, \lambda_3 = 2$

\item \stage{STAGE Y (Why this gives saddle):} One direction attracts, two repel. The solution:
\begin{equation}
\mathbf{x}(t) = c_1 e^{\lambda_1 t} \mathbf{v}_1 + c_2 e^{\lambda_2 t} \mathbf{v}_2 + c_3 e^{\lambda_3 t} \mathbf{v}_3
\end{equation}
\begin{itemize}
\item If $c_2 = c_3 = 0$: trajectory stays on $\mathbf{v}_1$ line and approaches origin
\item Otherwise: $e^{\lambda_2 t}$ and $e^{\lambda_3 t}$ terms dominate, trajectory escapes in 2D plane
\end{itemize}

\item \stage{STAGE Z (Classification):} \textbf{SADDLE} with 1D stable manifold, 2D unstable manifold
\end{itemize}

\textbf{Stability Manifolds:}
\begin{itemize}
\item Stable manifold: $W^s = \text{span}\{\mathbf{v}_1\}$ (1D line)
\item Unstable manifold: $W^u = \text{span}\{\mathbf{v}_2, \mathbf{v}_3\}$ (2D plane)
\item Dimensions: $\dim(W^s) = 1$, $\dim(W^u) = 2$
\end{itemize}

\textbf{Phase Portrait Description:}
\begin{itemize}
\item Trajectories starting on $W^s$ approach origin along the line
\item Trajectories in $W^u$ escape (except origin)
\item Generic trajectories: initially move toward the stable line, but get deflected and escape in the 2D unstable plane
\item Time-reversal of $(2,1)$-saddle
\end{itemize}

\begin{center}
\textit{[Phase portrait: 3D saddle - 1D stable line, 2D unstable plane perpendicular]}
\end{center}

\textbf{Notation:} Sometimes denoted as $(1,2)$-saddle (1 stable dimension, 2 unstable dimensions).

\end{solution}

\vspace{10pt}
\hrule
\vspace{10pt}

\section{Step 4: Case B - One Real + Complex Conjugate Pair}

\begin{solution}

For eigenvalues $\lambda_1 \in \RR$ and $\lambda_{2,3} = a \pm bi$ with $b \neq 0$:

\subsection*{Key Concepts for Complex Eigenvalues}

\begin{explanation}[Complex Eigenvalues and Spiraling]
From Lecture Notes (Section 7, pages 26-27):

When eigenvalues are complex, $\lambda = a \pm bi$:
\begin{itemize}
\item The real part $a = \text{Re}(\lambda)$ controls stability: $a < 0$ attracts, $a > 0$ repels
\item The imaginary part $b = \text{Im}(\lambda)$ controls rotation frequency: $\omega = |b|$
\item Solutions in the complex eigenspace spiral: $e^{(a+bi)t} = e^{at}(\cos(bt) + i\sin(bt))$
\item This corresponds to spiraling in a 2D real plane (the real and imaginary parts of the eigenvector)
\end{itemize}

The general solution in the 2D invariant plane is:
\begin{equation}
\mathbf{x}_{\text{plane}}(t) = e^{at}\begin{pmatrix} \cos(bt) & -\sin(bt) \\ \sin(bt) & \cos(bt) \end{pmatrix} \mathbf{x}_0
\end{equation}
This describes spiraling motion with exponential growth/decay.
\end{explanation}

\subsection*{Subcase B1: Real Negative, Complex with Negative Real Part}

\begin{itemize}[leftmargin=*]
\item \stage{STAGE X (Eigenvalue configuration):}
\begin{equation}
\lambda_1 < 0 \in \RR, \quad \lambda_{2,3} = a \pm bi \text{ with } a < 0, \, b \neq 0
\end{equation}
Example: $\lambda_1 = -2$, $\lambda_{2,3} = -1 \pm 3i$

\item \stage{STAGE Y (Why this gives stable spiral node):}
\begin{itemize}
\item Real eigenvalue $\lambda_1 < 0$: exponential decay along 1D line (direction $\mathbf{v}_1$)
\item Complex pair with $a < 0$: spiral decay in 2D plane (spanned by $\text{Re}(\mathbf{v}_2)$, $\text{Im}(\mathbf{v}_2)$)
\item Both components attract to origin
\end{itemize}

General solution:
\begin{equation}
\mathbf{x}(t) = c_1 e^{\lambda_1 t} \mathbf{v}_1 + e^{at}[c_2 \cos(bt) + c_3 \sin(bt)] \mathbf{u} + e^{at}[c_2 \sin(bt) - c_3 \cos(bt)] \mathbf{w}
\end{equation}
where $\mathbf{u}, \mathbf{w}$ span the 2D complex eigenspace.

\item \stage{STAGE Z (Classification):} \textbf{STABLE SPIRAL NODE} (or stable focus-node)
\end{itemize}

\textbf{Stability Manifolds:}
\begin{itemize}
\item Stable manifold: $W^s = \RR^3$ (entire space)
\item Unstable manifold: $W^u = \{\mathbf{0}\}$ (just the origin)
\item Dimensions: $\dim(W^s) = 3$, $\dim(W^u) = 0$
\end{itemize}

\textbf{Phase Portrait Description:}
\begin{itemize}
\item All trajectories spiral into the origin
\item In 2D plane: inward spiral (focus behavior)
\item Along 3rd direction: exponential decay (node behavior)
\item Combined: 3D spiral converging to origin
\item Frequency of rotation: $\omega = |b|$
\end{itemize}

\begin{center}
\textit{[Phase portrait: 3D stable spiral - trajectories spiral inward like a corkscrew toward origin]}
\end{center}

\vspace{10pt}
\hrule
\vspace{5pt}

\subsection*{Subcase B2: Real Positive, Complex with Positive Real Part}

\begin{itemize}[leftmargin=*]
\item \stage{STAGE X (Eigenvalue configuration):}
\begin{equation}
\lambda_1 > 0 \in \RR, \quad \lambda_{2,3} = a \pm bi \text{ with } a > 0, \, b \neq 0
\end{equation}
Example: $\lambda_1 = 2$, $\lambda_{2,3} = 1 \pm 3i$

\item \stage{STAGE Y (Why this gives unstable spiral node):}
\begin{itemize}
\item Real eigenvalue $\lambda_1 > 0$: exponential growth along 1D line
\item Complex pair with $a > 0$: spiral growth in 2D plane
\item Both components repel from origin
\end{itemize}

\item \stage{STAGE Z (Classification):} \textbf{UNSTABLE SPIRAL NODE} (or unstable focus-node)
\end{itemize}

\textbf{Stability Manifolds:}
\begin{itemize}
\item Stable manifold: $W^s = \{\mathbf{0}\}$ (just the origin)
\item Unstable manifold: $W^u = \RR^3$ (entire space)
\item Dimensions: $\dim(W^s) = 0$, $\dim(W^u) = 3$
\end{itemize}

\textbf{Phase Portrait Description:}
\begin{itemize}
\item All trajectories spiral away from the origin
\item In 2D plane: outward spiral
\item Along 3rd direction: exponential growth
\item Combined: 3D spiral diverging from origin
\item Time-reversal of stable spiral node
\end{itemize}

\begin{center}
\textit{[Phase portrait: 3D unstable spiral - trajectories spiral outward like expanding corkscrew]}
\end{center}

\vspace{10pt}
\hrule
\vspace{5pt}

\subsection*{Subcase B3: Real Negative, Complex with Positive Real Part}

\begin{itemize}[leftmargin=*]
\item \stage{STAGE X (Eigenvalue configuration):}
\begin{equation}
\lambda_1 < 0 \in \RR, \quad \lambda_{2,3} = a \pm bi \text{ with } a > 0, \, b \neq 0
\end{equation}
Example: $\lambda_1 = -2$, $\lambda_{2,3} = 1 \pm 3i$

\item \stage{STAGE Y (Why this gives saddle-focus):}
\begin{itemize}
\item Real eigenvalue $\lambda_1 < 0$: attracts along 1D line
\item Complex pair with $a > 0$: spirals outward in 2D plane
\item Mixed behavior: attraction in one direction, spiraling repulsion in plane
\end{itemize}

\item \stage{STAGE Z (Classification):} \textbf{SADDLE-FOCUS} with 1D stable, 2D unstable spiral
\end{itemize}

\textbf{Stability Manifolds:}
\begin{itemize}
\item Stable manifold: $W^s = \text{span}\{\mathbf{v}_1\}$ (1D line)
\item Unstable manifold: $W^u = \text{span}\{\text{Re}(\mathbf{v}_2), \text{Im}(\mathbf{v}_2)\}$ (2D plane, spiral structure)
\item Dimensions: $\dim(W^s) = 1$, $\dim(W^u) = 2$
\end{itemize}

\textbf{Phase Portrait Description:}
\begin{itemize}
\item Trajectories on $W^s$ approach origin along the line
\item Trajectories in $W^u$ spiral away from origin
\item Generic trajectories: initially attracted toward stable line, but deflected by spiraling unstable plane, eventually escape while spiraling
\item Creates characteristic "spiral saddle" or "saddle-focus"
\end{itemize}

\begin{center}
\textit{[Phase portrait: Saddle-focus - 1D stable line with 2D unstable spiral plane perpendicular]}
\end{center}

\textbf{Notation:} $(1,2)$-saddle-focus or saddle-focus with 1D stable manifold.

\begin{explanation}[Homoclinic Connections and Chaos]
From Lecture Notes (Section 10, page 36): The Shilnikov bifurcation involves a saddle-focus equilibrium where a 1D unstable manifold connects back to the 2D stable manifold (homoclinic connection). This configuration is associated with chaotic dynamics in certain parameter regimes.
\end{explanation}

\vspace{10pt}
\hrule
\vspace{5pt}

\subsection*{Subcase B4: Real Positive, Complex with Negative Real Part}

\begin{itemize}[leftmargin=*]
\item \stage{STAGE X (Eigenvalue configuration):}
\begin{equation}
\lambda_1 > 0 \in \RR, \quad \lambda_{2,3} = a \pm bi \text{ with } a < 0, \, b \neq 0
\end{equation}
Example: $\lambda_1 = 2$, $\lambda_{2,3} = -1 \pm 3i$

\item \stage{STAGE Y (Why this gives saddle-focus):}
\begin{itemize}
\item Real eigenvalue $\lambda_1 > 0$: repels along 1D line
\item Complex pair with $a < 0$: spirals inward in 2D plane
\item Mixed behavior: repulsion in one direction, spiraling attraction in plane
\end{itemize}

\item \stage{STAGE Z (Classification):} \textbf{SADDLE-FOCUS} with 2D stable spiral, 1D unstable
\end{itemize}

\textbf{Stability Manifolds:}
\begin{itemize}
\item Stable manifold: $W^s = \text{span}\{\text{Re}(\mathbf{v}_2), \text{Im}(\mathbf{v}_2)\}$ (2D plane, spiral structure)
\item Unstable manifold: $W^u = \text{span}\{\mathbf{v}_1\}$ (1D line)
\item Dimensions: $\dim(W^s) = 2$, $\dim(W^u) = 1$
\end{itemize}

\textbf{Phase Portrait Description:}
\begin{itemize}
\item Trajectories on $W^u$ escape from origin along the line
\item Trajectories in $W^s$ spiral into origin
\item Generic trajectories: initially repelled along unstable line, but attracted by spiraling stable plane, eventually spiral into origin
\item Time-reversal of $(1,2)$-saddle-focus
\end{itemize}

\begin{center}
\textit{[Phase portrait: Saddle-focus - 2D stable spiral plane with 1D unstable line perpendicular]}
\end{center}

\textbf{Notation:} $(2,1)$-saddle-focus or saddle-focus with 2D stable manifold.

\end{solution}

\vspace{10pt}
\hrule
\vspace{10pt}

\section{Step 5: Complete Classification Summary}

\begin{solution}

\subsection*{All Eight Hyperbolic Equilibrium Types in 3D}

\begin{center}
\renewcommand{\arraystretch}{1.5}
\begin{tabular}{|c|l|c|c|c|}
\hline
\textbf{Type} & \textbf{Eigenvalue Configuration} & $\dim(W^s)$ & $\dim(W^u)$ & \textbf{Name} \\
\hline
\hline
\multicolumn{5}{|c|}{\textbf{Case A: Three Real Eigenvalues}} \\
\hline
A1 & $\lambda_1, \lambda_2, \lambda_3 < 0$ & 3 & 0 & Stable Node \\
\hline
A2 & $\lambda_1, \lambda_2, \lambda_3 > 0$ & 0 & 3 & Unstable Node \\
\hline
A3 & $\lambda_1, \lambda_2 < 0, \, \lambda_3 > 0$ & 2 & 1 & $(2,1)$-Saddle \\
\hline
A4 & $\lambda_1 < 0, \, \lambda_2, \lambda_3 > 0$ & 1 & 2 & $(1,2)$-Saddle \\
\hline
\hline
\multicolumn{5}{|c|}{\textbf{Case B: One Real + Complex Conjugate Pair}} \\
\hline
B1 & $\lambda_1 < 0$, $\lambda_{2,3} = a \pm bi$, $a < 0$ & 3 & 0 & Stable Spiral Node \\
\hline
B2 & $\lambda_1 > 0$, $\lambda_{2,3} = a \pm bi$, $a > 0$ & 0 & 3 & Unstable Spiral Node \\
\hline
B3 & $\lambda_1 < 0$, $\lambda_{2,3} = a \pm bi$, $a > 0$ & 1 & 2 & Saddle-Focus $(1,2)$ \\
\hline
B4 & $\lambda_1 > 0$, $\lambda_{2,3} = a \pm bi$, $a < 0$ & 2 & 1 & Saddle-Focus $(2,1)$ \\
\hline
\end{tabular}
\end{center}

\subsection*{Dimensional Analysis Verification}

\begin{explanation}[Verification via Stable/Unstable Manifold Dimensions]
For each equilibrium, verify that dimensions sum correctly:
\begin{equation}
\dim(W^s) + \dim(W^u) = \text{dimension of phase space} = 3
\end{equation}

Checking each case:
\begin{itemize}
\item A1: $3 + 0 = 3$ \checkmark
\item A2: $0 + 3 = 3$ \checkmark
\item A3: $2 + 1 = 3$ \checkmark
\item A4: $1 + 2 = 3$ \checkmark
\item B1: $3 + 0 = 3$ \checkmark
\item B2: $0 + 3 = 3$ \checkmark
\item B3: $1 + 2 = 3$ \checkmark
\item B4: $2 + 1 = 3$ \checkmark
\end{itemize}

All cases satisfy the dimension requirement.
\end{explanation}

\subsection*{Topological Equivalence Classes}

\begin{explanation}[When Are Two Equilibria Topologically Equivalent?]
From Lecture Notes (Section 11, page 38), two linear systems $\dot{\mathbf{x}} = A\mathbf{x}$ and $\dot{\mathbf{y}} = B\mathbf{y}$ are topologically equivalent if and only if:
\begin{equation}
n_+(A) = n_+(B) \quad \text{and} \quad n_-(A) = n_-(B)
\end{equation}
where $n_+$ = number of eigenvalues with positive real part, $n_-$ = number with negative real part.

\textbf{This means:} Only the \textit{count} of positive/negative eigenvalues matters for topological equivalence, not:
\begin{itemize}
\item The specific values of eigenvalues
\item Whether eigenvalues are real or complex
\end{itemize}

However, we distinguish real vs. complex for \textit{qualitative behavior} (spiraling vs. not).
\end{explanation}

\subsection*{Relationship to 2D Classification}

From the hint in the problem and Lecture Notes (Section 8):

\textbf{2D Equilibrium Types:}
\begin{itemize}
\item Stable/Unstable Node: 2 real eigenvalues, same sign
\item Saddle: 2 real eigenvalues, opposite signs
\item Stable/Unstable Focus: 2 complex conjugate eigenvalues
\item Center: 2 purely imaginary eigenvalues (not hyperbolic)
\end{itemize}

\textbf{3D as Extension of 2D:}
\begin{itemize}
\item Types A1, A2: "Node" extended to 3D (all eigenvalues same sign)
\item Types A3, A4: "Saddle" extended to 3D (mixed eigenvalue signs)
\item Types B1, B2: "Focus/Spiral" extended to 3D (complex pair + real with same sign)
\item Types B3, B4: "Saddle-Focus" - unique to 3D+ (complex pair + real with opposite sign)
\end{itemize}

The saddle-focus types (B3, B4) \textbf{cannot occur in 2D} - they require at least 3 dimensions.

\end{solution}

\vspace{10pt}
\hrule
\vspace{10pt}

\section{Step 6: Geometric Visualization Guide}

\begin{solution}

\subsection*{How to Sketch Phase Portraits}

For each equilibrium type, follow this procedure:

\textbf{Step 1: Identify Eigenspaces}
\begin{itemize}
\item Real eigenvalues: draw eigenvector lines/planes
\item Complex eigenvalues: identify 2D invariant plane
\end{itemize}

\textbf{Step 2: Draw Stable/Unstable Manifolds}
\begin{itemize}
\item $W^s$: manifold where trajectories approach origin as $t \to +\infty$
\item $W^u$: manifold where trajectories approach origin as $t \to -\infty$ (equivalently, leave origin as $t \to +\infty$)
\end{itemize}

\textbf{Step 3: Add Trajectory Arrows}
\begin{itemize}
\item On $W^s$: arrows point toward origin
\item On $W^u$: arrows point away from origin
\item Off manifolds: show typical trajectory behavior
\end{itemize}

\textbf{Step 4: Indicate Spiraling (if complex eigenvalues present)}
\begin{itemize}
\item Draw spiral curves in the 2D invariant plane
\item Indicate frequency with spiral tightness (higher $|b|$ means more rotations)
\end{itemize}

\subsection*{Key Features to Highlight}

\begin{itemize}
\item \textbf{Stable Node (A1)}: All arrows inward, no preferred direction except rate differences
\item \textbf{Unstable Node (A2)}: All arrows outward, time-reversal of A1
\item \textbf{$(2,1)$-Saddle (A3)}: 2D attracting plane, 1D repelling line - classic saddle
\item \textbf{$(1,2)$-Saddle (A4)}: 1D attracting line, 2D repelling plane - inverted saddle
\item \textbf{Stable Spiral Node (B1)}: Inward spiraling corkscrew, all trajectories converge with rotation
\item \textbf{Unstable Spiral Node (B2)}: Outward spiraling corkscrew, all trajectories diverge with rotation
\item \textbf{Saddle-Focus $( 1,2)$ (B3)}: 1D stable line, 2D unstable spiral - trajectories escape while spiraling
\item \textbf{Saddle-Focus $(2,1)$ (B4)}: 2D stable spiral, 1D unstable line - trajectories approach while spiraling around unstable line
\end{itemize}

\end{solution}

\vspace{10pt}
\hrule
\vspace{10pt}

\section*{Final Summary and Key Insights}

\subsection*{Complete Answer to Question 5}

\begin{enumerate}[leftmargin=*]
\item \textbf{Total Count}: There are \textbf{exactly 8 topologically distinct types} of hyperbolic equilibria for 3D linear systems

\item \textbf{Eigenvalue Constraints}:
\begin{itemize}
\item 3D real matrix $\Rightarrow$ either 3 real eigenvalues OR 1 real + 2 complex conjugate
\item Hyperbolic $\Rightarrow$ all eigenvalues off imaginary axis ($\text{Re}(\lambda) \neq 0$)
\end{itemize}

\item \textbf{Classification Principle}: Determined by:
\begin{itemize}
\item Number of eigenvalues with $\text{Re}(\lambda) > 0$ vs. $\text{Re}(\lambda) < 0$
\item Whether eigenvalues are real or complex
\end{itemize}

\item \textbf{Stability Manifold Dimensions}: Always $\dim(W^s) + \dim(W^u) = 3$

\item \textbf{New Phenomena in 3D}: The saddle-focus types (B3, B4) are unique to dimensions $\geq 3$ and cannot occur in 2D systems
\end{enumerate}

\subsection*{Connection to Lecture Material}

\begin{itemize}[leftmargin=*]
\item \textbf{Section 7-8 (pages 24-31)}: Eigenvalue analysis, stable/unstable manifolds, node/saddle/focus classification
\item \textbf{Section 10 (page 34)}: Stable/unstable manifolds in higher dimensions
\item \textbf{Section 11 (pages 37-38)}: Topological equivalence and hyperbolicity
\item \textbf{Hartman-Grobman Theorem (page 38)}: Guarantees local topological equivalence to linearization for hyperbolic equilibria
\end{itemize}

\subsection*{Practical Importance}

Understanding 3D equilibrium classification is essential for:
\begin{itemize}
\item Analyzing dynamical systems in mechanics, electronics, population dynamics
\item Predicting long-term behavior from eigenvalue calculations
\item Identifying bifurcations (transitions between equilibrium types)
\item Understanding chaos (saddle-focus homoclinic connections)
\end{itemize}

\end{document}
