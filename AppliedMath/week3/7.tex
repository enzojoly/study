\documentclass[11pt,a4paper]{article}

% Packages
\usepackage{amsmath}
\usepackage{amssymb}
\usepackage{amsthm}
\usepackage[margin=1in]{geometry}
\usepackage{enumitem}
\usepackage{tikz}
\usepackage{pgfplots}
\usepackage{xcolor}
\pgfplotsset{compat=1.18}

% Custom commands
\newcommand{\stage}[1]{\textbf{\textcolor{blue}{#1}}}

% Title information
\title{Exercise Sheet 3: Bifurcations\\
Question 7 - Complete Solution}
\author{Methods of Applied Mathematics}
\date{}

\begin{document}

\maketitle

\section*{Problem Statement}

Determine what bifurcation happens as $\mu$ changes in the systems:
\begin{enumerate}[label=(\alph*)]
\item $\frac{dx}{dt} = \mu x - x^3$
\item $\frac{dx}{dt} = \mu x + (1 + \mu)x^2 - x^3$
\item $\frac{dx}{dt} = \tanh(x) - \mu x$
\item $\frac{d^2x}{dt^2} + \frac{dx}{dt} + \mu x + x^3 = 0$
\item $\frac{dx}{dt} = \mu y - x, \quad \frac{dy}{dt} = \frac{1}{3}y^3 + y^2 - y + x$
\end{enumerate}

\vspace{10pt}
\hrule
\vspace{10pt}

\section{Part (a): $\dot{x} = \mu x - x^3$}

\subsection*{Step 1: Find equilibria}

Set $\dot{x} = 0$:
\[
\mu x - x^3 = 0 \quad \Rightarrow \quad x(\mu - x^2) = 0
\]

Solutions: $x = 0$ or $x^2 = \mu$

\textbf{For $\mu < 0$:} Only $x = 0$ (real)

\textbf{For $\mu = 0$:} Only $x = 0$

\textbf{For $\mu > 0$:} Three equilibria: $x = 0, \pm\sqrt{\mu}$

\subsection*{Step 2: Analyze stability}

Compute $f'(x) = \mu - 3x^2$

\textbf{At $x = 0$:}
\[
f'(0) = \mu
\]
\begin{itemize}
\item $\mu < 0$: stable
\item $\mu = 0$: neutral
\item $\mu > 0$: unstable
\end{itemize}

\textbf{At $x = \pm\sqrt{\mu}$ (for $\mu > 0$):}
\[
f'(\pm\sqrt{\mu}) = \mu - 3\mu = -2\mu < 0 \quad \Rightarrow \quad \text{stable}
\]

\subsection*{Step 3: Identify bifurcation}

\textbf{Characteristics:}
\begin{itemize}
\item One equilibrium becomes three
\item Origin loses stability at $\mu = 0$
\item Two stable equilibria emerge symmetrically
\item System has symmetry: $f(-x) = -(\mu x - x^3) = -f(x)$
\end{itemize}

\[
\boxed{\text{SUPERCRITICAL PITCHFORK BIFURCATION at } \mu = 0}
\]

\subsection*{XYZ Analysis}

\begin{itemize}[leftmargin=*]
\item \stage{STAGE X (What we found):} Classic pitchfork structure: 1 → 3 equilibria, symmetric emergence of stable branches.

\item \stage{STAGE Y (Why pitchfork):} The odd symmetry $f(-x) = -f(x)$ forces equilibria to appear in $\pm$ pairs. The new equilibria are stable (supercritical) rather than unstable (subcritical). This is the canonical example from lecture notes (pages 48-49).

\item \stage{STAGE Z (Meaning):} Spontaneous symmetry breaking: for $\mu > 0$, system must "choose" between $x > 0$ or $x < 0$ states. Common in physics (ferromagnetism), mechanics (buckling), and biology (pattern formation).
\end{itemize}

\vspace{10pt}
\hrule
\vspace{10pt}

\section{Part (b): $\dot{x} = \mu x + (1 + \mu)x^2 - x^3$}

\subsection*{Step 1: Find equilibria}

Set $\dot{x} = 0$:
\[
\mu x + (1 + \mu)x^2 - x^3 = 0 \quad \Rightarrow \quad x[\mu + (1+\mu)x - x^2] = 0
\]

Either $x = 0$ or $x^2 - (1+\mu)x - \mu = 0$

For the quadratic: $x = \frac{(1+\mu) \pm \sqrt{(1+\mu)^2 + 4\mu}}{2}$

Simplify discriminant:
\[
(1+\mu)^2 + 4\mu = 1 + 2\mu + \mu^2 + 4\mu = \mu^2 + 6\mu + 1
\]

This is always positive for reasonable $\mu$ (discriminant of $\mu^2 + 6\mu + 1 = 0$ gives $\mu = -3 \pm 2\sqrt{2}$).

So we always have three real equilibria:
\[
x_0 = 0, \quad x_\pm = \frac{(1+\mu) \pm \sqrt{\mu^2 + 6\mu + 1}}{2}
\]

\subsection*{Step 2: Analyze stability}

Compute $f'(x) = \mu + 2(1+\mu)x - 3x^2$

\textbf{At $x = 0$:}
\[
f'(0) = \mu
\]
\begin{itemize}
\item $\mu < 0$: stable
\item $\mu = 0$: neutral (bifurcation point)
\item $\mu > 0$: unstable
\end{itemize}

\textbf{At $x = x_+$ (upper root):}

For $\mu = 0$: $x_+ = \frac{1 + 1}{2} = 1$

Check: $f'(1) = 0 + 2(1)(1) - 3(1)^2 = 2 - 3 = -1 < 0$ → stable

\textbf{At $x = x_-$ (lower root):}

For $\mu = 0$: $x_- = \frac{1 - 1}{2} = 0$ (coincides with origin)

For small $\mu < 0$: $x_- \approx -\mu/1 = -\mu > 0$ (small positive)

Check sign at $\mu = -0.1$: $x_- \approx 0.1$, and we need to verify stability.

\subsection*{Step 3: Behavior near $\mu = 0$}

At $\mu = 0$:
\begin{itemize}
\item $x_0 = 0$ and $x_- = 0$ coincide (two equilibria meet)
\item $x_+ = 1$ exists separately
\end{itemize}

For $\mu$ slightly negative: $x_0 = 0$ stable, $x_-$ unstable, $x_+$ stable

For $\mu$ slightly positive: $x_0 = 0$ unstable, $x_-$ stable (moved to negative), $x_+$ stable

\subsection*{Step 4: Identify bifurcation}

The equilibrium at origin is pinned (always exists). As $\mu$ varies, $x_-$ passes through the origin at $\mu = 0$, and they exchange stability.

\[
\boxed{\text{TRANSCRITICAL BIFURCATION at } \mu = 0}
\]

\subsection*{XYZ Analysis}

\begin{itemize}[leftmargin=*]
\item \stage{STAGE X (What we found):} The system always has three equilibria (for $\mu$ near 0), but one passes through the origin at $\mu = 0$, exchanging stability with it.

\item \stage{STAGE Y (Why transcritical):} The origin is pinned: $f(0, \mu) = 0$ for all $\mu$. The moving equilibrium $x_-$ passes through it, not annihilating (fold) or splitting (pitchfork). The number of equilibria stays constant at 3. This is transcritical behavior with a third "spectator" equilibrium at $x_+$.

\item \stage{STAGE Z (Meaning):} The additional quadratic term $(1+\mu)x^2$ breaks the pitchfork symmetry of part (a), converting it to transcritical. The third equilibrium $x_+ \approx 1$ provides an additional stable state that persists through the bifurcation.
\end{itemize}

\vspace{10pt}
\hrule
\vspace{10pt}

\section{Part (c): $\dot{x} = \tanh(x) - \mu x$}

\subsection*{Step 1: Find equilibria}

Set $\dot{x} = 0$:
\[
\tanh(x) = \mu x
\]

Graphically: intersections of $y = \tanh(x)$ (sigmoid, bounded by $\pm 1$) with $y = \mu x$ (line through origin, slope $\mu$).

\subsection*{Step 2: Analyze number of equilibria}

\textbf{For $\mu \leq 0$:} Since $\tanh(x)$ has slope 1 at $x=0$ and $\mu x$ has slope $\mu \leq 0$, and $\tanh(x) > \mu x$ for $x > 0$, $\tanh(x) < \mu x$ for $x < 0$, only intersection at origin.

\textbf{For small $\mu > 0$:} Line $y = \mu x$ has small positive slope. Since $\tanh'(0) = 1$, for $\mu < 1$, the line intersects $\tanh(x)$ three times: at origin and two points $\pm x^*$.

\textbf{For $\mu = 1$:} Critical case. The line $y = x$ is tangent to $\tanh(x)$ at origin (both have slope 1).

\textbf{For $\mu > 1$:} Line too steep, only intersection at origin.

\subsection*{Step 3: Stability analysis}

Compute $f'(x) = \text{sech}^2(x) - \mu = \frac{1}{\cosh^2(x)} - \mu$

\textbf{At origin:}
\[
f'(0) = 1 - \mu
\]
\begin{itemize}
\item $\mu < 1$: stable
\item $\mu = 1$: neutral
\item $\mu > 1$: unstable
\end{itemize}

Wait, this seems backwards. Let me reconsider.

Actually, for $\mu > 1$, we have $f'(0) = 1 - \mu < 0$, so origin is stable.

For $0 < \mu < 1$, we have $f'(0) = 1 - \mu > 0$, so origin is unstable.

Let me reconsider the equilibrium count:

For $\mu < 1$: The slope of $\tanh(x)$ at origin is 1, which exceeds the slope $\mu$ of the line. So near origin, $\tanh(x) > \mu x$ for small positive $x$, meaning $\dot{x} > 0$ just right of origin. But far from origin, $\tanh(x) \to 1$ while $\mu x \to \infty$, so eventually $\mu x > \tanh(x)$. Thus there must be an intersection at positive $x$. By symmetry (both functions are odd), three intersections total.

For $\mu > 1$: The line is steeper than $\tanh(x)$ at origin, so $\mu x > \tanh(x)$ for small positive $x$, meaning $\dot{x} < 0$ just right of origin. Thus only one intersection at origin.

\subsection*{Step 4: Identify bifurcation}

\textbf{Summary:}
\begin{itemize}
\item $\mu < 1$: Three equilibria (origin unstable, $\pm x^*$ stable)
\item $\mu = 1$: One equilibrium (origin, marginally stable)
\item $\mu > 1$: One equilibrium (origin stable)
\end{itemize}

This is a supercritical pitchfork in reverse (as $\mu$ increases, three equilibria merge into one).

\[
\boxed{\text{SUPERCRITICAL PITCHFORK BIFURCATION at } \mu = 1}
\]

(Direction reversed: stable equilibrium splits into three as $\mu$ decreases through 1)

\subsection*{XYZ Analysis}

\begin{itemize}[leftmargin=*]
\item \stage{STAGE X (What we found):} For $\mu < 1$: three equilibria. For $\mu > 1$: one equilibrium. Change at $\mu = 1$.

\item \stage{STAGE Y (Why pitchfork):} The system has odd symmetry: $f(-x) = \tanh(-x) - \mu(-x) = -\tanh(x) + \mu x = -f(x)$. This forces equilibria to appear in $\pm$ pairs. At $\mu = 1$, the tangency condition $\tanh'(0) = \mu$ is satisfied, marking the bifurcation point. For $\mu < 1$, the origin loses stability and two stable equilibria emerge. This is the "backwards" version of standard pitchfork - or equivalently, a standard supercritical pitchfork as $\mu$ decreases.

\item \stage{STAGE Z (Meaning):} The function $\tanh(x)$ represents a saturating nonlinearity (common in neural networks, control systems). The parameter $\mu$ represents linear damping/feedback. For weak feedback ($\mu < 1$), the nonlinearity dominates and system exhibits bistability (two stable states $\pm x^*$). For strong feedback ($\mu > 1$), linear damping dominates and only origin is stable.
\end{itemize}

\vspace{10pt}
\hrule
\vspace{10pt}

\section{Part (d): $\ddot{x} + \dot{x} + \mu x + x^3 = 0$}

\subsection*{Step 1: Convert to first-order system}

Let $y = \dot{x}$. Then:
\begin{align*}
\dot{x} &= y \\
\dot{y} &= -\dot{x} - \mu x - x^3 = -y - \mu x - x^3
\end{align*}

\subsection*{Step 2: Find equilibria}

Set $\dot{x} = 0$ and $\dot{y} = 0$:
\begin{align*}
y &= 0 \\
-y - \mu x - x^3 &= 0
\end{align*}

From second equation with $y = 0$:
\[
\mu x + x^3 = 0 \quad \Rightarrow \quad x(\mu + x^2) = 0
\]

Solutions: $x = 0$ or $x^2 = -\mu$

\textbf{For $\mu > 0$:} Only $(0, 0)$

\textbf{For $\mu = 0$:} Only $(0, 0)$

\textbf{For $\mu < 0$:} Three equilibria: $(0, 0)$ and $(\pm\sqrt{-\mu}, 0)$

\subsection*{Step 3: Jacobian analysis}

\[
J = \begin{pmatrix}
0 & 1 \\
-\mu - 3x^2 & -1
\end{pmatrix}
\]

\textbf{At origin:}
\[
J(0,0) = \begin{pmatrix}
0 & 1 \\
-\mu & -1
\end{pmatrix}
\]

Trace: $\tau = -1$, Determinant: $\Delta = \mu$

Eigenvalues: $\lambda^2 + \lambda + \mu = 0$, so $\lambda = \frac{-1 \pm \sqrt{1 - 4\mu}}{2}$

\textbf{For $\mu > 1/4$:} Complex eigenvalues $\lambda = -\frac{1}{2} \pm i\frac{\sqrt{4\mu - 1}}{2}$ with negative real part → stable spiral

\textbf{For $\mu = 0$:} $\lambda = \frac{-1 \pm 1}{2}$, so $\lambda = 0, -1$ → neutral

\textbf{For $\mu < 0$:} Real eigenvalues. $\Delta = \mu < 0$ means opposite signs → saddle

\textbf{At $(\pm\sqrt{-\mu}, 0)$ for $\mu < 0$:}
\[
J = \begin{pmatrix}
0 & 1 \\
-\mu - 3(-\mu) & -1
\end{pmatrix} = \begin{pmatrix}
0 & 1 \\
2\mu & -1
\end{pmatrix}
\]

Since $\mu < 0$: $\Delta = -2\mu > 0$ and $\tau = -1 < 0$ → stable (node or spiral depending on discriminant)

\subsection*{Step 4: Identify bifurcation}

\textbf{Characteristics:}
\begin{itemize}
\item One equilibrium becomes three as $\mu$ decreases through 0
\item Origin changes from stable to saddle
\item Two stable equilibria emerge symmetrically
\item System has Hamiltonian structure (conservative for $\dot{x} = 0$)
\end{itemize}

\[
\boxed{\text{SUPERCRITICAL PITCHFORK BIFURCATION at } \mu = 0}
\]

(In reverse direction: as $\mu$ decreases through 0)

\subsection*{XYZ Analysis}

\begin{itemize}[leftmargin=*]
\item \stage{STAGE X (What we found):} The second-order ODE exhibits pitchfork bifurcation when converted to phase-plane system. For $\mu > 0$: stable equilibrium at origin. For $\mu < 0$: saddle at origin, two stable equilibria at $\pm\sqrt{-\mu}$.

\item \stage{STAGE Y (Why pitchfork):} This is Duffing's equation with damping. The term $\mu x + x^3$ represents a potential $V(x) = \frac{\mu x^2}{2} + \frac{x^4}{4}$:
\begin{itemize}
\item For $\mu > 0$: Single-well potential (minimum at $x=0$)
\item For $\mu < 0$: Double-well potential (minima at $x = \pm\sqrt{-\mu}$, maximum at $x=0$)
\end{itemize}
The damping term $\dot{x}$ dissipates energy, causing trajectories to settle at potential minima. The pitchfork occurs when the single well bifurcates into double well at $\mu = 0$.

\item \stage{STAGE Z (Meaning):} This models mechanical systems like buckled beams, magnetic pendulums, or nonlinear springs. For $\mu > 0$ (stiff restoring force), unique stable position at origin. For $\mu < 0$ (negative stiffness + cubic hardening), two stable positions emerge - the system buckles. Common in structural mechanics and nonlinear oscillators.
\end{itemize}

\vspace{10pt}
\hrule
\vspace{10pt}

\section{Part (e): $\dot{x} = \mu y - x, \quad \dot{y} = \frac{1}{3}y^3 + y^2 - y + x$}

\subsection*{Step 1: Find equilibria}

Set $\dot{x} = 0$ and $\dot{y} = 0$:
\begin{align*}
\mu y - x &= 0 \quad \Rightarrow \quad x = \mu y \quad \cdots(1)\\
\frac{1}{3}y^3 + y^2 - y + x &= 0 \quad \cdots(2)
\end{align*}

Substitute (1) into (2):
\[
\frac{1}{3}y^3 + y^2 - y + \mu y = 0
\]
\[
\frac{1}{3}y^3 + y^2 + (\mu - 1)y = 0
\]
\[
y\left[\frac{1}{3}y^2 + y + (\mu - 1)\right] = 0
\]

Either $y = 0$ or $\frac{1}{3}y^2 + y + (\mu - 1) = 0$

\textbf{Equilibrium 1:} $y = 0$, then $x = 0$
\[
(x, y) = (0, 0)
\]

\textbf{Equilibria 2, 3:} From quadratic:
\[
y = \frac{-1 \pm \sqrt{1 - 4 \cdot \frac{1}{3}(\mu - 1)}}{2 \cdot \frac{1}{3}} = \frac{-3 \pm \sqrt{9 - 4(\mu - 1)}}{2}
\]
\[
y = \frac{-3 \pm \sqrt{13 - 4\mu}}{2}
\]

For real solutions: $13 - 4\mu \geq 0$, i.e., $\mu \leq 13/4$

\subsection*{Step 2: Analyze equilibrium count}

\textbf{For $\mu < 13/4$:} Three equilibria

\textbf{For $\mu = 13/4$:} Two equilibria coincide at $y = -3/2$

\textbf{For $\mu > 13/4$:} One equilibrium at origin

\subsection*{Step 3: Jacobian at origin}

\[
J = \begin{pmatrix}
-1 & \mu \\
1 & y^2 + 2y - 1
\end{pmatrix}
\]

At origin:
\[
J(0,0) = \begin{pmatrix}
-1 & \mu \\
1 & -1
\end{pmatrix}
\]

Trace: $\tau = -2$

Determinant: $\Delta = (-1)(-1) - \mu \cdot 1 = 1 - \mu$

Eigenvalues: $\lambda^2 + 2\lambda + (1-\mu) = 0$

\[
\lambda = \frac{-2 \pm \sqrt{4 - 4(1-\mu)}}{2} = -1 \pm \sqrt{\mu}
\]

\textbf{For $\mu < 0$:} Complex eigenvalues $\lambda = -1 \pm i\sqrt{|\mu|}$ → stable spiral

\textbf{For $\mu = 0$:} $\lambda = -1$ (repeated) → stable node

\textbf{For $0 < \mu < 1$:} Complex eigenvalues $\lambda = -1 \pm i\sqrt{\mu}$ → stable spiral

\textbf{For $\mu = 1$:} $\lambda = 0, -2$ → neutral (one zero eigenvalue)

\textbf{For $\mu > 1$:} Real eigenvalues, one positive ($\lambda = -1 + \sqrt{\mu} > 0$) → saddle

\subsection*{Step 4: Identify bifurcation}

At $\mu = 1$, the origin has one zero eigenvalue, suggesting a codimension-1 bifurcation.

From equilibrium structure: for $\mu$ slightly less than 1, origin is stable and there exist two other equilibria. For $\mu$ slightly greater than 1, origin becomes saddle.

The equilibria meet at $y = (-3 \pm \sqrt{13-4})/2 = (-3 \pm 3)/2$, giving $y = 0$ or $y = -3$.

At $\mu = 1$: one equilibrium at origin, another at $y = -3$ (so $x = -3$).

This doesn't look like a simple collision at origin. Let me recalculate where equilibria collide.

Actually, at $\mu = 13/4$, the discriminant vanishes, and the two non-origin equilibria collide at:
\[
y = -3/2, \quad x = \mu(-3/2) = (13/4)(-3/2) = -39/8
\]

This is a fold bifurcation of the two non-origin equilibria at $\mu = 13/4$.

But at $\mu = 1$, the origin changes stability (eigenvalue crosses zero). This could be a transcritical bifurcation where one of the non-origin equilibria passes through the origin.

Let me check: at $\mu = 1$, do any of the non-origin equilibria equal $(0, 0)$?
\[
y = \frac{-3 \pm \sqrt{13 - 4}}{2} = \frac{-3 \pm 3}{2}
\]

So $y = 0$ or $y = -3$.

Yes! At $\mu = 1$, one equilibrium is at $y = 0, x = 0$ (the origin), confirming a collision.

\[
\boxed{\text{TRANSCRITICAL BIFURCATION at } \mu = 1}
\]

and

\[
\boxed{\text{FOLD BIFURCATION at } \mu = 13/4}
\]

\subsection*{XYZ Analysis}

\begin{itemize}[leftmargin=*]
\item \stage{STAGE X (What we found):} Two bifurcations occur in this system as $\mu$ varies. At $\mu = 1$, a transcritical bifurcation where a moving equilibrium passes through the origin. At $\mu = 13/4$, a fold bifurcation where two equilibria collide and annihilate.

\item \stage{STAGE Y (Why this complexity):} The $y$-equation is cubic, giving up to three values of $y$ for equilibrium, hence up to three equilibria total. The parameter $\mu$ enters only through the coupling term $\mu y$ in the $x$-equation. As $\mu$ increases:
\begin{itemize}
\item $\mu < 1$: Three equilibria exist; origin stable
\item $\mu = 1$: Transcritical bifurcation; moving equilibrium passes through origin
\item $1 < \mu < 13/4$: Three equilibria; origin now saddle
\item $\mu = 13/4$: Fold bifurcation; two non-origin equilibria collide
\item $\mu > 13/4$: Only origin remains (saddle)
\end{itemize}

\item \stage{STAGE Z (Meaning):} This system exhibits a sequence of bifurcations, showing that complex dynamics can have multiple qualitative transitions. The cubic nonlinearity in $y$ combined with linear coupling creates a rich bifurcation structure. Such sequences appear in chemical reactors, neural networks, and ecological models where multiple mechanisms interact.
\end{itemize}

\vspace{10pt}
\hrule
\vspace{10pt}

\section{Summary Table}

\begin{center}
\begin{tabular}{|c|c|c|}
\hline
\textbf{System} & \textbf{Bifurcation Type} & \textbf{Critical Value} \\
\hline
(a) $\dot{x} = \mu x - x^3$ & Supercritical Pitchfork & $\mu = 0$ \\
\hline
(b) $\dot{x} = \mu x + (1+\mu)x^2 - x^3$ & Transcritical & $\mu = 0$ \\
\hline
(c) $\dot{x} = \tanh(x) - \mu x$ & Supercritical Pitchfork & $\mu = 1$ \\
\hline
(d) $\ddot{x} + \dot{x} + \mu x + x^3 = 0$ & Supercritical Pitchfork & $\mu = 0$ \\
\hline
\multirow{2}{*}{(e) $\dot{x} = \mu y - x$, $\dot{y} = \frac{y^3}{3} + y^2 - y + x$} & Transcritical & $\mu = 1$ \\
\cline{2-3}
& Fold & $\mu = 13/4$ \\
\hline
\end{tabular}
\end{center}

\subsection*{Key Insights}

\textbf{Identifying bifurcation types:}
\begin{itemize}
\item \textbf{Pitchfork}: Odd symmetry, 1 ↔ 3 equilibria, symmetric branches
\item \textbf{Transcritical}: Pinned equilibrium, passing equilibrium, stability exchange
\item \textbf{Fold}: Equilibria created/destroyed, tangent collision, 2 ↔ 0 equilibria
\end{itemize}

\textbf{Analysis strategy:}
\begin{enumerate}
\item Find equilibria as functions of parameter
\item Count equilibria for different parameter ranges
\item Compute stability (1D: $f'(x)$; 2D: eigenvalues of Jacobian)
\item Look for:
\begin{itemize}
\item Changes in equilibrium count → fold or pitchfork
\item Equilibria passing through each other → transcritical
\item Eigenvalues becoming complex/real → potential Hopf (if imaginary axis crossing)
\end{itemize}
\item Check for symmetries ($f(-x) = -f(x)$ suggests pitchfork)
\item Verify critical point: equilibrium exists, eigenvalue = 0
\end{enumerate}

\end{document}
