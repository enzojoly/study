\documentclass[11pt,a4paper]{article}

% Packages
\usepackage{amsmath}
\usepackage{amssymb}
\usepackage{amsthm}
\usepackage[margin=1in]{geometry}
\usepackage{enumitem}
\usepackage{xcolor}

% Custom commands
\newcommand{\stage}[1]{\textbf{\textcolor{blue}{#1}}}

% Title information
\title{Exercise Sheet 4: Maps\\
Question 1 - Complete Solution}
\author{Methods of Applied Mathematics}
\date{}

\begin{document}

\maketitle

\section*{Problem Statement}

Derive the discrete population model:
\[
N_{n+1} = N_n(1 + \beta - \gamma N_n)
\]
from the solution of the nonlinear ODE population model.

\vspace{10pt}
\hrule
\vspace{10pt}

\section{Step 1: Recall the Continuous Population Model}

\subsection*{The ODE}

The nonlinear population model is:
\[
\frac{dN}{dt} = N(\beta - \gamma N)
\]

where:
\begin{itemize}
\item $N(t)$ is the population at time $t$
\item $\beta > 0$ is the birth rate
\item $\gamma > 0$ is the death rate (proportional to population)
\end{itemize}

\subsection*{XYZ Analysis of the ODE}

\begin{itemize}[leftmargin=*]
\item \stage{STAGE X (What we have):} A first-order nonlinear ODE with quadratic nonlinearity in $N$. The right-hand side has two contributions: linear growth $\beta N$ and quadratic decay $-\gamma N^2$.

\item \stage{STAGE Y (Why this form):} The model assumes:
\begin{itemize}
\item Birth rate is proportional to population: births $= \beta N$
\item Death rate is proportional to population squared: deaths $= \gamma N^2$
\end{itemize}
The quadratic death term models resource competition - as population increases, individuals compete for limited resources, increasing mortality. This gives logistic-type growth with a carrying capacity at $N = \beta/\gamma$ where births balance deaths.

\item \stage{STAGE Z (What we need):} To derive a discrete map, we'll solve this ODE explicitly, then sample the solution at discrete time intervals $t = n\Delta t$ for integer $n$.
\end{itemize}

\vspace{10pt}
\hrule
\vspace{10pt}

\section{Step 2: Solve the ODE}

\subsection*{Separate variables}

Rearrange the ODE:
\[
\frac{dN}{N(\beta - \gamma N)} = dt
\]

\subsection*{Partial fraction decomposition}

Decompose the left side:
\[
\frac{1}{N(\beta - \gamma N)} = \frac{A}{N} + \frac{B}{\beta - \gamma N}
\]

Multiply both sides by $N(\beta - \gamma N)$:
\[
1 = A(\beta - \gamma N) + BN
\]

Setting $N = 0$: $1 = A\beta \Rightarrow A = \frac{1}{\beta}$

Setting $N = \beta/\gamma$: $1 = B \cdot \frac{\beta}{\gamma} \Rightarrow B = \frac{\gamma}{\beta}$

Therefore:
\[
\frac{1}{N(\beta - \gamma N)} = \frac{1}{\beta N} + \frac{\gamma}{\beta(\beta - \gamma N)}
\]

\subsection*{Integrate both sides}

\begin{align*}
\int \left[\frac{1}{\beta N} + \frac{\gamma}{\beta(\beta - \gamma N)}\right] dN &= \int dt \\
\frac{1}{\beta}\log|N| - \frac{\gamma}{\beta\gamma}\log|\beta - \gamma N| &= t + C \\
\frac{1}{\beta}\log|N| - \frac{1}{\beta}\log|\beta - \gamma N| &= t + C \\
\frac{1}{\beta}\log\left|\frac{N}{\beta - \gamma N}\right| &= t + C
\end{align*}

\subsection*{Solve for $N(t)$}

Exponentiate both sides:
\[
\frac{N}{\beta - \gamma N} = Ke^{\beta t}
\]
where $K = e^{\beta C}$ is determined by initial condition.

Solve for $N$:
\begin{align*}
N &= (\beta - \gamma N)Ke^{\beta t} \\
N &= \beta Ke^{\beta t} - \gamma N Ke^{\beta t} \\
N + \gamma N Ke^{\beta t} &= \beta Ke^{\beta t} \\
N(1 + \gamma Ke^{\beta t}) &= \beta Ke^{\beta t} \\
N(t) &= \frac{\beta Ke^{\beta t}}{1 + \gamma Ke^{\beta t}}
\end{align*}

\subsection*{Apply initial condition}

At $t = 0$, let $N(0) = N_0$:
\[
N_0 = \frac{\beta K}{1 + \gamma K}
\]

Solve for $K$:
\begin{align*}
N_0(1 + \gamma K) &= \beta K \\
N_0 + \gamma N_0 K &= \beta K \\
N_0 &= K(\beta - \gamma N_0) \\
K &= \frac{N_0}{\beta - \gamma N_0}
\end{align*}

\subsection*{General solution}

Substituting back:
\[
\boxed{N(t) = \frac{\beta N_0 e^{\beta t}}{\beta - \gamma N_0 + \gamma N_0 e^{\beta t}}}
\]

This can be rewritten as:
\[
N(t) = \frac{N_0 e^{\beta t}}{1 + \frac{\gamma N_0}{\beta}(e^{\beta t} - 1)}
\]

\subsection*{XYZ Analysis of the Solution}

\begin{itemize}[leftmargin=*]
\item \stage{STAGE X (What we derived):} An explicit solution for $N(t)$ in terms of initial population $N_0$, parameters $\beta, \gamma$, and time $t$.

\item \stage{STAGE Y (Why this form):} The solution is a logistic curve:
\begin{itemize}
\item As $t \to \infty$: $N(t) \to \beta/\gamma$ (carrying capacity)
\item Growth rate decreases as $N$ approaches carrying capacity
\item The exponential $e^{\beta t}$ drives growth, but is tempered by the denominator
\end{itemize}
The fraction structure emerges from the competition between linear growth and quadratic death terms in the original ODE.

\item \stage{STAGE Z (What's next):} We'll evaluate this solution at discrete time steps separated by interval $\Delta t$, then derive a relationship between successive populations.
\end{itemize}

\vspace{10pt}
\hrule
\vspace{10pt}

\section{Step 3: Discretize by Sampling at Fixed Time Intervals}

\subsection*{Define discrete time points}

Let $t_n = n \Delta t$ for $n = 0, 1, 2, \ldots$ where $\Delta t$ is a fixed time step.

Define:
\[
N_n = N(t_n) = N(n\Delta t)
\]

\subsection*{Evaluate solution at $t_n$ and $t_{n+1}$}

At time $t_n = n\Delta t$:
\[
N_n = \frac{\beta N_0 e^{\beta n \Delta t}}{\beta - \gamma N_0 + \gamma N_0 e^{\beta n \Delta t}}
\]

At time $t_{n+1} = (n+1)\Delta t = n\Delta t + \Delta t$:
\[
N_{n+1} = \frac{\beta N_0 e^{\beta(n+1)\Delta t}}{\beta - \gamma N_0 + \gamma N_0 e^{\beta(n+1)\Delta t}}
\]

Factor out $e^{\beta n\Delta t}$ from numerator and denominator of $N_{n+1}$:
\[
N_{n+1} = \frac{\beta N_0 e^{\beta n\Delta t} \cdot e^{\beta \Delta t}}{\beta - \gamma N_0 + \gamma N_0 e^{\beta n\Delta t} \cdot e^{\beta \Delta t}}
\]

\subsection*{Relate $N_{n+1}$ to $N_n$}

From the expression for $N_n$, we can write:
\[
\beta N_0 e^{\beta n\Delta t} = N_n(\beta - \gamma N_0 + \gamma N_0 e^{\beta n\Delta t})
\]

Let $\alpha = e^{\beta \Delta t}$. Then:
\[
N_{n+1} = \frac{\beta N_0 e^{\beta n\Delta t} \cdot \alpha}{\beta - \gamma N_0 + \gamma N_0 e^{\beta n\Delta t} \cdot \alpha}
\]

Factor the denominator:
\begin{align*}
N_{n+1} &= \frac{\beta N_0 e^{\beta n\Delta t} \cdot \alpha}{(\beta - \gamma N_0)(1 - \alpha) + \beta - \gamma N_0 + \gamma N_0 e^{\beta n\Delta t}(\alpha - 1) + \beta - \gamma N_0 + \gamma N_0 e^{\beta n\Delta t}} \\
\end{align*}

This is getting messy. Let's try a different approach.

\vspace{10pt}
\hrule
\vspace{10pt}

\section{Step 3 (Alternative): Use Relation Between Consecutive Times}

\subsection*{Direct ratio approach}

From the solution:
\[
N(t) = \frac{\beta N_0 e^{\beta t}}{\beta - \gamma N_0 + \gamma N_0 e^{\beta t}}
\]

We can invert this to get:
\[
\frac{1}{N(t)} = \frac{\beta - \gamma N_0 + \gamma N_0 e^{\beta t}}{\beta N_0 e^{\beta t}} = \frac{\beta - \gamma N_0}{\beta N_0 e^{\beta t}} + \frac{\gamma}{\beta}
\]

Therefore:
\[
\frac{1}{N(t)} - \frac{\gamma}{\beta} = \frac{\beta - \gamma N_0}{\beta N_0} e^{-\beta t}
\]

At $t = n\Delta t$ and $t = (n+1)\Delta t$:
\begin{align*}
\frac{1}{N_n} - \frac{\gamma}{\beta} &= \frac{\beta - \gamma N_0}{\beta N_0} e^{-\beta n\Delta t} \\
\frac{1}{N_{n+1}} - \frac{\gamma}{\beta} &= \frac{\beta - \gamma N_0}{\beta N_0} e^{-\beta(n+1)\Delta t}
\end{align*}

Divide the second by the first:
\[
\frac{\frac{1}{N_{n+1}} - \frac{\gamma}{\beta}}{\frac{1}{N_n} - \frac{\gamma}{\beta}} = e^{-\beta\Delta t}
\]

Let $\alpha = e^{-\beta\Delta t}$. Then:
\[
\frac{1}{N_{n+1}} - \frac{\gamma}{\beta} = \alpha\left(\frac{1}{N_n} - \frac{\gamma}{\beta}\right)
\]

Expanding:
\[
\frac{1}{N_{n+1}} = \alpha \cdot \frac{1}{N_n} - \alpha \cdot \frac{\gamma}{\beta} + \frac{\gamma}{\beta} = \frac{\alpha}{N_n} + \frac{\gamma}{\beta}(1-\alpha)
\]

Therefore:
\[
N_{n+1} = \frac{1}{\frac{\alpha}{N_n} + \frac{\gamma}{\beta}(1-\alpha)} = \frac{N_n}{\alpha + \frac{\gamma N_n}{\beta}(1-\alpha)}
\]

\subsection*{Simplify for $\Delta t = 1$}

For unit time step $\Delta t = 1$:
\[
\alpha = e^{-\beta}
\]

Let's expand for small $\beta$ (or equivalently, Taylor expand around the continuous limit). Actually, let's take a more direct approach.

\vspace{10pt}
\hrule
\vspace{10pt}

\section{Step 3 (Direct Approach): From First Principles}

\subsection*{Alternative derivation using ODE directly}

The definition of derivative gives:
\[
\frac{dN}{dt} = \lim_{\Delta t \to 0} \frac{N(t + \Delta t) - N(t)}{\Delta t}
\]

For finite $\Delta t$, we approximate:
\[
\frac{N(t + \Delta t) - N(t)}{\Delta t} \approx N(t)(\beta - \gamma N(t))
\]

Rearranging:
\[
N(t + \Delta t) \approx N(t) + \Delta t \cdot N(t)(\beta - \gamma N(t))
\]

\[
N(t + \Delta t) \approx N(t)[1 + \Delta t(\beta - \gamma N(t))]
\]

\subsection*{Discretize with $t = n\Delta t$ and $\Delta t = 1$}

Let $N_n = N(n \cdot 1) = N(n)$ and take $\Delta t = 1$:
\[
N_{n+1} = N_n[1 + 1 \cdot (\beta - \gamma N_n)]
\]

\[
\boxed{N_{n+1} = N_n(1 + \beta - \gamma N_n)}
\]

This is the desired discrete population map.

\subsection*{XYZ Analysis of Derivation}

\begin{itemize}[leftmargin=*]
\item \stage{STAGE X (What we did):} Started from the ODE's definition as a derivative, approximated the derivative for finite time step $\Delta t$, then set $\Delta t = 1$ to get discrete time intervals.

\item \stage{STAGE Y (Why this works):} The continuous ODE $\dot{N} = N(\beta - \gamma N)$ tells us the instantaneous rate of change. For a small (but finite) time step $\Delta t$, the change in population is approximately:
\[
\Delta N = N(t+\Delta t) - N(t) \approx \dot{N} \cdot \Delta t = N(\beta - \gamma N)\Delta t
\]
This is a first-order Euler approximation. When we take $\Delta t = 1$ as our fundamental time unit (e.g., one day, one year), we get:
\[
N_{n+1} = N_n + N_n(\beta - \gamma N_n) = N_n(1 + \beta - \gamma N_n)
\]
The map represents: current population + change over one time unit.

\item \stage{STAGE Z (What this means):} This discrete map is:
\begin{itemize}
\item An approximation to the continuous ODE when $\Delta t$ is small
\item A model in its own right when time naturally occurs in discrete steps
\item Valid when $1 + \beta - \gamma N_n > 0$ (otherwise population becomes negative)
\item Simpler to iterate than solving the ODE repeatedly
\end{itemize}
For $\beta$ small (slow growth), the map closely approximates the ODE. For larger $\beta$, the map can exhibit different behavior including oscillations and chaos.
\end{itemize}

\vspace{10pt}
\hrule
\vspace{10pt}

\section{Step 4: Verification from Exact Solution}

\subsection*{Alternative verification}

We can verify our discrete map by checking it's consistent with the exact solution in the limit of small $\Delta t$.

From the exact solution at $t$ and $t + \Delta t$:
\[
N(t) = \frac{\beta N_0 e^{\beta t}}{\beta - \gamma N_0 + \gamma N_0 e^{\beta t}}
\]

Taking the ratio $N(t+\Delta t)/N(t)$:
\[
\frac{N(t+\Delta t)}{N(t)} = \frac{e^{\beta(t+\Delta t)}}{e^{\beta t}} \cdot \frac{\beta - \gamma N_0 + \gamma N_0 e^{\beta t}}{\beta - \gamma N_0 + \gamma N_0 e^{\beta(t+\Delta t)}}
\]

For small $\Delta t$, using $e^{\beta\Delta t} \approx 1 + \beta\Delta t$:
\[
\frac{N(t+\Delta t)}{N(t)} \approx (1 + \beta\Delta t) \cdot \frac{\beta - \gamma N_0 + \gamma N_0 e^{\beta t}}{\beta - \gamma N_0 + \gamma N_0 e^{\beta t}(1 + \beta\Delta t)}
\]

The denominator:
\[
\beta - \gamma N_0 + \gamma N_0 e^{\beta t} + \gamma N_0 e^{\beta t}\beta\Delta t
\]

Factor out $(\beta - \gamma N_0 + \gamma N_0 e^{\beta t})$:
\[
(\beta - \gamma N_0 + \gamma N_0 e^{\beta t})\left(1 + \frac{\gamma N_0 e^{\beta t}\beta\Delta t}{\beta - \gamma N_0 + \gamma N_0 e^{\beta t}}\right)
\]

But from the solution: $N(t) = \frac{\beta N_0 e^{\beta t}}{\beta - \gamma N_0 + \gamma N_0 e^{\beta t}}$

So: $\beta - \gamma N_0 + \gamma N_0 e^{\beta t} = \frac{\beta N_0 e^{\beta t}}{N(t)}$

And: $\frac{\gamma N_0 e^{\beta t}}{\beta - \gamma N_0 + \gamma N_0 e^{\beta t}} = \frac{\gamma N(t)}{\beta}$

Therefore:
\[
\frac{N(t+\Delta t)}{N(t)} \approx (1 + \beta\Delta t) \cdot \frac{1}{1 + \frac{\gamma N(t)}{\beta}\beta\Delta t} \approx (1 + \beta\Delta t)(1 - \gamma N(t)\Delta t)
\]

Expanding:
\[
\frac{N(t+\Delta t)}{N(t)} \approx 1 + \beta\Delta t - \gamma N(t)\Delta t + O(\Delta t^2)
\]

Thus:
\[
N(t+\Delta t) \approx N(t)[1 + \Delta t(\beta - \gamma N(t))]
\]

Setting $\Delta t = 1$ and using discrete notation:
\[
\boxed{N_{n+1} = N_n(1 + \beta - \gamma N_n)} \quad \checkmark
\]

\subsection*{XYZ Analysis of Verification}

\begin{itemize}[leftmargin=*]
\item \stage{STAGE X (What we verified):} The discrete map derived from the derivative approximation matches the small-$\Delta t$ limit of the exact solution ratio.

\item \stage{STAGE Y (Why this consistency):} Both approaches use the same fundamental ODE. The Euler approximation (derivative approach) is exactly the first-order Taylor expansion of the exact solution. They must agree to $O(\Delta t)$.

\item \stage{STAGE Z (What this tells us):}
\begin{itemize}
\item The map $N_{n+1} = N_n(1 + \beta - \gamma N_n)$ is the natural discrete-time analog of the continuous ODE
\item For small time steps, map and ODE give nearly identical results
\item For larger time steps, the map and ODE can diverge significantly - the map is an independent model
\item The map is computationally simpler: no integration needed, just iteration
\end{itemize}
\end{itemize}

\vspace{10pt}
\hrule
\vspace{10pt}

\section{Summary}

\subsection*{Derivation pathway}

\begin{enumerate}
\item \textbf{Start:} Continuous ODE $\dot{N} = N(\beta - \gamma N)$

\item \textbf{Approximate derivative:} For finite $\Delta t$:
\[
\frac{N(t+\Delta t) - N(t)}{\Delta t} \approx N(t)(\beta - \gamma N(t))
\]

\item \textbf{Rearrange:}
\[
N(t+\Delta t) \approx N(t) + \Delta t \cdot N(t)(\beta - \gamma N(t))
\]

\item \textbf{Factor:}
\[
N(t+\Delta t) \approx N(t)[1 + \Delta t(\beta - \gamma N(t))]
\]

\item \textbf{Discretize with $\Delta t = 1$:}
\[
\boxed{N_{n+1} = N_n(1 + \beta - \gamma N_n)}
\]
\end{enumerate}

\subsection*{Physical interpretation}

The discrete map says:
\[
\text{Next population} = \text{Current population} \times \text{Growth factor}
\]

where the growth factor is:
\[
1 + \beta - \gamma N_n = 1 + \underbrace{\beta}_{\text{birth rate}} - \underbrace{\gamma N_n}_{\text{death rate}}
\]

\begin{itemize}
\item If $\beta > \gamma N_n$: growth factor $> 1$, population increases
\item If $\beta < \gamma N_n$: growth factor $< 1$, population decreases
\item If $\beta = \gamma N_n$: growth factor $= 1$, population stable
\end{itemize}

Fixed point: $N^* = N^*(1 + \beta - \gamma N^*)$ gives $N^* = 0$ or $N^* = \beta/\gamma$ (carrying capacity).

\subsection*{Connection to logistic map}

With rescaling $N_n = \frac{1+\beta}{\gamma}x_n$ and $r = 1 + \beta$:
\[
N_{n+1} = N_n(1 + \beta - \gamma N_n) \quad \Rightarrow \quad x_{n+1} = rx_n(1-x_n)
\]

This is the famous logistic map studied in chaos theory.

\end{document}
