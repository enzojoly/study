\documentclass[11pt,a4paper]{article}

% Packages
\usepackage{amsmath}
\usepackage{amssymb}
\usepackage{amsthm}
\usepackage[margin=1in]{geometry}
\usepackage{enumitem}
\usepackage{xcolor}

% Custom commands
\newcommand{\stage}[1]{\textbf{\textcolor{blue}{#1}}}

% Title information
\title{Exercise Sheet 4: Maps\\
Question 2 - Complete Solution}
\author{Methods of Applied Mathematics}
\date{}

\begin{document}

\maketitle

\section*{Problem Statement}

Derive a discrete map for the predator-prey system, in a similar way we did for the 1d population model.

\vspace{10pt}
\hrule
\vspace{10pt}

\section{Step 1: The Continuous Predator-Prey System}

\subsection*{The Lotka-Volterra equations}

The classical predator-prey model is:
\begin{align}
\frac{dx}{dt} &= ax - bxy \label{eq:prey}\\
\frac{dy}{dt} &= -cy + dxy \label{eq:predator}
\end{align}

where:
\begin{itemize}
\item $x(t)$ is the prey population at time $t$
\item $y(t)$ is the predator population at time $t$
\item $a > 0$ is the prey birth rate (in absence of predators)
\item $b > 0$ is the predation rate coefficient
\item $c > 0$ is the predator death rate (in absence of prey)
\item $d > 0$ is the predator growth rate from consumption
\end{itemize}

\subsection*{XYZ Analysis of the System}

\begin{itemize}[leftmargin=*]
\item \stage{STAGE X (What we have):} A coupled system of two first-order nonlinear ODEs. Each equation has a linear term (natural growth/death) and a nonlinear interaction term ($xy$).

\item \stage{STAGE Y (Why this structure):}
\begin{itemize}
\item \textbf{Prey equation} $\dot{x} = ax - bxy$:
\begin{itemize}
\item $ax$: Prey reproduce exponentially when alone
\item $-bxy$: Prey are consumed proportional to encounter rate (product of populations)
\end{itemize}
\item \textbf{Predator equation} $\dot{y} = -cy + dxy$:
\begin{itemize}
\item $-cy$: Predators die exponentially without food
\item $+dxy$: Predators grow proportional to prey consumed
\end{itemize}
\end{itemize}
The coupling through $xy$ creates the predator-prey dynamic: predators need prey to survive, prey are limited by predators.

\item \stage{STAGE Z (What we need):} Derive discrete-time versions by approximating derivatives for finite time step $\Delta t$, analogous to the single population case.
\end{itemize}

\vspace{10pt}
\hrule
\vspace{10pt}

\section{Step 2: Discretize Using Euler Approximation}

\subsection*{Approximate derivatives}

Recall the definition of derivative:
\[
\frac{dx}{dt} = \lim_{\Delta t \to 0} \frac{x(t+\Delta t) - x(t)}{\Delta t}
\]

For finite (small) $\Delta t$, we approximate:
\[
\frac{x(t+\Delta t) - x(t)}{\Delta t} \approx ax(t) - bx(t)y(t)
\]

Similarly for $y$:
\[
\frac{y(t+\Delta t) - y(t)}{\Delta t} \approx -cy(t) + dx(t)y(t)
\]

\subsection*{Rearrange to get update rules}

From the prey equation:
\begin{align*}
x(t+\Delta t) - x(t) &\approx \Delta t[ax(t) - bx(t)y(t)] \\
x(t+\Delta t) &\approx x(t) + \Delta t \cdot x(t)[a - by(t)] \\
x(t+\Delta t) &\approx x(t)[1 + \Delta t(a - by(t))]
\end{align*}

From the predator equation:
\begin{align*}
y(t+\Delta t) - y(t) &\approx \Delta t[-cy(t) + dx(t)y(t)] \\
y(t+\Delta t) &\approx y(t) + \Delta t \cdot y(t)[-c + dx(t)] \\
y(t+\Delta t) &\approx y(t)[1 + \Delta t(-c + dx(t))]
\end{align*}

\subsection*{Set time step $\Delta t = 1$}

Taking the fundamental time unit as $\Delta t = 1$ (e.g., one day, one generation), and using discrete notation $x_n = x(n)$, $y_n = y(n)$:

\begin{align}
x_{n+1} &= x_n[1 + a - by_n] \label{eq:prey_map}\\
y_{n+1} &= y_n[1 - c + dx_n] \label{eq:predator_map}
\end{align}

\subsection*{XYZ Analysis of Discretization}

\begin{itemize}[leftmargin=*]
\item \stage{STAGE X (What we derived):} A pair of coupled difference equations that map $(x_n, y_n) \to (x_{n+1}, y_{n+1})$.

\item \stage{STAGE Y (Why this works):} The Euler method approximates:
\[
\text{Population at next step} = \text{Current population} + \text{Change over } \Delta t
\]
For $\Delta t = 1$:
\begin{itemize}
\item \textbf{Prey:} $x_{n+1} = x_n + x_n(a - by_n) = x_n(1 + a - by_n)$
\begin{itemize}
\item Factor $(1+a)$ would give exponential growth alone
\item Factor $by_n$ represents reduction due to predation
\end{itemize}
\item \textbf{Predator:} $y_{n+1} = y_n + y_n(-c + dx_n) = y_n(1 - c + dx_n)$
\begin{itemize}
\item Factor $(1-c)$ would give exponential decay alone
\item Factor $dx_n$ represents growth from eating prey
\end{itemize}
\end{itemize}

\item \stage{STAGE Z (What this gives):} A 2D discrete dynamical system (map). Unlike the continuous ODE which requires solving differential equations, this map can be iterated directly:
\[
(x_0, y_0) \to (x_1, y_1) \to (x_2, y_2) \to \cdots
\]
Each iteration is algebraic, making it computationally simple.
\end{itemize}

\vspace{10pt}
\hrule
\vspace{10pt}

\section{Step 3: Standard Form of the Discrete Map}

\subsection*{Present as a map}

We can write the discrete predator-prey system as:
\[
\boxed{\begin{aligned}
x_{n+1} &= x_n(1 + a - by_n) \\
y_{n+1} &= y_n(1 - c + dx_n)
\end{aligned}}
\]

or in vector form as $\mathbf{z}_{n+1} = \mathbf{F}(\mathbf{z}_n)$ where $\mathbf{z}_n = (x_n, y_n)$ and:
\[
\mathbf{F}(x,y) = \begin{pmatrix} x(1 + a - by) \\ y(1 - c + dx) \end{pmatrix}
\]

\subsection*{Alternative formulation}

We can also write this emphasizing the change:
\begin{align*}
x_{n+1} - x_n &= x_n(a - by_n) \\
y_{n+1} - y_n &= y_n(-c + dx_n)
\end{align*}

This makes clear that:
\begin{itemize}
\item Prey increase when $a > by_n$ (birth rate exceeds predation rate)
\item Predators increase when $dx_n > c$ (consumption exceeds death rate)
\end{itemize}

\subsection*{XYZ Analysis of Form}

\begin{itemize}[leftmargin=*]
\item \stage{STAGE X (What the form shows):} The map is \textit{multiplicative} - each population is multiplied by a growth factor that depends on the other population.

\item \stage{STAGE Y (Why multiplicative):} Because the original ODEs are:
\begin{itemize}
\item Linear in each variable separately: $\dot{x} = x(\cdots)$ and $\dot{y} = y(\cdots)$
\item This "factorizable" structure is preserved under Euler discretization
\item Each population's next value is current value $\times$ (1 + change rate)
\end{itemize}
If either population is zero, it remains zero (extinction is permanent). The interaction terms $xy$ couple the equations but maintain the multiplicative structure.

\item \stage{STAGE Z (What this means for analysis):}
\begin{itemize}
\item The map has fixed points where $x_{n+1} = x_n$ and $y_{n+1} = y_n$
\item We can analyze stability using the Jacobian matrix
\item For small time steps, behavior mimics continuous system
\item For larger time steps, discrete map can show different dynamics
\end{itemize}
\end{itemize}

\vspace{10pt}
\hrule
\vspace{10pt}

\section{Step 4: Properties of the Discrete Map}

\subsection*{Fixed points}

Fixed points satisfy $x_{n+1} = x_n$ and $y_{n+1} = y_n$:
\begin{align*}
x^* &= x^*(1 + a - by^*) \\
y^* &= y^*(1 - c + dx^*)
\end{align*}

This gives:
\begin{align*}
0 &= x^*(a - by^*) \\
0 &= y^*(-c + dx^*)
\end{align*}

\textbf{Fixed point 1:} $(x^*, y^*) = (0, 0)$ - extinction of both species

\textbf{Fixed point 2:} $x^* = 0$, $y^* \neq 0$ gives $0 = y^*(-c) \Rightarrow$ no solution (unless $y^*=0$)

\textbf{Fixed point 3:} $y^* = 0$, $x^* \neq 0$ gives $0 = x^* \cdot a \Rightarrow$ no solution (unless $x^*=0$)

\textbf{Fixed point 4:} $x^*, y^* \neq 0$ requires:
\begin{align*}
a - by^* &= 0 \quad \Rightarrow \quad y^* = \frac{a}{b} \\
-c + dx^* &= 0 \quad \Rightarrow \quad x^* = \frac{c}{d}
\end{align*}

Therefore: $\boxed{(x^*, y^*) = \left(\frac{c}{d}, \frac{a}{b}\right)}$ - coexistence equilibrium

\subsection*{Jacobian matrix}

The Jacobian of $\mathbf{F}(x,y)$ is:
\[
J = \begin{pmatrix}
\frac{\partial}{\partial x}[x(1+a-by)] & \frac{\partial}{\partial y}[x(1+a-by)] \\[10pt]
\frac{\partial}{\partial x}[y(1-c+dx)] & \frac{\partial}{\partial y}[y(1-c+dx)]
\end{pmatrix}
\]

Computing derivatives:
\[
J = \begin{pmatrix}
1 + a - by & -bx \\
dy & 1 - c + dx
\end{pmatrix}
\]

At the coexistence fixed point $(x^*, y^*) = (c/d, a/b)$:
\[
J^* = \begin{pmatrix}
1 + a - b \cdot \frac{a}{b} & -b \cdot \frac{c}{d} \\[10pt]
d \cdot \frac{a}{b} & 1 - c + d \cdot \frac{c}{d}
\end{pmatrix} = \begin{pmatrix}
1 & -\frac{bc}{d} \\[10pt]
\frac{ad}{b} & 1
\end{pmatrix}
\]

\subsection*{Eigenvalue analysis}

For the Jacobian $J^* = \begin{pmatrix} 1 & -\frac{bc}{d} \\ \frac{ad}{b} & 1 \end{pmatrix}$:

Characteristic equation:
\[
\det(J^* - \lambda I) = (1-\lambda)^2 + \frac{bc}{d} \cdot \frac{ad}{b} = 0
\]

\[
(1-\lambda)^2 + ac = 0
\]

\[
\lambda = 1 \pm i\sqrt{ac}
\]

The eigenvalues are complex with:
\begin{itemize}
\item Real part: $\text{Re}(\lambda) = 1$
\item Imaginary part: $\text{Im}(\lambda) = \pm\sqrt{ac}$
\item Modulus: $|\lambda| = \sqrt{1 + ac}$
\end{itemize}

Since $|\lambda| = \sqrt{1 + ac} > 1$ (for $a, c > 0$), the fixed point is \textbf{unstable}.

\subsection*{XYZ Analysis of Fixed Points}

\begin{itemize}[leftmargin=*]
\item \stage{STAGE X (What we found):} Two fixed points: $(0,0)$ (trivial) and $(c/d, a/b)$ (coexistence). The coexistence point has complex eigenvalues with modulus $>1$.

\item \stage{STAGE Y (Why this instability):}
\begin{itemize}
\item In the continuous Lotka-Volterra system, the coexistence point is a \textit{center} with purely imaginary eigenvalues - neutral stability with closed orbits
\item The discrete map shifts eigenvalues: $\lambda_{\text{map}} \approx e^{\lambda_{\text{ODE}} \Delta t}$
\item For the ODE with $\lambda_{\text{ODE}} = \pm i\sqrt{ac}$, we get $\lambda_{\text{map}} = e^{\pm i\sqrt{ac}}$ which has $|\lambda| = 1$
\item However, our Euler discretization introduces additional terms that push $|\lambda|$ slightly above 1
\item The spiral instability means orbits slowly diverge outward from the fixed point
\end{itemize}

\item \stage{STAGE Z (What this means):} The discrete map has fundamentally different stability than the continuous system:
\begin{itemize}
\item Continuous: Neutral stability, periodic orbits (conservative system)
\item Discrete (with $\Delta t = 1$): Unstable spiral (trajectories diverge)
\end{itemize}
This illustrates that discrete maps are \textit{not} just approximations to ODEs - they can exhibit genuinely different dynamics. For predator-prey, the discretization breaks the conservation law that existed in the continuous case.
\end{itemize}

\vspace{10pt}
\hrule
\vspace{10pt}

\section{Step 5: Comparison with Continuous System}

\subsection*{Continuous system properties}

The Lotka-Volterra ODE has:
\begin{itemize}
\item Fixed point at $(c/d, a/b)$ with purely imaginary eigenvalues $\pm i\sqrt{ac}$
\item This is a \textbf{center} - neutrally stable
\item Solutions are closed periodic orbits around the fixed point
\item System is \textbf{conservative}: has a conserved quantity $H(x,y) = d x - c\log x + b y - a\log y$
\end{itemize}

\subsection*{Discrete system properties}

The discrete map has:
\begin{itemize}
\item Same fixed point at $(c/d, a/b)$
\item But eigenvalues $1 \pm i\sqrt{ac}$ have modulus $\sqrt{1+ac} > 1$
\item This is an \textbf{unstable spiral}
\item Solutions spiral outward (for $\Delta t = 1$)
\item System is \textbf{not conservative}: no preserved quantity
\end{itemize}

\subsection*{Why the difference?}

\begin{enumerate}
\item \textbf{Euler method is first-order}: It only captures behavior to $O(\Delta t)$

\item \textbf{Time step too large}: For $\Delta t = 1$, discrete approximation introduces significant error

\item \textbf{Conservation broken}: Euler method doesn't preserve the Hamiltonian structure

\item \textbf{Eigenvalue transformation}: The map $\lambda_{\text{map}} = e^{\lambda_{\text{ODE}}\Delta t}$ takes $\pm i\omega \to e^{\pm i\omega}$ which has $|e^{\pm i\omega}| = 1$, but the Euler approximation $\lambda \approx 1 + \lambda_{\text{ODE}}\Delta t$ gives $1 \pm i\omega$ with $|1 \pm i\omega| = \sqrt{1+\omega^2} > 1$
\end{enumerate}

\subsection*{XYZ Analysis of Comparison}

\begin{itemize}[leftmargin=*]
\item \stage{STAGE X (What differs):} The continuous system has periodic orbits (center), while the naive discrete system has spiraling unstable orbits.

\item \stage{STAGE Y (Why this happens):}
\begin{itemize}
\item The continuous predator-prey system is \textit{Hamiltonian} - it conserves energy-like quantities
\item Euler discretization is \textit{not symplectic} - it doesn't preserve Hamiltonian structure
\item Each iteration adds a small numerical dissipation/excitation
\item Over many iterations, these errors accumulate, causing spiraling
\item The magnitude $|\lambda| = \sqrt{1 + ac}$ quantifies the "per-iteration drift"
\end{itemize}
Better discretization schemes (like symplectic integrators) can preserve the center structure.

\item \stage{STAGE Z (What we learn):}
\begin{enumerate}
\item \textbf{Maps $\neq$ ODEs}: Discrete maps are independent models with their own dynamics
\item \textbf{Discretization matters}: Choice of scheme affects qualitative behavior
\item \textbf{Time step critical}: Smaller $\Delta t$ improves agreement with ODE
\item \textbf{Both are valid}: The map models discrete-time processes (generations), the ODE models continuous time
\end{enumerate}
In applications where time is naturally discrete (insect populations with distinct generations), the map may be more appropriate than the ODE.
\end{itemize}

\vspace{10pt}
\hrule
\vspace{10pt}

\section{Step 6: Improved Discretization (Optional)}

\subsection*{Better time step}

For smaller time step $\Delta t \ll 1$, the discrete map becomes:
\begin{align*}
x_{n+1} &= x_n[1 + \Delta t(a - by_n)] \\
y_{n+1} &= y_n[1 + \Delta t(-c + dx_n)]
\end{align*}

At the fixed point $(c/d, a/b)$, eigenvalues:
\[
\lambda \approx 1 \pm i\sqrt{ac}\,\Delta t
\]

with modulus:
\[
|\lambda| = \sqrt{1 + ac(\Delta t)^2} \approx 1 + \frac{ac(\Delta t)^2}{2}
\]

As $\Delta t \to 0$, we have $|\lambda| \to 1$, recovering the center behavior.

\subsection*{Stroboscopic interpretation}

The discrete map with $\Delta t = 1$ can be viewed as:
\begin{itemize}
\item A \textbf{stroboscopic map} of the continuous system
\item Sampling the ODE solution at times $t = 0, 1, 2, 3, \ldots$
\item Each "flash" captures the instantaneous populations
\item The sequence $\{(x_n, y_n)\}$ traces out points on the continuous trajectory
\end{itemize}

For systems with natural periodicity (seasonal breeding), stroboscopic sampling at appropriate intervals gives meaningful discrete models.

\subsection*{XYZ Analysis of Improvements}

\begin{itemize}[leftmargin=*]
\item \stage{STAGE X (What improves):} Using smaller $\Delta t$ makes the discrete map better approximate the continuous ODE's qualitative behavior.

\item \stage{STAGE Y (Why smaller is better):} Taylor expansion shows:
\[
x(t + \Delta t) = x(t) + \dot{x}(t)\Delta t + \frac{1}{2}\ddot{x}(t)(\Delta t)^2 + O(\Delta t^3)
\]
Euler method only uses first two terms, so error is $O(\Delta t^2)$ per step. Over time interval $T$, we take $n = T/\Delta t$ steps, accumulating error $\sim n \cdot (\Delta t)^2 = T \cdot \Delta t$. Thus error $\to 0$ as $\Delta t \to 0$.

\item \stage{STAGE Z (What this teaches):}
\begin{itemize}
\item \textbf{Resolution vs. accuracy}: Smaller $\Delta t$ requires more iterations but gives better accuracy
\item \textbf{Map as model}: When $\Delta t$ is naturally determined (breeding season), accept the map's own dynamics
\item \textbf{Map as algorithm}: When approximating ODE, choose $\Delta t$ carefully
\end{itemize}
The "right" choice depends on whether we're modeling inherently discrete-time processes or approximating continuous-time processes.
\end{itemize}

\vspace{10pt}
\hrule
\vspace{10pt}

\section{Summary}

\subsection*{Derivation}

Starting from continuous Lotka-Volterra equations:
\[
\dot{x} = ax - bxy, \quad \dot{y} = -cy + dxy
\]

Euler approximation with $\Delta t = 1$ gives discrete predator-prey map:
\[
\boxed{\begin{aligned}
x_{n+1} &= x_n(1 + a - by_n) \\
y_{n+1} &= y_n(1 - c + dx_n)
\end{aligned}}
\]

\subsection*{Key properties}

\begin{itemize}
\item \textbf{Fixed points:}
\begin{itemize}
\item Extinction: $(0, 0)$
\item Coexistence: $(c/d, a/b)$
\end{itemize}

\item \textbf{Stability:} Coexistence point has eigenvalues $\lambda = 1 \pm i\sqrt{ac}$ with $|\lambda| = \sqrt{1+ac} > 1$ (unstable spiral)

\item \textbf{Dynamics:} Unlike continuous system (neutral center with closed orbits), discrete system has spiraling trajectories
\end{itemize}

\subsection*{Biological interpretation}

The discrete map models populations measured at regular intervals:
\begin{itemize}
\item $x_n$ = number of prey at generation $n$
\item $y_n$ = number of predators at generation $n$
\item Updates depend on current populations through:
\begin{itemize}
\item Prey growth rate: $1 + a - by_n$ (high predators $\Rightarrow$ low growth)
\item Predator growth rate: $1 - c + dx_n$ (high prey $\Rightarrow$ high growth)
\end{itemize}
\end{itemize}

\subsection*{Continuous vs. discrete}

\begin{center}
\begin{tabular}{|l|c|c|}
\hline
\textbf{Property} & \textbf{Continuous ODE} & \textbf{Discrete Map ($\Delta t=1$)} \\
\hline
Fixed point & $(c/d, a/b)$ & $(c/d, a/b)$ \\
Eigenvalues & $\pm i\sqrt{ac}$ & $1 \pm i\sqrt{ac}$ \\
$|\lambda|$ & $\sqrt{ac}$ & $\sqrt{1 + ac}$ \\
Stability type & Center (neutral) & Unstable spiral \\
Orbits & Closed periodic & Outward spiraling \\
Conservative? & Yes & No \\
\hline
\end{tabular}
\end{center}

\textbf{Conclusion:} The discrete predator-prey map is a valid model in its own right for discrete-generation populations, but exhibits different long-term behavior than the continuous model due to broken conservation and numerical artifacts of Euler discretization.

\end{document}
