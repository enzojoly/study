\documentclass[11pt,a4paper]{article}

% Packages
\usepackage{amsmath}
\usepackage{amssymb}
\usepackage{amsthm}
\usepackage[margin=1in]{geometry}
\usepackage{enumitem}
\usepackage{tikz}
\usepackage{pgfplots}
\usepackage{xcolor}
\pgfplotsset{compat=1.18}

% Custom commands
\newcommand{\stage}[1]{\textbf{\textcolor{blue}{#1}}}

% Title information
\title{Exercise Sheet 4: Maps\\
Question 3 - Complete Solution}
\author{Methods of Applied Mathematics}
\date{}

\begin{document}

\maketitle

\section*{Problem Statement}

The logistic map is given by:
\[
x_{n+1} = rx_n(1-x_n)
\]

with fixed points at:
\[
x_1^* = 0 \quad \text{and} \quad x_2^* = \frac{r-1}{r}
\]

\textbf{Tasks:}
\begin{itemize}
\item Derive the linearization of the map about each fixed point
\item Show that $x_1^*$ is unstable for $r > 1$
\item Show that $x_2^*$ is stable for $r > 1$
\end{itemize}

\vspace{10pt}
\hrule
\vspace{10pt}

\section{Step 1: Verify Fixed Points}

\subsection*{Definition of fixed point}

A fixed point $x^*$ of map $x_{n+1} = f(x_n)$ satisfies:
\[
x^* = f(x^*)
\]

For the logistic map, $f(x) = rx(1-x)$.

\subsection*{Check $x_1^* = 0$}

\begin{align*}
f(0) &= r \cdot 0 \cdot (1-0) \\
&= 0 \quad \checkmark
\end{align*}

\subsection*{Check $x_2^* = (r-1)/r$}

\begin{align*}
f\left(\frac{r-1}{r}\right) &= r \cdot \frac{r-1}{r} \cdot \left(1 - \frac{r-1}{r}\right) \\
&= (r-1) \cdot \left(\frac{r - (r-1)}{r}\right) \\
&= (r-1) \cdot \frac{1}{r} \\
&= \frac{r-1}{r} \quad \checkmark
\end{align*}

\subsection*{XYZ Analysis of Fixed Points}

\begin{itemize}[leftmargin=*]
\item \stage{STAGE X (What we have):} Two fixed points: $x_1^* = 0$ (always exists) and $x_2^* = (r-1)/r$ (exists for all $r$, coincides with $x_1^*$ at $r=1$).

\item \stage{STAGE Y (Why these are the only fixed points):} Solving $x = rx(1-x)$:
\begin{align*}
x &= rx - rx^2 \\
0 &= x(r(1-x) - 1)
\end{align*}
This gives $x=0$ or $r(1-x) = 1 \Rightarrow x = (r-1)/r$.

\item \stage{STAGE Z (What this means):} $x_1^* = 0$ represents extinction. $x_2^*$ represents steady-state population. For $r>1$, we have $x_2^* > 0$ (positive equilibrium exists). Stability determines which state the system reaches.
\end{itemize}

\vspace{10pt}
\hrule
\vspace{10pt}

\section{Step 2: Compute Derivative}

\subsection*{Linearization formula}

For map $x_{n+1} = f(x_n)$, Taylor expansion near $x^*$:
\[
f(x^* + \epsilon) = f(x^*) + f'(x^*) \epsilon + O(\epsilon^2)
\]

Since $f(x^*) = x^*$:
\[
\boxed{x_{n+1} - x^* \approx f'(x^*)(x_n - x^*)}
\]

\subsection*{Compute $f'(x)$}

For $f(x) = rx(1-x) = rx - rx^2$:
\[
f'(x) = r - 2rx = r(1 - 2x)
\]

Therefore:
\[
\boxed{f'(x) = r(1-2x)}
\]

\subsection*{XYZ Analysis}

\begin{itemize}[leftmargin=*]
\item \stage{STAGE X (What we computed):} Derivative $f'(x) = r(1-2x)$ is linear in $x$, depends on parameter $r$.

\item \stage{STAGE Y (Why this form):} The logistic map is a parabola with:
\begin{itemize}
\item Maximum at $x = 1/2$ where $f'(1/2) = 0$
\item $f'(0) = r$ (slope at origin)
\item $f'(1) = -r$ (slope at boundary)
\end{itemize}
The derivative measures local stretching/contraction - key for stability.

\item \stage{STAGE Z (What this means):} If $|f'(x^*)| < 1$: nearby points contract toward $x^*$ (stable). If $|f'(x^*)| > 1$: nearby points stretch away from $x^*$ (unstable).
\end{itemize}

\vspace{10pt}
\hrule
\vspace{10pt}

\section{Step 3: Stability of $x_1^* = 0$}

\subsection*{Evaluate derivative}

\[
\lambda_1 = f'(0) = r(1 - 0) = r
\]

\subsection*{Linearized map}

Let $\epsilon_n = x_n - 0 = x_n$:
\[
\epsilon_{n+1} = r \epsilon_n
\]

Solution by iteration:
\[
\epsilon_n = r^n \epsilon_0 \quad \Rightarrow \quad x_n = r^n x_0
\]

\subsection*{Stability criterion}

Fixed point stable if $|\lambda| < 1$, unstable if $|\lambda| > 1$.

For $x_1^*$: $|\lambda_1| = r$ (assuming $r > 0$)

\begin{align*}
r < 1: \quad & |\lambda_1| < 1 \quad \Rightarrow \quad \text{STABLE} \\
r = 1: \quad & |\lambda_1| = 1 \quad \Rightarrow \quad \text{NEUTRAL} \\
r > 1: \quad & |\lambda_1| > 1 \quad \Rightarrow \quad \boxed{\text{UNSTABLE}}
\end{align*}

\subsection*{Conclusion}

\[
\boxed{\text{For } r > 1: \quad x_1^* = 0 \text{ is UNSTABLE}}
\]

\subsection*{XYZ Analysis}

\begin{itemize}[leftmargin=*]
\item \stage{STAGE X (What we found):} Eigenvalue $\lambda_1 = r$. For $r>1$: $|\lambda_1| > 1 \Rightarrow$ unstable.

\item \stage{STAGE Y (Why instability):} Solution $x_n = r^n x_0$ shows exponential growth. Each iteration multiplies by $r$. For $r>1$, births exceed deaths, so any nonzero population grows away from extinction. Geometrically, parabola slope at origin ($f'(0) = r$) exceeds diagonal slope (1), pushing trajectories away.

\item \stage{STAGE Z (Biological meaning):} For $r>1$, extinction is unstable - any small population grows. Makes sense: reproduction rate exceeds replacement, so population cannot stay at zero. System escapes toward $x_2^*$.
\end{itemize}

\vspace{10pt}
\hrule
\vspace{10pt}

\section{Step 4: Stability of $x_2^* = (r-1)/r$}

\subsection*{Evaluate derivative}

\begin{align*}
\lambda_2 &= f'\left(\frac{r-1}{r}\right) = r\left(1 - 2 \cdot \frac{r-1}{r}\right) \\
&= r\left(\frac{r - 2(r-1)}{r}\right) = r \cdot \frac{2-r}{r} = 2 - r
\end{align*}

\[
\boxed{\lambda_2 = 2 - r}
\]

\subsection*{Linearized map}

Let $\epsilon_n = x_n - x_2^*$:
\[
\epsilon_{n+1} = (2-r) \epsilon_n
\]

Solution:
\[
\epsilon_n = (2-r)^n \epsilon_0
\]

\subsection*{Stability analysis}

Need $|\lambda_2| = |2-r| < 1$.

\textbf{For $1 < r < 2$:} $\lambda_2 = 2-r \in (0,1)$
\[
|\lambda_2| < 1 \quad \Rightarrow \quad \text{STABLE (monotonic)}
\]

\textbf{For $r = 2$:} $\lambda_2 = 0$
\[
|\lambda_2| = 0 \quad \Rightarrow \quad \text{SUPERSTABLE}
\]

\textbf{For $2 < r < 3$:} $\lambda_2 = 2-r \in (-1,0)$
\[
|\lambda_2| < 1 \quad \Rightarrow \quad \text{STABLE (oscillatory)}
\]

\textbf{For $r = 3$:} $\lambda_2 = -1$
\[
|\lambda_2| = 1 \quad \Rightarrow \quad \text{BIFURCATION}
\]

\textbf{For $r > 3$:} $\lambda_2 < -1$
\[
|\lambda_2| > 1 \quad \Rightarrow \quad \text{UNSTABLE}
\]

\subsection*{Conclusion}

\[
\boxed{\text{For } 1 < r < 3: \quad x_2^* = \frac{r-1}{r} \text{ is STABLE}}
\]

\subsection*{XYZ Analysis}

\begin{itemize}[leftmargin=*]
\item \stage{STAGE X (What we found):} Eigenvalue $\lambda_2 = 2-r$. For $1 < r < 3$: $|\lambda_2| < 1 \Rightarrow$ stable.

\item \stage{STAGE Y (Why stability):} Solution $\epsilon_n = (2-r)^n \epsilon_0$ shows exponential decay toward $x_2^*$:
\begin{itemize}
\item For $1 < r < 2$: $\lambda_2 \in (0,1)$ positive. Monotonic convergence (no oscillation).
\item For $2 < r < 3$: $\lambda_2 \in (-1,0)$ negative. Oscillatory convergence (alternates above/below).
\item At $r=2$: $\lambda_2 = 0$. Instantaneous convergence (superstable).
\end{itemize}

\item \stage{STAGE Z (Dynamic meaning):} For $1 < r < 3$, population converges to equilibrium $x_2^* = 1 - 1/r$. Bifurcations occur at:
\begin{itemize}
\item $r=1$: Transcritical bifurcation (extinction destabilizes, $x_2^*$ born)
\item $r=3$: Flip bifurcation ($x_2^*$ destabilizes, period-2 orbit created)
\end{itemize}
\end{itemize}

\vspace{10pt}
\hrule
\vspace{10pt}

\section{Summary}

\subsection*{Main Results}

For logistic map $x_{n+1} = rx_n(1-x_n)$:

\textbf{Linearization:} $f'(x) = r(1-2x)$

\textbf{Eigenvalues:}
\begin{align*}
\lambda_1 &= r \\
\lambda_2 &= 2 - r
\end{align*}

\textbf{Stability for $r > 1$:}

\begin{center}
\begin{tabular}{ccc}
\textbf{Fixed Point} & \textbf{Eigenvalue} & \textbf{Stability} \\
\hline
$x_1^* = 0$ & $\lambda_1 = r > 1$ & Unstable \\[5pt]
$x_2^* = \dfrac{r-1}{r}$ & $\lambda_2 = 2-r$ & Stable ($1 < r < 3$)
\end{tabular}
\end{center}

\end{document}
