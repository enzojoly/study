\documentclass[11pt,a4paper]{article}

% Packages
\usepackage{amsmath}
\usepackage{amssymb}
\usepackage{amsthm}
\usepackage[margin=1in]{geometry}
\usepackage{enumitem}
\usepackage{tikz}
\usepackage{pgfplots}
\usepackage{xcolor}
\pgfplotsset{compat=1.18}

% Custom commands
\newcommand{\stage}[1]{\textbf{\textcolor{blue}{#1}}}

% Title information
\title{Exercise Sheet 4: Maps\\
Question 4 - Complete Solution}
\author{Methods of Applied Mathematics}
\date{}

\begin{document}

\maketitle

\section*{Problem Statement}

Newton's Method is a numerical tool for finding roots of functions (i.e., $x$ such that $f(x) = 0$). Starting from an initial value $x_0$, we repeatedly apply the mapping:
\[
x_{n+1} = x_n - \frac{f(x_n)}{f'(x_n)}
\]
where $f' \equiv df/dx$, until the method converges on a root. We are iterating a map until it reaches a stable fixed point.

\textbf{Tasks:}
\begin{enumerate}[label=(\alph*)]
\item Show that all fixed points of the map are points where $f(x) = 0$
\item Show that certain roots cannot be found with this method, using the concept of local stability to derive the condition that must be met by a fixed point for it to be reachable by Newton's Method
\end{enumerate}

\vspace{10pt}
\hrule
\vspace{10pt}

\section{Step 1: Define the Newton Map}

\subsection*{Map structure}

Newton's Method defines a map $g: \mathbb{R} \to \mathbb{R}$ given by:
\[
g(x) = x - \frac{f(x)}{f'(x)}
\]

The iterative scheme is:
\[
x_{n+1} = g(x_n)
\]

\subsection*{XYZ Analysis of Newton Map Structure}

\begin{itemize}[leftmargin=*]
\item \stage{STAGE X (What we have):} Newton's method is a discrete dynamical system (a map). Each iteration updates the current guess $x_n$ to a new guess $x_{n+1}$ using the function value $f(x_n)$ and its derivative $f'(x_n)$.

\item \stage{STAGE Y (Why this formula):} The Newton map comes from linear approximation. Near a point $x_n$, we approximate $f(x)$ by its tangent line:
\[
f(x) \approx f(x_n) + f'(x_n)(x - x_n)
\]
To find where this linear approximation crosses zero:
\begin{align*}
0 &= f(x_n) + f'(x_n)(x_{n+1} - x_n) \\
f'(x_n)(x_{n+1} - x_n) &= -f(x_n) \\
x_{n+1} - x_n &= -\frac{f(x_n)}{f'(x_n)} \\
x_{n+1} &= x_n - \frac{f(x_n)}{f'(x_n)}
\end{align*}
Geometrically: draw the tangent line at $(x_n, f(x_n))$, find where it hits the $x$-axis, and that's $x_{n+1}$.

\item \stage{STAGE Z (What this means):} Newton's method is trying to "chase" the root by following tangent lines. If it converges, it converges to a fixed point of the map $g$. Understanding when this works requires analyzing fixed points and their stability.
\end{itemize}

\vspace{10pt}
\hrule
\vspace{10pt}

\section{Step 2: Part (a) - Fixed Points of Newton Map}

\subsection*{Find fixed points}

A fixed point $x^*$ satisfies:
\[
x^* = g(x^*) = x^* - \frac{f(x^*)}{f'(x^*)}
\]

\subsection*{Solve for fixed point condition}

Rearranging:
\begin{align*}
x^* &= x^* - \frac{f(x^*)}{f'(x^*)} \\
0 &= -\frac{f(x^*)}{f'(x^*)} \\
0 &= f(x^*)
\end{align*}

(assuming $f'(x^*) \neq 0$)

\subsection*{Conclusion}

\[
\boxed{\text{Fixed points of } g(x) \text{ are exactly the roots of } f(x)}
\]

\subsection*{XYZ Analysis of Fixed Points}

\begin{itemize}[leftmargin=*]
\item \stage{STAGE X (What we proved):} Every fixed point of the Newton map corresponds to a root of $f$, and every root of $f$ (where $f' \neq 0$) is a fixed point of the Newton map.

\item \stage{STAGE Y (Why this is fundamental):} The equation $x^* = g(x^*)$ means the map doesn't change the value - we've converged. The algebra shows this happens precisely when $f(x^*) = 0$. This is beautiful: we've transformed the root-finding problem "find $x$ such that $f(x) = 0$" into the fixed-point problem "find $x^*$ such that $g(x^*) = x^*$".

The key insight: Newton's method is designed so that its fixed points are exactly the objects we're looking for (roots of $f$). This is not accidental - the map was constructed with this property.

\item \stage{STAGE Z (What this means):} Part (a) tells us WHERE the method converges if it converges (to roots of $f$). But it doesn't tell us WHETHER it converges, or FROM WHICH initial conditions. That's what part (b) addresses through stability analysis.
\end{itemize}

\vspace{10pt}
\hrule
\vspace{10pt}

\section{Step 3: Part (b) - Stability Analysis}

\subsection*{Compute derivative of Newton map}

To determine stability of fixed points, we need $g'(x)$ where:
\[
g(x) = x - \frac{f(x)}{f'(x)}
\]

Using the quotient rule:
\begin{align*}
g'(x) &= \frac{d}{dx}\left[x - \frac{f(x)}{f'(x)}\right] \\
&= 1 - \frac{d}{dx}\left[\frac{f(x)}{f'(x)}\right] \\
&= 1 - \frac{f'(x) \cdot f'(x) - f(x) \cdot f''(x)}{[f'(x)]^2} \\
&= 1 - \frac{[f'(x)]^2 - f(x) f''(x)}{[f'(x)]^2} \\
&= 1 - 1 + \frac{f(x) f''(x)}{[f'(x)]^2} \\
&= \frac{f(x) f''(x)}{[f'(x)]^2}
\end{align*}

Therefore:
\[
\boxed{g'(x) = \frac{f(x) f''(x)}{[f'(x)]^2}}
\]

\subsection*{Evaluate at fixed point}

At a fixed point $x^*$ where $f(x^*) = 0$:
\[
g'(x^*) = \frac{f(x^*) f''(x^*)}{[f'(x^*)]^2} = \frac{0 \cdot f''(x^*)}{[f'(x^*)]^2} = 0
\]

(assuming $f'(x^*) \neq 0$)

\subsection*{Stability criterion from map theory}

From lecture notes (Section 20), a fixed point $x^*$ of a map is:
\begin{itemize}
\item \textbf{Stable} (attracting) if $|g'(x^*)| < 1$
\item \textbf{Unstable} (repelling) if $|g'(x^*)| > 1$
\item \textbf{Marginal} if $|g'(x^*)| = 1$
\end{itemize}

\subsection*{Apply to Newton's method}

At a simple root $x^*$ (where $f(x^*) = 0$ and $f'(x^*) \neq 0$):
\[
|g'(x^*)| = 0 < 1
\]

Therefore:
\[
\boxed{\text{All simple roots are STABLE fixed points of Newton's method}}
\]

\subsection*{XYZ Analysis of Stability}

\begin{itemize}[leftmargin=*]
\item \stage{STAGE X (What we found):} The derivative of the Newton map at any simple root is zero: $g'(x^*) = 0$. This is less than 1, so simple roots are stable (attracting) fixed points.

\item \stage{STAGE Y (Why this works):} The calculation reveals why Newton's method converges so fast (quadratically) near simple roots. The derivative $g'(x^*) = 0$ means the map is "super-stable" at simple roots - not just $|g'| < 1$ but actually $g' = 0$.

Geometrically, near a simple root, the Newton map has a horizontal tangent at the fixed point. The cobweb diagram shows iterates converging very rapidly - much faster than if $0 < |g'| < 1$.

The formula $g'(x) = \frac{f(x)f''(x)}{[f'(x)]^2}$ tells us:
\begin{itemize}
\item Numerator has $f(x)$: this vanishes at roots, giving $g' = 0$
\item Denominator has $[f'(x)]^2$: this is always positive (when $f' \neq 0$)
\item Factor $f''(x)$: the second derivative, which we'll analyze below
\end{itemize}

\item \stage{STAGE Z (What this means):} Simple roots (where $f'(x^*) \neq 0$) are always reachable by Newton's method, provided we start close enough. But what about roots where $f'(x^*) = 0$? Those are next.
\end{itemize}

\vspace{10pt}
\hrule
\vspace{10pt}

\section{Step 4: Special Case - Multiple Roots}

\subsection*{What if $f'(x^*) = 0$?}

If $x^*$ is a root where $f(x^*) = 0$ AND $f'(x^*) = 0$, then $x^*$ is a \textbf{multiple root} (at least double).

Near such a root, write $f(x) = (x - x^*)^m h(x)$ where $m \geq 2$ and $h(x^*) \neq 0$.

\subsection*{Recompute Newton map behavior}

For a multiple root of order $m$:
\begin{align*}
f(x) &= (x - x^*)^m h(x) \\
f'(x) &= m(x - x^*)^{m-1} h(x) + (x - x^*)^m h'(x) \\
&= (x - x^*)^{m-1}[mh(x) + (x-x^*)h'(x)]
\end{align*}

Near $x^*$:
\[
\frac{f(x)}{f'(x)} = \frac{(x-x^*)^m h(x)}{(x-x^*)^{m-1}[mh(x) + (x-x^*)h'(x)]} \approx \frac{(x-x^*) h(x^*)}{m h(x^*)} = \frac{x - x^*}{m}
\]

Therefore:
\[
g(x) = x - \frac{f(x)}{f'(x)} \approx x - \frac{x - x^*}{m} = x^* + \left(1 - \frac{1}{m}\right)(x - x^*)
\]

The derivative at the fixed point:
\[
g'(x^*) = 1 - \frac{1}{m}
\]

\subsection*{Stability of multiple roots}

For a root of multiplicity $m \geq 2$:
\[
|g'(x^*)| = \left|1 - \frac{1}{m}\right| = \frac{m-1}{m}
\]

Analysis:
\begin{itemize}
\item $m = 1$ (simple root): $g'(x^*) = 0$ $\Rightarrow$ stable ✓
\item $m = 2$ (double root): $g'(x^*) = 1/2$ $\Rightarrow$ stable, but slower convergence
\item $m = 3$ (triple root): $g'(x^*) = 2/3$ $\Rightarrow$ stable, but even slower
\item As $m \to \infty$: $g'(x^*) \to 1$ $\Rightarrow$ marginally stable
\end{itemize}

All multiple roots remain stable ($|g'| < 1$), but convergence slows down as multiplicity increases.

\subsection*{XYZ Analysis of Multiple Roots}

\begin{itemize}[leftmargin=*]
\item \stage{STAGE X (What we found):} Multiple roots are still attracting fixed points, but with $g'(x^*) = (m-1)/m$ instead of $0$. Convergence is linear rather than quadratic.

\item \stage{STAGE Y (Why convergence slows):} For a simple root, the numerator $f(x)$ in Newton's formula vanishes faster than the denominator $f'(x)$ as $x \to x^*$, giving a large correction step. For multiple roots, both numerator and denominator vanish at the same rate, so the correction step is smaller - only $(x-x^*)/m$ instead of $(x-x^*)$.

The derivative $g' = (m-1)/m$ measures the "contraction rate": each iteration reduces the error by a factor of $(m-1)/m$. For $m=2$, error is halved each step. For $m=10$, error is only reduced by 10%.

\item \stage{STAGE Z (What this means practically):} Multiple roots can still be found but require many more iterations. Modified Newton's method (multiplying by $m$ if multiplicity is known) can restore quadratic convergence: $x_{n+1} = x_n - m \frac{f(x_n)}{f'(x_n)}$.
\end{itemize}

\vspace{10pt}
\hrule
\vspace{10pt}

\section{Step 5: When Newton's Method Fails}

\subsection*{Conditions for failure}

Newton's method can fail to reach a root when:

\textbf{(1) Division by zero:} If $f'(x_n) = 0$ at any iteration but $f(x_n) \neq 0$, the map is undefined:
\[
x_{n+1} = x_n - \frac{f(x_n)}{0} = \text{undefined}
\]

\textbf{(2) Unstable dynamics:} Even if all roots are stable fixed points, the initial condition $x_0$ might lie in a basin of repulsion or lead to chaotic behavior.

\textbf{(3) Cycles:} The iterates might enter a periodic orbit that doesn't include any root.

\subsection*{Example of failure}

Consider $f(x) = x^{1/3}$ with root at $x^* = 0$.

Near $x = 0$:
\begin{align*}
f(x) &= x^{1/3} \\
f'(x) &= \frac{1}{3}x^{-2/3}
\end{align*}

Newton map:
\[
g(x) = x - \frac{x^{1/3}}{(1/3)x^{-2/3}} = x - 3x = -2x
\]

Therefore:
\begin{align*}
x_1 &= -2x_0 \\
x_2 &= -2x_1 = 4x_0 \\
x_3 &= -2x_2 = -8x_0 \\
&\vdots
\end{align*}

The iterates diverge: $|x_n| = 2^n |x_0| \to \infty$!

\subsection*{Why this fails}

The derivative at the root:
\[
g'(0) = \lim_{x \to 0} \frac{d}{dx}[-2x] = -2
\]

Since $|g'(0)| = 2 > 1$, the root $x^* = 0$ is an \textbf{unstable} fixed point.

Why? Because $f'(0)$ doesn't exist (or is infinite), violating our assumption of a simple root.

\subsection*{XYZ Analysis of Failure Modes}

\begin{itemize}[leftmargin=*]
\item \stage{STAGE X (What we discovered):} Not all roots can be found by Newton's method. Roots where $f'(x^*) = 0$ or where $f'$ doesn't exist can be unstable fixed points.

\item \stage{STAGE Y (Why instability occurs):} The formula $g'(x^*) = \frac{f(x^*)f''(x^*)}{[f'(x^*)]^2}$ only gives $g' = 0$ when:
\begin{itemize}
\item $f(x^*) = 0$ (it's a root)
\item $f'(x^*) \neq 0$ (denominator is finite)
\end{itemize}
If $f'(x^*) = 0$, the formula is indeterminate $0/0$. More careful analysis (like the multiple root case) is needed. If $f'$ has a singularity or vanishes in pathological ways, $g'$ can exceed 1 in magnitude, making the fixed point repelling.

The $x^{1/3}$ example: $f'(0)$ is infinite, so the denominator in $g'$ vanishes, and the Newton map becomes $g(x) = -2x$ with slope $-2$ at the origin.

\item \stage{STAGE Z (What this means for applications):} Before applying Newton's method:
\begin{enumerate}
\item Check that $f'(x) \neq 0$ near roots (no critical points)
\item Start with a good initial guess (in the basin of attraction)
\item Monitor for cycles or divergence
\item Consider modified Newton for multiple roots
\end{enumerate}
The method is not universally convergent, but when it works (simple roots, good initial guess), it's exceptionally fast.
\end{itemize}

\vspace{10pt}
\hrule
\vspace{10pt}

\section{Step 6: Reachability Condition}

\subsection*{Formal statement}

A root $x^*$ of $f(x)$ is \textbf{reachable} by Newton's method if and only if:
\[
\boxed{|g'(x^*)| < 1 \quad \text{where} \quad g'(x) = \frac{f(x)f''(x)}{[f'(x)]^2}}
\]

For a simple root (where $f(x^*) = 0$ and $f'(x^*) \neq 0$):
\[
|g'(x^*)| = 0 < 1 \quad \Rightarrow \quad \boxed{\text{REACHABLE}}
\]

For a multiple root of order $m$ (where $f(x^*) = f'(x^*) = \cdots = f^{(m-1)}(x^*) = 0$ and $f^{(m)}(x^*) \neq 0$):
\[
|g'(x^*)| = \frac{m-1}{m} < 1 \quad \Rightarrow \quad \boxed{\text{REACHABLE (but slow)}}
\]

For pathological roots (where $f'$ is undefined or behaves badly):
\[
|g'(x^*)| \geq 1 \quad \Rightarrow \quad \boxed{\text{NOT REACHABLE}}
\]

\subsection*{Basin of attraction}

Even for reachable roots, convergence depends on initial condition $x_0$ being in the \textbf{basin of attraction}:
\[
\mathcal{B}(x^*) = \{x_0 : \lim_{n \to \infty} x_n = x^* \text{ under Newton iteration}\}
\]

The basin structure can be complex when multiple roots exist.

\vspace{10pt}
\hrule
\vspace{10pt}

\section{Summary}

\subsection*{Part (a): Fixed Points Are Roots}

Starting from $x^* = g(x^*)$ where $g(x) = x - f(x)/f'(x)$:
\[
x^* = x^* - \frac{f(x^*)}{f'(x^*)} \quad \Rightarrow \quad f(x^*) = 0
\]

\textbf{Conclusion:} Fixed points of Newton map $\Leftrightarrow$ roots of $f$

\subsection*{Part (b): Stability Condition for Reachability}

The derivative of the Newton map:
\[
g'(x) = \frac{f(x) f''(x)}{[f'(x)]^2}
\]

At a simple root ($f(x^*) = 0$, $f'(x^*) \neq 0$):
\[
g'(x^*) = 0 < 1 \quad \Rightarrow \quad \text{STABLE (reachable)}
\]

\textbf{Reachability condition:} A root is reachable if $|g'(x^*)| < 1$

This holds for:
\begin{itemize}
\item All simple roots: $g'(x^*) = 0$
\item All multiple roots: $g'(x^*) = (m-1)/m < 1$
\end{itemize}

This fails when:
\begin{itemize}
\item $f'(x^*)$ vanishes or is undefined in pathological ways
\item $f$ has singularities near the root
\end{itemize}

\textbf{Key insight:} Newton's method is designed so roots are automatically stable fixed points (when $f'$ is well-behaved), explaining its widespread success in practice.

\end{document}
