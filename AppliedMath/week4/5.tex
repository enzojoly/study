\documentclass[11pt,a4paper]{article}

% Packages
\usepackage{amsmath}
\usepackage{amssymb}
\usepackage{amsthm}
\usepackage[margin=0.4in]{geometry}
\usepackage{enumitem}
\usepackage{xcolor}

% Custom commands
\newcommand{\stage}[1]{\textbf{\textcolor{blue}{#1}}}

% Title information
\title{Exercise Sheet 4: Maps\\
Question 5 - Complete Solution}
\author{Methods of Applied Mathematics}
\date{}

\begin{document}

\maketitle

\section*{Problem Statement}

Solve the map:
\begin{align*}
x_{n+1} &= 2x_n - y_n \\
y_{n+1} &= 2y_n - x_n
\end{align*}

with initial condition $x_0 = 1$, $y_0 = 0$.

\textbf{Part (a):} Iterate the map repeatedly until a pattern emerges.

\textbf{Part (b):} Use eigenvalue decomposition as for ODEs, but with $\lambda^n$ powers instead of $e^{\lambda t}$ terms.

\vspace{10pt}
\hrule
\vspace{10pt}

\section{Part (a): Direct Iteration Method}

\subsection*{Set up iteration procedure}

The map is a discrete-time dynamical system. Starting from $(x_0, y_0) = (1, 0)$, we repeatedly apply:
\begin{align*}
x_{n+1} &= 2x_n - y_n \\
y_{n+1} &= 2y_n - x_n
\end{align*}

\subsection*{Compute successive iterates}

\textbf{Iteration 0 $\to$ 1:}
\begin{align*}
x_1 &= 2x_0 - y_0 = 2(1) - 0 = 2 \\
y_1 &= 2y_0 - x_0 = 2(0) - 1 = -1
\end{align*}

\textbf{Iteration 1 $\to$ 2:}
\begin{align*}
x_2 &= 2x_1 - y_1 = 2(2) - (-1) = 5 \\
y_2 &= 2y_1 - x_1 = 2(-1) - 2 = -4
\end{align*}

\textbf{Iteration 2 $\to$ 3:}
\begin{align*}
x_3 &= 2x_2 - y_2 = 2(5) - (-4) = 14 \\
y_3 &= 2y_2 - x_2 = 2(-4) - 5 = -13
\end{align*}

\textbf{Iteration 3 $\to$ 4:}
\begin{align*}
x_4 &= 2x_3 - y_3 = 2(14) - (-13) = 41 \\
y_4 &= 2y_3 - x_3 = 2(-13) - 14 = -40
\end{align*}

\textbf{Iteration 4 $\to$ 5:}
\begin{align*}
x_5 &= 2x_4 - y_4 = 2(41) - (-40) = 122 \\
y_5 &= 2y_4 - x_4 = 2(-40) - 41 = -121
\end{align*}

\subsection*{Tabulate results}

\begin{center}
\begin{tabular}{|c|c|c|}
\hline
$n$ & $x_n$ & $y_n$ \\
\hline
0 & 1 & 0 \\
1 & 2 & $-1$ \\
2 & 5 & $-4$ \\
3 & 14 & $-13$ \\
4 & 41 & $-40$ \\
5 & 122 & $-121$ \\
\hline
\end{tabular}
\end{center}

\subsection*{XYZ Analysis of Iteration Pattern}

\begin{itemize}[leftmargin=*]
\item \stage{STAGE X (What we observe):} The values grow rapidly. The $x_n$ sequence is $1, 2, 5, 14, 41, 122, \ldots$ and the $y_n$ sequence is $0, -1, -4, -13, -40, -121, \ldots$. Note that $x_n$ and $y_n$ are always close: $x_n - y_n = 1, 3, 9, 27, 81, 243, \ldots = 3^n$.

\item \stage{STAGE Y (Why this pattern):} Looking at combinations:
\begin{itemize}
\item $x_n + y_n$: $1, 1, 1, 1, 1, 1, \ldots$ (constant!)
\item $x_n - y_n$: $1, 3, 9, 27, 81, 243, \ldots$ (powers of 3: $3^n$)
\end{itemize}
The map preserves $x_n + y_n = 1$ but amplifies the difference $x_n - y_n$ by factor of 3 each iteration. This suggests the system has two eigenvalues: $\lambda_1 = 3$ (exponential growth in the difference direction) and $\lambda_2 = 1$ (constant in the sum direction).

\item \stage{STAGE Z (What this means):} From the pattern $x_n - y_n = 3^n$ and $x_n + y_n = 1$, we can solve:
\[
x_n = \frac{(x_n + y_n) + (x_n - y_n)}{2} = \frac{1 + 3^n}{2}
\]
\[
y_n = \frac{(x_n + y_n) - (x_n - y_n)}{2} = \frac{1 - 3^n}{2}
\]
This gives explicit formulas, which we'll verify rigorously in part (b).
\end{itemize}

\subsection*{Conjectured solution from iteration}

\[
\boxed{x_n = \frac{3^n + 1}{2}, \quad y_n = \frac{1 - 3^n}{2}}
\]

\vspace{10pt}
\hrule
\vspace{10pt}

\section{Part (b): Eigenvalue Decomposition Method}

\subsection*{Step 1: Write system in matrix form}

The map can be written as:
\[
\begin{pmatrix} x_{n+1} \\ y_{n+1} \end{pmatrix} = \begin{pmatrix} 2 & -1 \\ -1 & 2 \end{pmatrix} \begin{pmatrix} x_n \\ y_n \end{pmatrix}
\]

Define the matrix:
\[
A = \begin{pmatrix} 2 & -1 \\ -1 & 2 \end{pmatrix}
\]

\subsection*{XYZ Analysis of Matrix Structure}

\begin{itemize}[leftmargin=*]
\item \stage{STAGE X (What we have):} A $2 \times 2$ symmetric matrix with 2's on the diagonal and $-1$'s off-diagonal.

\item \stage{STAGE Y (Why this form matters):} Symmetry guarantees real eigenvalues and orthogonal eigenvectors. The structure $A = 2I - J$ where $J$ is the all-ones off-diagonal suggests eigenvalues related to sum and difference coordinates.

\item \stage{STAGE Z (What to compute):} Find eigenvalues $\lambda$ and eigenvectors $\mathbf{v}$ to decompose the solution as $\mathbf{x}_n = \alpha_1 \lambda_1^n \mathbf{v}_1 + \alpha_2 \lambda_2^n \mathbf{v}_2$.
\end{itemize}

\vspace{10pt}
\hrule
\vspace{10pt}

\subsection*{Step 2: Find eigenvalues}

Solve the characteristic equation $\det(A - \lambda I) = 0$:
\[
\det\begin{pmatrix} 2-\lambda & -1 \\ -1 & 2-\lambda \end{pmatrix} = 0
\]

Expand:
\begin{align*}
(2-\lambda)^2 - (-1)(-1) &= 0 \\
(2-\lambda)^2 - 1 &= 0 \\
4 - 4\lambda + \lambda^2 - 1 &= 0 \\
\lambda^2 - 4\lambda + 3 &= 0
\end{align*}

Factor:
\[
(\lambda - 3)(\lambda - 1) = 0
\]

Therefore:
\[
\boxed{\lambda_1 = 3, \quad \lambda_2 = 1}
\]

\subsection*{XYZ Analysis of Eigenvalues}

\begin{itemize}[leftmargin=*]
\item \stage{STAGE X (What we found):} Two positive real eigenvalues: $\lambda_1 = 3 > 1$ and $\lambda_2 = 1$.

\item \stage{STAGE Y (Why these values):}
\begin{itemize}
\item $\lambda_1 = 3 > 1$: This eigenvalue causes exponential growth. Points in this eigendirection grow by factor 3 each iteration.
\item $\lambda_2 = 1$: This eigenvalue preserves magnitude. Points in this eigendirection remain at constant distance from origin.
\end{itemize}
For stability analysis: $|\lambda_1| = 3 > 1$ means unstable (exponential growth), $|\lambda_2| = 1$ means marginally stable (neutral). Any initial condition with nonzero component in the $\lambda_1$ eigendirection will grow to infinity.

\item \stage{STAGE Z (What this means dynamically):} The system has one unstable direction (growing like $3^n$) and one neutral direction (constant). Our initial condition $(1,0)$ must have components in both directions, explaining why the iterations showed both growth and a constant pattern.
\end{itemize}

\vspace{10pt}
\hrule
\vspace{10pt}

\subsection*{Step 3: Find eigenvectors}

\textbf{For $\lambda_1 = 3$:}

Solve $(A - 3I)\mathbf{v}_1 = \mathbf{0}$:
\[
\begin{pmatrix} 2-3 & -1 \\ -1 & 2-3 \end{pmatrix} \begin{pmatrix} v_1 \\ v_2 \end{pmatrix} = \begin{pmatrix} -1 & -1 \\ -1 & -1 \end{pmatrix} \begin{pmatrix} v_1 \\ v_2 \end{pmatrix} = \begin{pmatrix} 0 \\ 0 \end{pmatrix}
\]

From first row: $-v_1 - v_2 = 0 \Rightarrow v_2 = -v_1$

Choose $v_1 = 1$, then $v_2 = -1$:
\[
\boxed{\mathbf{v}_1 = \begin{pmatrix} 1 \\ -1 \end{pmatrix}}
\]

\textbf{For $\lambda_2 = 1$:}

Solve $(A - I)\mathbf{v}_2 = \mathbf{0}$:
\[
\begin{pmatrix} 2-1 & -1 \\ -1 & 2-1 \end{pmatrix} \begin{pmatrix} v_1 \\ v_2 \end{pmatrix} = \begin{pmatrix} 1 & -1 \\ -1 & 1 \end{pmatrix} \begin{pmatrix} v_1 \\ v_2 \end{pmatrix} = \begin{pmatrix} 0 \\ 0 \end{pmatrix}
\]

From first row: $v_1 - v_2 = 0 \Rightarrow v_2 = v_1$

Choose $v_1 = 1$, then $v_2 = 1$:
\[
\boxed{\mathbf{v}_2 = \begin{pmatrix} 1 \\ 1 \end{pmatrix}}
\]

\subsection*{Verify orthogonality}

\[
\mathbf{v}_1 \cdot \mathbf{v}_2 = (1)(1) + (-1)(1) = 1 - 1 = 0 \quad \checkmark
\]

The eigenvectors are orthogonal, as expected for a symmetric matrix.

\subsection*{XYZ Analysis of Eigenvectors}

\begin{itemize}[leftmargin=*]
\item \stage{STAGE X (What we found):}
\begin{itemize}
\item $\mathbf{v}_1 = (1, -1)^T$: the "difference" direction
\item $\mathbf{v}_2 = (1, 1)^T$: the "sum" direction
\end{itemize}

\item \stage{STAGE Y (Why these directions):}
\begin{itemize}
\item $\mathbf{v}_1 = (1, -1)$: Points in this direction have $x = -y$ (opposite signs). This is the direction where $x - y$ is maximized. Along this direction, the map scales by $\lambda_1 = 3$.
\item $\mathbf{v}_2 = (1, 1)$: Points in this direction have $x = y$ (same values). This is the direction where $x + y$ is constant. Along this direction, the map scales by $\lambda_2 = 1$ (unchanged).
\end{itemize}
These match the observed pattern from part (a): $x_n - y_n$ grows like $3^n$, while $x_n + y_n$ stays constant at 1.

\item \stage{STAGE Z (What this geometric picture means):} Any initial point can be decomposed into components along these two perpendicular axes. The component along $\mathbf{v}_1$ grows exponentially, while the component along $\mathbf{v}_2$ remains constant. This explains the long-term behavior: trajectories move along lines parallel to $\mathbf{v}_1$ while maintaining their projection onto $\mathbf{v}_2$.
\end{itemize}

\vspace{10pt}
\hrule
\vspace{10pt}

\subsection*{Step 4: General solution}

The general solution has the form:
\[
\begin{pmatrix} x_n \\ y_n \end{pmatrix} = \alpha_1 \lambda_1^n \mathbf{v}_1 + \alpha_2 \lambda_2^n \mathbf{v}_2
\]

Substituting our eigenvalues and eigenvectors:
\[
\begin{pmatrix} x_n \\ y_n \end{pmatrix} = \alpha_1 (3)^n \begin{pmatrix} 1 \\ -1 \end{pmatrix} + \alpha_2 (1)^n \begin{pmatrix} 1 \\ 1 \end{pmatrix}
\]

Simplify:
\[
\begin{pmatrix} x_n \\ y_n \end{pmatrix} = \alpha_1 \cdot 3^n \begin{pmatrix} 1 \\ -1 \end{pmatrix} + \alpha_2 \begin{pmatrix} 1 \\ 1 \end{pmatrix}
\]

In component form:
\begin{align*}
x_n &= \alpha_1 \cdot 3^n + \alpha_2 \\
y_n &= -\alpha_1 \cdot 3^n + \alpha_2
\end{align*}

\subsection*{XYZ Analysis of General Solution Form}

\begin{itemize}[leftmargin=*]
\item \stage{STAGE X (What the formula shows):} The solution is a linear combination of two modes: one that grows exponentially ($3^n$ term) and one that is constant (the $\alpha_2$ term).

\item \stage{STAGE Y (Why this structure):} Unlike ODEs where solutions involve $e^{\lambda t}$ (continuous exponential growth), maps have discrete time steps, so solutions involve $\lambda^n$ (discrete exponential growth). Each iteration multiplies by $\lambda$ rather than adding $\lambda \, dt$. The constants $\alpha_1, \alpha_2$ weight how much of each eigenmode is present, determined by projecting the initial condition onto the eigenvectors.

\item \stage{STAGE Z (What remains):} We need two equations (the initial conditions) to determine two unknowns ($\alpha_1, \alpha_2$). Once found, we have the complete explicit solution for all time steps $n$.
\end{itemize}

\vspace{10pt}
\hrule
\vspace{10pt}

\subsection*{Step 5: Apply initial conditions}

At $n = 0$: $(x_0, y_0) = (1, 0)$

\[
\begin{pmatrix} 1 \\ 0 \end{pmatrix} = \alpha_1 (3)^0 \begin{pmatrix} 1 \\ -1 \end{pmatrix} + \alpha_2 (1)^0 \begin{pmatrix} 1 \\ 1 \end{pmatrix} = \alpha_1 \begin{pmatrix} 1 \\ -1 \end{pmatrix} + \alpha_2 \begin{pmatrix} 1 \\ 1 \end{pmatrix}
\]

This gives two equations:
\begin{align}
1 &= \alpha_1 + \alpha_2 \label{eq:ic1} \\
0 &= -\alpha_1 + \alpha_2 \label{eq:ic2}
\end{align}

From equation (\ref{eq:ic2}):
\[
\alpha_2 = \alpha_1
\]

Substitute into equation (\ref{eq:ic1}):
\[
1 = \alpha_1 + \alpha_1 = 2\alpha_1 \quad \Rightarrow \quad \alpha_1 = \frac{1}{2}
\]

Therefore:
\[
\boxed{\alpha_1 = \frac{1}{2}, \quad \alpha_2 = \frac{1}{2}}
\]

\subsection*{XYZ Analysis of Initial Condition Decomposition}

\begin{itemize}[leftmargin=*]
\item \stage{STAGE X (What we found):} The initial condition $(1, 0)$ has equal weight ($1/2$) in both eigendirections.

\item \stage{STAGE Y (Why equal weights):} The initial point $(1, 0)$ lies exactly halfway between the two eigenvector directions:
\[
\begin{pmatrix} 1 \\ 0 \end{pmatrix} = \frac{1}{2}\begin{pmatrix} 1 \\ -1 \end{pmatrix} + \frac{1}{2}\begin{pmatrix} 1 \\ 1 \end{pmatrix}
\]
Geometrically, $(1, 0)$ is the average of $(1, -1)$ and $(1, 1)$. Since both components are present, the trajectory will show both exponential growth (from the $\lambda_1=3$ mode) and a constant background (from the $\lambda_2=1$ mode).

\item \stage{STAGE Z (What this predicts):} With equal weights, the solution will be $x_n = \frac{1}{2}3^n + \frac{1}{2}$ and $y_n = -\frac{1}{2}3^n + \frac{1}{2}$. For large $n$, the $3^n$ terms dominate, and the trajectory approaches the unstable eigendirection $(1, -1)$ (moving along the line $y = -x + 1$).
\end{itemize}

\vspace{10pt}
\hrule
\vspace{10pt}

\subsection*{Step 6: Write explicit solution}

Substitute $\alpha_1 = \alpha_2 = 1/2$ into the general solution:
\begin{align*}
x_n &= \frac{1}{2} \cdot 3^n + \frac{1}{2} = \frac{3^n + 1}{2} \\
y_n &= -\frac{1}{2} \cdot 3^n + \frac{1}{2} = \frac{1 - 3^n}{2}
\end{align*}

\[
\boxed{x_n = \frac{3^n + 1}{2}, \quad y_n = \frac{1 - 3^n}{2}}
\]

\vspace{10pt}
\hrule
\vspace{10pt}

\subsection*{Step 7: Verify solution}

\textbf{Check initial condition at $n=0$:}
\begin{align*}
x_0 &= \frac{3^0 + 1}{2} = \frac{1 + 1}{2} = 1 \quad \checkmark \\
y_0 &= \frac{1 - 3^0}{2} = \frac{1 - 1}{2} = 0 \quad \checkmark
\end{align*}

\textbf{Check map is satisfied at $n=0 \to 1$:}
\begin{align*}
x_1 &= \frac{3^1 + 1}{2} = \frac{4}{2} = 2 \\
\text{From map: } x_1 &= 2x_0 - y_0 = 2(1) - 0 = 2 \quad \checkmark \\[5pt]
y_1 &= \frac{1 - 3^1}{2} = \frac{-2}{2} = -1 \\
\text{From map: } y_1 &= 2y_0 - x_0 = 2(0) - 1 = -1 \quad \checkmark
\end{align*}

\textbf{Check map is satisfied at $n=1 \to 2$:}
\begin{align*}
x_2 &= \frac{3^2 + 1}{2} = \frac{10}{2} = 5 \\
\text{From map: } x_2 &= 2x_1 - y_1 = 2(2) - (-1) = 5 \quad \checkmark \\[5pt]
y_2 &= \frac{1 - 3^2}{2} = \frac{-8}{2} = -4 \\
\text{From map: } y_2 &= 2y_1 - x_1 = 2(-1) - 2 = -4 \quad \checkmark
\end{align*}

\textbf{General verification:}

We verify that $x_n, y_n$ satisfy the original map equations. From our solution:
\begin{align*}
2x_n - y_n &= 2 \cdot \frac{3^n + 1}{2} - \frac{1 - 3^n}{2} \\
&= \frac{2(3^n + 1) - (1 - 3^n)}{2} \\
&= \frac{2 \cdot 3^n + 2 - 1 + 3^n}{2} \\
&= \frac{3 \cdot 3^n + 1}{2} \\
&= \frac{3^{n+1} + 1}{2} = x_{n+1} \quad \checkmark
\end{align*}

\begin{align*}
2y_n - x_n &= 2 \cdot \frac{1 - 3^n}{2} - \frac{3^n + 1}{2} \\
&= \frac{2(1 - 3^n) - (3^n + 1)}{2} \\
&= \frac{2 - 2 \cdot 3^n - 3^n - 1}{2} \\
&= \frac{1 - 3 \cdot 3^n}{2} \\
&= \frac{1 - 3^{n+1}}{2} = y_{n+1} \quad \checkmark
\end{align*}

The solution is verified!

\vspace{10pt}
\hrule
\vspace{10pt}

\section{Summary and Comparison}

\subsection*{Both methods yield the same result}

\[
\boxed{x_n = \frac{3^n + 1}{2}, \quad y_n = \frac{1 - 3^n}{2}}
\]

\subsection*{Method comparison}

\begin{center}
\begin{tabular}{p{7cm}|p{7cm}}
\textbf{Method (a): Direct Iteration} & \textbf{Method (b): Eigenvalue Decomposition} \\
\hline
\textbf{Pros:} & \textbf{Pros:} \\
- Simple to implement & - Gives explicit closed-form solution \\
- No linear algebra required & - Reveals system structure (eigenmodes) \\
- Easy to program & - Efficient for computing $x_n$ for large $n$ \\
- Good for short-term behavior & - Explains long-term behavior analytically \\[5pt]
\textbf{Cons:} & \textbf{Cons:} \\
- Must compute every step & - Requires eigenvalue computation \\
- Impractical for large $n$ & - More complex setup \\
- Pattern recognition needed & - Requires understanding of linear algebra \\
- No insight into system structure & - Can be difficult for large systems \\
\end{tabular}
\end{center}

\subsection*{XYZ Analysis of Solution Structure}

\begin{itemize}[leftmargin=*]
\item \stage{STAGE X (What the solution tells us):}
\begin{itemize}
\item $x_n$ grows like $3^n/2$ for large $n$
\item $y_n$ grows like $-3^n/2$ for large $n$ (same magnitude, opposite sign)
\item The ratio $y_n/x_n \to -1$ as $n \to \infty$
\end{itemize}

\item \stage{STAGE Y (Why this behavior):} The dominant eigenvalue $\lambda_1 = 3$ with eigenvector $(1, -1)$ controls long-term dynamics. The system is unstable: trajectories escape to infinity along the unstable eigendirection. The $\lambda_2 = 1$ mode contributes a constant background $(1/2, 1/2)$, but becomes negligible relative to the growing $3^n$ terms. This is characteristic of linear maps where $|\lambda_{\text{max}}| > 1$.

\item \stage{STAGE Z (What this means):}
\begin{itemize}
\item \textbf{Asymptotic behavior:} $(x_n, y_n) \approx (3^n/2, -3^n/2)$ as $n \to \infty$
\item \textbf{Trajectory shape:} Points move along lines parallel to $(1, -1)$
\item \textbf{Growth rate:} Distance from origin grows like $\sqrt{x_n^2 + y_n^2} \sim 3^n/\sqrt{2}$
\item \textbf{Doubling time:} Since $3^n = e^{n \ln 3}$, the system grows by factor $e$ every $1/\ln 3 \approx 0.91$ iterations
\end{itemize}
\end{itemize}

\subsection*{Connection to ODEs}

The key difference between maps and ODEs:
\begin{center}
\begin{tabular}{c|c}
\textbf{ODEs} & \textbf{Maps} \\
\hline
$\mathbf{x}(t) = \alpha_1 e^{\lambda_1 t} \mathbf{v}_1 + \alpha_2 e^{\lambda_2 t} \mathbf{v}_2$ & $\mathbf{x}_n = \alpha_1 \lambda_1^n \mathbf{v}_1 + \alpha_2 \lambda_2^n \mathbf{v}_2$ \\
Continuous time & Discrete time \\
Stability: $\text{Re}(\lambda) < 0$ & Stability: $|\lambda| < 1$ \\
$e^{\lambda t}$ terms & $\lambda^n$ terms \\
\end{tabular}
\end{center}

For our system: $\lambda_1 = 3 > 1$ (unstable), $\lambda_2 = 1$ (marginally stable). The origin is an unstable fixed point.

\end{document}
