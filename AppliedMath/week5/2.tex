\documentclass[12pt,a4paper]{article}

% Packages
\usepackage{amsmath}
\usepackage{amssymb}
\usepackage{amsthm}
\usepackage[margin=1in]{geometry}
\usepackage{enumitem}

% Custom environments
\newtheorem{explanation}{Explanation}
\theoremstyle{definition}
\newtheorem{solution}{Solution}

% Title information
\title{Exercise 5, Question 2: Sawtooth Map\\
Periodic Orbits and Chaotic Dynamics}
\author{}
\date{}

\begin{document}

\maketitle

\section*{Problem Statement}

Sketch or graph the sawtooth map
$$x_{n+1} = \begin{cases}
2x_n & \text{for } 0 \leq x_n < 1/2 \\
2x_n - 1 & \text{for } 1/2 < x_n \leq 1
\end{cases}$$

Either by hand or computer, investigate its dynamics with cobweb diagrams.

\textbf{(a)} Show there is a period two orbit with an iterate at $x = 1/3$, and find the other iterate.

\textbf{(b)} Show there is a period three orbit with an iterate at $x = 1/7$, and find the other iterates.

\textbf{(c)} Show that the orbits from (a)-(b) are unstable.

\textbf{(d)} Argue why this map cannot have any stable periodic orbits, and conjecture what kind of dynamics you will see from a typical initial point.

\vspace{10pt}
\hrule
\vspace{10pt}

\section{Understanding the Sawtooth Map}

\begin{solution}

\subsection*{Step 1: Recognize the Map Structure}

The sawtooth map is a piecewise linear map on the interval $[0, 1]$.

\begin{explanation}[What is a piecewise map?]
The map has different formulas depending on which region $x_n$ lies in. We need to track which piece applies at each iteration.
\end{explanation}

\subsection*{Step 2: Analyze the First Piece ($0 \leq x_n < 1/2$)}

For $x_n$ in $[0, 1/2)$:
$$x_{n+1} = 2x_n$$

This doubles the value. Since $x_n < 1/2$, we have $x_{n+1} = 2x_n < 2(1/2) = 1$.

Also, $x_{n+1} \geq 2(0) = 0$.

So if $x_n \in [0, 1/2)$, then $x_{n+1} \in [0, 1)$.

\subsection*{Step 3: Analyze the Second Piece ($1/2 < x_n \leq 1$)}

For $x_n$ in $(1/2, 1]$:
$$x_{n+1} = 2x_n - 1$$

This doubles and then subtracts 1.

If $x_n = 1/2$: $x_{n+1} = 2(1/2) - 1 = 0$.

If $x_n = 1$: $x_{n+1} = 2(1) - 1 = 1$.

So if $x_n \in (1/2, 1]$, then $x_{n+1} \in (0, 1]$.

\begin{explanation}[Geometric picture]
The map "unfolds" the interval $[0,1]$. It stretches each half by a factor of 2. The left half $[0, 1/2)$ maps to $[0, 1)$, and the right half $(1/2, 1]$ also maps to $(0, 1]$. This is why it's called a "sawtooth" - the graph has two rising linear pieces.
\end{explanation}

\subsection*{Step 4: Write a Compact Notation}

We can write:
$$f(x) = \begin{cases}
2x & \text{if } x < 1/2 \\
2x - 1 & \text{if } x \geq 1/2
\end{cases}$$

This can also be written as:
$$f(x) = 2x \mod 1$$

where "$\mod 1$" means take the fractional part (subtract the integer part).

\end{solution}

\vspace{10pt}
\hrule
\vspace{10pt}

\section{Part (a): Period Two Orbit with Iterate at $x = 1/3$}

\begin{solution}

\subsection*{Step 1: Understand What a Period Two Orbit Means}

A period two orbit consists of two points $\{x_1, x_2\}$ such that:
$$f(x_1) = x_2 \quad \text{and} \quad f(x_2) = x_1$$

This means $f(f(x_1)) = x_1$, i.e., $f^2(x_1) = x_1$.

\begin{explanation}[Why period two?]
From lecture notes Section 22, page 80: A period $m$ orbit satisfies $x = f^m(x)$. For $m=2$, the orbit has two distinct points that map to each other.
\end{explanation}

\subsection*{Step 2: Apply the Map to $x_1 = 1/3$}

We have $x_1 = 1/3$. Check which piece of the map applies:

Since $1/3 < 1/2$, we use $f(x) = 2x$:
$$x_2 = f(1/3) = 2 \cdot \frac{1}{3} = \frac{2}{3}$$

\subsection*{Step 3: Apply the Map to $x_2 = 2/3$}

Now check $x_2 = 2/3$. Which piece applies?

Since $2/3 > 1/2$, we use $f(x) = 2x - 1$:
$$f(2/3) = 2 \cdot \frac{2}{3} - 1 = \frac{4}{3} - 1 = \frac{4 - 3}{3} = \frac{1}{3}$$

So $f(x_2) = f(2/3) = 1/3 = x_1$.

\subsection*{Step 4: Verify the Period Two Orbit}

We have:
$$f(1/3) = 2/3 \quad \text{and} \quad f(2/3) = 1/3$$

This confirms a period two orbit: $1/3 \to 2/3 \to 1/3 \to \cdots$

\subsection*{Step 5: Verify Using $f^2$}

Check that $f^2(1/3) = 1/3$:
$$f^2(1/3) = f(f(1/3)) = f(2/3) = 1/3 \quad \checkmark$$

Also check $f^2(2/3) = 2/3$:
$$f^2(2/3) = f(f(2/3)) = f(1/3) = 2/3 \quad \checkmark$$

\textbf{Answer:}
$$\boxed{\text{Period 2 orbit: } \{1/3, 2/3\}}$$

The other iterate is $x_2 = 2/3$.

\end{solution}

\vspace{10pt}
\hrule
\vspace{10pt}

\section{Part (b): Period Three Orbit with Iterate at $x = 1/7$}

\begin{solution}

\subsection*{Step 1: Understand What a Period Three Orbit Means}

A period three orbit consists of three points $\{x_1, x_2, x_3\}$ such that:
$$f(x_1) = x_2, \quad f(x_2) = x_3, \quad f(x_3) = x_1$$

This means $f^3(x_1) = x_1$.

\subsection*{Step 2: Apply the Map to $x_1 = 1/7$}

We have $x_1 = 1/7$. Check which piece applies:

Since $1/7 \approx 0.143 < 1/2$, we use $f(x) = 2x$:
$$x_2 = f(1/7) = 2 \cdot \frac{1}{7} = \frac{2}{7}$$

\subsection*{Step 3: Apply the Map to $x_2 = 2/7$}

Check $x_2 = 2/7$. Which piece applies?

Since $2/7 \approx 0.286 < 1/2$, we use $f(x) = 2x$:
$$x_3 = f(2/7) = 2 \cdot \frac{2}{7} = \frac{4}{7}$$

\subsection*{Step 4: Apply the Map to $x_3 = 4/7$}

Check $x_3 = 4/7$. Which piece applies?

Since $4/7 \approx 0.571 > 1/2$, we use $f(x) = 2x - 1$:
$$f(4/7) = 2 \cdot \frac{4}{7} - 1 = \frac{8}{7} - 1 = \frac{8 - 7}{7} = \frac{1}{7}$$

So $f(x_3) = 1/7 = x_1$.

\subsection*{Step 5: Verify the Period Three Orbit}

We have:
$$f(1/7) = 2/7, \quad f(2/7) = 4/7, \quad f(4/7) = 1/7$$

This confirms a period three orbit: $1/7 \to 2/7 \to 4/7 \to 1/7 \to \cdots$

\subsection*{Step 6: Verify Using $f^3$}

Check that $f^3(1/7) = 1/7$:
$$f^3(1/7) = f(f(f(1/7))) = f(f(2/7)) = f(4/7) = 1/7 \quad \checkmark$$

\textbf{Answer:}
$$\boxed{\text{Period 3 orbit: } \{1/7, 2/7, 4/7\}}$$

The other iterates are $x_2 = 2/7$ and $x_3 = 4/7$.

\begin{explanation}[Pattern observation]
Notice the pattern: starting from $1/7$, each iterate doubles the numerator:
$$\frac{1}{7} \to \frac{2}{7} \to \frac{4}{7} \to \frac{8}{7} = \frac{1}{7} + 1 \equiv \frac{1}{7} \pmod{1}$$

This reflects the structure $f(x) = 2x \mod 1$.
\end{explanation}

\end{solution}

\vspace{10pt}
\hrule
\vspace{10pt}

\section{Part (c): Stability of Periodic Orbits}

\begin{solution}

\subsection*{Overview of Stability for Periodic Orbits}

From lecture notes Section 22, page 82: For a period $p$ orbit, stability is determined by the product of derivatives at each point in the orbit:
$$\frac{dx_{n+p}}{dx_n} = f'(x_{n+p-1}) \cdots f'(x_{n+1}) f'(x_n)$$

The orbit is stable if this product has absolute value less than 1.

\subsection*{Step 1: Compute the Derivative of $f$}

For the sawtooth map:
$$f(x) = \begin{cases}
2x & \text{if } x < 1/2 \\
2x - 1 & \text{if } x \geq 1/2
\end{cases}$$

The derivative is:
$$f'(x) = \begin{cases}
2 & \text{if } x < 1/2 \\
2 & \text{if } x > 1/2
\end{cases}$$

\begin{explanation}[Key observation]
The derivative is constant and equals 2 everywhere (except at $x = 1/2$ where the map is not differentiable, but we won't evaluate it there).
\end{explanation}

\subsection*{Step 2: Apply Chain Rule for Period Two Orbit}

For the period two orbit $\{1/3, 2/3\}$, the stability is:
$$\frac{dx_{n+2}}{dx_n}\bigg|_{x=1/3} = f'(2/3) \cdot f'(1/3)$$

\subsection*{Step 3: Evaluate Derivatives at Orbit Points}

At $x = 1/3 < 1/2$: $f'(1/3) = 2$

At $x = 2/3 > 1/2$: $f'(2/3) = 2$

Therefore:
$$\frac{dx_{n+2}}{dx_n}\bigg|_{x=1/3} = 2 \cdot 2 = 4$$

\subsection*{Step 4: Determine Stability of Period Two Orbit}

Since $|4| = 4 > 1$, the period two orbit is \textbf{unstable}.

\begin{explanation}[Why unstable?]
A small perturbation away from the orbit grows by a factor of 4 after two iterations. The orbit repels nearby trajectories.
\end{explanation}

$$\boxed{\text{Period 2 orbit is unstable: } \left|\frac{dx_{n+2}}{dx_n}\right| = 4 > 1}$$

\subsection*{Step 5: Apply Chain Rule for Period Three Orbit}

For the period three orbit $\{1/7, 2/7, 4/7\}$, the stability is:
$$\frac{dx_{n+3}}{dx_n}\bigg|_{x=1/7} = f'(4/7) \cdot f'(2/7) \cdot f'(1/7)$$

\subsection*{Step 6: Evaluate Derivatives at Orbit Points}

At $x = 1/7 < 1/2$: $f'(1/7) = 2$

At $x = 2/7 < 1/2$: $f'(2/7) = 2$

At $x = 4/7 > 1/2$: $f'(4/7) = 2$

Therefore:
$$\frac{dx_{n+3}}{dx_n}\bigg|_{x=1/7} = 2 \cdot 2 \cdot 2 = 8$$

\subsection*{Step 7: Determine Stability of Period Three Orbit}

Since $|8| = 8 > 1$, the period three orbit is \textbf{unstable}.

$$\boxed{\text{Period 3 orbit is unstable: } \left|\frac{dx_{n+3}}{dx_n}\right| = 8 > 1}$$

\begin{explanation}[General pattern]
For any period $p$ orbit, the stability multiplier is:
$$\prod_{i=1}^{p} f'(x_i) = 2^p$$

Since $2^p > 1$ for all $p \geq 1$, all periodic orbits are unstable.
\end{explanation}

\end{solution}

\vspace{10pt}
\hrule
\vspace{10pt}

\section{Part (d): No Stable Periodic Orbits and Chaotic Dynamics}

\begin{solution}

\subsection*{Step 1: Generalize the Stability Argument}

Consider any period $p$ orbit with points $\{x_1, x_2, \ldots, x_p\}$.

The stability is determined by:
$$\left|\frac{dx_{n+p}}{dx_n}\right| = |f'(x_p) \cdot f'(x_{p-1}) \cdots f'(x_1)|$$

\subsection*{Step 2: Evaluate the Product of Derivatives}

Since $f'(x) = 2$ for all $x \neq 1/2$ (and periodic orbits generically avoid the non-differentiable point):
$$|f'(x_p) \cdot f'(x_{p-1}) \cdots f'(x_1)| = |2 \cdot 2 \cdots 2| = 2^p$$

\subsection*{Step 3: Check Stability Condition}

For stability, we need $2^p < 1$.

But $2^p \geq 2^1 = 2 > 1$ for all $p \geq 1$.

Therefore, $2^p > 1$ for all positive integers $p$.

\subsection*{Step 4: Conclude About Periodic Orbits}

Since every periodic orbit has stability multiplier $2^p > 1$, there are \textbf{no stable periodic orbits}.

\begin{explanation}[Why this happens]
The sawtooth map stretches by a factor of 2 everywhere. Any small perturbation gets amplified, so no orbit can be attracting. This is a key property of expanding maps.
\end{explanation}

$$\boxed{\text{The sawtooth map has no stable periodic orbits.}}$$

\subsection*{Step 5: Connect to Lyapunov Exponents}

From lecture notes Section 23, page 85-86, the Lyapunov exponent is:
$$\lambda = \lim_{n \to \infty} \frac{1}{n} \sum_{i=0}^{n-1} \ln|f'(x_i)|$$

For the sawtooth map, $f'(x) = 2$ almost everywhere, so:
$$\lambda = \lim_{n \to \infty} \frac{1}{n} \sum_{i=0}^{n-1} \ln(2) = \ln(2) > 0$$

\begin{explanation}[Positive Lyapunov exponent]
A positive Lyapunov exponent indicates \textbf{chaos}. Nearby trajectories diverge exponentially at rate $e^{\lambda} = e^{\ln 2} = 2$ per iteration.
\end{explanation}

\subsection*{Step 6: Identify Chaotic Properties}

From lecture notes Section 23, page 86, chaotic systems have:
\begin{enumerate}
\item \textbf{Sensitive dependence on initial conditions}: Small changes in $x_0$ lead to exponentially different trajectories
\item \textbf{Topological transitivity}: The orbit can visit any region of $[0,1]$
\item \textbf{Dense periodic orbits}: Every point is arbitrarily close to a periodic orbit
\end{enumerate}

The sawtooth map satisfies all three properties.

\subsection*{Step 7: Analyze the Binary Representation}

The sawtooth map has a beautiful interpretation in binary:

Any $x \in [0,1]$ can be written in binary as:
$$x = 0.b_1b_2b_3b_4\ldots \quad \text{(binary)}$$

where $b_i \in \{0, 1\}$.

Applying $f$ (which is $2x \mod 1$) shifts the binary digits left:
$$f(x) = 0.b_2b_3b_4b_5\ldots$$

\begin{explanation}[Shift map interpretation]
The sawtooth map is equivalent to a left shift on binary sequences. This is the Bernoulli shift map, a canonical example of chaos.
\end{explanation}

\subsection*{Step 8: Describe Typical Orbit Behavior}

For a typical initial point $x_0$:

\begin{itemize}
\item The orbit $\{x_0, f(x_0), f^2(x_0), \ldots\}$ will \textbf{never settle} to a fixed point or periodic orbit

\item The orbit will appear to \textbf{randomly} visit different parts of $[0,1]$

\item The orbit will \textbf{eventually come arbitrarily close} to any given point in $[0,1]$ (ergodicity)

\item Two orbits starting very close together will \textbf{diverge exponentially} and become uncorrelated after a few iterations
\end{itemize}

\subsection*{Step 9: Connect to Sharkovskii's Theorem}

From lecture notes Section 22, page 84: The existence of a period 3 orbit implies the existence of periodic orbits of all periods (Sharkovskii's theorem).

We found a period 3 orbit in part (b), which means:
$$\boxed{\text{The sawtooth map has periodic orbits of every period.}}$$

But from Li-Yorke's theorem (page 84): "period three implies chaos."

\subsection*{Step 10: Final Characterization}

\textbf{Answer to part (d):}

\begin{enumerate}
\item \textbf{Why no stable periodic orbits:}

The map has constant derivative $f'(x) = 2 > 1$ everywhere. Any periodic orbit of period $p$ has stability multiplier $2^p > 1$, making it unstable. There is no contracting region, only expansion.

\item \textbf{Typical dynamics:}

From a typical initial point, the orbit exhibits \textbf{deterministic chaos}:
\begin{itemize}
\item The orbit never repeats
\item It densely fills the interval $[0,1]$
\item It has sensitive dependence on initial conditions
\item The Lyapunov exponent is $\lambda = \ln(2) > 0$
\item The dynamics appear random despite being deterministic
\end{itemize}
\end{enumerate}

$$\boxed{\text{The sawtooth map is chaotic with Lyapunov exponent } \lambda = \ln(2)}$$

\begin{explanation}[Physical meaning]
The sawtooth map is a simple example of deterministic chaos. Despite having a simple, explicit formula, its long-term behavior is unpredictable. This map models situations where a system repeatedly stretches and folds, like kneading dough or mixing fluids.

The map serves as a prototype for understanding chaos in more complicated systems. Its analysis demonstrates that:
\begin{itemize}
\item Chaos can arise in very simple systems
\item Expansion (derivative $> 1$) prevents stable orbits
\item Dense periodic orbits coexist with chaotic trajectories
\item Predictability breaks down due to sensitivity to initial conditions
\end{itemize}

This connects to the broader theory of chaos covered in Section 23 of the lecture notes, including the period-doubling route to chaos and strange attractors.
\end{explanation}

\end{solution}

\vspace{10pt}
\hrule
\vspace{10pt}

\section{Summary}

\subsection*{Complete Results}

\textbf{Part (a):} Period 2 orbit: $\boxed{\{1/3, 2/3\}}$

\textbf{Part (b):} Period 3 orbit: $\boxed{\{1/7, 2/7, 4/7\}}$

\textbf{Part (c):} Both orbits are unstable:
\begin{itemize}
\item Period 2: $\left|\frac{dx_{n+2}}{dx_n}\right| = 4 > 1$
\item Period 3: $\left|\frac{dx_{n+3}}{dx_n}\right| = 8 > 1$
\end{itemize}

\textbf{Part (d):} No stable periodic orbits exist because $f'(x) = 2$ everywhere implies all stability multipliers satisfy $2^p > 1$.

Typical dynamics: \textbf{Deterministic chaos} with Lyapunov exponent $\lambda = \ln(2) > 0$.

\subsection*{Connection to Course Material}

This problem illustrates key concepts from Sections 22-23:
\begin{itemize}
\item Finding periodic orbits by iteration (Section 22)
\item Using the chain rule for stability (Section 22, page 82)
\item Identifying chaos through Lyapunov exponents (Section 23, page 85-86)
\item Sharkovskii ordering and "period three implies chaos" (Section 22, page 84)
\item Sensitive dependence on initial conditions (Section 23, page 85)
\end{itemize}

\end{document}
