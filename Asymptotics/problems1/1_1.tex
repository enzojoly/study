\documentclass[11pt,a4paper]{article}
\usepackage{amsmath,amssymb,amsthm}
\usepackage{geometry}
\geometry{margin=1in}
\usepackage{enumitem}
\usepackage{xcolor}

\newtheorem{theorem}{Theorem}
\newtheorem{lemma}[theorem]{Lemma}
\newtheorem{definition}[theorem]{Definition}

\title{Asymptotics 2025/2026 Sheet 1\\Problem 1: Complete Solutions with Full Justification}
\author{}
\date{}

\begin{document}

\maketitle

\section*{Problem 1(a)}

\subsection*{Problem Statement}

For $\epsilon \ll 1$, obtain two-term expansions for the solutions of
\[
(x-1)(x-2)(x-3) + \epsilon = 0.
\]

\subsection*{Complete Solution}

\subsubsection*{Phase I: Problem Classification}

\textbf{Step 1.1: Identify the structure of the equation.}

\textit{What we observe:} The equation has the form
\[
F(x) + \epsilon = 0,
\]
where $F(x) = (x-1)(x-2)(x-3)$ is a polynomial of degree 3, and $\epsilon$ is a small parameter that appears additively (not multiplying the highest degree term).

\textit{Why this matters:} According to Lecture Notes Section 2.1, when a small parameter appears additively in an algebraic equation, we must first examine the unperturbed equation (obtained by setting $\epsilon = 0$) to classify whether this is a regular or singular perturbation problem.

\textit{Theoretical foundation:} The lecture notes define:
\begin{itemize}
\item \textbf{Regular perturbation problem:} ``The exact solution for small but finite $\epsilon$ approaches the unperturbed solution(s) $x_0$ as $\epsilon \to 0$. Consequently, all solutions of the perturbed system can be expressed as well-defined power series expansions around the unperturbed solution.''
\item \textbf{Singular perturbation problem:} ``The perturbed and unperturbed problem differ in an essential way: Not all solutions of the perturbed problem can be expressed as an expansion of the form $x(\epsilon) = x_0 + x_1\epsilon + x_2\epsilon^2 + \cdots$ around the unperturbed solution(s) $x_0$.''
\end{itemize}

\textbf{Step 1.2: Solve the unperturbed equation.}

\textit{What we do:} Set $\epsilon = 0$ in the original equation:
\[
(x-1)(x-2)(x-3) + 0 = 0.
\]

\textit{Why we do this:} The unperturbed equation reveals the ``baseline'' solutions around which we will attempt to construct perturbative expansions. This is the starting point of any perturbative analysis.

\textit{Solution of unperturbed equation:}
\[
(x-1)(x-2)(x-3) = 0.
\]

This factored form immediately gives us three solutions:
\[
x_0^{(1)} = 1, \quad x_0^{(2)} = 2, \quad x_0^{(3)} = 3.
\]

\textit{Why these are the solutions:} A product of factors equals zero if and only if at least one factor equals zero. Thus $x-1=0$ gives $x=1$, $x-2=0$ gives $x=2$, and $x-3=0$ gives $x=3$.

\textbf{Step 1.3: Count degrees of freedom.}

\textit{What we observe:}
\begin{itemize}
\item The unperturbed equation (degree 3 polynomial) has exactly 3 solutions.
\item The perturbed equation (also degree 3 polynomial) must also have exactly 3 solutions (counting multiplicities, over $\mathbb{C}$).
\end{itemize}

\textit{Why this matters:} Since the number of solutions is preserved, and each unperturbed solution is simple (non-degenerate), we expect that each perturbed solution will smoothly approach one of the unperturbed solutions as $\epsilon \to 0$.

\textbf{Step 1.4: Check for degeneracy.}

\textit{What we check:} Are any of the unperturbed roots repeated?

\textit{Finding:} All three roots $x = 1, 2, 3$ are distinct (simple roots).

\textit{Why this matters:} According to Lecture Notes Section 2.3, degenerate roots often require non-integer power expansions. Since all our roots are simple, we expect regular behavior with integer power expansions of the form $x(\epsilon) = x_0 + x_1\epsilon + x_2\epsilon^2 + \cdots$.

\textbf{Step 1.5: Classify the problem.}

\textit{Conclusion:} This is a \textbf{regular perturbation problem}.

\textit{Justification:}
\begin{enumerate}
\item The number of solutions is preserved (3 solutions in both cases).
\item All unperturbed roots are simple (non-degenerate).
\item The perturbation is additive and small.
\item We expect each perturbed solution to approach exactly one unperturbed solution as $\epsilon \to 0$.
\end{enumerate}

\textit{Method to use:} According to Lecture Notes Section 2.1.1, we will use the \textbf{expansion method}, making the ansatz
\[
x(\epsilon) = x_0 + x_1\epsilon + x_2\epsilon^2 + \cdots
\]
for each unperturbed solution $x_0$.

\subsubsection*{Phase II: Solution Near $x_0 = 1$}

\textbf{Step 2.1: Make the expansion ansatz.}

\textit{What we assume:} For the root near $x_0 = 1$, we write:
\[
x(\epsilon) = 1 + x_1\epsilon + O(\epsilon^2).
\]

\textbf{Step 2.2: Substitute into the equation.}

\textit{Original equation:}
\[
(x-1)(x-2)(x-3) + \epsilon = 0.
\]

\textit{Substitution:} Replace $x$ with $1 + x_1\epsilon + \cdots$:
\begin{align*}
\text{First factor:} & \quad (x - 1) = x_1\epsilon + O(\epsilon^2), \\
\text{Second factor:} & \quad (x - 2) = -1 + x_1\epsilon + O(\epsilon^2), \\
\text{Third factor:} & \quad (x - 3) = -2 + x_1\epsilon + O(\epsilon^2).
\end{align*}

\textbf{Step 2.3: Expand the product.}

Multiplying the second and third factors first:
\[
(-1 + x_1\epsilon)(-2 + x_1\epsilon) = 2 - 3x_1\epsilon + O(\epsilon^2).
\]

Now multiply by the first factor:
\[
(x_1\epsilon)(2 - 3x_1\epsilon + \cdots) = 2x_1\epsilon + O(\epsilon^2).
\]

\textbf{Step 2.4: Apply the equation.}

\[
2x_1\epsilon + O(\epsilon^2) + \epsilon = 0 \implies (2x_1 + 1)\epsilon + O(\epsilon^2) = 0.
\]

\textit{At $O(\epsilon)$:}
\[
2x_1 + 1 = 0 \implies x_1 = -\frac{1}{2}.
\]

\textbf{Final answer for root near $x_0 = 1$:}
\[
\boxed{x(\epsilon) = 1 - \frac{1}{2}\epsilon + O(\epsilon^2)}.
\]

\subsubsection*{Phase III: Solution Near $x_0 = 2$}

\textbf{Ansatz:} $x(\epsilon) = 2 + x_1\epsilon + O(\epsilon^2)$.

\textbf{Factors:}
\begin{align*}
(x - 1) &= 1 + x_1\epsilon + O(\epsilon^2), \\
(x - 2) &= x_1\epsilon + O(\epsilon^2), \\
(x - 3) &= -1 + x_1\epsilon + O(\epsilon^2).
\end{align*}

\textbf{Product:}
\[
(1 + x_1\epsilon)(-1 + x_1\epsilon) = -1 + O(\epsilon^2).
\]
\[
(x_1\epsilon)(-1) = -x_1\epsilon.
\]

\textbf{Equation:}
\[
-x_1\epsilon + \epsilon = 0 \implies (-x_1 + 1)\epsilon = 0 \implies x_1 = 1.
\]

\textbf{Final answer for root near $x_0 = 2$:}
\[
\boxed{x(\epsilon) = 2 + \epsilon + O(\epsilon^2)}.
\]

\subsubsection*{Phase IV: Solution Near $x_0 = 3$}

\textbf{Ansatz:} $x(\epsilon) = 3 + x_1\epsilon + O(\epsilon^2)$.

\textbf{Factors:}
\begin{align*}
(x - 1) &= 2 + x_1\epsilon + O(\epsilon^2), \\
(x - 2) &= 1 + x_1\epsilon + O(\epsilon^2), \\
(x - 3) &= x_1\epsilon + O(\epsilon^2).
\end{align*}

\textbf{Product:}
\[
(2 + x_1\epsilon)(1 + x_1\epsilon) = 2 + 3x_1\epsilon + O(\epsilon^2).
\]
\[
(x_1\epsilon)(2 + 3x_1\epsilon) = 2x_1\epsilon + O(\epsilon^2).
\]

\textbf{Equation:}
\[
2x_1\epsilon + \epsilon = 0 \implies (2x_1 + 1)\epsilon = 0 \implies x_1 = -\frac{1}{2}.
\]

\textbf{Final answer for root near $x_0 = 3$:}
\[
\boxed{x(\epsilon) = 3 - \frac{1}{2}\epsilon + O(\epsilon^2)}.
\]

\subsubsection*{Summary for Problem 1(a)}

The three roots of $(x-1)(x-2)(x-3) + \epsilon = 0$ are:
\begin{align*}
x_1(\epsilon) &= 1 - \frac{1}{2}\epsilon + O(\epsilon^2), \\
x_2(\epsilon) &= 2 + \epsilon + O(\epsilon^2), \\
x_3(\epsilon) &= 3 - \frac{1}{2}\epsilon + O(\epsilon^2).
\end{align*}

\newpage
\section*{Problem 1(b)}

\subsection*{Problem Statement}

For $\epsilon \ll 1$, obtain two-term expansions for the solutions of
\[
x^3 + x^2 - \epsilon = 0.
\]

\subsection*{Complete Solution}

\subsubsection*{Phase I: Problem Classification and Structure}

\textbf{Step 1.1: Examine the equation structure.}

\textit{What we observe:} The equation can be written as
\[
x^2(x + 1) = \epsilon.
\]

\textit{Form:} This has the structure $F(x) = \epsilon$ where $F(x) = x^2(x+1)$ is a cubic polynomial.

\textbf{Step 1.2: Solve the unperturbed equation.}

\textit{Setting $\epsilon = 0$:}
\[
x^3 + x^2 = x^2(x + 1) = 0.
\]

\textit{Solutions:}
\[
x^2 = 0 \implies x = 0 \text{ (double root)},
\]
\[
x + 1 = 0 \implies x = -1 \text{ (simple root)}.
\]

\textit{Critical observation:} The unperturbed equation has a \textbf{degenerate root} at $x = 0$ (multiplicity 2).

\textbf{Step 1.3: Classify the problem.}

\textit{Conclusion:} This is a \textbf{singular perturbation problem with non-integer power expansions} (as discussed in Lecture Notes Section 2.3).

\textit{Key insight from the lecture notes:} ``In cases where unperturbed solutions are degenerate, their behavior as $\epsilon \to 0$ may sometimes not be captured by a power series expansion of integer powers.''

\textit{Strategy:}
\begin{enumerate}
\item Find the regular solution near $x = -1$ using standard integer power expansion.
\item Find the singular solutions near $x = 0$ using fractional power expansion with exponent determined by dominant balance.
\end{enumerate}

\subsubsection*{Phase II: Regular Solution Near $x_0 = -1$}

\textbf{Step 2.1: Justification for regular expansion.}

\textit{Observation:} $x = -1$ is a \textbf{simple root} of the unperturbed equation.

\textit{Theory:} Simple roots typically give rise to regular perturbative expansions with integer powers of $\epsilon$.

\textbf{Step 2.2: Make the standard ansatz.}

\[
x(\epsilon) = -1 + x_1\epsilon + x_2\epsilon^2 + O(\epsilon^3).
\]

\textbf{Step 2.3: Substitute into the equation.}

\textit{Original equation:} $x^3 + x^2 - \epsilon = 0$.

\textit{Compute $x^2$ and $x^3$:}
\begin{align*}
x^2 &= (-1 + x_1\epsilon + \cdots)^2 = 1 - 2x_1\epsilon + O(\epsilon^2), \\
x^3 &= (-1 + x_1\epsilon + \cdots)^3 = -1 + 3x_1\epsilon + O(\epsilon^2).
\end{align*}

\textbf{Step 2.4: Combine and solve.}

\[
x^3 + x^2 = (-1 + 3x_1\epsilon) + (1 - 2x_1\epsilon) + O(\epsilon^2) = x_1\epsilon + O(\epsilon^2).
\]

Setting equal to $\epsilon$:
\[
x_1\epsilon + O(\epsilon^2) = \epsilon.
\]

\textit{At $O(\epsilon)$:}
\[
x_1 = 1.
\]

\textbf{Final answer for regular root:}
\[
\boxed{x(\epsilon) = -1 + \epsilon + O(\epsilon^2)}.
\]

\subsubsection*{Phase III: Singular Solutions Near $x_0 = 0$ (Double Root)}

\textbf{Step 3.1: Why we need fractional powers.}

\textit{Key observation:} The unperturbed root $x = 0$ is degenerate (double root).

\textit{Theory from Lecture Notes (Section 2.3):} For degenerate roots, we use the ansatz
\[
x = x_1\epsilon^\alpha + x_2\epsilon^{2\alpha} + \cdots
\]
where $\alpha > 0$ is determined by \textbf{dominant balance analysis}.

\textbf{Step 3.2: Make the fractional power ansatz.}

\[
x(\epsilon) = x_1\epsilon^\alpha + x_2\epsilon^{2\alpha} + \cdots
\]

\textbf{Step 3.3: Substitute into the equation.}

\textit{Original equation:} $x^3 + x^2 = \epsilon$.

\textit{Compute powers of $x$:}
\begin{align*}
x^2 &= x_1^2\epsilon^{2\alpha} + 2x_1x_2\epsilon^{3\alpha} + O(\epsilon^{4\alpha}), \\
x^3 &= x_1^3\epsilon^{3\alpha} + 3x_1^2x_2\epsilon^{4\alpha} + O(\epsilon^{5\alpha}).
\end{align*}

\textit{Sum:}
\[
x^3 + x^2 = x_1^2\epsilon^{2\alpha} + x_1^3\epsilon^{3\alpha} + 2x_1x_2\epsilon^{3\alpha} + O(\epsilon^{4\alpha}).
\]

\textbf{Step 3.4: Determine $\alpha$ by dominant balance.}

\textit{The fundamental principle:} We must balance the left-hand side against $\epsilon$ on the right-hand side. The leading term on the left must match $\epsilon$ in order of magnitude.

\textit{Comparison of terms:}
\begin{itemize}
\item $x^2$ term: $x_1^2\epsilon^{2\alpha}$
\item $x^3$ term: $x_1^3\epsilon^{3\alpha}$
\item RHS: $\epsilon$
\end{itemize}

\textit{Since $\alpha > 0$, we have $2\alpha < 3\alpha$, so $\epsilon^{2\alpha} \gg \epsilon^{3\alpha}$ as $\epsilon \to 0$.}

\textit{Therefore, the dominant term on the LHS is $x_1^2\epsilon^{2\alpha}$.}

\textit{Dominant balance condition:} The dominant term must balance the RHS:
\[
x_1^2\epsilon^{2\alpha} \sim \epsilon.
\]

\textit{Matching powers of $\epsilon$:}
\[
2\alpha = 1 \implies \boxed{\alpha = \frac{1}{2}}.
\]

\textit{Verification of consistency:} With $\alpha = 1/2$:
\begin{itemize}
\item $x^2 \sim \epsilon^1$ (leading order, balances RHS)
\item $x^3 \sim \epsilon^{3/2}$ (subdominant, since $3/2 > 1$)
\end{itemize}
This is consistent: the $x^3$ term is indeed smaller than the $x^2$ term.

\textbf{Step 3.5: Solve at $O(\epsilon)$ (leading order).}

With $\alpha = 1/2$, the leading-order equation is:
\[
x_1^2\epsilon = \epsilon \implies x_1^2 = 1 \implies x_1 = \pm 1.
\]

\textit{Interpretation:} The double root at $x = 0$ splits into two roots, one going as $+\epsilon^{1/2}$ and one as $-\epsilon^{1/2}$.

\textbf{Step 3.6: Find the next-order correction.}

\textit{Ansatz with known leading order:}
\[
x = \pm\epsilon^{1/2} + x_2\epsilon + O(\epsilon^{3/2}).
\]

\textit{Compute $x^2$ and $x^3$ to higher order:}

For $x = x_1\epsilon^{1/2} + x_2\epsilon$ with $x_1 = \pm 1$:
\begin{align*}
x^2 &= x_1^2\epsilon + 2x_1x_2\epsilon^{3/2} + O(\epsilon^2) = \epsilon + 2x_1x_2\epsilon^{3/2} + O(\epsilon^2), \\
x^3 &= x_1^3\epsilon^{3/2} + 3x_1^2x_2\epsilon^2 + O(\epsilon^{5/2}) = x_1^3\epsilon^{3/2} + O(\epsilon^2).
\end{align*}

\textit{Note:} We used $x_1^2 = 1$ in simplifying $x^2$.

\textit{Sum:}
\[
x^3 + x^2 = \epsilon + x_1^3\epsilon^{3/2} + 2x_1x_2\epsilon^{3/2} + O(\epsilon^2).
\]

\textit{Setting equal to $\epsilon$:}
\[
\epsilon + (x_1^3 + 2x_1x_2)\epsilon^{3/2} + O(\epsilon^2) = \epsilon.
\]

\textit{At $O(\epsilon^{3/2})$:}
\[
x_1^3 + 2x_1x_2 = 0 \implies x_2 = -\frac{x_1^3}{2x_1} = -\frac{x_1^2}{2} = -\frac{1}{2}.
\]

\textit{Key observation:} The coefficient $x_2 = -1/2$ is the same for both branches ($x_1 = +1$ and $x_1 = -1$).

\textbf{Step 3.7: Final answers for singular roots.}

\[
\boxed{x(\epsilon) = +\epsilon^{1/2} - \frac{1}{2}\epsilon + O(\epsilon^{3/2})},
\]
\[
\boxed{x(\epsilon) = -\epsilon^{1/2} - \frac{1}{2}\epsilon + O(\epsilon^{3/2})}.
\]

\subsubsection*{Summary for Problem 1(b)}

The three roots of $x^3 + x^2 - \epsilon = 0$ are:

\begin{enumerate}
\item \textbf{Regular root} (from simple root at $x = -1$):
\[
x(\epsilon) = -1 + \epsilon + O(\epsilon^2).
\]

\item \textbf{Singular roots} (from double root at $x = 0$):
\[
x(\epsilon) = \pm\epsilon^{1/2} - \frac{1}{2}\epsilon + O(\epsilon^{3/2}).
\]
\end{enumerate}

\textit{Physical interpretation:} The double root at $x = 0$ undergoes a ``splitting'' into two distinct roots when $\epsilon \neq 0$, with the separation growing as $2\epsilon^{1/2}$ for small $\epsilon$.

\newpage
\section*{Problem 1(c)}

\subsection*{Problem Statement}

For $\epsilon \ll 1$, obtain two-term expansions for the solutions of
\[
\epsilon x^3 + x^2 + 2x + 1 = 0.
\]

\subsection*{Complete Solution}

\subsubsection*{Phase I: Problem Classification}

\textbf{Step 1.1: Examine the equation structure.}

\textit{What we observe:} The small parameter $\epsilon$ multiplies the highest-degree term $x^3$.

\textit{Why this matters:} When $\epsilon$ multiplies the highest power, setting $\epsilon = 0$ reduces the degree of the polynomial. This is a hallmark of a \textbf{singular perturbation problem}.

\textbf{Step 1.2: Solve the unperturbed equation.}

\textit{Setting $\epsilon = 0$:}
\[
x^2 + 2x + 1 = 0 \implies (x + 1)^2 = 0 \implies x = -1 \text{ (double root)}.
\]

\textit{Critical observations:}
\begin{itemize}
\item The unperturbed equation (quadratic) has only 2 roots (both equal to $-1$).
\item The perturbed equation (cubic) has 3 roots.
\item One root must ``come from infinity'' as $\epsilon \to 0$ --- this is the hallmark of a singular perturbation.
\item The double root at $x = -1$ suggests non-integer power expansions for the roots near $-1$.
\end{itemize}

\textbf{Step 1.3: Strategy.}

We will find:
\begin{enumerate}
\item Two roots near $x = -1$ (from the double root) using fractional power expansion.
\item One root going to infinity as $\epsilon \to 0$ (the singular root) using dominant balance.
\end{enumerate}

\subsubsection*{Phase II: Roots Near $x_0 = -1$ (Double Root)}

\textbf{Step 2.1: Motivation for fractional powers.}

\textit{Theory:} Since $x = -1$ is a double root of the unperturbed equation, we expect the perturbation to split this into two distinct roots. Based on Lecture Notes Section 2.3 (see Eq. (17)--(20)), we try an expansion of the form:
\[
x = -1 + x_1\epsilon^{1/2} + x_2\epsilon + \cdots
\]

\textbf{Step 2.2: Substitute the ansatz.}

\textit{Original equation:} $\epsilon x^3 + x^2 + 2x + 1 = 0$.

\textit{Let} $x = -1 + x_1\epsilon^{1/2} + x_2\epsilon + \cdots$

\textit{Compute each term:}

\textit{Term 1: $\epsilon x^3$}
\begin{align*}
x^3 &= (-1 + x_1\epsilon^{1/2} + x_2\epsilon + \cdots)^3 \\
&= -1 + 3x_1\epsilon^{1/2} + O(\epsilon).
\end{align*}
Therefore:
\[
\epsilon x^3 = \epsilon(-1 + 3x_1\epsilon^{1/2} + \cdots) = -\epsilon + O(\epsilon^{3/2}).
\]

\textit{Term 2: $x^2$}
\begin{align*}
x^2 &= (-1 + x_1\epsilon^{1/2} + x_2\epsilon)^2 \\
&= 1 - 2x_1\epsilon^{1/2} + (x_1^2 - 2x_2)\epsilon + O(\epsilon^{3/2}).
\end{align*}

\textit{Term 3: $2x$}
\[
2x = 2(-1 + x_1\epsilon^{1/2} + x_2\epsilon + \cdots) = -2 + 2x_1\epsilon^{1/2} + 2x_2\epsilon + O(\epsilon^{3/2}).
\]

\textit{Term 4: $+1$}
\[
+1.
\]

\textbf{Step 2.3: Collect terms by order.}

\[
\epsilon x^3 + x^2 + 2x + 1 = [1 - 2 + 1] + [-2x_1 + 2x_1]\epsilon^{1/2} + [-1 + x_1^2 - 2x_2 + 2x_2]\epsilon + O(\epsilon^{3/2}).
\]

\textit{Simplify:}
\begin{itemize}
\item $O(1)$: $1 - 2 + 1 = 0$ \checkmark (automatically satisfied)
\item $O(\epsilon^{1/2})$: $-2x_1 + 2x_1 = 0$ \checkmark (automatically satisfied)
\item $O(\epsilon)$: $-1 + x_1^2 - 2x_2 + 2x_2 = -1 + x_1^2 = 0$
\end{itemize}

\textbf{Step 2.4: Solve the balance equations.}

\textit{At $O(\epsilon)$:}
\[
x_1^2 - 1 = 0 \implies x_1^2 = 1 \implies x_1 = \pm 1.
\]

\textbf{Step 2.5: Final answers for roots near $x = -1$.}

\[
\boxed{x(\epsilon) = -1 + \epsilon^{1/2} + O(\epsilon) = -1 + \sqrt{\epsilon} + O(\epsilon)},
\]
\[
\boxed{x(\epsilon) = -1 - \epsilon^{1/2} + O(\epsilon) = -1 - \sqrt{\epsilon} + O(\epsilon)}.
\]

\subsubsection*{Phase III: Singular Root Going to Infinity}

\textbf{Step 3.1: Physical intuition.}

\textit{The perturbed equation is cubic (3 roots), but the unperturbed is quadratic (2 roots). Where does the third root ``go'' as $\epsilon \to 0$?}

\textit{Answer:} It must go to infinity. As $\epsilon \to 0$, one root escapes to $\pm\infty$.

\textbf{Step 3.2: Dominant balance analysis.}

\textit{For large $|x|$, compare terms in $\epsilon x^3 + x^2 + 2x + 1 = 0$:}
\begin{itemize}
\item $\epsilon x^3$: grows as $|x|^3$
\item $x^2$: grows as $|x|^2$
\item $2x$: grows as $|x|$
\item $1$: constant
\end{itemize}

\textit{For very large $|x|$, the dominant terms are $\epsilon x^3$ and $x^2$. The terms $2x$ and $1$ become negligible.}

\textit{Dominant balance:}
\[
\epsilon x^3 + x^2 \approx 0 \implies \epsilon x^3 \sim -x^2 \implies x \sim -\frac{1}{\epsilon}.
\]

\textbf{Step 3.3: Make the ansatz for the singular root.}

Based on the dominant balance, we try:
\[
x = -\frac{1}{\epsilon} + x_0 + x_1\epsilon + \cdots
\]

\textbf{Step 3.4: Substitute and expand.}

\textit{Compute $x^2$:}
\begin{align*}
x^2 &= \left(-\frac{1}{\epsilon} + x_0 + \cdots\right)^2 = \frac{1}{\epsilon^2} - \frac{2x_0}{\epsilon} + x_0^2 + O(1).
\end{align*}

\textit{Compute $x^3$:}
\begin{align*}
x^3 &= \left(-\frac{1}{\epsilon} + x_0 + \cdots\right)^3 = -\frac{1}{\epsilon^3} + \frac{3x_0}{\epsilon^2} - \frac{3x_0^2}{\epsilon} + x_0^3 + O(\epsilon).
\end{align*}

\textit{Compute $\epsilon x^3$:}
\[
\epsilon x^3 = -\frac{1}{\epsilon^2} + \frac{3x_0}{\epsilon} - 3x_0^2 + O(\epsilon).
\]

\textit{Compute $2x$:}
\[
2x = -\frac{2}{\epsilon} + 2x_0 + O(\epsilon).
\]

\textbf{Step 3.5: Collect terms.}

\[
\epsilon x^3 + x^2 + 2x + 1 = \left(-\frac{1}{\epsilon^2} + \frac{1}{\epsilon^2}\right) + \left(\frac{3x_0}{\epsilon} - \frac{2x_0}{\epsilon} - \frac{2}{\epsilon}\right) + O(1).
\]

\textit{At $O(\epsilon^{-2})$:}
\[
-\frac{1}{\epsilon^2} + \frac{1}{\epsilon^2} = 0. \quad \checkmark
\]

\textit{At $O(\epsilon^{-1})$:}
\[
\frac{3x_0 - 2x_0 - 2}{\epsilon} = \frac{x_0 - 2}{\epsilon} = 0 \implies x_0 = 2.
\]

\textbf{Step 3.6: Final answer for singular root.}

\[
\boxed{x(\epsilon) = -\frac{1}{\epsilon} + 2 + O(\epsilon)}.
\]

\subsubsection*{Summary for Problem 1(c)}

The three roots of $\epsilon x^3 + x^2 + 2x + 1 = 0$ are:

\begin{enumerate}
\item \textbf{Two roots from the double root at $x = -1$:}
\[
x(\epsilon) = -1 \pm \epsilon^{1/2} + O(\epsilon).
\]

\item \textbf{Singular root (goes to $-\infty$ as $\epsilon \to 0$):}
\[
x(\epsilon) = -\frac{1}{\epsilon} + 2 + O(\epsilon).
\]
\end{enumerate}

\newpage
\section*{Problem 1(d)}

\subsection*{Problem Statement}

For $\epsilon \ll 1$, obtain a two-term expansion for the solution near $x = 0$ of
\[
\sqrt{2}\sin\left(x + \frac{\pi}{4}\right) - 1 - x + \frac{1}{2}x^2 = -\frac{1}{6}\epsilon.
\]

\subsection*{Complete Solution}

\subsubsection*{Phase I: Simplification and Taylor Expansion}

\textbf{Step 1.1: Expand the left-hand side for small $x$.}

\textit{The key insight:} We are looking for solutions near $x = 0$, so we should Taylor expand the LHS around $x = 0$.

\textit{Use the angle addition formula:}
\[
\sqrt{2}\sin\left(x + \frac{\pi}{4}\right) = \sqrt{2}\left[\sin x \cos\frac{\pi}{4} + \cos x \sin\frac{\pi}{4}\right] = \sqrt{2}\left[\frac{\sin x}{\sqrt{2}} + \frac{\cos x}{\sqrt{2}}\right] = \sin x + \cos x.
\]

\textit{Taylor expand $\sin x$ and $\cos x$:}
\begin{align*}
\sin x &= x - \frac{x^3}{6} + \frac{x^5}{120} - \cdots, \\
\cos x &= 1 - \frac{x^2}{2} + \frac{x^4}{24} - \cdots.
\end{align*}

\textit{Therefore:}
\[
\sin x + \cos x = 1 + x - \frac{x^2}{2} - \frac{x^3}{6} + O(x^4).
\]

\textbf{Step 1.2: Substitute into the equation.}

\textit{LHS of equation:}
\begin{align*}
&\sqrt{2}\sin\left(x + \frac{\pi}{4}\right) - 1 - x + \frac{1}{2}x^2 \\
&= (\sin x + \cos x) - 1 - x + \frac{1}{2}x^2 \\
&= \left(1 + x - \frac{x^2}{2} - \frac{x^3}{6} + O(x^4)\right) - 1 - x + \frac{1}{2}x^2 \\
&= 1 + x - \frac{x^2}{2} - \frac{x^3}{6} - 1 - x + \frac{x^2}{2} + O(x^4) \\
&= -\frac{x^3}{6} + \frac{x^4}{24} + O(x^5).
\end{align*}

\textit{Remarkable cancellation:} The constant terms, $x$ terms, and $x^2$ terms all cancel exactly!

\textbf{Step 1.3: Write the simplified equation.}

The equation becomes:
\[
-\frac{x^3}{6} + \frac{x^4}{24} + O(x^5) = -\frac{\epsilon}{6}.
\]

Multiply through by $-6$:
\[
x^3 - \frac{x^4}{4} + O(x^5) = \epsilon.
\]

\subsubsection*{Phase II: Leading-Order Solution}

\textbf{Step 2.1: Dominant balance for leading order.}

\textit{For small $x$, compare terms:}
\begin{itemize}
\item $x^3$: leading term
\item $x^4/4$: subdominant (since $x^4 \ll x^3$ for small $x$)
\end{itemize}

\textit{Leading-order balance:}
\[
x^3 \approx \epsilon \implies x \approx \epsilon^{1/3}.
\]

\textbf{Step 2.2: Leading-order solution.}

\[
x_0 = \epsilon^{1/3}.
\]

\textit{Verification:} With $x \sim \epsilon^{1/3}$, we have $x^4 \sim \epsilon^{4/3}$, which is indeed $o(\epsilon)$ as $\epsilon \to 0$. So neglecting $x^4$ at leading order is justified.

\subsubsection*{Phase III: Next-Order Correction}

\textbf{Step 3.1: Make the ansatz for next order.}

We seek:
\[
x = \epsilon^{1/3} + \alpha\epsilon^\beta + \cdots
\]
where $\beta > 1/3$ (since this is a correction to the leading term) and $\alpha$ is a constant to be determined.

\textbf{Step 3.2: Substitute and expand.}

\textit{Let} $x = \epsilon^{1/3} + \alpha\epsilon^\beta$ with $\beta > 1/3$.

\textit{Compute $x^3$:}
\begin{align*}
x^3 &= (\epsilon^{1/3} + \alpha\epsilon^\beta)^3 \\
&= \epsilon + 3\epsilon^{2/3}(\alpha\epsilon^\beta) + 3\epsilon^{1/3}(\alpha\epsilon^\beta)^2 + (\alpha\epsilon^\beta)^3 \\
&= \epsilon + 3\alpha\epsilon^{2/3 + \beta} + O(\epsilon^{1/3 + 2\beta}).
\end{align*}

\textit{Compute $x^4$:}
\begin{align*}
x^4 &= (\epsilon^{1/3})^4 + 4(\epsilon^{1/3})^3(\alpha\epsilon^\beta) + \cdots \\
&= \epsilon^{4/3} + 4\alpha\epsilon^{1+\beta} + \cdots.
\end{align*}

\textit{Therefore:}
\[
x^3 - \frac{x^4}{4} = \epsilon + 3\alpha\epsilon^{2/3+\beta} - \frac{\epsilon^{4/3}}{4} + O(\text{higher order}).
\]

\textbf{Step 3.3: Balance the equation.}

The equation is:
\[
x^3 - \frac{x^4}{4} + O(x^5) = \epsilon.
\]

Substituting:
\[
\epsilon + 3\alpha\epsilon^{2/3+\beta} - \frac{\epsilon^{4/3}}{4} + \cdots = \epsilon.
\]

\textit{The $\epsilon$ terms cancel. We need to balance the remaining terms.}

\textit{The next largest terms are:}
\begin{itemize}
\item $3\alpha\epsilon^{2/3+\beta}$
\item $-\frac{1}{4}\epsilon^{4/3}$
\end{itemize}

\textit{For these to balance:}
\[
\frac{2}{3} + \beta = \frac{4}{3} \implies \beta = \frac{2}{3}.
\]

\textit{With $\beta = 2/3$:}
\[
3\alpha\epsilon^{4/3} - \frac{1}{4}\epsilon^{4/3} = 0 \implies 3\alpha = \frac{1}{4} \implies \alpha = \frac{1}{12}.
\]

\textbf{Step 3.4: Final answer.}

\[
\boxed{x(\epsilon) = \epsilon^{1/3} + \frac{1}{12}\epsilon^{2/3} + o(\epsilon^{2/3})}.
\]

\subsubsection*{Verification}

\textit{Let's verify the first few terms of our expansion.}

With $x = \epsilon^{1/3} + \frac{1}{12}\epsilon^{2/3}$:

\begin{align*}
x^3 &= \left(\epsilon^{1/3}\right)^3 + 3\left(\epsilon^{1/3}\right)^2 \cdot \frac{1}{12}\epsilon^{2/3} + O(\epsilon^{5/3}) \\
&= \epsilon + \frac{1}{4}\epsilon^{4/3} + O(\epsilon^{5/3}).
\end{align*}

\begin{align*}
x^4 &= \left(\epsilon^{1/3}\right)^4 + O(\epsilon^{5/3}) = \epsilon^{4/3} + O(\epsilon^{5/3}).
\end{align*}

\begin{align*}
x^3 - \frac{x^4}{4} &= \epsilon + \frac{1}{4}\epsilon^{4/3} - \frac{1}{4}\epsilon^{4/3} + O(\epsilon^{5/3}) \\
&= \epsilon + O(\epsilon^{5/3}).
\end{align*}

This equals $\epsilon$ to the required order. \checkmark

\subsubsection*{Summary for Problem 1(d)}

The solution near $x = 0$ of $\sqrt{2}\sin(x + \pi/4) - 1 - x + \frac{1}{2}x^2 = -\frac{\epsilon}{6}$ is:
\[
x(\epsilon) = \epsilon^{1/3} + \frac{1}{12}\epsilon^{2/3} + o(\epsilon^{2/3}) \quad \text{as } \epsilon \to 0.
\]

\textit{Key insight:} The remarkable cancellation of lower-order terms in the Taylor expansion of the LHS leads to the dominant balance $x^3 \sim \epsilon$, giving the unusual scaling $x \sim \epsilon^{1/3}$.

\end{document}
