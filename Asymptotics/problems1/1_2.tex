\documentclass[11pt,a4paper]{article}
\usepackage{amsmath,amssymb,amsthm}
\usepackage{geometry}
\geometry{margin=1in}
\usepackage{enumitem}
\usepackage{xcolor}

\newtheorem{theorem}{Theorem}
\newtheorem{lemma}[theorem]{Lemma}
\newtheorem{definition}[theorem]{Definition}

\title{}
\author{}
\date{}

\begin{document}

\section*{Problem Sheet 1, Question 2}
\section*{Problem Statement}

Find two-term expansions for each root of
\begin{equation}
\epsilon^2 x^3 + x^2 + 2x + \epsilon = 0, \quad \epsilon \ll 1.
\end{equation}

\section*{Complete Solution}

\subsection*{Phase I: Problem Classification}

\textbf{Step 1.1: Identify the structure of the equation.}

\textit{What we observe:} The equation has the form $F(x;\epsilon) = 0$ where $\epsilon$ appears both as an additive perturbation (the $+\epsilon$ term) and multiplying the highest-degree term ($\epsilon^2 x^3$).

\textit{Why this matters:} According to Lecture Notes Section 2.2, when a small parameter multiplies the highest-degree term, the problem is potentially singular because setting $\epsilon = 0$ reduces the degree of the equation.

\textbf{Step 1.2: Solve the unperturbed equation.}

\textit{Setting $\epsilon = 0$:}
\[
x^2 + 2x = x(x + 2) = 0.
\]

\textit{Solutions of unperturbed equation:}
\[
x_0^{(1)} = 0, \quad x_0^{(2)} = -2.
\]

\textbf{Step 1.3: Count degrees of freedom.}

\textit{What we observe:}
\begin{itemize}
\item The perturbed equation (1) is cubic (degree 3), so it has 3 roots.
\item The unperturbed equation is quadratic (degree 2), with only 2 roots.
\end{itemize}

\textit{Why this matters:} The mismatch in the number of roots confirms this is a \textbf{singular perturbation problem}. One root must ``escape to infinity'' as $\epsilon \to 0$.

\textbf{Step 1.4: Classify the problem.}

\textit{Conclusion:} This is a \textbf{singular perturbation problem}.

\textit{Method to use:}
\begin{enumerate}
\item For the two roots near finite values: use standard expansion method (Section 2.1.1)
\item For the ``missing'' third root: use dominant balance analysis (Section 2.2.2)
\end{enumerate}

\subsection*{Phase II: Solution Near $x_0 = 0$}

\textbf{Step 2.1: Make the expansion ansatz.}

\textit{What we assume:} Since $x \to 0$ as $\epsilon \to 0$, and there is no constant term from the unperturbed root, we write:
\[
x(\epsilon) = a_1\epsilon + a_2\epsilon^2 + O(\epsilon^3).
\]

\textbf{Step 2.2: Substitute into the equation.}

\textit{The cubic term:}
\[
\epsilon^2 x^3 = \epsilon^2(a_1\epsilon + \cdots)^3 = a_1^3\epsilon^5 + O(\epsilon^6).
\]
This is $O(\epsilon^5)$, negligible at the orders we need.

\textit{The quadratic term:}
\[
x^2 = (a_1\epsilon + a_2\epsilon^2 + \cdots)^2 = a_1^2\epsilon^2 + 2a_1a_2\epsilon^3 + O(\epsilon^4).
\]

\textit{The linear term:}
\[
2x = 2a_1\epsilon + 2a_2\epsilon^2 + O(\epsilon^3).
\]

\textit{The constant term:} $\epsilon$.

\textbf{Step 2.3: Collect terms by powers of $\epsilon$.}

Adding all terms:
\[
\underbrace{(2a_1 + 1)}_{O(\epsilon)}\epsilon + \underbrace{(a_1^2 + 2a_2)}_{O(\epsilon^2)}\epsilon^2 + O(\epsilon^3) = 0.
\]

For this to hold for all small $\epsilon$, each coefficient must vanish.

\textbf{Step 2.4: Solve order by order.}

\textit{At $O(\epsilon)$:}
\[
2a_1 + 1 = 0 \implies a_1 = -\frac{1}{2}.
\]

\textit{At $O(\epsilon^2)$:}
\[
a_1^2 + 2a_2 = 0 \implies \frac{1}{4} + 2a_2 = 0 \implies a_2 = -\frac{1}{8}.
\]

\textbf{Final answer for root near $x_0 = 0$:}
\[
\boxed{x^{(1)}(\epsilon) = -\frac{1}{2}\epsilon - \frac{1}{8}\epsilon^2 + O(\epsilon^3)}.
\]

\subsection*{Phase III: Solution Near $x_0 = -2$}

\textbf{Step 3.1: Make the expansion ansatz.}

\textit{What we assume:} Since $x \to -2$ as $\epsilon \to 0$:
\[
x(\epsilon) = -2 + b_1\epsilon + O(\epsilon^2).
\]

\textbf{Step 3.2: Substitute into the equation.}

\textit{The cubic term:}
\[
x^3 = (-2 + b_1\epsilon + \cdots)^3 = -8 + 12b_1\epsilon + O(\epsilon^2).
\]
\[
\epsilon^2 x^3 = -8\epsilon^2 + O(\epsilon^3).
\]

\textit{The quadratic term:}
\[
x^2 = (-2)^2 + 2(-2)(b_1\epsilon) + O(\epsilon^2) = 4 - 4b_1\epsilon + O(\epsilon^2).
\]

\textit{The linear term:}
\[
2x = -4 + 2b_1\epsilon + O(\epsilon^2).
\]

\textit{The constant term:} $\epsilon$.

\textbf{Step 3.3: Collect terms by powers of $\epsilon$.}

Adding all terms:
\[
\underbrace{(4 - 4)}_{O(1)} + \underbrace{(-4b_1 + 2b_1 + 1)}_{O(\epsilon)}\epsilon + O(\epsilon^2) = 0.
\]

\textbf{Step 3.4: Solve order by order.}

\textit{At $O(1)$:}
\[
4 - 4 = 0. \quad \checkmark
\]
This confirms $x_0 = -2$ is a root of the unperturbed equation.

\textit{At $O(\epsilon)$:}
\[
-4b_1 + 2b_1 + 1 = 0 \implies -2b_1 + 1 = 0 \implies b_1 = \frac{1}{2}.
\]

\textbf{Final answer for root near $x_0 = -2$:}
\[
\boxed{x^{(2)}(\epsilon) = -2 + \frac{1}{2}\epsilon + O(\epsilon^2)}.
\]

\textit{Note:} For a two-term expansion, the two terms are $-2$ and $\frac{1}{2}\epsilon$.

\subsection*{Phase IV: Singular Solution via Dominant Balance}

\textbf{Step 4.1: Why dominant balance is needed.}

\textit{The situation:} We have found 2 roots, but a cubic equation has 3 roots. The third root cannot be found by expanding around any finite unperturbed value---it must escape to infinity as $\epsilon \to 0$.

\textit{The question:} How does this root scale with $\epsilon$? That is, what power of $\epsilon$ describes its size?

\textbf{Step 4.2: The dominant balance principle.}

\textit{Key insight:} For an equation to be satisfied, terms cannot simply ``blow up'' to infinity---they must \textbf{cancel}. When $|x| \to \infty$, at least two terms must be of the same order of magnitude and opposite in sign, while all other terms are smaller (subdominant).

\textit{The method:} Assume the singular root scales as
\[
x \sim \epsilon^{-\alpha} \quad \text{for some } \alpha > 0.
\]
Then determine $\alpha$ by requiring that:
\begin{enumerate}
\item[(i)] At least two terms have the same order in $\epsilon$ (they balance).
\item[(ii)] These balanced terms are the \textbf{largest} terms in the equation.
\item[(iii)] All other terms are smaller (subdominant).
\end{enumerate}

\textbf{Step 4.3: Compute the order of each term.}

\textit{The equation:}
\[
\epsilon^2 x^3 + x^2 + 2x + \epsilon = 0.
\]

\textit{With $x = O(\epsilon^{-\alpha})$, compute the $\epsilon$-order of each term.}

\textit{The rule:} If $x = O(\epsilon^{-\alpha})$, then $x^n = O(\epsilon^{-n\alpha})$ (powers multiply). For a general term $\epsilon^m x^n$:
\[
\epsilon^m x^n = O(\epsilon^m) \cdot O(\epsilon^{-n\alpha}) = O(\epsilon^{m - n\alpha}).
\]

\textit{Applying this rule to each term:}
\begin{align*}
\epsilon^2 x^3 &= O(\epsilon^2) \cdot O(\epsilon^{-3\alpha}) = O(\epsilon^{2-3\alpha}), \\
x^2 &= O(\epsilon^{-2\alpha}), \\
2x &= O(\epsilon^{-\alpha}), \\
\epsilon &= O(\epsilon^1).
\end{align*}

\textbf{Step 4.4: Determine which terms can balance.}

\textit{Systematic analysis:} For large $|x|$ (i.e., $\alpha > 0$), rank the terms from largest to smallest. The exponent of $\epsilon$ determines size: \textbf{more negative = larger}.

\begin{center}
\begin{tabular}{c|c|c}
\textbf{Term} & \textbf{Order} & \textbf{Exponent} \\
\hline
$\epsilon^2 x^3$ & $O(\epsilon^{2-3\alpha})$ & $2-3\alpha$ \\
$x^2$ & $O(\epsilon^{-2\alpha})$ & $-2\alpha$ \\
$2x$ & $O(\epsilon^{-\alpha})$ & $-\alpha$ \\
$\epsilon$ & $O(\epsilon)$ & $1$
\end{tabular}
\end{center}

\textit{For the two largest terms to balance:} Set their exponents equal. The natural candidates are $\epsilon^2 x^3$ and $x^2$ (both involve powers of $x$):
\[
2 - 3\alpha = -2\alpha \implies \alpha = 2.
\]

\textbf{Step 4.5: Verify the balance is consistent.}

\textit{With $\alpha = 2$, compute all exponents:}
\begin{align*}
\epsilon^2 x^3 &= O(\epsilon^{2-6}) = O(\epsilon^{-4}), \\
x^2 &= O(\epsilon^{-4}), \\
2x &= O(\epsilon^{-2}), \\
\epsilon &= O(\epsilon).
\end{align*}

\textit{Check the hierarchy:}
\[
\underbrace{O(\epsilon^{-4})}_{\epsilon^2 x^3,\, x^2} \gg \underbrace{O(\epsilon^{-2})}_{2x} \gg \underbrace{O(\epsilon)}_{\epsilon}.
\]

\textit{Conclusion:} The balance $\epsilon^2 x^3 \sim x^2$ is \textbf{consistent}---these are indeed the two largest terms, and $2x$ and $\epsilon$ are subdominant. The scaling $x \sim \epsilon^{-2}$ is correct.

\textbf{Step 4.6: Extract the leading coefficient.}

\textit{From the dominant balance:}
\[
\epsilon^2 x^3 + x^2 \approx 0 \implies x^2(\epsilon^2 x + 1) = 0.
\]

Since $x \neq 0$ for this root:
\[
\epsilon^2 x + 1 = 0 \implies x = -\frac{1}{\epsilon^2}.
\]

\textbf{Step 4.7: Find the next-order correction.}

\textit{Ansatz:} Based on dominant balance, write:
\[
x = -\frac{1}{\epsilon^2} + c_0 + O(\epsilon),
\]
where $c_0$ is a constant to be determined.

\textit{Substitute and expand each term:}

\textit{Compute $x^3$:}
\[
x^3 = \left(-\frac{1}{\epsilon^2} + c_0\right)^3 = -\frac{1}{\epsilon^6} + \frac{3c_0}{\epsilon^4} - \frac{3c_0^2}{\epsilon^2} + c_0^3.
\]

\textit{Compute $\epsilon^2 x^3$:}
\[
\epsilon^2 x^3 = -\frac{1}{\epsilon^4} + \frac{3c_0}{\epsilon^2} - 3c_0^2 + O(\epsilon^2).
\]

\textit{Compute $x^2$:}
\[
x^2 = \frac{1}{\epsilon^4} - \frac{2c_0}{\epsilon^2} + c_0^2.
\]

\textit{Compute $2x$:}
\[
2x = -\frac{2}{\epsilon^2} + 2c_0.
\]

\textbf{Step 4.8: Collect terms by powers of $\epsilon$.}

Adding all terms:
\[
\underbrace{\left(-\frac{1}{\epsilon^4} + \frac{1}{\epsilon^4}\right)}_{O(\epsilon^{-4})} + \underbrace{\left(\frac{3c_0 - 2c_0 - 2}{\epsilon^2}\right)}_{O(\epsilon^{-2})} + \underbrace{(-3c_0^2 + c_0^2 + 2c_0)}_{O(1)} + O(\epsilon) = 0.
\]

\textbf{Step 4.9: Solve order by order.}

\textit{At $O(\epsilon^{-4})$:}
\[
-\frac{1}{\epsilon^4} + \frac{1}{\epsilon^4} = 0. \quad \checkmark
\]
This confirms the leading term $-1/\epsilon^2$ is correct.

\textit{At $O(\epsilon^{-2})$:}
\[
\frac{3c_0 - 2c_0 - 2}{\epsilon^2} = \frac{c_0 - 2}{\epsilon^2} = 0 \implies c_0 = 2.
\]

\textbf{Final answer for singular root:}
\[
\boxed{x^{(3)}(\epsilon) = -\frac{1}{\epsilon^2} + 2 + O(\epsilon)}.
\]

\subsection*{Summary}

The three roots of $\epsilon^2 x^3 + x^2 + 2x + \epsilon = 0$ are:

\begin{align*}
x^{(1)}(\epsilon) &= -\frac{1}{2}\epsilon - \frac{1}{8}\epsilon^2 + O(\epsilon^3), \\[0.5em]
x^{(2)}(\epsilon) &= -2 + \frac{1}{2}\epsilon + O(\epsilon^2), \\[0.5em]
x^{(3)}(\epsilon) &= -\frac{1}{\epsilon^2} + 2 + O(\epsilon).
\end{align*}

\textit{Classification:}
\begin{itemize}
\item Root 1: Regular solution near $x_0 = 0$. Two terms: $-\frac{1}{2}\epsilon$ and $-\frac{1}{8}\epsilon^2$.
\item Root 2: Regular solution near $x_0 = -2$. Two terms: $-2$ and $+\frac{1}{2}\epsilon$.
\item Root 3: Singular solution (escapes to $-\infty$ as $\epsilon \to 0$). Two terms: $-\frac{1}{\epsilon^2}$ and $+2$.
\end{itemize}

\subsection*{General Method: Finding Singular Roots via Dominant Balance}

For any polynomial equation where $\epsilon \to 0$ causes the degree to drop (losing roots to infinity):

\begin{enumerate}
\item \textbf{Assume scaling:} Let $x \sim \epsilon^{-\alpha}$ for unknown $\alpha > 0$.

\item \textbf{Compute orders:} For each term $\epsilon^m x^n$, apply the rule:
\[
\epsilon^m x^n = O(\epsilon^{m - n\alpha}).
\]
The exponent $m - n\alpha$ determines the size of the term.

\item \textbf{Find $\alpha$:} Set the exponents of the two largest terms equal and solve for $\alpha$. Terms are ``largest'' when their exponent is most negative.

\item \textbf{Verify consistency:} Confirm these two terms are indeed the largest (most negative exponent), and all others are subdominant (less negative or positive exponent).

\item \textbf{Extract leading behavior:} From the balanced terms, solve for the leading coefficient of $x$.

\item \textbf{Iterate:} Substitute $x = (\text{leading}) + c_0 + \cdots$ and collect terms to find corrections.
\end{enumerate}

This method works for \textbf{any} singular perturbation problem where roots escape to infinity, regardless of the specific equation.

\subsection*{Verification}

\textbf{Verification of Root 1:}

Substitute $x = -\frac{1}{2}\epsilon - \frac{1}{8}\epsilon^2$ into $\epsilon^2 x^3 + x^2 + 2x + \epsilon$:
\begin{align*}
\epsilon^2 x^3 &= O(\epsilon^5), \\
x^2 &= \frac{\epsilon^2}{4} + \frac{\epsilon^3}{8} + O(\epsilon^4), \\
2x &= -\epsilon - \frac{\epsilon^2}{4}, \\
\epsilon &= \epsilon.
\end{align*}

Sum: $\frac{\epsilon^2}{4} - \frac{\epsilon^2}{4} - \epsilon + \epsilon + O(\epsilon^3) = O(\epsilon^3)$. \checkmark

\textbf{Verification of Root 2:}

Substitute $x = -2 + \frac{1}{2}\epsilon$ into the equation:
\begin{align*}
\epsilon^2 x^3 &= -8\epsilon^2 + O(\epsilon^3), \\
x^2 &= 4 - 2\epsilon + O(\epsilon^2), \\
2x &= -4 + \epsilon, \\
\epsilon &= \epsilon.
\end{align*}

Sum: $(4-4) + (-2\epsilon + \epsilon + \epsilon) + O(\epsilon^2) = O(\epsilon^2)$. \checkmark

\textbf{Verification of Root 3:}

Substitute $x = -\frac{1}{\epsilon^2} + 2$ into the equation:
\begin{align*}
\epsilon^2 x^3 &= -\frac{1}{\epsilon^4} + \frac{6}{\epsilon^2} - 12 + O(\epsilon^2), \\
x^2 &= \frac{1}{\epsilon^4} - \frac{4}{\epsilon^2} + 4, \\
2x &= -\frac{2}{\epsilon^2} + 4, \\
\epsilon &= \epsilon.
\end{align*}

At $O(\epsilon^{-4})$: $-\frac{1}{\epsilon^4} + \frac{1}{\epsilon^4} = 0$. \checkmark

At $O(\epsilon^{-2})$: $\frac{6}{\epsilon^2} - \frac{4}{\epsilon^2} - \frac{2}{\epsilon^2} = 0$. \checkmark

At $O(1)$: $-12 + 4 + 4 = -4 \neq 0$.

The $O(1)$ residual confirms we need higher-order corrections beyond the two-term expansion. \checkmark

\end{document}
