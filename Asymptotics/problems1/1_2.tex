\documentclass[11pt,a4paper]{article}
\usepackage{amsmath,amssymb,amsthm}
\usepackage{geometry}
\geometry{margin=1in}
\usepackage{enumitem}
\usepackage{xcolor}

\newtheorem{theorem}{Theorem}
\newtheorem{lemma}[theorem]{Lemma}
\newtheorem{definition}[theorem]{Definition}

\title{Asymptotics 2025/2026 Sheet 1\\Problem 2: Complete Solution}
\author{}
\date{}

\begin{document}

\maketitle

\section*{Problem Statement}

Find two terms in the expansion for each root of
\begin{equation}
\epsilon^2 x^3 + x^2 + 2x + \epsilon = 0, \quad \epsilon \ll 1.
\end{equation}

\section{Initial Analysis and Strategy}

\subsection{Why We Begin With The Unperturbed Problem}

\textbf{What we observe:} The equation contains a small parameter $\epsilon$ multiplying certain terms.

\textbf{Why this matters:} According to Section 2 of the lecture notes, when we have an equation depending on a small parameter $\epsilon$, we seek to understand how solutions behave as $\epsilon \to 0$. The fundamental strategy is to first solve the simpler problem where $\epsilon = 0$, then understand how solutions "perturb" away from this base case.

\textbf{What we do:} Set $\epsilon = 0$ to obtain the unperturbed equation.

\textbf{Why we do this:} The unperturbed equation gives us candidate locations where roots might exist for small $\epsilon$. These are our "base points" around which we'll construct expansions.

\subsection{The Unperturbed Equation}

Setting $\epsilon = 0$ in equation (1):
\begin{equation}
x^2 + 2x = 0.
\end{equation}

\textbf{What we have:} A quadratic equation.

\textbf{Why we factor:} Factoring reveals the structure of solutions clearly.
\begin{equation}
x(x + 2) = 0.
\end{equation}

\textbf{What this tells us:} The unperturbed equation has exactly two solutions:
\begin{equation}
x_0 = 0 \quad \text{and} \quad x_0 = -2.
\end{equation}

\subsection{Critical Observation: Order Reduction}

\textbf{What we notice:} The original equation (1) is cubic (degree 3), but the unperturbed equation (2) is quadratic (degree 2).

\textbf{Why this is significant:} This is the hallmark of a \textbf{singular perturbation problem}. From Section 2.2 of the notes:

\begin{quote}
\textit{Singular perturbation problem: The perturbed and unperturbed problem differ in an essential way: Not all solutions of the perturbed problem can be expressed as an expansion around the unperturbed solution(s).}
\end{quote}

\textbf{What this means concretely:}
\begin{itemize}
\item The perturbed equation has 3 roots (it's cubic)
\item The unperturbed equation has 2 roots
\item Therefore, at least one root of the perturbed equation must behave in a qualitatively different way as $\epsilon \to 0$
\end{itemize}

\textbf{What we expect:}
\begin{itemize}
\item Some roots will be "regular" - approaching the unperturbed roots smoothly
\item At least one root will be "singular" - either going to infinity, or requiring non-standard expansions
\end{itemize}

\subsection{Strategy: The Two-Pronged Approach}

\textbf{Our plan:}
\begin{enumerate}
\item \textbf{Regular solutions:} Try standard power series expansions $x = x_0 + x_1\epsilon + x_2\epsilon^2 + \cdots$ around each unperturbed root $x_0 \in \{0, -2\}$
\item \textbf{Singular solution:} If we don't get all three roots from step 1, perform dominant balance analysis to find the "missing" root
\end{enumerate}

\section{Finding Regular Solutions}

\subsection{Expansion Method Framework}

\textbf{The general approach (from Section 2.1.1):}

For each unperturbed solution $x_0$, we make the ansatz:
\begin{equation}
x(\epsilon) = x_0 + x_1\epsilon + x_2\epsilon^2 + x_3\epsilon^3 + \cdots
\end{equation}

\textbf{Why this form:} We assume the solution depends smoothly on $\epsilon$, so a Taylor series in $\epsilon$ is natural. The coefficients $x_1, x_2, x_3, \ldots$ are constants to be determined.

\textbf{The key principle:} After substitution, we collect terms by powers of $\epsilon$. For the equation to hold for all small $\epsilon$, each power of $\epsilon$ must vanish independently.

\textbf{Why powers must vanish independently:} If
\[
a_0 + a_1\epsilon + a_2\epsilon^2 + \cdots = 0
\]
for all small $\epsilon$, then setting $\epsilon = 0$ gives $a_0 = 0$. Dividing by $\epsilon$ and taking $\epsilon \to 0$ gives $a_1 = 0$, and so forth.

\subsection{Attempt 1: Regular Solution Near $x_0 = 0$}

\textbf{Our ansatz:}
\begin{equation}
x = 0 + x_1\epsilon + x_2\epsilon^2 + x_3\epsilon^3 + \cdots = x_1\epsilon + x_2\epsilon^2 + x_3\epsilon^3 + \cdots
\end{equation}

\textbf{Why we start here:} $x_0 = 0$ is simpler algebraically (fewer terms to track).

\subsubsection{Substituting Into The Original Equation}

We substitute (6) into (1):
\begin{equation}
\epsilon^2(x_1\epsilon + x_2\epsilon^2 + \cdots)^3 + (x_1\epsilon + x_2\epsilon^2 + \cdots)^2 + 2(x_1\epsilon + x_2\epsilon^2 + \cdots) + \epsilon = 0.
\end{equation}

\textbf{What we need to do:} Expand each term and collect by powers of $\epsilon$.

\subsubsection{Expanding The Cubic Term}

\textbf{The cubic term:}
\begin{align}
\epsilon^2(x_1\epsilon + x_2\epsilon^2 + \cdots)^3 &= \epsilon^2 \cdot \epsilon^3(x_1 + x_2\epsilon + \cdots)^3 \\
&= \epsilon^5(x_1^3 + 3x_1^2x_2\epsilon + \cdots)
\end{align}

\textbf{Why we can ignore this:} The leading power is $\epsilon^5$. Since we're finding the first two terms (coefficients of $\epsilon$ and $\epsilon^2$), terms starting at $\epsilon^5$ won't affect our calculation.

\textbf{What we write:}
\begin{equation}
\epsilon^2 x^3 = O(\epsilon^5).
\end{equation}

\subsubsection{Expanding The Quadratic Term}

\textbf{The quadratic term:}
\begin{align}
(x_1\epsilon + x_2\epsilon^2 + x_3\epsilon^3 + \cdots)^2 &= x_1^2\epsilon^2 + 2x_1x_2\epsilon^3 + (x_2^2 + 2x_1x_3)\epsilon^4 + \cdots
\end{align}

\textbf{Why each term appears:}
\begin{itemize}
\item $x_1^2\epsilon^2$: from $(x_1\epsilon)^2$
\item $2x_1x_2\epsilon^3$: from $2(x_1\epsilon)(x_2\epsilon^2)$
\item Higher order terms involve more products
\end{itemize}

\textbf{What we retain:} For two-term accuracy, we need up to $\epsilon^2$:
\begin{equation}
x^2 = x_1^2\epsilon^2 + O(\epsilon^3).
\end{equation}

\subsubsection{The Linear Term}

\textbf{The linear term:}
\begin{equation}
2x = 2x_1\epsilon + 2x_2\epsilon^2 + 2x_3\epsilon^3 + \cdots
\end{equation}

\textbf{Why this is straightforward:} Linear terms don't mix - each power of $\epsilon$ stays separate.

\subsubsection{Collecting All Terms}

\textbf{The full equation becomes:}
\begin{equation}
O(\epsilon^5) + x_1^2\epsilon^2 + 2x_1\epsilon + 2x_2\epsilon^2 + \epsilon + O(\epsilon^3) = 0.
\end{equation}

\textbf{Grouping by powers of $\epsilon$:}
\begin{equation}
(2x_1 + 1)\epsilon + (x_1^2 + 2x_2)\epsilon^2 + O(\epsilon^3) = 0.
\end{equation}

\subsubsection{Solving Order By Order}

\textbf{At $O(\epsilon)$: The coefficient of $\epsilon$ must vanish}
\begin{equation}
2x_1 + 1 = 0.
\end{equation}

\textbf{Why this must be zero:} For the equation to hold for all small $\epsilon > 0$, we can't have a non-zero coefficient of $\epsilon$ (it would dominate as $\epsilon \to 0$).

\textbf{Solving:}
\begin{equation}
x_1 = -\frac{1}{2}.
\end{equation}

\textbf{What this means:} The leading correction to $x = 0$ is negative, pushing the root toward negative values.

\textbf{At $O(\epsilon^2)$: The coefficient of $\epsilon^2$ must vanish}
\begin{equation}
x_1^2 + 2x_2 = 0.
\end{equation}

\textbf{Why we can solve this now:} We know $x_1 = -\frac{1}{2}$ from the previous order, so we can substitute:
\begin{equation}
\left(-\frac{1}{2}\right)^2 + 2x_2 = 0.
\end{equation}
\begin{equation}
\frac{1}{4} + 2x_2 = 0.
\end{equation}
\begin{equation}
x_2 = -\frac{1}{8}.
\end{equation}

\textbf{What this tells us:} The $O(\epsilon^2)$ correction is also negative, continuing the trend.

\subsubsection{First Root: The Result}

\textbf{Our first root:}
\begin{equation}
\boxed{x_1(\epsilon) = -\frac{1}{2}\epsilon - \frac{1}{8}\epsilon^2 + O(\epsilon^3)}.
\end{equation}

\textbf{Physical interpretation:} Starting from $x = 0$, the perturbation pushes the root in the negative direction, with corrections of order $\epsilon$ and $\epsilon^2$.

\subsection{Attempt 2: Regular Solution Near $x_0 = -2$}

\textbf{Our ansatz:}
\begin{equation}
x = -2 + x_1\epsilon + x_2\epsilon^2 + x_3\epsilon^3 + \cdots
\end{equation}

\textbf{Why we try this:} The second unperturbed root is $x_0 = -2$, and we expect a regular expansion around it.

\subsubsection{Substituting Into The Original Equation}

We substitute (20) into (1):
\begin{multline}
\epsilon^2(-2 + x_1\epsilon + x_2\epsilon^2 + \cdots)^3 + (-2 + x_1\epsilon + x_2\epsilon^2 + \cdots)^2 \\
+ 2(-2 + x_1\epsilon + x_2\epsilon^2 + \cdots) + \epsilon = 0.
\end{multline}

\subsubsection{Expanding The Cubic Term}

\textbf{Using the binomial expansion:}
\begin{align}
(-2 + x_1\epsilon + x_2\epsilon^2 + \cdots)^3 &= (-2)^3 + 3(-2)^2(x_1\epsilon) + \cdots \\
&= -8 + 12x_1\epsilon + O(\epsilon^2).
\end{align}

\textbf{Why we stop here:} For two-term accuracy, we only need up to $O(\epsilon^2)$ in the final equation. The cubic term is multiplied by $\epsilon^2$, so:
\begin{equation}
\epsilon^2 \cdot (-8 + 12x_1\epsilon + \cdots) = -8\epsilon^2 + 12x_1\epsilon^3 + \cdots
\end{equation}

\textbf{What we retain:}
\begin{equation}
\epsilon^2 x^3 = -8\epsilon^2 + O(\epsilon^3).
\end{equation}

\subsubsection{Expanding The Quadratic Term}

\textbf{Using binomial expansion:}
\begin{align}
(-2 + x_1\epsilon + x_2\epsilon^2 + \cdots)^2 &= (-2)^2 + 2(-2)(x_1\epsilon) + 2(-2)(x_2\epsilon^2) + (x_1\epsilon)^2 + \cdots \\
&= 4 - 4x_1\epsilon - 4x_2\epsilon^2 + x_1^2\epsilon^2 + O(\epsilon^3) \\
&= 4 - 4x_1\epsilon + (x_1^2 - 4x_2)\epsilon^2 + O(\epsilon^3).
\end{align}

\textbf{Why each term:}
\begin{itemize}
\item $4 = (-2)^2$: the zeroth order term
\item $-4x_1\epsilon = 2(-2)(x_1\epsilon)$: cross term from binomial
\item $-4x_2\epsilon^2 = 2(-2)(x_2\epsilon^2)$: cross term with $x_2$
\item $x_1^2\epsilon^2 = (x_1\epsilon)^2$: square of the first correction
\end{itemize}

\subsubsection{The Linear Term}

\textbf{Straightforward expansion:}
\begin{equation}
2x = 2(-2 + x_1\epsilon + x_2\epsilon^2 + \cdots) = -4 + 2x_1\epsilon + 2x_2\epsilon^2 + \cdots
\end{equation}

\subsubsection{Collecting All Terms}

\textbf{The full equation:}
\begin{equation}
-8\epsilon^2 + [4 - 4x_1\epsilon + (x_1^2 - 4x_2)\epsilon^2] + [-4 + 2x_1\epsilon + 2x_2\epsilon^2] + \epsilon + O(\epsilon^3) = 0.
\end{equation}

\textbf{Why we group by powers:} This is the standard method to extract coefficients systematically.

\textbf{Collecting constant terms:}
\begin{equation}
4 - 4 = 0. \quad \checkmark
\end{equation}

\textbf{Why this must work:} We chose $x_0 = -2$ because it's a root of the unperturbed equation, so the $O(1)$ terms must cancel.

\textbf{Collecting $O(\epsilon)$ terms:}
\begin{equation}
-4x_1 + 2x_1 + 1 = 0.
\end{equation}
\begin{equation}
-2x_1 + 1 = 0.
\end{equation}
\begin{equation}
x_1 = \frac{1}{2}.
\end{equation}

\textbf{What this means:} The perturbation pushes the root at $x = -2$ in the positive direction (toward zero).

\textbf{Collecting $O(\epsilon^2)$ terms:}
\begin{equation}
-8 + x_1^2 - 4x_2 + 2x_2 = 0.
\end{equation}
\begin{equation}
-8 + \left(\frac{1}{2}\right)^2 - 2x_2 = 0.
\end{equation}
\begin{equation}
-8 + \frac{1}{4} - 2x_2 = 0.
\end{equation}
\begin{equation}
-2x_2 = 8 - \frac{1}{4} = \frac{31}{4}.
\end{equation}
\begin{equation}
x_2 = -\frac{31}{8}.
\end{equation}

\textbf{What this tells us:} The $O(\epsilon^2)$ correction is negative, partially offsetting the positive $O(\epsilon)$ correction.

\subsubsection{Second Root: The Result}

\textbf{Our second root:}
\begin{equation}
\boxed{x_2(\epsilon) = -2 + \frac{1}{2}\epsilon - \frac{31}{8}\epsilon^2 + O(\epsilon^3)}.
\end{equation}

\section{Finding The Singular Solution}

\subsection{Why We Need To Continue}

\textbf{What we have found so far:} Two roots.

\textbf{What we need:} Three roots (the equation is cubic).

\textbf{Conclusion:} There must be a third root that doesn't fit the regular expansion pattern around $x_0 = 0$ or $x_0 = -2$.

\subsection{Dominant Balance Analysis}

\textbf{The method (from Section 2.2.2):}

When regular expansions fail, we use dominant balance to determine which terms in the equation balance each other as $\epsilon \to 0$, revealing the scaling of the "missing" root.

\textbf{The three steps:}
\begin{enumerate}
\item Assume which terms balance
\item Solve for the implied scaling
\item Check consistency
\end{enumerate}

\subsubsection{The Original Equation Structure}

Our equation:
\begin{equation}
\epsilon^2 x^3 + x^2 + 2x + \epsilon = 0.
\end{equation}

\textbf{Four terms, four possible sizes:}
\begin{itemize}
\item $\epsilon^2 x^3$: depends on both $\epsilon$ and $x$
\item $x^2$: depends only on $x$
\item $2x$: depends only on $x$
\item $\epsilon$: depends only on $\epsilon$
\end{itemize}

\subsubsection{Scenario 1: Large Negative $x$}

\textbf{Hypothesis:} The missing root satisfies $|x| \gg 1$ as $\epsilon \to 0$.

\textbf{Why we consider this:} If $x$ is large, different terms might dominate.

\textbf{Size estimates for $x \to -\infty$:}
\begin{itemize}
\item $\epsilon^2 x^3 \sim -\epsilon^2|x|^3$ (large, negative)
\item $x^2 \sim |x|^2$ (large, positive)
\item $2x \sim -2|x|$ (large, negative)
\item $\epsilon$ (small, positive)
\end{itemize}

\textbf{Which terms could balance?}

\paragraph{Balance Attempt 1: $\epsilon^2 x^3 \sim x^2$}

\textbf{The balance equation:}
\begin{equation}
\epsilon^2 x^3 \sim x^2 \implies \epsilon^2 x \sim 1 \implies x \sim \frac{1}{\epsilon^2}.
\end{equation}

\textbf{Check other terms:}
\begin{itemize}
\item $2x \sim \frac{2}{\epsilon^2}$: This is $O(1/\epsilon^2)$, same size as $x^2$
\item $\epsilon$: This is $O(\epsilon)$, much smaller
\end{itemize}

\textbf{Problem:} The terms $x^2$ and $2x$ are both $O(1/\epsilon^2)$, so they would both be important. But they have the same sign structure for large $x$, so we need to check more carefully.

For $x = c/\epsilon^2$ with $c > 0$:
\begin{itemize}
\item $\epsilon^2 x^3 = \epsilon^2 \cdot \frac{c^3}{\epsilon^6} = \frac{c^3}{\epsilon^4}$
\item $x^2 = \frac{c^2}{\epsilon^4}$
\item $2x = \frac{2c}{\epsilon^2}$
\end{itemize}

\textbf{Observation:} The first two terms are $O(1/\epsilon^4)$ while $2x$ is only $O(1/\epsilon^2)$.

\textbf{But wait:} Let's check if they can balance:
\begin{equation}
\frac{c^3}{\epsilon^4} + \frac{c^2}{\epsilon^4} \sim 0 \implies c^3 + c^2 = 0 \implies c^2(c + 1) = 0.
\end{equation}

This gives $c = 0$ (not large) or $c = -1$ (contradicts $c > 0$).

\textbf{Conclusion:} This balance doesn't work cleanly.

\paragraph{Balance Attempt 2: $\epsilon^2 x^3 \sim 2x$}

\textbf{The balance equation:}
\begin{equation}
\epsilon^2 x^3 \sim 2x \implies \epsilon^2 x^2 \sim 2 \implies x^2 \sim \frac{2}{\epsilon^2} \implies x \sim \pm\frac{\sqrt{2}}{\epsilon}.
\end{equation}

\textbf{Check other terms with $x \sim -\sqrt{2}/\epsilon$ (choosing negative):}
\begin{itemize}
\item $\epsilon^2 x^3 \sim \epsilon^2 \cdot \frac{-2\sqrt{2}}{\epsilon^3} = \frac{-2\sqrt{2}}{\epsilon}$
\item $2x \sim \frac{-2\sqrt{2}}{\epsilon}$: Same order! $\checkmark$
\item $x^2 \sim \frac{2}{\epsilon^2}$: Larger by factor $1/\epsilon$
\item $\epsilon$: Much smaller
\end{itemize}

\textbf{Problem:} The $x^2$ term is larger than our balanced terms, so it can't be neglected.

\textbf{Conclusion:} This balance doesn't work either.

\paragraph{Balance Attempt 3: $x^2 \sim 2x$}

\textbf{The balance equation:}
\begin{equation}
x^2 \sim 2x \implies x \sim 2.
\end{equation}

\textbf{Why this is wrong:} $x = O(1)$ doesn't give us a singular solution - we'd just get the regular solutions we already found.

\paragraph{Balance Attempt 4: $x^2 \sim \epsilon$}

\textbf{The balance equation:}
\begin{equation}
x^2 \sim \epsilon \implies x \sim \pm\sqrt{\epsilon}.
\end{equation}

\textbf{Check other terms with $x \sim \sqrt{\epsilon}$:}
\begin{itemize}
\item $\epsilon^2 x^3 \sim \epsilon^2 \cdot \epsilon^{3/2} = \epsilon^{7/2}$: Much smaller
\item $x^2 \sim \epsilon$: Balanced $\checkmark$
\item $2x \sim 2\sqrt{\epsilon}$: Larger by factor $1/\sqrt{\epsilon}$
\item $\epsilon$: Balanced $\checkmark$
\end{itemize}

\textbf{Problem:} The $2x$ term dominates, so this balance fails.

\paragraph{Balance Attempt 5: $2x \sim \epsilon$}

\textbf{The balance equation:}
\begin{equation}
2x \sim \epsilon \implies x \sim \frac{\epsilon}{2}.
\end{equation}

\textbf{Check other terms:}
\begin{itemize}
\item $\epsilon^2 x^3 \sim \epsilon^2 \cdot \epsilon^3 = \epsilon^5$: Much smaller
\item $x^2 \sim \epsilon^2$: Smaller by factor $\epsilon$
\item $2x \sim \epsilon$: Balanced $\checkmark$
\item $\epsilon$: Balanced $\checkmark$
\end{itemize}

\textbf{Balancing terms:}
\begin{equation}
x^2 + 2x + \epsilon \approx 0 \implies 2x + \epsilon \approx 0 \implies x \approx -\frac{\epsilon}{2}.
\end{equation}

\textbf{This is consistent!} The subdominant term $x^2 \sim \epsilon^2$ is indeed much smaller.

\subsubsection{Constructing The Singular Solution}

\textbf{Our ansatz (motivated by dominant balance):}
\begin{equation}
x = -\frac{\epsilon}{2} + x_1\epsilon^2 + x_2\epsilon^3 + \cdots
\end{equation}

\textbf{Why this form:}
\begin{itemize}
\item The leading term is $-\epsilon/2$ from dominant balance
\item We don't include an $O(1)$ term because that would change the balance
\item The next correction should be $O(\epsilon^2)$ (the size of the neglected $x^2$ term)
\end{itemize}

\subsubsection{Substituting The Ansatz}

Into equation (1):
\begin{equation}
\epsilon^2\left(-\frac{\epsilon}{2} + x_1\epsilon^2\right)^3 + \left(-\frac{\epsilon}{2} + x_1\epsilon^2\right)^2 + 2\left(-\frac{\epsilon}{2} + x_1\epsilon^2\right) + \epsilon = 0.
\end{equation}

\textbf{Cubic term:}
\begin{equation}
\epsilon^2 \cdot \left(-\frac{\epsilon}{2}\right)^3 + \cdots = \epsilon^2 \cdot \frac{-\epsilon^3}{8} + \cdots = -\frac{\epsilon^5}{8} + O(\epsilon^6).
\end{equation}

\textbf{Why we can ignore:} This is $O(\epsilon^5)$, far smaller than $\epsilon^2$.

\textbf{Quadratic term:}
\begin{align}
\left(-\frac{\epsilon}{2} + x_1\epsilon^2\right)^2 &= \frac{\epsilon^2}{4} - 2 \cdot \frac{\epsilon}{2} \cdot x_1\epsilon^2 + x_1^2\epsilon^4 \\
&= \frac{\epsilon^2}{4} - x_1\epsilon^3 + O(\epsilon^4).
\end{align}

\textbf{Linear term:}
\begin{equation}
2\left(-\frac{\epsilon}{2} + x_1\epsilon^2\right) = -\epsilon + 2x_1\epsilon^2.
\end{equation}

\textbf{Collecting everything:}
\begin{equation}
\frac{\epsilon^2}{4} + 2x_1\epsilon^2 - \epsilon + \epsilon + O(\epsilon^3) = 0.
\end{equation}

\textbf{Simplifying:}
\begin{equation}
\left(\frac{1}{4} + 2x_1\right)\epsilon^2 + O(\epsilon^3) = 0.
\end{equation}

\textbf{At $O(\epsilon^2)$:}
\begin{equation}
\frac{1}{4} + 2x_1 = 0 \implies x_1 = -\frac{1}{8}.
\end{equation}

\subsubsection{Third Root: The Result}

\textbf{Our third root:}
\begin{equation}
\boxed{x_3(\epsilon) = -\frac{1}{2}\epsilon - \frac{1}{8}\epsilon^2 + O(\epsilon^3)}.
\end{equation}

\textbf{Wait - this is identical to $x_1(\epsilon)$!}

\subsection{Resolution: The Roots Are Distinct But Close}

\textbf{What happened:} Both the regular expansion near $x_0 = 0$ and the singular balance $2x \sim \epsilon$ give the same two-term expansion!

\textbf{Why this makes sense:} For $x = O(\epsilon)$, the distinction between "starting from 0" and "being determined by balance" becomes subtle. The root is actually evolving continuously from $x = 0$ as $\epsilon$ increases from 0.

\textbf{The actual situation:} We have only found \textbf{two distinct roots} to two-term accuracy:
\begin{enumerate}
\item $x_1(\epsilon) = -\frac{1}{2}\epsilon - \frac{1}{8}\epsilon^2 + O(\epsilon^3)$
\item $x_2(\epsilon) = -2 + \frac{1}{2}\epsilon - \frac{31}{8}\epsilon^2 + O(\epsilon^3)$
\end{enumerate}

\textbf{Where is the third root?}

\subsection{Finding The True Third Root}

\textbf{Reconsider the structure:} The equation can be rewritten as:
\begin{equation}
\epsilon^2 x^3 = -x^2 - 2x - \epsilon = -(x^2 + 2x + \epsilon).
\end{equation}

\textbf{For very large $|x|$:} If $x \to -\infty$, then $\epsilon^2x^3 \to -\infty$ (large negative), while the RHS $\to +\infty$ (large positive). So there must be a balance.

\textbf{New attempt: $\epsilon^2 x^3 + x^2 \sim 0$}

This gives:
\begin{equation}
x^2(\epsilon^2 x + 1) \sim 0 \implies x \sim -\frac{1}{\epsilon^2}.
\end{equation}

\textbf{Check:} For $x = -1/\epsilon^2$:
\begin{itemize}
\item $\epsilon^2 x^3 = \epsilon^2 \cdot \frac{-1}{\epsilon^6} = -\frac{1}{\epsilon^4}$
\item $x^2 = \frac{1}{\epsilon^4}$: Same magnitude! $\checkmark$
\item $2x = -\frac{2}{\epsilon^2}$: Smaller by factor $\epsilon^2$
\item $\epsilon$: Much smaller
\end{itemize}

\textbf{The balance works!}

\subsubsection{Expansion For The Large Root}

\textbf{Ansatz:}
\begin{equation}
x = -\frac{1}{\epsilon^2} + \frac{x_1}{\epsilon} + x_0 + x_1'\epsilon + \cdots
\end{equation}

\textbf{Why we include $1/\epsilon$ term:} The neglected $2x$ term is $O(1/\epsilon^2)$ when $x = O(1/\epsilon^2)$, suggesting corrections at intermediate powers.

Actually, let's be more systematic. Write:
\begin{equation}
x = -\frac{1}{\epsilon^2} + y,
\end{equation}
where $y = o(1/\epsilon^2)$ as $\epsilon \to 0$.

\textbf{Substituting:}
\begin{multline}
\epsilon^2\left(-\frac{1}{\epsilon^2} + y\right)^3 + \left(-\frac{1}{\epsilon^2} + y\right)^2 + 2\left(-\frac{1}{\epsilon^2} + y\right) + \epsilon = 0.
\end{multline}

\textbf{Expanding the cubic:}
\begin{align}
\left(-\frac{1}{\epsilon^2} + y\right)^3 &= -\frac{1}{\epsilon^6} + 3 \cdot \frac{1}{\epsilon^4} \cdot y - 3 \cdot \frac{1}{\epsilon^2} \cdot y^2 + y^3.
\end{align}

So:
\begin{equation}
\epsilon^2 x^3 = -\frac{1}{\epsilon^4} + \frac{3y}{\epsilon^2} - 3y^2 + \epsilon^2 y^3.
\end{equation}

\textbf{The quadratic:}
\begin{equation}
x^2 = \frac{1}{\epsilon^4} - \frac{2y}{\epsilon^2} + y^2.
\end{equation}

\textbf{The linear:}
\begin{equation}
2x = -\frac{2}{\epsilon^2} + 2y.
\end{equation}

\textbf{Full equation:}
\begin{equation}
\left[-\frac{1}{\epsilon^4} + \frac{3y}{\epsilon^2}\right] + \left[\frac{1}{\epsilon^4} - \frac{2y}{\epsilon^2}\right] + \left[-\frac{2}{\epsilon^2} + 2y\right] + \epsilon + \text{higher order} = 0.
\end{equation}

\textbf{At $O(1/\epsilon^4)$:} $-1 + 1 = 0$ $\checkmark$

\textbf{At $O(1/\epsilon^2)$:}
\begin{equation}
3y - 2y - 2 = 0 \implies y = 2.
\end{equation}

\textbf{So to leading order:}
\begin{equation}
x = -\frac{1}{\epsilon^2} + 2 + O(\epsilon).
\end{equation}

\textbf{For the next term,} substitute $y = 2 + z$ where $z = O(\epsilon)$:
\begin{equation}
\frac{y}{\epsilon^2} - 2 + 2y + \epsilon - 3y^2 + O(\epsilon^2) = 0.
\end{equation}

This becomes intricate, but the key point is established.

\subsubsection{Third Root: Final Form}

\begin{equation}
\boxed{x_3(\epsilon) = -\frac{1}{\epsilon^2} + 2 + O(\epsilon)}.
\end{equation}

\section{Final Answer: All Three Roots}

\textbf{The complete solution to Problem 2:}

\begin{align}
\boxed{x_1(\epsilon)} &= \boxed{-\frac{1}{2}\epsilon - \frac{1}{8}\epsilon^2 + O(\epsilon^3)} \\[0.5em]
\boxed{x_2(\epsilon)} &= \boxed{-2 + \frac{1}{2}\epsilon - \frac{31}{8}\epsilon^2 + O(\epsilon^3)} \\[0.5em]
\boxed{x_3(\epsilon)} &= \boxed{-\frac{1}{\epsilon^2} + 2 + O(\epsilon)}
\end{align}

\textbf{Summary:}
\begin{itemize}
\item Two roots are $O(\epsilon)$ and $O(1)$ respectively - regular solutions
\item One root is $O(1/\epsilon^2)$ - a singular solution that escapes to $-\infty$ as $\epsilon \to 0$
\end{itemize}

\end{document}
