\documentclass[11pt,a4paper]{article}
\usepackage{amsmath,amssymb,amsthm}
\usepackage{geometry}
\geometry{margin=1in}
\usepackage{enumitem}

\newtheorem{theorem}{Theorem}
\newtheorem{lemma}[theorem]{Lemma}
\newtheorem{definition}[theorem]{Definition}

\title{Asymptotics 2025/2026 Sheet 1\\Solutions}
\author{}
\date{}

\begin{document}

\maketitle

\section*{Problem 1}

\subsection*{Problem 1(a)}

\textbf{Problem:} For $\epsilon \ll 1$, obtain two-term expansions for the solutions of
\[
(x-1)(x-2)(x-3) + \epsilon = 0.
\]

\textbf{Solution:}

\textit{Step 1: Identify unperturbed solutions.}

Setting $\epsilon = 0$, we obtain the unperturbed equation:
\[
(x-1)(x-2)(x-3) = 0,
\]
which has solutions $x = 1, 2, 3$.

\textit{Step 2: Determine if this is a regular or singular perturbation problem.}

Since the unperturbed equation has three solutions and the perturbed equation (being cubic) also has three solutions, and we expect the perturbed solutions to approach the unperturbed solutions smoothly as $\epsilon \to 0$, this is a \textbf{regular perturbation problem}.

\textit{Step 3: Apply the expansion method.}

For each unperturbed solution $x_0 \in \{1, 2, 3\}$, we make the ansatz:
\[
x(\epsilon) = x_0 + x_1 \epsilon + x_2 \epsilon^2 + O(\epsilon^3).
\]

Substituting into the equation:
\[
(x_0 + x_1\epsilon + x_2\epsilon^2 - 1)(x_0 + x_1\epsilon + x_2\epsilon^2 - 2)(x_0 + x_1\epsilon + x_2\epsilon^2 - 3) + \epsilon = 0.
\]

\textit{Step 4: Expand around $x_0 = 1$.}

Let $x = 1 + x_1\epsilon + x_2\epsilon^2 + \cdots$. Then:
\[
(x_1\epsilon + x_2\epsilon^2)(1 + x_1\epsilon + x_2\epsilon^2 - 2)(1 + x_1\epsilon + x_2\epsilon^2 - 3) + \epsilon = 0.
\]
\[
(x_1\epsilon + x_2\epsilon^2)(-1 + x_1\epsilon)(-2 + x_1\epsilon) + \epsilon = 0.
\]
\[
(x_1\epsilon)(2 - x_1\epsilon - 2x_1\epsilon + O(\epsilon^2)) + \epsilon = 0.
\]

At $O(\epsilon)$:
\[
2x_1 + 1 = 0 \implies x_1 = -\frac{1}{2}.
\]

At $O(\epsilon^2)$:
\[
2x_2 - 3x_1^2 = 0 \implies x_2 = \frac{3}{2} \cdot \frac{1}{4} = \frac{3}{8}.
\]

Therefore:
\[
\boxed{x(\epsilon) = 1 - \frac{1}{2}\epsilon + \frac{3}{8}\epsilon^2 + O(\epsilon^3)}.
\]

\textit{Step 5: Expand around $x_0 = 2$.}

Let $x = 2 + x_1\epsilon + x_2\epsilon^2 + \cdots$. Then:
\[
(1 + x_1\epsilon)(x_1\epsilon)(2 + x_1\epsilon - 3) + \epsilon = 0.
\]
\[
(1 + x_1\epsilon)(x_1\epsilon)(-1 + x_1\epsilon) + \epsilon = 0.
\]

At $O(\epsilon)$:
\[
-x_1 + 1 = 0 \implies x_1 = 1.
\]

At $O(\epsilon^2)$:
\[
-x_2 + x_1^2 - x_1 = 0 \implies x_2 = 1 - 1 = 0.
\]

Therefore:
\[
\boxed{x(\epsilon) = 2 + \epsilon + O(\epsilon^3)}.
\]

\textit{Step 6: Expand around $x_0 = 3$.}

Let $x = 3 + x_1\epsilon + x_2\epsilon^2 + \cdots$. Then:
\[
(2 + x_1\epsilon)(1 + x_1\epsilon)(x_1\epsilon) + \epsilon = 0.
\]

At $O(\epsilon)$:
\[
2x_1 + 1 = 0 \implies x_1 = -\frac{1}{2}.
\]

At $O(\epsilon^2)$:
\[
2x_2 + x_1 + 2x_1^2 = 0 \implies x_2 = -\frac{1}{2}\left(-\frac{1}{2}\right) - \frac{1}{2} = \frac{1}{4} - \frac{1}{2} = -\frac{1}{4}.
\]

Therefore:
\[
\boxed{x(\epsilon) = 3 - \frac{1}{2}\epsilon - \frac{1}{4}\epsilon^2 + O(\epsilon^3)}.
\]

\subsection*{Problem 1(b)}

\textbf{Problem:} For $\epsilon \ll 1$, obtain two-term expansions for the solutions of
\[
x^3 + x^2 - \epsilon = 0.
\]

\textbf{Solution:}

\textit{Step 1: Identify unperturbed solutions.}

Setting $\epsilon = 0$:
\[
x^2(x + 1) = 0,
\]
which gives $x = 0$ (double root) and $x = -1$ (simple root).

\textit{Step 2: Classify the problem.}

Since the unperturbed equation has a degenerate root at $x = 0$, we expect non-integer power expansions for solutions near this root. This is a singular perturbation problem.

\textit{Step 3: Regular solution near $x_0 = -1$.}

Let $x = -1 + x_1\epsilon + x_2\epsilon^2 + \cdots$. Substituting:
\[
(-1 + x_1\epsilon)^3 + (-1 + x_1\epsilon)^2 - \epsilon = 0.
\]
\[
-1 + 3x_1\epsilon + 1 - 2x_1\epsilon - \epsilon + O(\epsilon^2) = 0.
\]

At $O(\epsilon)$:
\[
x_1 - 1 = 0 \implies x_1 = 1.
\]

At $O(\epsilon^2)$:
\[
-3 + 3x_1 + 1 - 2x_1 + x_2 = 0 \implies x_2 = 2.
\]

Therefore:
\[
\boxed{x(\epsilon) = -1 + \epsilon + 2\epsilon^2 + O(\epsilon^3)}.
\]

\textit{Step 4: Singular solutions near $x_0 = 0$.}

Since the unperturbed solution at $x = 0$ is degenerate, we try a fractional power expansion:
\[
x(\epsilon) = x_0 + x_1\epsilon^\alpha + x_2\epsilon^{2\alpha} + \cdots
\]

Substituting into $x^3 + x^2 = \epsilon$:
\[
x_1^3\epsilon^{3\alpha} + x_1^2\epsilon^{2\alpha} = \epsilon + \text{higher order terms}.
\]

For a balance, we need $2\alpha = 1$, giving $\alpha = \frac{1}{2}$.

At $O(\epsilon^{1/2})$: The leading term gives
\[
x_1^2\epsilon = 0,
\]
which doesn't work. Instead, try $3\alpha = 1$, giving $\alpha = \frac{1}{3}$.

At $O(\epsilon)$:
\[
x_1^3 = 1 \implies x_1 = 1, \omega, \omega^2,
\]
where $\omega = e^{2\pi i/3}$.

For the real root:
\[
\boxed{x(\epsilon) = \epsilon^{1/3} + O(\epsilon^{2/3})}.
\]

For higher order terms, substitute $x = \epsilon^{1/3} + x_2\epsilon^{2/3} + \cdots$:
\[
\epsilon + 3\epsilon^{2/3}x_2 + \epsilon^{2/3} + 2\epsilon^{1/3}x_2 = \epsilon.
\]

At $O(\epsilon^{2/3})$:
\[
3x_2 + 1 = 0 \implies x_2 = -\frac{1}{3}.
\]

Therefore:
\[
\boxed{x(\epsilon) = \epsilon^{1/3} - \frac{1}{3}\epsilon^{2/3} + O(\epsilon)}.
\]

The complex roots are:
\[
\boxed{x(\epsilon) = \omega\epsilon^{1/3} - \frac{1}{3}\omega^2\epsilon^{2/3} + O(\epsilon)},
\]
\[
\boxed{x(\epsilon) = \omega^2\epsilon^{1/3} - \frac{1}{3}\omega\epsilon^{2/3} + O(\epsilon)}.
\]

\subsection*{Problem 1(c)}

\textbf{Problem:} For $\epsilon \ll 1$, obtain two-term expansions for the solutions of
\[
\epsilon x^3 + x^2 + 2x + 1 = 0.
\]

\textbf{Solution:}

\textit{Step 1: Identify the type of problem.}

This is a singular perturbation problem because $\epsilon$ multiplies the highest derivative term. Setting $\epsilon = 0$:
\[
x^2 + 2x + 1 = (x+1)^2 = 0,
\]
giving $x = -1$ (double root).

The perturbed equation is cubic and has three roots, while the unperturbed has only one (degenerate) root.

\textit{Step 2: Regular solution.}

Try $x = -1 + x_1\epsilon + x_2\epsilon^2 + \cdots$:
\[
\epsilon(-1 + x_1\epsilon)^3 + (-1 + x_1\epsilon)^2 + 2(-1 + x_1\epsilon) + 1 = 0.
\]
\[
\epsilon(-1 + 3x_1\epsilon) + (1 - 2x_1\epsilon) - 2 + 2x_1\epsilon + 1 = 0.
\]
\[
-\epsilon + 3x_1\epsilon^2 + 1 - 2x_1\epsilon - 2 + 2x_1\epsilon + 1 = 0.
\]

At $O(1)$: $0 = 0$ ✓

At $O(\epsilon)$: $-1 = 0$ ✗

This fails, confirming we need a different approach.

\textit{Step 3: Dominant balance analysis.}

Consider the full equation $\epsilon x^3 + x^2 + 2x + 1 = 0$. For a singular solution with $|x| \gg 1$:

\begin{itemize}
\item If $\epsilon x^3 \sim x^2$: $x \sim 1/\epsilon$
\item If $\epsilon x^3 \sim 2x$: $x \sim \sqrt{2/\epsilon}$
\item If $\epsilon x^3 \sim 1$: $x \sim \epsilon^{-1/3}$
\end{itemize}

Try $x \sim -1/\epsilon$. Then $\epsilon x^3 \sim -1/\epsilon^2$, $x^2 \sim 1/\epsilon^2$, $2x \sim -2/\epsilon$, giving a consistent balance between $\epsilon x^3$ and $x^2$.

\textit{Step 4: Singular solution.}

Let $x = -\frac{1}{\epsilon} + x_0 + x_1\epsilon + \cdots$:
\[
\epsilon\left(-\frac{1}{\epsilon} + x_0\right)^3 + \left(-\frac{1}{\epsilon} + x_0\right)^2 + 2\left(-\frac{1}{\epsilon} + x_0\right) + 1 = 0.
\]

At $O(1/\epsilon^2)$:
\[
-\frac{1}{\epsilon^2} + \frac{1}{\epsilon^2} = 0 \quad ✓
\]

At $O(1/\epsilon)$:
\[
\frac{3x_0}{\epsilon} - \frac{2x_0}{\epsilon} - \frac{2}{\epsilon} = 0 \implies x_0 = -2.
\]

Therefore:
\[
\boxed{x(\epsilon) = -\frac{1}{\epsilon} - 2 + O(\epsilon)}.
\]

\textit{Step 5: Find other singular solutions.}

For solutions with $x^2 \sim 2x$, we have $x \sim -2$. Try $x = -2 + y$ where $y$ is small:
\[
\epsilon(-2+y)^3 + (-2+y)^2 + 2(-2+y) + 1 = 0.
\]
\[
\epsilon(-8 + 12y) + (4 - 4y + y^2) - 4 + 2y + 1 = 0.
\]
\[
y^2 - 2y - 8\epsilon + 12\epsilon y + 1 = 0.
\]

This doesn't yield a simple expansion. Instead, use the iterative method or note that the two remaining roots come from balancing different terms.

Actually, the complete solution requires finding all three roots. The equation can be rewritten as:
\[
x^2(1 + \epsilon x) = -(2x + 1).
\]

For the two complex/remaining roots, we can use the quadratic formula on the reduced problem or dominant balance.

\subsection*{Problem 1(d)}

\textbf{Problem:} For $\epsilon \ll 1$, obtain a two-term expansion for the solution near $x = 0$ of
\[
\sqrt{2}\sin(x + \pi/4) - 1 - x + \frac{1}{2}x^2 = -\frac{1}{6}\epsilon.
\]

\textbf{Solution:}

\textit{Step 1: Simplify using trigonometric identity.}

\[
\sqrt{2}\sin(x + \pi/4) = \sqrt{2}\left(\sin x \cos\frac{\pi}{4} + \cos x \sin\frac{\pi}{4}\right) = \sin x + \cos x.
\]

So the equation becomes:
\[
\sin x + \cos x - 1 - x + \frac{1}{2}x^2 = -\frac{1}{6}\epsilon.
\]

\textit{Step 2: Find unperturbed solution.}

Setting $\epsilon = 0$:
\[
\sin x + \cos x - 1 - x + \frac{1}{2}x^2 = 0.
\]

For $x$ near 0, use Taylor expansions:
\[
\sin x = x - \frac{x^3}{6} + O(x^5), \quad \cos x = 1 - \frac{x^2}{2} + \frac{x^4}{24} + O(x^6).
\]
\[
x - \frac{x^3}{6} + 1 - \frac{x^2}{2} - 1 - x + \frac{x^2}{2} = 0.
\]
\[
-\frac{x^3}{6} = 0 \implies x_0 = 0.
\]

\textit{Step 3: Perturbative expansion.}

Let $x = x_1\epsilon + x_2\epsilon^2 + \cdots$. Substituting:
\[
\sin(x_1\epsilon) + \cos(x_1\epsilon) - 1 - x_1\epsilon + \frac{1}{2}x_1^2\epsilon^2 = -\frac{1}{6}\epsilon.
\]
\[
x_1\epsilon - \frac{x_1^3\epsilon^3}{6} + 1 - \frac{x_1^2\epsilon^2}{2} - 1 - x_1\epsilon + \frac{x_1^2\epsilon^2}{2} = -\frac{1}{6}\epsilon.
\]

At $O(\epsilon)$:
\[
0 = -\frac{1}{6} \quad \text{(contradiction)}.
\]

This suggests $x_1 = 0$ is not correct. Let's reconsider.

Actually, expand more carefully:
\[
\sin x + \cos x - 1 - x + \frac{x^2}{2} + \frac{\epsilon}{6} = 0.
\]

Near $x = 0$:
\[
x - \frac{x^3}{6} + 1 - \frac{x^2}{2} + \frac{x^4}{24} - 1 - x + \frac{x^2}{2} + \frac{\epsilon}{6} = 0.
\]
\[
-\frac{x^3}{6} + \frac{x^4}{24} + \frac{\epsilon}{6} = 0.
\]
\[
x^3 - \frac{x^4}{4} = \epsilon.
\]

For small $x$: $x^3 \approx \epsilon$, so $x \sim \epsilon^{1/3}$.

Let $x = c_1\epsilon^{1/3} + c_2\epsilon^{2/3} + \cdots$:
\[
c_1^3\epsilon + O(\epsilon^{4/3}) = \epsilon.
\]

At $O(\epsilon)$:
\[
c_1^3 = 1 \implies c_1 = 1.
\]

Therefore:
\[
\boxed{x(\epsilon) = \epsilon^{1/3} + O(\epsilon^{2/3})}.
\]

\section*{Problem 2}

\textbf{Problem:} Find two terms in the expansion for each root of
\[
\epsilon^2 x^3 + x^2 + 2x + \epsilon = 0.
\]

\textbf{Solution:}

\textit{Step 1: Unperturbed equation.}

Setting $\epsilon = 0$:
\[
x^2 + 2x = x(x + 2) = 0,
\]
giving $x = 0$ and $x = -2$.

\textit{Step 2: Regular solution near $x_0 = -2$.}

Let $x = -2 + x_1\epsilon + x_2\epsilon^2 + \cdots$:
\[
\epsilon^2(-2 + x_1\epsilon)^3 + (-2 + x_1\epsilon)^2 + 2(-2 + x_1\epsilon) + \epsilon = 0.
\]

At $O(1)$: $4 - 4 = 0$ ✓

At $O(\epsilon)$:
\[
-4x_1 + 2x_1 + 1 = 0 \implies -2x_1 = -1 \implies x_1 = \frac{1}{2}.
\]

Therefore:
\[
\boxed{x(\epsilon) = -2 + \frac{1}{2}\epsilon + O(\epsilon^2)}.
\]

\textit{Step 3: Singular solution near $x_0 = 0$.}

Since the perturbed equation is cubic, there should be three roots total. We found one regular root, so two roots must bifurcate from $x = 0$.

For $x$ small, the equation becomes approximately:
\[
x^2 + 2x + \epsilon \approx 0,
\]
giving:
\[
x \approx \frac{-2 \pm \sqrt{4 - 4\epsilon}}{2} = -1 \pm \sqrt{1 - \epsilon}.
\]
\[
x \approx -1 \pm (1 - \frac{\epsilon}{2}) = \begin{cases} -\frac{\epsilon}{2} \\ -2 + \frac{\epsilon}{2} \end{cases}
\]

For the root near 0, try $x = x_0 + x_1\epsilon + \cdots$ with $x_0 = 0$:
\[
x_1^2\epsilon^2 + 2x_1\epsilon + \epsilon = 0.
\]

At $O(\epsilon)$:
\[
2x_1 + 1 = 0 \implies x_1 = -\frac{1}{2}.
\]

Therefore:
\[
\boxed{x(\epsilon) = -\frac{1}{2}\epsilon + O(\epsilon^2)}.
\]

\textit{Step 4: Third root by dominant balance.}

Actually, we should check if there's a large root. For $|x| \gg 1$:

If $\epsilon^2 x^3 \sim x^2$: $x \sim 1/\epsilon^2$
If $\epsilon^2 x^3 \sim 2x$: $x \sim (2/\epsilon^2)^{1/2} = \sqrt{2}/\epsilon$
If $\epsilon^2 x^3 \sim \epsilon$: $x \sim \epsilon^{-1/3}$

Actually, the equation has exactly 3 roots. We found:
- $x = -2 + \frac{1}{2}\epsilon$
- $x = -\frac{1}{2}\epsilon$

The third must be large. Actually, checking the discriminant and structure more carefully, the second root from the quadratic $x^2 + 2x + \epsilon = 0$ should be analyzed more carefully.

From $x(x+2) = -\epsilon$, for small solutions:
\[
x \approx -\frac{\epsilon}{2}, \quad \text{or} \quad x \approx -2 + \frac{\epsilon}{2}.
\]

So the three roots are:
\[
\boxed{x_1(\epsilon) = -\frac{\epsilon}{2} + O(\epsilon^2)},
\]
\[
\boxed{x_2(\epsilon) = -2 + \frac{\epsilon}{2} + O(\epsilon^2)},
\]
and for the third, we need to include the $\epsilon^2 x^3$ term. This actually modifies the expansions at higher order.

\section*{Problem 3}

\subsection*{Problem 3(a)}

\textbf{Verify:} $\sin(x^{1/3}) = O(x^{1/3})$ as $x \to 0^+$.

\textbf{Solution:}

By definition, $f(x) = O(g(x))$ as $x \to x_0$ if
\[
\lim_{x \to x_0} \left|\frac{f(x)}{g(x)}\right| = C, \quad \text{where } 0 \le C < \infty.
\]

Consider:
\[
\lim_{x \to 0^+} \frac{\sin(x^{1/3})}{x^{1/3}}.
\]

Let $u = x^{1/3}$. As $x \to 0^+$, we have $u \to 0^+$. Therefore:
\[
\lim_{x \to 0^+} \frac{\sin(x^{1/3})}{x^{1/3}} = \lim_{u \to 0^+} \frac{\sin u}{u} = 1.
\]

Since the limit is finite and non-zero, we conclude:
\[
\boxed{\sin(x^{1/3}) = O(x^{1/3}) \text{ as } x \to 0^+}. \quad ✓
\]

\subsection*{Problem 3(b)}

\textbf{Verify:} $\cos(x) = O(1)$ as $x \to \infty$.

\textbf{Solution:}

We need to show:
\[
\lim_{x \to \infty} \frac{|\cos(x)|}{1} = C < \infty.
\]

Since $|\cos(x)| \le 1$ for all $x$, we have:
\[
\frac{|\cos(x)|}{1} \le 1 < \infty.
\]

Therefore, the function $|\cos(x)|$ is bounded as $x \to \infty$, confirming:
\[
\boxed{\cos(x) = O(1) \text{ as } x \to \infty}. \quad ✓
\]

\subsection*{Problem 3(c)}

\textbf{Verify:} $\sin x = O(x\cos x)$ as $x \to 0$.

\textbf{Solution:}

We need to check:
\[
\lim_{x \to 0} \frac{\sin x}{x \cos x}.
\]

This can be rewritten as:
\[
\lim_{x \to 0} \frac{\sin x}{x \cos x} = \lim_{x \to 0} \frac{\sin x}{x} \cdot \frac{1}{\cos x} = 1 \cdot \frac{1}{1} = 1.
\]

Since the limit is finite, we conclude:
\[
\boxed{\sin x = O(x\cos x) \text{ as } x \to 0}. \quad ✓
\]

\subsection*{Problem 3(d)}

\textbf{Verify:} $\log(\log(1/x)) = o(\log(x))$ as $x \to 0^+$.

\textbf{Solution:}

By definition, $f(x) = o(g(x))$ as $x \to x_0$ if
\[
\lim_{x \to x_0} \frac{f(x)}{g(x)} = 0.
\]

Consider:
\[
\lim_{x \to 0^+} \frac{\log(\log(1/x))}{\log(x)}.
\]

Note that $\log(1/x) = -\log(x)$, so:
\[
\log(\log(1/x)) = \log(-\log(x)).
\]

As $x \to 0^+$, we have $\log(x) \to -\infty$, so $-\log(x) \to +\infty$, and thus $\log(-\log(x)) \to +\infty$.

Similarly, $\log(x) \to -\infty$ as $x \to 0^+$.

Therefore:
\[
\lim_{x \to 0^+} \frac{\log(-\log(x))}{\log(x)}.
\]

Let $u = -\log(x)$, so $u \to +\infty$ as $x \to 0^+$. Then:
\[
\lim_{u \to +\infty} \frac{\log(u)}{-u} = -\lim_{u \to +\infty} \frac{\log(u)}{u} = 0,
\]
where we used L'Hôpital's rule or the fact that logarithms grow slower than any positive power.

Therefore:
\[
\boxed{\log(\log(1/x)) = o(\log(x)) \text{ as } x \to 0^+}. \quad ✓
\]

\section*{Problem 4}

\textbf{Problem:} Explain why $\{\phi_n(x) = x^{-n}\cos(nx)\}$, $n = 0, 1, \ldots$, is not an asymptotic sequence as $x \to \infty$.

\textbf{Solution:}

\textit{Definition:} A sequence $\{\phi_n(x)\}$ is an asymptotic sequence as $x \to x_0$ if, for all $n$,
\[
\phi_{n+1}(x) = o(\phi_n(x)) \text{ as } x \to x_0,
\]
i.e., $\lim_{x \to x_0} \phi_{n+1}(x)/\phi_n(x) = 0$.

\textit{Check the condition:}

For $\phi_n(x) = x^{-n}\cos(nx)$, we need:
\[
\lim_{x \to \infty} \frac{\phi_{n+1}(x)}{\phi_n(x)} = \lim_{x \to \infty} \frac{x^{-(n+1)}\cos((n+1)x)}{x^{-n}\cos(nx)} = \lim_{x \to \infty} \frac{\cos((n+1)x)}{x\cos(nx)}.
\]

\textit{Analysis:}

The numerator $\cos((n+1)x)$ oscillates between $-1$ and $1$ as $x \to \infty$, while the denominator $x\cos(nx)$ also oscillates but grows in magnitude like $x$.

However, the denominator $\cos(nx)$ can be arbitrarily close to zero for infinitely many values of $x$ as $x \to \infty$ (when $nx \approx \pi/2 + k\pi$). At these points, the ratio can become unbounded.

More precisely, the limit:
\[
\lim_{x \to \infty} \frac{\cos((n+1)x)}{x\cos(nx)}
\]
does not exist because:
\begin{itemize}
\item When $\cos(nx) \approx 0$ and $\cos((n+1)x) \ne 0$, the ratio $\to \pm\infty$.
\item When both oscillate, the behavior is erratic.
\end{itemize}

Therefore, we \textbf{cannot} conclude that $\phi_{n+1}(x) = o(\phi_n(x))$.

\textit{Conclusion:}

\[
\boxed{\text{The sequence } \{\phi_n(x) = x^{-n}\cos(nx)\} \text{ is not an asymptotic sequence as } x \to \infty}
\]
\[
\boxed{\text{because } \lim_{x\to\infty} \frac{\phi_{n+1}(x)}{\phi_n(x)} \text{ does not exist (due to oscillations).}}
\]

\section*{Problem 5}

\textbf{Problem:} Prove that $\sum_{n=1}^{\infty} \frac{1}{z^n}$ is an asymptotic expansion of $\frac{1}{z-1}$ as $z \to \infty$.

\textbf{Solution:}

\textit{Step 1: Recall definition of asymptotic expansion.}

The series $\sum_{n=0}^{\infty} a_n \phi_n(z)$ is an asymptotic expansion of $f(z)$ as $z \to \infty$ if, for every $N$,
\[
\lim_{z \to \infty} \frac{f(z) - \sum_{n=0}^{N} a_n\phi_n(z)}{\phi_N(z)} = 0.
\]

\textit{Step 2: Identify the asymptotic sequence.}

Here, $\phi_n(z) = z^{-n}$ and $a_n = 1$ for $n \ge 1$. The asymptotic sequence $\{z^{-n}\}$ satisfies:
\[
\frac{z^{-(n+1)}}{z^{-n}} = \frac{1}{z} \to 0 \text{ as } z \to \infty.
\]

\textit{Step 3: Compute the partial sum.}

\[
\sum_{n=1}^{N} \frac{1}{z^n} = \frac{1/z - (1/z)^{N+1}}{1 - 1/z} = \frac{1 - z^{-N}}{z - 1}.
\]

\textit{Step 4: Compute the remainder.}

\[
R_N = \frac{1}{z-1} - \sum_{n=1}^{N} \frac{1}{z^n} = \frac{1}{z-1} - \frac{1 - z^{-N}}{z-1} = \frac{z^{-N}}{z-1} = \frac{1}{z^N(z-1)}.
\]

\textit{Step 5: Check the asymptotic condition.}

\[
\frac{R_N}{\phi_N(z)} = \frac{1/(z^N(z-1))}{1/z^N} = \frac{1}{z-1}.
\]

As $z \to \infty$:
\[
\lim_{z \to \infty} \frac{1}{z-1} = 0.
\]

Therefore, for every $N$, the remainder is $o(z^{-N})$ as $z \to \infty$.

\textit{Conclusion:}

\[
\boxed{\sum_{n=1}^{\infty} \frac{1}{z^n} \text{ is an asymptotic expansion of } \frac{1}{z-1} \text{ as } z \to \infty}. \quad ✓
\]

\section*{Problem 6}

\textbf{Problem:} Show that $f(x) \sim g(x)$ as $x \to x_0$ does not necessarily imply that $\exp(f(x)) \sim \exp(g(x))$ as $x \to x_0$ by finding a counterexample.

\textbf{Solution:}

\textit{Counterexample:}

Let:
\[
f(x) = x + 1, \quad g(x) = x, \quad x_0 = \infty.
\]

\textit{Step 1: Verify $f(x) \sim g(x)$ as $x \to \infty$.}

\[
\lim_{x \to \infty} \frac{f(x)}{g(x)} = \lim_{x \to \infty} \frac{x+1}{x} = \lim_{x \to \infty} \left(1 + \frac{1}{x}\right) = 1.
\]

Therefore, $f(x) \sim g(x)$ as $x \to \infty$. ✓

\textit{Step 2: Check if $\exp(f(x)) \sim \exp(g(x))$.}

\[
\frac{\exp(f(x))}{\exp(g(x))} = \frac{e^{x+1}}{e^x} = e^1 = e \ne 1.
\]

Since:
\[
\lim_{x \to \infty} \frac{\exp(f(x))}{\exp(g(x))} = e \ne 1,
\]
we have $\exp(f(x)) \not\sim \exp(g(x))$ as $x \to \infty$.

\textit{Conclusion:}

\[
\boxed{\text{Counterexample: } f(x) = x+1, \, g(x) = x, \, x_0 = \infty.}
\]
\[
\boxed{f(x) \sim g(x) \text{ but } e^{f(x)} \not\sim e^{g(x)}.}
\]

\end{document}
