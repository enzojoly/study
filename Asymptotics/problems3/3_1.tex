\documentclass[11pt,a4paper]{article}
\usepackage[margin=2.5cm]{geometry}
\usepackage{amsmath,amssymb,amsthm}
\usepackage{enumitem}
\usepackage{xcolor}
\usepackage{mathtools}

% Custom theorem environments
\newtheorem{problem}{Problem}
\newtheorem*{solution}{Solution}

% Custom commands
\newcommand{\dd}{\mathrm{d}}
\newcommand{\R}{\mathbb{R}}
\newcommand{\asym}{\sim}

\title{Asymptotics 2025/2026 \\ Problem Sheet 3 \\ Question 1: Integration by Parts}
\author{Solution with XYZ Methodology}
\date{}

\begin{document}

\maketitle

\begin{problem}
Use integration by parts to obtain the first two terms in the asymptotic expansion of
\[
I(X) = \int_1^\infty e^{-X(t^2+1)} \dd t
\]
as $X \to \infty$.
\end{problem}

\begin{solution}

\section*{Overview and Strategy}

\textbf{What do we see?} We have an integral of the form $\int_a^\infty (\text{function}) \cdot e^{-X(\text{something})} \dd t$ where $X$ is a large parameter.

\textbf{Why is this significant?} This is a \emph{Laplace-type integral} (Section 4.2 of lecture notes). As $X \to \infty$, the exponential factor $e^{-X(t^2+1)}$ decays extremely rapidly, causing the integral to be dominated by behavior near the \emph{lower limit} of integration where the exponent is smallest.

\textbf{What method do we use?} The lecture notes (Section 4.2.1, ``Integration by parts of Laplace integrals'') provide the systematic approach: we must first transform the integral into standard form, then apply repeated integration by parts.

\section*{Step 1: Transformation to Standard Form}

\subsection*{Step 1.1: Identifying the Structure}

\textbf{What do we have?} The exponent is $-X(t^2+1)$, which we write as $-X\phi(t)$ where
\[
\phi(t) = t^2 + 1.
\]

\textbf{Why this decomposition?} The standard form for Laplace integrals (Equation 163 in notes) is
\[
I(x) = \int_a^b f(t) e^{-xt} \dd t.
\]
Our integral has $e^{-X\phi(t)}$ instead of $e^{-Xt}$, so we must use a change of variables to achieve the standard form.

\subsection*{Step 1.2: The Substitution}

\textbf{What substitution do we make?} Following Section 4.2.1, when $\phi'(t) \neq 0$ in the integration interval, we set
\[
\tau = \phi(t) = t^2 + 1.
\]

\textbf{Why this choice?} This substitution transforms $e^{-X\phi(t)}$ into $e^{-X\tau}$, giving us the standard Laplace form. Let's verify $\phi'(t) \neq 0$:
\[
\phi'(t) = 2t.
\]
Since $t \in [1,\infty)$, we have $\phi'(t) = 2t > 0$ throughout the domain. \textbf{Therefore the substitution is valid.}

\subsection*{Step 1.3: Computing the Differential}

\textbf{What do we need?} We need to express $\dd t$ in terms of $\dd\tau$.

\textbf{How do we find it?} From $\tau = t^2 + 1$, we differentiate:
\[
\dd\tau = 2t \dd t \quad \Rightarrow \quad \dd t = \frac{\dd\tau}{2t}.
\]

\textbf{What about $t$ as a function of $\tau$?} Since $\tau = t^2 + 1$, we have
\[
t^2 = \tau - 1 \quad \Rightarrow \quad t = \sqrt{\tau - 1},
\]
where we take the positive root because $t \geq 1$ in our integration domain.

\subsection*{Step 1.4: Transforming the Limits}

\textbf{What are the new limits?}
\begin{itemize}
\item When $t = 1$: $\tau = 1^2 + 1 = 2$.
\item When $t \to \infty$: $\tau = t^2 + 1 \to \infty$.
\end{itemize}

\textbf{Why is this important?} The transformation preserves the infinite upper limit, and we now know the integral starts at $\tau = 2$.

\subsection*{Step 1.5: The Transformed Integral}

\textbf{Putting it all together:}
\begin{align*}
I(X) &= \int_1^\infty e^{-X(t^2+1)} \dd t \\
&= \int_2^\infty e^{-X\tau} \cdot \frac{\dd\tau}{2t} \\
&= \int_2^\infty \frac{1}{2\sqrt{\tau-1}} e^{-X\tau} \dd\tau.
\end{align*}

\textbf{What have we achieved?} We now have the standard form
\[
I(X) = \int_2^\infty f(\tau) e^{-X\tau} \dd\tau
\]
where
\[
f(\tau) = \frac{1}{2\sqrt{\tau-1}} = \frac{1}{2}(\tau-1)^{-1/2}.
\]

\textbf{Why is this form useful?} This is precisely the setup for the integration by parts formula (Equation 165 in notes).

\section*{Step 2: First Integration by Parts}

\subsection*{Step 2.1: Setting Up Integration by Parts}

\textbf{What is our formula?} From calculus, $\int u \dd v = uv - \int v \dd u$.

\textbf{How do we choose $u$ and $\dd v$?} Following the lecture notes method (Section 4.2.1):
\begin{itemize}
\item Let $\dd v = e^{-X\tau} \dd\tau$, so $v = -\frac{1}{X}e^{-X\tau}$.
\item Let $u = f(\tau) = \frac{1}{2\sqrt{\tau-1}}$, so $\dd u = f'(\tau) \dd\tau$.
\end{itemize}

\textbf{Why this assignment?} We want to ``peel off'' the exponential and generate a boundary term that will give us the leading asymptotic behavior.

\subsection*{Step 2.2: Applying the Formula}

\textbf{Computing the parts:}
\begin{align*}
I(X) &= \int_2^\infty f(\tau) e^{-X\tau} \dd\tau \\
&= \left[ f(\tau) \cdot \left(-\frac{e^{-X\tau}}{X}\right) \right]_2^\infty - \int_2^\infty \left(-\frac{e^{-X\tau}}{X}\right) f'(\tau) \dd\tau \\
&= -\frac{1}{X}\left[ f(\tau) e^{-X\tau} \right]_2^\infty + \frac{1}{X}\int_2^\infty f'(\tau) e^{-X\tau} \dd\tau.
\end{align*}

\subsection*{Step 2.3: Evaluating the Boundary Term}

\textbf{What happens at $\tau \to \infty$?}

We need to evaluate $\lim_{\tau \to \infty} f(\tau) e^{-X\tau}$.

\textbf{Analyzing the behavior:}
\begin{itemize}
\item $f(\tau) = \frac{1}{2\sqrt{\tau-1}} \sim \frac{1}{2\sqrt{\tau}}$ as $\tau \to \infty$, which grows like $\tau^{-1/2}$.
\item $e^{-X\tau}$ decays exponentially.
\end{itemize}

\textbf{Why does it vanish?} Exponential decay always dominates polynomial growth, so
\[
\lim_{\tau \to \infty} \frac{1}{2\sqrt{\tau-1}} e^{-X\tau} = 0.
\]

\textbf{What about at $\tau = 2$?}
\[
f(2) = \frac{1}{2\sqrt{2-1}} = \frac{1}{2\sqrt{1}} = \frac{1}{2}.
\]

\textbf{Therefore:}
\[
-\frac{1}{X}\left[ f(\tau) e^{-X\tau} \right]_2^\infty = -\frac{1}{X}(0 - f(2)e^{-2X}) = \frac{1}{X} \cdot \frac{1}{2} \cdot e^{-2X} = \frac{e^{-2X}}{2X}.
\]

\subsection*{Step 2.4: Intermediate Result}

\textbf{What have we found?}
\[
I(X) = \frac{e^{-2X}}{2X} + \frac{1}{X}\int_2^\infty f'(\tau) e^{-X\tau} \dd\tau.
\]

\textbf{What is the significance?}
\begin{itemize}
\item The first term $\frac{e^{-2X}}{2X}$ is our \emph{leading order term}.
\item The integral contains $f'(\tau)$, which will yield the next correction.
\item The factor $\frac{1}{X}$ in front means this integral contributes at lower order.
\end{itemize}

\section*{Step 3: Second Integration by Parts}

\subsection*{Step 3.1: Computing $f'(\tau)$}

\textbf{What is the derivative of $f$?}
\[
f(\tau) = \frac{1}{2}(\tau-1)^{-1/2}.
\]

Using the power rule:
\[
f'(\tau) = \frac{1}{2} \cdot \left(-\frac{1}{2}\right)(\tau-1)^{-3/2} \cdot 1 = -\frac{1}{4}(\tau-1)^{-3/2}.
\]

\textbf{At $\tau = 2$:}
\[
f'(2) = -\frac{1}{4}(2-1)^{-3/2} = -\frac{1}{4}.
\]

\subsection*{Step 3.2: Applying Integration by Parts Again}

\textbf{What do we do?} We apply the same technique to $\int_2^\infty f'(\tau) e^{-X\tau} \dd\tau$:

\begin{align*}
\int_2^\infty f'(\tau) e^{-X\tau} \dd\tau &= \left[ f'(\tau) \cdot \left(-\frac{e^{-X\tau}}{X}\right) \right]_2^\infty + \frac{1}{X}\int_2^\infty f''(\tau) e^{-X\tau} \dd\tau \\
&= -\frac{1}{X}\left[ f'(\tau) e^{-X\tau} \right]_2^\infty + \frac{1}{X}\int_2^\infty f''(\tau) e^{-X\tau} \dd\tau.
\end{align*}

\subsection*{Step 3.3: Evaluating the Boundary Term}

\textbf{At $\tau \to \infty$:} Again, exponential decay dominates, so the term vanishes.

\textbf{At $\tau = 2$:}
\[
-\frac{1}{X}\left[ f'(\tau) e^{-X\tau} \right]_2^\infty = -\frac{1}{X}(0 - f'(2)e^{-2X}) = \frac{1}{X} \cdot \left(-\frac{1}{4}\right) \cdot e^{-2X} = -\frac{e^{-2X}}{4X}.
\]

\subsection*{Step 3.4: Substituting Back}

\textbf{Combining results:}
\begin{align*}
I(X) &= \frac{e^{-2X}}{2X} + \frac{1}{X}\left[ -\frac{e^{-2X}}{4X} + \frac{1}{X}\int_2^\infty f''(\tau) e^{-X\tau} \dd\tau \right] \\
&= \frac{e^{-2X}}{2X} - \frac{e^{-2X}}{4X^2} + \frac{1}{X^2}\int_2^\infty f''(\tau) e^{-X\tau} \dd\tau.
\end{align*}

\section*{Step 4: Asymptotic Interpretation}

\subsection*{Step 4.1: Understanding the Remainder}

\textbf{What about the remaining integral?} The lecture notes (proof of Equation 167) show that
\[
\int_2^\infty f''(\tau) e^{-X\tau} \dd\tau = o\left(\frac{e^{-2X}}{X^2}\right) \text{ as } X \to \infty.
\]

\textbf{Why?} The integral can be bounded, and the exponential decay at $\tau = 2$ determines its order.

\subsection*{Step 4.2: The Asymptotic Expansion}

\textbf{Conclusion:} The first two terms in the asymptotic expansion are:
\[
\boxed{I(X) \sim \frac{e^{-2X}}{2X} - \frac{e^{-2X}}{4X^2} \text{ as } X \to \infty}
\]

\textbf{Alternative form:} Factoring out $e^{-2X}/X$:
\[
\boxed{I(X) \sim \frac{e^{-2X}}{X}\left(\frac{1}{2} - \frac{1}{4X}\right) \text{ as } X \to \infty}
\]

\subsection*{Step 4.3: Interpretation}

\textbf{What does this tell us?}
\begin{itemize}
\item The factor $e^{-2X}$ comes from the exponential evaluated at $\tau = 2$ (equivalently $t = 1$), the lower limit where the exponent $t^2+1$ is smallest.
\item The powers of $X$ in the denominator come from repeated differentiation and integration by parts.
\item Each successive term is smaller by a factor of order $1/X$.
\item The expansion is asymptotic but not necessarily convergent for any fixed $X$.
\end{itemize}

\section*{Verification and Consistency}

\textbf{Does our answer make sense?}
\begin{itemize}
\item As $X \to \infty$, the integral $I(X) \to 0$ because the exponential kills the integrand. ✓
\item The leading behavior is $\sim e^{-2X}/X$, which goes to zero. ✓
\item The next term provides a correction of order $1/X$ smaller. ✓
\item The form matches the general result (Equation 164) from the lecture notes. ✓
\end{itemize}

\end{solution}

\end{document}
