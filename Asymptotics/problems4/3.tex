\documentclass[11pt,a4paper]{article}
\usepackage[margin=1in]{geometry}
\usepackage{amsmath,amssymb,amsthm}
\usepackage{mathtools}
\usepackage{enumitem}
\usepackage{xcolor}

\newcommand{\stage}[1]{\textbf{\textcolor{blue}{#1}}}

\title{Question 3: Irregular Singular Points and Asymptotic Solutions\\
Complete Analysis with Controlling Factor Method}
\author{Asymptotics Course — Sheet 4}
\date{}

\begin{document}

\maketitle

\section*{Problem Statement}

Consider the differential equation:
\[
\frac{d^2y}{dx^2} - \left(1 + \frac{1}{x}\right)y = 0
\]

\textbf{(a)} Show that it has an irregular singular point at $x = \infty$.

\textbf{(b)} Compute the two linearly independent solutions at leading order as $x \to \infty$.

\section{Part (a): Verify Irregular Singular Point at $x = \infty$}

\subsection{Step 1: Strategy for Analyzing Point at Infinity}

\begin{itemize}[leftmargin=*]
\item \stage{STAGE X (What we need):} To analyze whether $x = \infty$ is a singular point, we must transform it to a finite point and apply the classification from Section 3.1.

\item \stage{STAGE Y (Why this approach):} The standard classification (regular vs. irregular singular point) is defined for finite points. We use the transformation $x = 1/t$ to map $x = \infty$ to $t = 0$.

\item \stage{STAGE Z (What this means):} If $t = 0$ is an irregular singular point after transformation, then $x = \infty$ is an irregular singular point.
\end{itemize}

\subsection{Step 2: Transform the ODE}

\subsubsection*{Change of Variable}

Set:
\[
x = \frac{1}{t} \quad \Rightarrow \quad t = \frac{1}{x}
\]

As $x \to \infty$, we have $t \to 0$.

\subsubsection*{Transform Derivatives (ESSENTIAL)}

Using the chain rule:
\[
\frac{d}{dx} = \frac{dt}{dx} \cdot \frac{d}{dt} = -t^2 \frac{d}{dt}
\]

For the second derivative:
\begin{align*}
\frac{d^2}{dx^2} &= \frac{d}{dx}\left(\frac{d}{dx}\right) = \frac{d}{dx}\left(-t^2 \frac{d}{dt}\right) \\
&= -\frac{d(t^2)}{dx} \cdot \frac{d}{dt} - t^2 \frac{d}{dx}\left(\frac{d}{dt}\right) \\
&= -(-2t^3) \frac{d}{dt} - t^2 \left(-t^2 \frac{d^2}{dt^2}\right) \\
&= 2t^3 \frac{d}{dt} + t^4 \frac{d^2}{dt^2}
\end{align*}

Therefore:
\[
\frac{d^2y}{dx^2} = t^4 \frac{d^2y}{dt^2} + 2t^3 \frac{dy}{dt}
\]

\subsubsection*{Transform the Coefficient}

\[
1 + \frac{1}{x} = 1 + t
\]

\subsection{Step 3: Write the Transformed ODE}

Substituting into the original equation:
\[
t^4 \frac{d^2y}{dt^2} + 2t^3 \frac{dy}{dt} - (1 + t)y = 0
\]

Divide through by $t^4$ to obtain standard form:
\[
\frac{d^2y}{dt^2} + \frac{2}{t} \frac{dy}{dt} - \frac{1 + t}{t^4} y = 0
\]

\subsection{Step 4: Apply Classification Criteria}

\subsubsection*{Standard Form Comparison (Section 3.1)}

The general second-order linear ODE is:
\[
\frac{d^2y}{dt^2} + p(t)\frac{dy}{dt} + q(t)y = 0
\]

In our transformed equation:
\[
p(t) = \frac{2}{t}, \qquad q(t) = -\frac{1 + t}{t^4}
\]

\subsubsection*{Classification Criteria for Regular Singular Point}

\textit{From Section 3.1 of lecture notes:}

A point $t_0$ is a \textbf{regular singular point} if:
\begin{enumerate}
\item $(t - t_0)p(t)$ is analytic at $t_0$
\item $(t - t_0)^2 q(t)$ is analytic at $t_0$
\end{enumerate}

For our case with $t_0 = 0$:

\textbf{Check Condition 1:}
\[
t \cdot p(t) = t \cdot \frac{2}{t} = 2
\]

This is analytic at $t = 0$ (it's a constant). $\checkmark$

\textbf{Check Condition 2:}
\[
t^2 \cdot q(t) = t^2 \cdot \left(-\frac{1 + t}{t^4}\right) = -\frac{1 + t}{t^2} = -\frac{1}{t^2} - \frac{1}{t}
\]

This is \textbf{NOT analytic} at $t = 0$ because it has terms $-1/t^2$ and $-1/t$ that diverge as $t \to 0$. $\times$

\subsection{Step 5: Conclusion}

\begin{itemize}[leftmargin=*]
\item \stage{STAGE X (What we found):} Condition 1 is satisfied, but Condition 2 fails.

\item \stage{STAGE Y (Why this matters):} Since $(t-0)^2 q(t)$ is not analytic at $t = 0$, the point $t = 0$ is NOT a regular singular point. Therefore, it must be an irregular singular point.

\item \stage{STAGE Z (Final conclusion):} Since $t = 0$ (corresponding to $x = \infty$ in the original coordinates) is an irregular singular point, we have proven:
\end{itemize}

\begin{center}
\fbox{%
\parbox{0.85\textwidth}{%
\textbf{Answer to Part (a):}

The differential equation $y'' - (1 + 1/x)y = 0$ has an \textbf{irregular singular point at $x = \infty$} because:
\begin{enumerate}
\item The transformation $x = 1/t$ maps $x = \infty$ to $t = 0$
\item The transformed ODE has $t^2 q(t) = -(1+t)/t^2$, which is not analytic at $t = 0$
\item Therefore $t = 0$ is an irregular singular point, making $x = \infty$ an irregular singular point
\end{enumerate}
}}
\end{center}

\section{Part (b): Compute Leading Order Solutions as $x \to \infty$}

\subsection{Step 1: Strategy — Controlling Factor Method}

\begin{itemize}[leftmargin=*]
\item \stage{STAGE X (What we have):} An ODE with irregular singular point at $x = \infty$, which means standard Frobenius series fails.

\item \stage{STAGE Y (Why controlling factor):} For irregular singular points, we use the \textbf{controlling factor method} (Section 3.2) which seeks solutions of the form $y = e^{S(x)}$.

\item \stage{STAGE Z (What this means):} We'll find $S(x)$ by successive approximation, starting with dominant balance.
\end{itemize}

\subsection{Step 2: Set Up the Controlling Factor Ansatz}

\subsubsection*{Ansatz (Section 3.2.1)}

Set:
\[
y(x) = e^{S(x)}
\]

Then:
\begin{align*}
y'(x) &= S'(x) e^{S(x)} = S'(x) y(x) \\
y''(x) &= \left[S''(x) + \left(S'(x)\right)^2\right] e^{S(x)} = \left[S''(x) + \left(S'(x)\right)^2\right] y(x)
\end{align*}

\subsubsection*{Substitute into ODE}

The original ODE is:
\[
y'' - \left(1 + \frac{1}{x}\right)y = 0
\]

Substituting:
\[
\left[S'' + (S')^2\right]y - \left(1 + \frac{1}{x}\right)y = 0
\]

Since $y = e^{S(x)} \neq 0$, divide by $y$:
\[
S'' + (S')^2 - 1 - \frac{1}{x} = 0
\]

Rearrange:
\[
S'' + (S')^2 = 1 + \frac{1}{x}
\]

\subsection{Step 3: Leading Order — Dominant Balance}

\subsubsection*{Standard Assumption (Equation 121)}

\textit{From Section 3.2.1:} Near irregular singular points, we typically assume:
\[
S'' = o\left((S')^2\right) \quad \text{as } x \to \infty
\]

\subsubsection*{Apply Dominant Balance}

If $S'' \ll (S')^2$, then to leading order:
\[
(S')^2 \approx 1 + \frac{1}{x} \quad \text{as } x \to \infty
\]

For large $x$, $1/x \to 0$, so:
\[
(S')^2 \approx 1 \quad \Rightarrow \quad S' \approx \pm 1
\]

Integrating:
\[
S_0(x) = \pm x + \text{const}
\]

We can absorb the constant into the overall multiplicative constant of the solution, so:
\[
\boxed{S_0(x) = \pm x}
\]

\begin{itemize}[leftmargin=*]
\item \stage{STAGE X (What we found):} Two leading order solutions: $S_0(x) = x$ and $S_0(x) = -x$.

\item \stage{STAGE Y (Verify assumption):} With $S_0' = \pm 1$, we have $S_0'' = 0$, so indeed $S_0'' = o((S_0')^2)$ ✓

\item \stage{STAGE Z (Next step):} Find the next correction to determine the complete leading order behavior including prefactors.
\end{itemize}

\subsection{Step 4: Next Order Correction}

\subsubsection*{Refinement (Section 3.2.3)}

Set:
\[
S(x) = \pm x + C(x)
\]
where $C(x) = o(x)$ as $x \to \infty$ (i.e., $C(x)$ grows slower than linearly).

\subsubsection*{Compute Derivatives}

\begin{align*}
S'(x) &= \pm 1 + C'(x) \\
S''(x) &= C''(x)
\end{align*}

\subsubsection*{Substitute Back into ODE Equation}

\[
C'' + (\pm 1 + C')^2 = 1 + \frac{1}{x}
\]

Expand:
\[
C'' + 1 \pm 2C' + (C')^2 = 1 + \frac{1}{x}
\]

Simplify:
\[
C'' \pm 2C' + (C')^2 = \frac{1}{x}
\]

\subsubsection*{Apply Dominant Balance for $C(x)$}

Since $C(x) = o(x)$, we have:
\begin{itemize}
\item $C'(x) = o(1)$ as $x \to \infty$
\item $C''(x) = o(1/x)$ as $x \to \infty$
\item $(C')^2 = o(1)$ as $x \to \infty$
\end{itemize}

Therefore, to leading order in the equation for $C$:
\[
\pm 2C' \approx \frac{1}{x}
\]

This gives:
\[
C'(x) \approx \pm \frac{1}{2x}
\]

Integrating:
\[
C(x) = \pm \frac{1}{2}\log x + \text{const}
\]

Again, absorbing the constant:
\[
\boxed{C(x) = \pm \frac{1}{2}\log x}
\]

\subsection{Step 5: Combine Leading and Next Order}

\subsubsection*{Complete Expression for $S(x)$}

\[
S(x) = \pm x \pm \frac{1}{2}\log x + o(\log x) \quad \text{as } x \to \infty
\]

Note: The two $\pm$ signs are independent. For the two linearly independent solutions, we take:
\[
S_+(x) = x + \frac{1}{2}\log x \quad \text{and} \quad S_-(x) = -x - \frac{1}{2}\log x
\]

\subsubsection*{Solutions}

\begin{align*}
y_+(x) &= e^{S_+(x)} = e^{x + \frac{1}{2}\log x} = e^x \cdot e^{\log x^{1/2}} = e^x \sqrt{x} \\
y_-(x) &= e^{S_-(x)} = e^{-x - \frac{1}{2}\log x} = e^{-x} \cdot e^{-\log x^{1/2}} = \frac{e^{-x}}{\sqrt{x}}
\end{align*}

\subsection{Step 6: State Final Answer}

\begin{center}
\fbox{%
\parbox{0.85\textwidth}{%
\textbf{Answer to Part (b):}

The two linearly independent solutions at leading order as $x \to \infty$ are:
\[
y_1(x) \sim \sqrt{x} \, e^x \quad \text{as } x \to \infty
\]
\[
y_2(x) \sim \frac{e^{-x}}{\sqrt{x}} \quad \text{as } x \to \infty
\]

Or equivalently:
\[
y_1(x) \sim A x^{1/2} e^x, \qquad y_2(x) \sim B x^{-1/2} e^{-x} \quad \text{as } x \to \infty
\]
where $A$ and $B$ are arbitrary constants.
}}
\end{center}

\subsection{Step 7: Verification}

\subsubsection*{Check Linear Independence}

The Wronskian:
\[
W = y_1 y_2' - y_1' y_2
\]

For $y_1 \sim \sqrt{x}e^x$ and $y_2 \sim \frac{e^{-x}}{\sqrt{x}}$:
\begin{align*}
W &\sim \left(\sqrt{x}e^x\right) \cdot \left(-\frac{e^{-x}}{\sqrt{x}} - \frac{e^{-x}}{2x^{3/2}}\right) - \left(\sqrt{x}e^x + \frac{e^x}{2\sqrt{x}}\right) \cdot \frac{e^{-x}}{\sqrt{x}} \\
&\sim -1 - \frac{1}{2x} - 1 - \frac{1}{2x} = -2 + O(1/x) \neq 0 \quad \checkmark
\end{align*}

The solutions are linearly independent.

\subsubsection*{Verify Dominant Balance Consistency}

For $y_1 \sim \sqrt{x}e^x$:
\begin{align*}
y_1' &\sim \sqrt{x}e^x + \frac{e^x}{2\sqrt{x}} \sim \sqrt{x}e^x \quad (\text{leading order}) \\
y_1'' &\sim \sqrt{x}e^x \quad (\text{leading order})
\end{align*}

Substitute into ODE:
\[
y_1'' - (1 + 1/x)y_1 \sim \sqrt{x}e^x - (1 + 1/x)\sqrt{x}e^x \sim \sqrt{x}e^x(1 - 1 - 1/x)
\]

The leading terms cancel, leaving $O(x^{-1/2}e^x) = o(\sqrt{x}e^x)$ $\checkmark$

\section{Verification Checklist}

\begin{enumerate}[leftmargin=*]
\item[$\checkmark$] \textbf{Part (a) — Transformation:} $x = 1/t$ correctly transforms derivatives
\item[$\checkmark$] \textbf{Part (a) — Classification:} Checked both conditions for regular singular point
\item[$\checkmark$] \textbf{Part (a) — Conclusion:} $t^2 q(t)$ not analytic $\Rightarrow$ irregular singular point
\item[$\checkmark$] \textbf{Part (b) — Method:} Used controlling factor ansatz $y = e^{S(x)}$
\item[$\checkmark$] \textbf{Part (b) — Leading order:} Found $S_0' \approx \pm 1$ via dominant balance
\item[$\checkmark$] \textbf{Part (b) — Next order:} Found $C(x) \approx \pm \frac{1}{2}\log x$
\item[$\checkmark$] \textbf{Part (b) — Solutions:} Combined to get $y_1 \sim \sqrt{x}e^x$ and $y_2 \sim e^{-x}/\sqrt{x}$
\item[$\checkmark$] \textbf{Verification:} Checked linear independence via Wronskian
\item[$\checkmark$] \textbf{Consistency:} Verified dominant balance assumption holds
\end{enumerate}

\vspace{1em}
\textit{This solution follows the methodology of Section 3.1 (classification of singular points) and Section 3.2 (controlling factor method for irregular singular points) from the lecture notes.}

\end{document}
