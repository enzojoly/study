\documentclass[11pt]{article}
\usepackage{amsmath, amssymb}
\usepackage[margin=1in]{geometry}

\title{Asymptotics Problem Sheet 5 - Question 1(d)}
\author{}
\date{}

\begin{document}
\maketitle

\section*{Problem Statement}
Use the method of stationary phase to obtain the leading asymptotic behaviour as $X \to \infty$ of:
$$I(X) = \int_0^\infty e^{iX(2t-t^2)} \ln(1+t^2)\, dt$$

\section*{Solution}

\subsection*{Step 1: Identify the integral structure}
This is a Fourier-type integral of the form:
$$I(X) = \int_a^b f(t)\, e^{iX\phi(t)}\, dt$$

where:
\begin{itemize}
    \item $f(t) = \ln(1+t^2)$
    \item $\phi(t) = 2t - t^2$
    \item Integration domain: $[a,b] = [0, \infty)$
\end{itemize}

\subsection*{Step 2: Locate stationary points}
Following Section 4.3.2 of the lecture notes, we search for critical points where $\phi'(t) = 0$ in the integration domain.

Computing the derivative:
$$\phi'(t) = \frac{d}{dt}(2t - t^2) = 2 - 2t$$

Setting $\phi'(t) = 0$:
$$2 - 2t = 0 \implies t = 1$$

\textbf{Check:} Is $t = 1 \in (0, \infty)$? Yes, this is a valid interior stationary point.

\subsection*{Step 3: Classify the stationary point}
Compute the second derivative:
$$\phi''(t) = -2$$

At $t = 1$:
$$\phi''(1) = -2 < 0$$

Since $\phi''(1) \neq 0$, this is a \textbf{non-degenerate stationary point}. The negative second derivative indicates $t = 1$ is a \textbf{maximum} of $\phi(t)$.

\subsection*{Step 4: Evaluate quantities at the stationary point}
At $t = 1$:
\begin{align*}
f(1) &= \ln(1 + 1^2) = \ln(2) \\
\phi(1) &= 2(1) - 1^2 = 1 \\
\phi''(1) &= -2
\end{align*}

\subsection*{Step 5: Apply the stationary phase formula}
From equation (235) in the lecture notes, for a stationary point $c$ with $\phi'(c) = 0$ and $\phi''(c) \neq 0$:

$$I(X) \sim \sqrt{\frac{2\pi i}{X\phi''(c)}}\, f(c)\, e^{iX\phi(c)} \quad \text{as } X \to \infty$$

Substituting our values with $c = 1$:
$$I(X) \sim \sqrt{\frac{2\pi i}{X \cdot (-2)}}\, \ln(2)\, e^{iX}$$

\subsection*{Step 6: Simplify the square root}
$$\sqrt{\frac{2\pi i}{-2X}} = \sqrt{\frac{-\pi i}{X}} = \sqrt{\frac{\pi}{X}} \cdot \sqrt{-i}$$

To evaluate $\sqrt{-i}$, write $-i$ in exponential form:
$$-i = e^{-i\pi/2}$$

Therefore:
$$\sqrt{-i} = e^{-i\pi/4}$$

This gives:
$$\sqrt{\frac{2\pi i}{-2X}} = \sqrt{\frac{\pi}{X}}\, e^{-i\pi/4}$$

\subsection*{Step 7: Write the final result}
Combining all terms:
$$I(X) \sim \sqrt{\frac{\pi}{X}}\, e^{-i\pi/4} \cdot \ln(2) \cdot e^{iX}$$

$$I(X) \sim \ln(2)\sqrt{\frac{\pi}{X}}\, e^{iX - i\pi/4}$$

\section*{Final Answer}
$$\boxed{I(X) \sim \ln(2)\sqrt{\frac{\pi}{X}}\, e^{i(X - \pi/4)} \quad \text{as } X \to \infty}$$

Alternatively, this can be written as:
$$\boxed{I(X) \sim \frac{\ln(2)}{\sqrt{X}}\sqrt{\pi}\, e^{iX}\, e^{-i\pi/4} \quad \text{as } X \to \infty}$$

\subsection*{Remarks}
\begin{itemize}
    \item The leading order behavior is $O(X^{-1/2})$, which is characteristic of contributions from non-degenerate stationary points.
    \item The phase shift of $-\pi/4$ arises from the negative second derivative $\phi''(1) = -2 < 0$.
    \item No contributions from the endpoints $t = 0$ or $t \to \infty$ compete at this order, as they contribute at higher order in $X^{-1}$ (see equation 232).
\end{itemize}

\end{document}
