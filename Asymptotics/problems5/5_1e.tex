\documentclass[11pt]{article}
\usepackage{amsmath,amssymb,amsthm}
\usepackage[margin=1in]{geometry}

\begin{document}

\section*{Solution 5.1(e)}

\textbf{Problem:} Find the asymptotic behavior of
\[
I(X) = \int_0^{\pi} \sin(X \cos t)\, e^{-t^2}\, dt
\]
as $X \to \infty$.

\bigskip

\textbf{Solution:}

\textbf{Step 1: Express as imaginary part of complex integral.}

We write the integral in complex exponential form:
\[
I(X) = \operatorname{Im} \int_0^{\pi} e^{-t^2} e^{iX \cos t}\, dt.
\]

\textbf{Step 2: Identify the phase function and locate stationary points.}

The phase function is $\Phi(t) = \cos t$, with derivatives:
\[
\Phi'(t) = -\sin t, \qquad \Phi''(t) = -\cos t.
\]

The stationary points occur where $\Phi'(t) = 0$, i.e., where $\sin t = 0$. Within the integration interval $[0, \pi]$, this gives stationary points at the two endpoints:
\[
t = 0 \quad \text{and} \quad t = \pi.
\]

\textbf{Step 3: Evaluate the phase function at stationary points.}

At the endpoints:
\begin{align*}
\Phi(0) &= \cos 0 = 1, \qquad &\Phi''(0) &= -\cos 0 = -1, \\
\Phi(\pi) &= \cos \pi = -1, \qquad &\Phi''(\pi) &= -\cos \pi = 1.
\end{align*}

\textbf{Step 4: Apply the stationary phase formula for boundary stationary points.}

For a stationary point at an endpoint $t = c$ of the integration interval, the contribution to the integral is half that of an interior stationary point. The asymptotic contribution from an endpoint stationary point is given by:
\[
\frac{1}{2} \sqrt{\frac{2\pi i}{X |\Phi''(c)|}} \, e^{\pm i\pi/4} \, f(c) \, e^{iX\Phi(c)},
\]
where the sign in the exponential $e^{\pm i\pi/4}$ depends on the sign of $\Phi''(c)$, and $f(t) = e^{-t^2}$ is the amplitude function.

\textbf{Step 5: Compute the contribution from $t = 0$.}

At $t = 0$:
\begin{itemize}
    \item $f(0) = e^{0} = 1$
    \item $\Phi(0) = 1$
    \item $\Phi''(0) = -1 < 0$
\end{itemize}

The contribution is:
\[
\frac{1}{2} \sqrt{\frac{2\pi i}{X \cdot 1}} \cdot e^{-i\pi/4} \cdot 1 \cdot e^{iX} = \frac{1}{2} \sqrt{\frac{2\pi}{X}} \cdot e^{i\pi/4} \cdot e^{-i\pi/4} \cdot e^{iX} = \frac{1}{2} \sqrt{\frac{2\pi}{X}} \, e^{iX}.
\]

More carefully, using $\sqrt{i} = e^{i\pi/4}$ and accounting for $\Phi''(0) = -1$:
\[
\sqrt{\frac{i}{\Phi''(0)}} = \sqrt{\frac{i}{-1}} = \sqrt{-i} = e^{-i\pi/4}.
\]

Thus the contribution from $t = 0$ is:
\[
I_0(X) \sim \frac{1}{2} \sqrt{\frac{2\pi}{X}} \, e^{-i\pi/4} \, e^{iX} = \frac{1}{2} \sqrt{\frac{2\pi}{X}} \, e^{i(X - \pi/4)}.
\]

\textbf{Step 6: Compute the contribution from $t = \pi$.}

At $t = \pi$:
\begin{itemize}
    \item $f(\pi) = e^{-\pi^2}$
    \item $\Phi(\pi) = -1$
    \item $\Phi''(\pi) = 1 > 0$
\end{itemize}

Using $\sqrt{i/\Phi''(\pi)} = \sqrt{i} = e^{i\pi/4}$:
\[
I_\pi(X) \sim \frac{1}{2} \sqrt{\frac{2\pi}{X}} \, e^{i\pi/4} \, e^{-\pi^2} \, e^{-iX} = \frac{1}{2} \sqrt{\frac{2\pi}{X}} \, e^{-\pi^2} \, e^{-i(X - \pi/4)}.
\]

\textbf{Step 7: Combine contributions from both endpoints.}

The total asymptotic contribution is:
\begin{align*}
\int_0^{\pi} e^{-t^2} e^{iX \cos t}\, dt &\sim I_0(X) + I_\pi(X) \\
&= \frac{1}{2} \sqrt{\frac{2\pi}{X}} \left[ e^{i(X - \pi/4)} + e^{-\pi^2} e^{-i(X - \pi/4)} \right].
\end{align*}

\textbf{Step 8: Extract the imaginary part.}

Taking the imaginary part:
\begin{align*}
I(X) &= \operatorname{Im} \left\{ \frac{1}{2} \sqrt{\frac{2\pi}{X}} \left[ e^{i(X - \pi/4)} + e^{-\pi^2} e^{-i(X - \pi/4)} \right] \right\} \\
&= \frac{1}{2} \sqrt{\frac{2\pi}{X}} \left[ \sin\left(X - \frac{\pi}{4}\right) - e^{-\pi^2} \sin\left(X - \frac{\pi}{4}\right) \right] \\
&= \frac{1}{2} \sqrt{\frac{2\pi}{X}} \left(1 - e^{-\pi^2}\right) \sin\left(X - \frac{\pi}{4}\right).
\end{align*}

\textbf{Step 9: Final result.}

Therefore, the asymptotic behavior is:
\[
\boxed{I(X) \sim \sqrt{\frac{\pi}{2X}} \sin\left(X - \frac{\pi}{4}\right) \left(1 - e^{-\pi^2}\right) \quad \text{as } X \to \infty.}
\]

\bigskip

\textbf{Remark:} Although $e^{-\pi^2} \approx 5.2 \times 10^{-5}$ is numerically small, the factor $(1 - e^{-\pi^2})$ is $O(1)$ as $X \to \infty$ and must be retained in the asymptotic expression. Both endpoint stationary points contribute at the same asymptotic order $O(X^{-1/2})$, and their interference produces this factor. The amplitude $e^{-t^2}$ evaluates to $1$ at $t = 0$ and to $e^{-\pi^2}$ at $t = \pi$, giving rise to the difference in contributions.

\end{document}
