\documentclass[11pt]{article}
\usepackage{amsmath,amssymb,amsthm}
\usepackage{geometry}
\geometry{a4paper, margin=0.7in}
\usepackage{enumitem}

\title{Asymptotics Problem 2(b)}
\author{Solution using Method of Steepest Descent}
\date{}

\begin{document}
\maketitle

\section*{Problem Statement}
Use the method of steepest descent to find the leading asymptotic behaviour as $X \to \infty$ of:
$$I(X) = \int_{-\infty}^{\infty} \frac{t e^{iX(t^5/5+t)}}{1+t^2}\, dt$$

\section*{Solution}

\subsection*{Step 1: Identify the integral structure}
We have a complex integral of the form (Eq.~239 from notes):
$$I(X) = \int_C f(z)\, e^{X\phi(z)}\, dz$$
where:
\begin{itemize}
\item $f(t) = \dfrac{t}{1+t^2}$
\item $\phi(t) = i\left(\dfrac{t^5}{5} + t\right)$
\item Original contour $C$: the real axis from $-\infty$ to $\infty$
\end{itemize}

\subsection*{Step 2: Find saddle points}
Saddle points occur where $\phi'(t) = 0$.

Computing the derivative:
$$\phi'(t) = i(t^4 + 1)$$

Setting $\phi'(t) = 0$:
$$i(t^4 + 1) = 0 \implies t^4 = -1 = e^{i\pi}$$

The four saddle points are:
$$t_k = e^{i\pi(2k+1)/4}, \quad k = 0, 1, 2, 3$$

Explicitly:
\begin{align*}
t_0 &= e^{i\pi/4} = \frac{1+i}{\sqrt{2}} \\
t_1 &= e^{i3\pi/4} = \frac{-1+i}{\sqrt{2}} \\
t_2 &= e^{i5\pi/4} = \frac{-1-i}{\sqrt{2}} \\
t_3 &= e^{i7\pi/4} = \frac{1-i}{\sqrt{2}}
\end{align*}

\subsection*{Step 3: Evaluate $\phi(t)$ at saddle points}
For any saddle point where $t^4 = -1$:
$$\phi(t) = i\left(\frac{t^5}{5} + t\right) = i\left(\frac{t \cdot t^4}{5} + t\right) = i\left(\frac{-t}{5} + t\right) = i\cdot\frac{4t}{5}$$

Therefore:
\begin{align*}
\phi(t_0) &= i\cdot\frac{4e^{i\pi/4}}{5} = \frac{4i}{\sqrt{2}}e^{i\pi/4} = \frac{2\sqrt{2}i}{5}(1+i) = \frac{2\sqrt{2}}{5}(i-1) \\
\phi(t_1) &= i\cdot\frac{4e^{i3\pi/4}}{5} = \frac{2\sqrt{2}}{5}(i+1) \\
\phi(t_2) &= i\cdot\frac{4e^{i5\pi/4}}{5} = \frac{2\sqrt{2}}{5}(-i+1) \\
\phi(t_3) &= i\cdot\frac{4e^{i7\pi/4}}{5} = \frac{2\sqrt{2}}{5}(-i-1)
\end{align*}

Computing the real parts:
\begin{align*}
\text{Re}[\phi(t_0)] &= -\frac{2\sqrt{2}}{5} \\
\text{Re}[\phi(t_1)] &= +\frac{2\sqrt{2}}{5} \\
\text{Re}[\phi(t_2)] &= +\frac{2\sqrt{2}}{5} \\
\text{Re}[\phi(t_3)] &= -\frac{2\sqrt{2}}{5}
\end{align*}

\subsection*{Step 4: Determine relevant saddle point}
For steepest descent, we seek saddle points with **maximum** real part of $\phi(t)$ (since the integrand contains $e^{X\phi(t)}$).

The saddle points $t_1$ and $t_2$ both have $\text{Re}[\phi] = +\frac{2\sqrt{2}}{5}$, which is the maximum.

\subsection*{Step 5: Check contour deformation}
The original contour is the real axis. We need to determine which saddle point(s) the real axis can be deformed through using Cauchy's theorem.

By examining the geometry and applying Cauchy's theorem (Section 4.4), the real axis can be deformed to pass through saddle points in the lower half-plane as $t \to -\infty$ and upper half-plane as $t \to +\infty$.

Given the symmetry of the problem and the location of saddle points:
\begin{itemize}
\item $t_1 = \frac{-1+i}{\sqrt{2}}$ (upper left quadrant)
\item $t_2 = \frac{-1-i}{\sqrt{2}}$ (lower left quadrant)
\end{itemize}

However, we must check which gives a steepest **descent** path compatible with the original contour.

\subsection*{Step 6: Analyze steepest descent paths}
For a saddle point where $\phi''(t_c) = ae^{i\alpha}$, steepest descent directions are (from notes, Section 4.4.2):
$$\theta_{\text{descent}} = -\frac{\alpha}{2} + \frac{(2p+1)\pi}{2}, \quad p = 0, 1$$

Computing $\phi''(t) = 4it^3$:

At $t_1 = e^{i3\pi/4}$:
$$\phi''(t_1) = 4i \cdot e^{i9\pi/4} = 4i \cdot e^{i\pi/4} = 4e^{i3\pi/4}$$

So $\alpha = 3\pi/4$, giving descent directions:
\begin{align*}
p=0: \quad \theta &= -\frac{3\pi/4}{2} + \frac{\pi}{2} = -\frac{3\pi}{8} + \frac{\pi}{2} = \frac{\pi}{8} \\
p=1: \quad \theta &= -\frac{3\pi}{8} + \frac{3\pi}{2} = \frac{9\pi}{8}
\end{align*}

The contour can be deformed through $t_1$ along steepest descent paths.

\subsection*{Step 7: Evaluate contribution from $t_1$}
Near the saddle point $t_1$, we use the expansion (Laplace's method, Eq.~205):
$$\phi(t) \approx \phi(t_1) + \frac{1}{2}\phi''(t_1)(t-t_1)^2$$

The leading asymptotic contribution is (from Eq.~207 for $n=2$):
$$I(X) \sim \sqrt{\frac{2\pi i}{X\phi''(t_1)}}\, f(t_1)\, e^{X\phi(t_1)}$$

\subsection*{Step 8: Compute the leading term}
First, evaluate $f(t_1)$:
$$f(t_1) = \frac{t_1}{1+t_1^2} = \frac{e^{i3\pi/4}}{1+e^{i3\pi/2}} = \frac{e^{i3\pi/4}}{1-i}$$

Simplifying:
$$f(t_1) = \frac{e^{i3\pi/4}}{1-i} \cdot \frac{1+i}{1+i} = \frac{e^{i3\pi/4}(1+i)}{2} = \frac{e^{i3\pi/4} \cdot \sqrt{2}e^{i\pi/4}}{2} = \frac{\sqrt{2}e^{i\pi}}{2} = -\frac{\sqrt{2}}{2}$$

Now evaluate the coefficient:
$$\phi''(t_1) = 4e^{i3\pi/4}$$

Therefore:
$$\sqrt{\frac{2\pi i}{X\phi''(t_1)}} = \sqrt{\frac{2\pi i}{4X e^{i3\pi/4}}} = \sqrt{\frac{\pi i}{2X}} \cdot e^{-i3\pi/8}$$

Note: $\sqrt{i} = e^{i\pi/4}$, so:
$$\sqrt{\frac{\pi i}{2X}} = \sqrt{\frac{\pi}{2X}} \cdot e^{i\pi/4}$$

Thus:
$$\sqrt{\frac{2\pi i}{X\phi''(t_1)}} = \sqrt{\frac{\pi}{2X}} \cdot e^{i\pi/4} \cdot e^{-i3\pi/8} = \sqrt{\frac{\pi}{2X}} \cdot e^{-i\pi/8}$$

\subsection*{Step 9: Final answer}
Combining all terms:
$$I(X) \sim -\frac{\sqrt{2}}{2} \cdot \sqrt{\frac{\pi}{2X}} \cdot e^{-i\pi/8} \cdot \exp\left(\frac{2\sqrt{2}X}{5}(i+1)\right)$$

Simplifying:
$$\boxed{I(X) \sim -\frac{1}{2}\sqrt{\frac{\pi}{X}} \cdot e^{-i\pi/8} \cdot e^{2\sqrt{2}X/5} \cdot e^{i2\sqrt{2}X/5} \quad \text{as } X \to \infty}$$

Or more compactly:
$$\boxed{I(X) \sim -\frac{1}{2}\sqrt{\frac{\pi}{X}} \cdot \exp\left[\frac{2\sqrt{2}X}{5}(1+i) - \frac{i\pi}{8}\right] \quad \text{as } X \to \infty}$$

The leading order behaviour is dominated by $e^{2\sqrt{2}X/5} \cdot X^{-1/2}$.

\end{document}
