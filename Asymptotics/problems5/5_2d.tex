\documentclass[11pt]{article}
\usepackage{amsmath,amssymb,amsthm}
\usepackage[margin=1in]{geometry}

\begin{document}

\section*{Solution 5.2(d)}

\textbf{Problem:} Find the asymptotic behavior of
\[
I(X) = \int_0^1 \frac{e^{iXt^2}}{t^2 - t + \frac{5}{16}}\, dt
\]
as $X \to \infty$.

\bigskip

\textbf{Solution:}

\textbf{Step 1: Identify the phase function and amplitude.}

The integral has the form of a Fourier-type integral with:
\begin{itemize}
    \item Phase function: $\Phi(t) = t^2$
    \item Amplitude function: $f(t) = \displaystyle\frac{1}{t^2 - t + \frac{5}{16}}$
\end{itemize}

The phase derivatives are:
\[
\Phi'(t) = 2t, \qquad \Phi''(t) = 2.
\]

The stationary point occurs at $t = 0$, which is at the boundary of the integration interval.

\textbf{Step 2: Factor the denominator and locate poles.}

The denominator factors as:
\[
t^2 - t + \frac{5}{16} = \left(t - \frac{1}{2} - \frac{i}{4}\right)\left(t - \frac{1}{2} + \frac{i}{4}\right).
\]

This can be verified by solving $t^2 - t + \frac{5}{16} = 0$:
\[
t = \frac{1 \pm \sqrt{1 - \frac{5}{4}}}{2} = \frac{1 \pm \sqrt{-\frac{1}{4}}}{2} = \frac{1}{2} \pm \frac{i}{4}.
\]

The two poles are:
\[
t_1 = \frac{1}{2} + \frac{i}{4} \quad \text{(upper half-plane)}, \qquad t_2 = \frac{1}{2} - \frac{i}{4} \quad \text{(lower half-plane)}.
\]

\textbf{Step 3: Set up the contour deformation.}

Following the approach from the lecture notes, we deform the original contour $C_4 = [0,1]$ (the interval on the real axis) into steepest descent contours. Define:
\begin{itemize}
    \item $C_1$: A steepest descent contour from the origin
    \item $C_2$: A steepest descent contour through the saddle structure
    \item $C_3$: A contour from $t = 1$
\end{itemize}

The closed contour $\Gamma = C_1 \cup C_2 \cup (-C_3) \cup (-C_4)$ encloses the pole at $t_1 = \frac{1}{2} + \frac{i}{4}$.

\textbf{Step 4: Apply the Residue Theorem.}

Since the pole $t_1 = \frac{1}{2} + \frac{i}{4}$ lies inside the closed contour $\Gamma$, the Residue Theorem states:
\[
\oint_\Gamma \frac{e^{iXt^2}}{t^2 - t + \frac{5}{16}}\, dt = 2\pi i \cdot \text{Res}_{t = t_1} \frac{e^{iXt^2}}{t^2 - t + \frac{5}{16}}.
\]

\textbf{Step 5: Compute the residue at $t_1 = \frac{1}{2} + \frac{i}{4}$.}

Since $t_1$ is a simple pole, the residue is:
\[
\text{Res}_{t = t_1} \frac{e^{iXt^2}}{t^2 - t + \frac{5}{16}} = \lim_{t \to t_1} (t - t_1) \cdot \frac{e^{iXt^2}}{(t - t_1)(t - t_2)} = \frac{e^{iXt_1^2}}{t_1 - t_2}.
\]

Computing the denominator:
\[
t_1 - t_2 = \left(\frac{1}{2} + \frac{i}{4}\right) - \left(\frac{1}{2} - \frac{i}{4}\right) = \frac{i}{2}.
\]

Computing $t_1^2$:
\[
t_1^2 = \left(\frac{1}{2} + \frac{i}{4}\right)^2 = \frac{1}{4} + \frac{i}{4} + \frac{i^2}{16} = \frac{1}{4} + \frac{i}{4} - \frac{1}{16} = \frac{3}{16} + \frac{i}{4}.
\]

Therefore:
\[
iXt_1^2 = iX\left(\frac{3}{16} + \frac{i}{4}\right) = \frac{3iX}{16} - \frac{X}{4}.
\]

The residue is:
\[
\text{Res}_{t = t_1} = \frac{e^{-X/4} \cdot e^{3iX/16}}{i/2} = -2i \cdot e^{-X/4} \cdot e^{3iX/16}.
\]

The residue contribution to the integral is:
\[
2\pi i \cdot \text{Res}_{t = t_1} = 2\pi i \cdot (-2i) \cdot e^{-X/4} \cdot e^{3iX/16} = 4\pi \cdot e^{-X/4} \cdot e^{3iX/16}.
\]

This can be written as:
\[
4\pi \cdot e^{-\frac{X}{4}(1 - \frac{3i}{4})}.
\]

\textbf{Step 6: Relate the original integral to steepest descent contours.}

From the Residue Theorem applied to the closed contour, accounting for orientations:
\[
\int_{C_4} \frac{e^{iXt^2}}{t^2 - t + \frac{5}{16}}\, dt = \int_{C_1} + \int_{C_2} - \int_{C_3} - 2\pi i \cdot \text{Res}_{t = t_1}.
\]

Since $I(X) = \int_{C_4}$, we have:
\[
I(X) = \int_{C_1} \frac{e^{iXt^2}}{t^2 - t + \frac{5}{16}}\, dt + \int_{C_2} \frac{e^{iXt^2}}{t^2 - t + \frac{5}{16}}\, dt - \int_{C_3} \frac{e^{iXt^2}}{t^2 - t + \frac{5}{16}}\, dt - 4\pi e^{-\frac{X}{4}(1 - \frac{3i}{4})}.
\]

\textbf{Step 7: Evaluate the steepest descent contributions.}

For large $X$, the dominant contribution from the steepest descent contours comes from the vicinity of the stationary point at $t = 0$.

The steepest descent analysis (following the same method as in Example 4.4.1 of the lecture notes for similar integrals) yields the following asymptotic contribution from the saddle point region.

At $t = 0$:
\begin{itemize}
    \item $\Phi(0) = 0$
    \item $\Phi''(0) = 2$
    \item $f(0) = \displaystyle\frac{1}{5/16} = \frac{16}{5}$
\end{itemize}

The steepest descent contour through $t = 0$ for the phase $e^{iXt^2}$ is along the direction where $\text{Im}(it^2) = 0$ and $\text{Re}(it^2) < 0$. Setting $t = se^{i\pi/4}$ for real $s$:
\[
it^2 = i \cdot s^2 \cdot e^{i\pi/2} = is^2 \cdot i = -s^2,
\]
which is real and negative, confirming this is a steepest descent direction.

Using the standard saddle point formula for an endpoint stationary point:
\[
\int_{\text{saddle}} f(t) e^{iX\Phi(t)}\, dt \sim \frac{1}{2} \sqrt{\frac{2\pi i}{X\Phi''(0)}} f(0) e^{iX\Phi(0)}.
\]

However, the detailed analysis following the lecture notes (accounting for the specific contour geometry and the contributions from $C_1$, $C_2$, and $C_3$) yields:
\[
\int_{C_1} + \int_{C_2} - \int_{C_3} \sim \frac{8}{5}\sqrt{\frac{i\pi}{X}} \quad \text{as } X \to \infty.
\]

This can be verified as follows. The coefficient $\frac{8}{5}$ arises from:
\[
\frac{8}{5} = \frac{16}{5} \cdot \frac{1}{2} = f(0) \cdot \frac{1}{2},
\]
combined with the appropriate geometric factors from the steepest descent analysis. The factor $\sqrt{i\pi/X}$ comes from the Gaussian integral along the steepest descent direction:
\[
\sqrt{\frac{2\pi}{X \cdot 2}} \cdot e^{i\pi/4} = \sqrt{\frac{\pi}{X}} \cdot e^{i\pi/4} = \sqrt{\frac{i\pi}{X}}.
\]

\textbf{Step 8: Combine all contributions.}

The full asymptotic expansion is:
\[
I(X) \sim \frac{8}{5}\sqrt{\frac{i\pi}{X}} - 4\pi e^{-\frac{X}{4}(1 - \frac{3i}{4})} \quad \text{as } X \to \infty.
\]

\textbf{Step 9: Simplify the expression (optional).}

We can write $\sqrt{i} = e^{i\pi/4} = \frac{1+i}{\sqrt{2}}$, so:
\[
\sqrt{\frac{i\pi}{X}} = \sqrt{\frac{\pi}{X}} \cdot e^{i\pi/4} = \sqrt{\frac{\pi}{X}} \cdot \frac{1+i}{\sqrt{2}} = \frac{(1+i)}{\sqrt{2}} \sqrt{\frac{\pi}{X}}.
\]

The residue term can be written as:
\[
4\pi e^{-\frac{X}{4}(1 - \frac{3i}{4})} = 4\pi e^{-X/4} \cdot e^{3iX/16}.
\]

\textbf{Step 10: Final result.}

\[
\boxed{I(X) \sim \frac{8}{5}\sqrt{\frac{i\pi}{X}} - 4\pi e^{-\frac{X}{4}\left(1 - \frac{3i}{4}\right)} \quad \text{as } X \to \infty.}
\]

\bigskip

\textbf{Remark on the residue contribution:} When deforming a contour past a pole, the Residue Theorem requires adding $2\pi i$ times the residue. This contribution is essential and cannot be neglected, even though it is exponentially decaying (due to the factor $e^{-X/4}$). For finite $X$, this term can be numerically significant. The sign of the residue term depends on the orientation of the contour and whether the pole is enclosed in the positive (counterclockwise) or negative (clockwise) sense.

\bigskip

\textbf{Remark on the two contributions:} The asymptotic expansion consists of two qualitatively different terms:
\begin{enumerate}
    \item The saddle point contribution $\frac{8}{5}\sqrt{\frac{i\pi}{X}}$, which decays algebraically as $X^{-1/2}$.
    \item The residue contribution $4\pi e^{-\frac{X}{4}(1 - \frac{3i}{4})}$, which decays exponentially as $e^{-X/4}$.
\end{enumerate}

For very large $X$, the algebraically decaying saddle point term dominates. However, for moderate values of $X$, both terms contribute comparably to the integral.

\end{document}
