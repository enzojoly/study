\documentclass[11pt]{article}
\usepackage{amsmath,amssymb,amsthm}
\usepackage[margin=1in]{geometry}

\begin{document}

\section*{Solution 5.2(e)}

\textbf{Problem:} Find the asymptotic behavior of
\[
I(X) = \int_{-1}^{\infty} \sqrt{1+t}\, \cos(Xt^2)\, e^{X(t - t^3/3)}\, dt
\]
as $X \to \infty$.

\bigskip

\textbf{Solution:}

\textbf{Step 1: Recast as the real part of a complex integral.}

We write the integral in complex form by expressing $\cos(Xt^2) = \text{Re}[e^{iXt^2}]$:
\[
I(X) = \text{Re} \int_{-1}^{\infty} \sqrt{1+z}\, e^{X(z + iz^2 - z^3/3)}\, dz.
\]

Define the phase function:
\[
\Phi(z) = z + iz^2 - \frac{z^3}{3}.
\]

Thus:
\[
I(X) = \text{Re} \int_{-1}^{\infty} \sqrt{1+z}\, e^{X\Phi(z)}\, dz.
\]

\textbf{Step 2: Find the saddle points.}

The derivative of the phase function is:
\[
\Phi'(z) = 1 + 2iz - z^2 = -(z^2 - 2iz - 1) = -(z - i)^2.
\]

Setting $\Phi'(z) = 0$:
\[
-(z - i)^2 = 0 \implies z = i.
\]

This is a saddle point of \textbf{order two} (a double root), which means $\Phi'(i) = 0$ and $\Phi''(i) = 0$, but $\Phi'''(i) \neq 0$.

We verify:
\[
\Phi''(z) = 2i - 2z \implies \Phi''(i) = 2i - 2i = 0.
\]
\[
\Phi'''(z) = -2 \implies \Phi'''(i) = -2 \neq 0.
\]

\textbf{Step 3: Evaluate $\Phi(z)$ at the saddle point.}

At $z = i$:
\[
\Phi(i) = i + i(i)^2 - \frac{(i)^3}{3} = i + i(-1) - \frac{-i}{3} = i - i + \frac{i}{3} = \frac{i}{3}.
\]

\textbf{Step 4: Determine the steepest descent paths through $z = i$.}

The real and imaginary parts of $\Phi(z)$ for $z = x + iy$ are:
\[
\Phi(x + iy) = \left(-\frac{x^3}{3} + xy^2 - 2xy + x\right) + i\left(\frac{(y-1)^3}{3} - x^2(y-1) + \frac{1}{3}\right).
\]

More directly, expanding around the saddle point:
\[
\Phi(z) = \Phi(i) + \frac{\Phi'''(i)}{6}(z - i)^3 + O((z-i)^4) = \frac{i}{3} - \frac{1}{3}(z - i)^3 + O((z-i)^4).
\]

The steepest descent/ascent contours through $z = i$ are determined by $\text{Im}[\Phi(z)] = \text{Im}[\Phi(i)] = \frac{1}{3}$.

For a saddle point of order two (where the first non-vanishing derivative is the third), the steepest contours consist of \textbf{three lines} meeting at equal angles of $120°$ at the saddle point. These alternate between steepest descent and steepest ascent paths.

From the condition $\text{Im}[\Phi(z)] = \frac{1}{3}$, the steepest contours are:
\begin{itemize}
    \item $y = 1$ (the horizontal line through $z = i$)
    \item $y = \sqrt{3}x + 1$ (line at angle $60°$ from horizontal)
    \item $y = -\sqrt{3}x + 1$ (line at angle $-60°$ from horizontal, i.e., $120°$)
\end{itemize}

Examining the real part $\text{Re}[\Phi(z)]$ along these curves determines which are descent and which are ascent:
\begin{itemize}
    \item Along $y = 1$, $x > 0$: This is a steepest \textbf{descent} curve.
    \item Along $y = \sqrt{3}x + 1$, $x < 0$: This is also a steepest \textbf{descent} curve.
    \item The remaining directions are steepest ascent curves.
\end{itemize}

\textbf{Step 5: Deform the integration contour.}

The original contour is the real axis from $z = -1$ to $z = +\infty$. We need to deform this contour to pass through the saddle point at $z = i$ along steepest descent paths.

The deformation proceeds as follows:
\begin{enumerate}
    \item Start at $z = -1$ and follow the steepest descent path determined by $\text{Im}[\Phi(z)] = \text{Im}[\Phi(-1)] = 1$ into the third quadrant until it approaches the asymptote $y = \sqrt{3}x + 1$.
    \item This asymptote is itself a steepest descent path of the saddle point $z = i$, so we follow it to reach $z = i$.
    \item From $z = i$, we follow the steepest descent path along $y = 1$, $x > 0$ (i.e., the line $\text{Im}(z) = 1$) to infinity.
\end{enumerate}

The contributions from paths far from the saddle point are subdominant. The leading asymptotic contribution comes from the vicinity of the saddle point $z = i$.

\textbf{Step 6: Evaluate the contribution from the saddle point.}

For an order-two saddle point (where $\Phi'''(z_0) \neq 0$ is the first non-vanishing derivative), the local behavior is:
\[
\Phi(z) \approx \Phi(i) + \frac{\Phi'''(i)}{6}(z - i)^3 = \frac{i}{3} - \frac{1}{3}(z - i)^3.
\]

\textbf{Contribution from the path $y = 1$, $x > 0$:}

Along this path, set $z = i + s$ where $s \in [0, \infty)$ is real. Then:
\[
\Phi(z) \approx \frac{i}{3} - \frac{s^3}{3}.
\]

The amplitude function at the saddle is:
\[
\sqrt{1 + i} = (1 + i)^{1/2} = 2^{1/4} e^{i\pi/8}.
\]

The integral contribution from this path is:
\[
I_1(X) \sim \text{Re}\left[\sqrt{1+i} \int_0^{\infty} e^{X(i/3 - s^3/3)}\, ds\right] = \text{Re}\left[2^{1/4} e^{i\pi/8} e^{iX/3} \int_0^{\infty} e^{-Xs^3/3}\, ds\right].
\]

Using the substitution $t = Xs^3/3$, so $s = (3t/X)^{1/3}$ and $ds = \frac{1}{3}(3/X)^{1/3} t^{-2/3}\, dt$:
\[
\int_0^{\infty} e^{-Xs^3/3}\, ds = \frac{1}{3}\left(\frac{3}{X}\right)^{1/3} \int_0^{\infty} t^{-2/3} e^{-t}\, dt = \frac{3^{1/3}}{3 X^{1/3}} \Gamma\left(\frac{1}{3}\right) = \frac{\Gamma(1/3)}{3^{2/3} X^{1/3}}.
\]

Thus:
\[
I_1(X) \sim \text{Re}\left[\frac{2^{1/4} \Gamma(1/3)}{3^{2/3} X^{1/3}} e^{i(X/3 + \pi/8)}\right].
\]

\textbf{Contribution from the path $y = \sqrt{3}x + 1$, $x < 0$:}

Along this path, set $z = i + e^{-2\pi i/3} s$ where $s \in [0, \infty)$ is real (note the direction is into the third quadrant). Then:
\[
(z - i)^3 = e^{-2\pi i} s^3 = s^3.
\]

So:
\[
\Phi(z) \approx \frac{i}{3} - \frac{s^3}{3}.
\]

The integral contribution is:
\[
I_2(X) \sim \text{Re}\left[\sqrt{1+i} \int_{\infty}^{0} e^{X(i/3 - s^3/3)} e^{-2\pi i/3}\, ds\right] = -\text{Re}\left[2^{1/4} e^{i\pi/8} e^{-2\pi i/3} e^{iX/3} \int_0^{\infty} e^{-Xs^3/3}\, ds\right].
\]

Thus:
\[
I_2(X) \sim -\text{Re}\left[\frac{2^{1/4} \Gamma(1/3)}{3^{2/3} X^{1/3}} e^{i(X/3 + \pi/8 - 2\pi/3)}\right].
\]

\textbf{Step 7: Combine the contributions.}

The total asymptotic contribution is:
\[
I(X) = I_1(X) + I_2(X) \sim \frac{2^{1/4} \Gamma(1/3)}{3^{2/3} X^{1/3}} \text{Re}\left[e^{i(X/3 + \pi/8)} - e^{i(X/3 + \pi/8 - 2\pi/3)}\right].
\]

Let $\theta = X/3 + \pi/8$. Then:
\[
e^{i\theta} - e^{i(\theta - 2\pi/3)} = e^{i\theta}\left(1 - e^{-2\pi i/3}\right).
\]

Now:
\[
1 - e^{-2\pi i/3} = 1 - \left(-\frac{1}{2} - \frac{\sqrt{3}}{2}i\right) = \frac{3}{2} + \frac{\sqrt{3}}{2}i = \sqrt{3}\left(\frac{\sqrt{3}}{2} + \frac{1}{2}i\right) = \sqrt{3}\, e^{i\pi/6}.
\]

Therefore:
\[
e^{i\theta} - e^{i(\theta - 2\pi/3)} = \sqrt{3}\, e^{i(\theta + \pi/6)} = \sqrt{3}\, e^{i(X/3 + \pi/8 + \pi/6)}.
\]

Simplifying the phase:
\[
\frac{\pi}{8} + \frac{\pi}{6} = \frac{3\pi + 4\pi}{24} = \frac{7\pi}{24}.
\]

So:
\[
I(X) \sim \frac{2^{1/4} \Gamma(1/3)}{3^{2/3} X^{1/3}} \cdot \sqrt{3} \cdot \text{Re}\left[e^{i(X/3 + 7\pi/24)}\right] = \frac{2^{1/4} \sqrt{3}\, \Gamma(1/3)}{3^{2/3} X^{1/3}} \cos\left(\frac{X}{3} + \frac{7\pi}{24}\right).
\]

\textbf{Step 8: Simplify the coefficient.}

\[
\frac{\sqrt{3}}{3^{2/3}} = \frac{3^{1/2}}{3^{2/3}} = 3^{1/2 - 2/3} = 3^{-1/6} = \frac{1}{3^{1/6}}.
\]

Therefore, the coefficient becomes:
\[
\frac{2^{1/4} \Gamma(1/3)}{3^{1/6} X^{1/3}}.
\]

\textbf{Step 9: Final result.}

\[
\boxed{I(X) \sim \frac{2^{1/4} \Gamma(1/3)}{3^{1/6} X^{1/3}} \cos\left(\frac{X}{3} + \frac{7\pi}{24}\right) \quad \text{as } X \to \infty.}
\]

\bigskip

\textbf{Remarks:}

\begin{enumerate}
    \item \textbf{Order of the saddle point:} The saddle point at $z = i$ is of order two because $\Phi'(i) = \Phi''(i) = 0$ but $\Phi'''(i) = -2 \neq 0$. This is sometimes called a ``monkey saddle'' or a degenerate saddle point.

    \item \textbf{Asymptotic order:} For a standard (order-one) saddle point, the asymptotic expansion gives $O(X^{-1/2})$ behavior. For an order-two saddle point, the expansion instead gives $O(X^{-1/3})$ behavior, which decays more slowly. This is reflected in the appearance of $X^{1/3}$ in the denominator and $\Gamma(1/3)$ from the integral $\int_0^\infty t^{-2/3} e^{-t}\, dt$.

    \item \textbf{Phase of the oscillation:} The argument of the cosine is $X/3 + 7\pi/24$, not $X$ as one might naively expect. The factor of $1/3$ comes from the value $\text{Im}[\Phi(i)] = 1/3$, and the constant phase shift $7\pi/24$ arises from combining the phase of $\sqrt{1+i} = 2^{1/4} e^{i\pi/8}$ with the geometric factor $e^{i\pi/6}$ from the steepest descent analysis.

    \item \textbf{Why the naive approach fails:} Treating this as a standard Laplace integral with a maximum at $t = 1$ (or any other point on the real axis) would give incorrect results because the $\cos(Xt^2)$ term creates rapid oscillations that fundamentally change the character of the integral. The proper treatment requires the method of steepest descent in the complex plane.

    \item \textbf{Numerical value:} For reference, $\Gamma(1/3) \approx 2.6789$, $2^{1/4} \approx 1.1892$, and $3^{1/6} \approx 1.2009$, so the coefficient $\frac{2^{1/4}\Gamma(1/3)}{3^{1/6}} \approx 2.653$.
\end{enumerate}

\end{document}
