\documentclass[11pt,a4paper]{article}
\usepackage[margin=1in]{geometry}
\usepackage{amsmath,amssymb,amsthm}
\usepackage{mathtools}
\usepackage{enumitem}
\usepackage{xcolor}

% Custom commands
\newcommand{\stage}[1]{\textbf{\textcolor{blue}{#1}}}

\title{Question 2: Regular Perturbation of Linear Oscillator\\
Complete Solution with Exact Comparison}
\author{Asymptotics Course --- Sheet 6}
\date{}

\begin{document}

\maketitle

\section*{Problem Statement}
Obtain a two-term expansion when $\epsilon \ll 1$ for the solution of:
\begin{equation}
\frac{d^2 f}{dt^2} + f = \epsilon \frac{df}{dt}, \quad f(0) = 1, \quad \frac{df}{dt}(0) = \frac{\epsilon}{2}
\end{equation}
Compare this expansion with that obtained by expanding the exact solution.

\section{Step 1: Problem Classification and Method Selection}

\subsection*{Form Recognition}

The ODE has the structure:
\begin{equation}
\frac{d^2 f}{dt^2} + f = \epsilon \frac{df}{dt}
\end{equation}
with initial conditions:
\begin{align}
f(0) &= 1 \\
f'(0) &= \frac{\epsilon}{2}
\end{align}

\begin{itemize}[leftmargin=*]
\item \stage{STAGE X (What we have):} This is a second-order linear ODE with a small parameter $\epsilon$ multiplying the first derivative term on the right-hand side. The initial velocity $f'(0) = \epsilon/2$ is also of order $\epsilon$. The unperturbed equation ($\epsilon = 0$) is the simple harmonic oscillator $f'' + f = 0$, which has well-defined solutions.

\item \stage{STAGE Y (Why this classification):} When $\epsilon = 0$, we obtain $f'' + f = 0$ with $f(0) = 1$ and $f'(0) = 0$. This has the solution $f_0(t) = \cos t$. The perturbed problem should smoothly approach this solution as $\epsilon \to 0$. The highest derivative is NOT multiplied by $\epsilon$, so this is a \textbf{regular perturbation problem}, not a singular one.

\item \stage{STAGE Z (What this means):} We can use a standard power series expansion in $\epsilon$. The solution will remain smooth and bounded as $\epsilon \to 0$. We expect the expansion to be uniformly valid for moderate times (though we must check for secular terms).
\end{itemize}

\section{Step 2: Perturbation Expansion Setup}

\subsection*{Expansion Ansatz}

We seek a solution of the form:
\begin{equation}
f(t) = f_0(t) + \epsilon f_1(t) + \epsilon^2 f_2(t) + \mathcal{O}(\epsilon^3)
\end{equation}

\begin{itemize}[leftmargin=*]
\item \stage{STAGE X (What we're doing):} We assume the solution can be expanded as a power series in $\epsilon$, where each coefficient function $f_n(t)$ is independent of $\epsilon$.

\item \stage{STAGE Y (Why this works):} For regular perturbation problems, the solution depends smoothly on $\epsilon$. By Taylor's theorem for functions of a parameter, such a series expansion exists and converges for sufficiently small $\epsilon$.

\item \stage{STAGE Z (Next step):} Substitute this ansatz into both the ODE and the initial conditions, then collect terms by powers of $\epsilon$.
\end{itemize}

\section{Step 3: Substitute and Collect Terms}

\subsection*{Derivatives of the Ansatz}

From equation (6):
\begin{align}
f'(t) &= f_0'(t) + \epsilon f_1'(t) + \epsilon^2 f_2'(t) + \mathcal{O}(\epsilon^3) \\
f''(t) &= f_0''(t) + \epsilon f_1''(t) + \epsilon^2 f_2''(t) + \mathcal{O}(\epsilon^3)
\end{align}

\subsection*{Substitution into ODE}

Insert into $f'' + f = \epsilon f'$:
\begin{multline}
\left[f_0'' + \epsilon f_1'' + \epsilon^2 f_2'' + \mathcal{O}(\epsilon^3)\right] + \left[f_0 + \epsilon f_1 + \epsilon^2 f_2 + \mathcal{O}(\epsilon^3)\right] \\
= \epsilon\left[f_0' + \epsilon f_1' + \epsilon^2 f_2' + \mathcal{O}(\epsilon^3)\right]
\end{multline}

Rearranging:
\begin{equation}
(f_0'' + f_0) + \epsilon(f_1'' + f_1 - f_0') + \epsilon^2(f_2'' + f_2 - f_1') + \mathcal{O}(\epsilon^3) = 0
\end{equation}

\begin{itemize}[leftmargin=*]
\item \stage{STAGE X (What we observe):} After substitution, we have a polynomial in $\epsilon$ equal to zero.

\item \stage{STAGE Y (Why we can equate coefficients):} For the equation to hold for all small $\epsilon$, each power of $\epsilon$ must independently vanish. This is because polynomials can only be identically zero if all coefficients are zero.

\item \stage{STAGE Z (Result):} We obtain a hierarchy of ODEs, one for each order of $\epsilon$.
\end{itemize}

\subsection*{Initial Conditions Hierarchy}

Expand $f(0) = 1$:
\begin{equation}
f_0(0) + \epsilon f_1(0) + \epsilon^2 f_2(0) + \cdots = 1
\end{equation}

Expand $f'(0) = \frac{\epsilon}{2}$:
\begin{equation}
f_0'(0) + \epsilon f_1'(0) + \epsilon^2 f_2'(0) + \cdots = \frac{\epsilon}{2}
\end{equation}

\begin{itemize}[leftmargin=*]
\item \stage{STAGE X (Initial conditions by order):} Equating powers of $\epsilon$ in the initial conditions:
\begin{align*}
\mathcal{O}(1): &\quad f_0(0) = 1, \quad f_0'(0) = 0 \\
\mathcal{O}(\epsilon): &\quad f_1(0) = 0, \quad f_1'(0) = \frac{1}{2} \\
\mathcal{O}(\epsilon^2): &\quad f_2(0) = 0, \quad f_2'(0) = 0
\end{align*}

\item \stage{STAGE Y (Why this structure):} The condition $f'(0) = \epsilon/2$ means the initial velocity is of order $\epsilon$. When we expand this in powers of $\epsilon$, the coefficient $1/2$ appears at order $\mathcal{O}(\epsilon)$, giving $f_1'(0) = 1/2$. This is crucial for determining the correct initial conditions at each order.

\item \stage{STAGE Z (Ready to solve):} We now have complete ODE problems at each order.
\end{itemize}

\section{Step 4: Solve Order by Order}

\subsection*{Order $\mathcal{O}(1)$: Leading Order Problem}

\textbf{ODE and Initial Conditions:}
\begin{equation}
f_0'' + f_0 = 0, \quad f_0(0) = 1, \quad f_0'(0) = 0
\end{equation}

\textbf{General Solution:}
The characteristic equation is $r^2 + 1 = 0 \Rightarrow r = \pm i$, giving:
\begin{equation}
f_0(t) = A\cos t + B\sin t
\end{equation}

\textbf{Apply Initial Conditions:}
\begin{align}
f_0(0) = A = 1 &\Rightarrow A = 1 \\
f_0'(t) = -A\sin t + B\cos t \Rightarrow f_0'(0) = B = 0 &\Rightarrow B = 0
\end{align}

\textbf{Leading Order Solution:}
\begin{equation}
\boxed{f_0(t) = \cos t}
\end{equation}

\begin{itemize}[leftmargin=*]
\item \stage{STAGE X (What we found):} The unperturbed solution is pure harmonic oscillation with unit amplitude and zero initial velocity.

\item \stage{STAGE Y (Physical meaning):} This represents an undamped oscillator starting from maximum displacement. The perturbation term $\epsilon f'$ represents weak forcing proportional to velocity.

\item \stage{STAGE Z (Next order):} This solution becomes the forcing term for the next order.
\end{itemize}

\subsection*{Order $\mathcal{O}(\epsilon)$: First Correction}

\textbf{ODE and Initial Conditions:}
\begin{equation}
f_1'' + f_1 = f_0' = -\sin t, \quad f_1(0) = 0, \quad f_1'(0) = \frac{1}{2}
\end{equation}

\begin{itemize}[leftmargin=*]
\item \stage{STAGE X (What we have):} An inhomogeneous ODE where the forcing term is $-\sin t$, which is a solution of the homogeneous equation.

\item \stage{STAGE Y (Why this matters - Resonance):} The forcing frequency equals the natural frequency. This is a resonance condition that will produce a \textbf{secular term} (a term growing linearly with time).

\item \stage{STAGE Z (Expect secular growth):} The particular solution will contain a term proportional to $t\cos t$.
\end{itemize}

\textbf{Solution Method:}

The homogeneous solution is:
\begin{equation}
f_{1,h}(t) = C\cos t + D\sin t
\end{equation}

For the particular solution, since $\sin t$ is a homogeneous solution, we use the ansatz:
\begin{equation}
f_{1,p}(t) = t(E\cos t + F\sin t)
\end{equation}

\textbf{Compute derivatives:}
\begin{align}
f_{1,p}'(t) &= (E\cos t + F\sin t) + t(-E\sin t + F\cos t) \\
&= E\cos t + F\sin t - Et\sin t + Ft\cos t
\end{align}

\begin{align}
f_{1,p}''(t) &= -E\sin t + F\cos t + [-E\sin t - Et\cos t + F\cos t - Ft\sin t] \\
&= -2E\sin t + 2F\cos t - Et\cos t - Ft\sin t
\end{align}

\textbf{Substitute into ODE:}
\begin{align}
f_{1,p}'' + f_{1,p} &= -2E\sin t + 2F\cos t - Et\cos t - Ft\sin t + Et\cos t + Ft\sin t \\
&= -2E\sin t + 2F\cos t
\end{align}

Setting this equal to $-\sin t$:
\begin{align}
-2E\sin t + 2F\cos t &= -\sin t \\
\Rightarrow \quad E = \frac{1}{2}, \quad F &= 0
\end{align}

\textbf{Particular solution:}
\begin{equation}
f_{1,p}(t) = \frac{t}{2}\cos t
\end{equation}

\textbf{General solution for $f_1$:}
\begin{equation}
f_1(t) = C\cos t + D\sin t + \frac{t}{2}\cos t
\end{equation}

\textbf{Apply Initial Conditions:}
\begin{align}
f_1(0) &= C = 0 \Rightarrow C = 0 \\
f_1'(t) &= -C\sin t + D\cos t + \frac{1}{2}\cos t - \frac{t}{2}\sin t \\
f_1'(0) &= D + \frac{1}{2} = \frac{1}{2} \Rightarrow D = 0
\end{align}

\textbf{First Order Correction:}
\begin{equation}
\boxed{f_1(t) = \frac{t}{2}\cos t}
\end{equation}

\begin{itemize}[leftmargin=*]
\item \stage{STAGE X (What we found):} The first correction is purely a secular term---a resonant response that grows linearly with time.

\item \stage{STAGE Y (Why $D = 0$):} The non-zero initial velocity $f_1'(0) = 1/2$ exactly cancels with the $1/2$ contribution from the particular solution's derivative at $t = 0$, leaving $D = 0$.

\item \stage{STAGE Z (Simplification):} The solution is remarkably clean: no oscillatory correction at order $\epsilon$, only secular growth.
\end{itemize}

\section{Step 5: Combine the Two-Term Expansion}

\textbf{Two-term perturbation expansion:}
\begin{align}
f(t) &= f_0(t) + \epsilon f_1(t) + \mathcal{O}(\epsilon^2) \\
&= \cos t + \epsilon \cdot \frac{t}{2}\cos t + \mathcal{O}(\epsilon^2)
\end{align}

\begin{equation}
\boxed{f(t) = \cos t + \frac{\epsilon t}{2}\cos t + \mathcal{O}(\epsilon^2)}
\end{equation}

This can also be written as:
\begin{equation}
f(t) = \left(1 + \frac{\epsilon t}{2}\right)\cos t + \mathcal{O}(\epsilon^2)
\end{equation}

\begin{itemize}[leftmargin=*]
\item \stage{STAGE X (What we have):} A two-term expansion with a secular term that represents amplitude growth.

\item \stage{STAGE Y (Secular term meaning):} The factor $(1 + \epsilon t/2)$ suggests exponential growth $e^{\epsilon t/2} \approx 1 + \epsilon t/2$ for small $\epsilon t$.

\item \stage{STAGE Z (Validity):} This expansion is valid for $\epsilon t \ll 1$, i.e., $t \ll 1/\epsilon$.
\end{itemize}

\section{Step 6: Find the Exact Solution}

\subsection*{Characteristic Equation}

Rewrite the ODE:
\begin{equation}
f'' - \epsilon f' + f = 0
\end{equation}

The characteristic equation is:
\begin{equation}
m^2 - \epsilon m + 1 = 0
\end{equation}

Using the quadratic formula:
\begin{equation}
m = \frac{\epsilon \pm \sqrt{\epsilon^2 - 4}}{2} = \frac{\epsilon}{2} \pm \frac{1}{2}\sqrt{\epsilon^2 - 4}
\end{equation}

For small $\epsilon$ where $\epsilon^2 < 4$:
\begin{equation}
m = \frac{\epsilon}{2} \pm \frac{i}{2}\sqrt{4 - \epsilon^2}
\end{equation}

\begin{itemize}[leftmargin=*]
\item \stage{STAGE X (Complex roots):} The roots are complex conjugates for $|\epsilon| < 2$.

\item \stage{STAGE Y (Oscillatory solution):} Complex roots indicate oscillatory behavior with exponential amplitude modulation.

\item \stage{STAGE Z (Exact form):} We can write the solution in terms of exponentials or trigonometric functions.
\end{itemize}

Let:
\begin{align}
\alpha &= \frac{\epsilon}{2} \\
\omega &= \frac{\sqrt{4 - \epsilon^2}}{2} = \frac{1}{2}\sqrt{4 - \epsilon^2}
\end{align}

\textbf{General solution:}
\begin{equation}
f(t) = e^{\alpha t}(A\cos(\omega t) + B\sin(\omega t))
\end{equation}

\subsection*{Apply Initial Conditions}

\textbf{Condition 1:} $f(0) = 1$
\begin{equation}
f(0) = e^0(A\cos 0 + B\sin 0) = A = 1 \Rightarrow A = 1
\end{equation}

\textbf{Condition 2:} $f'(0) = \frac{\epsilon}{2}$

First compute $f'(t)$:
\begin{align}
f'(t) &= \alpha e^{\alpha t}(A\cos(\omega t) + B\sin(\omega t)) \\
&\quad + e^{\alpha t}(-A\omega\sin(\omega t) + B\omega\cos(\omega t))
\end{align}

At $t = 0$:
\begin{equation}
f'(0) = \alpha A + B\omega = \frac{\epsilon}{2} \cdot 1 + B\omega = \frac{\epsilon}{2}
\end{equation}

Solving for $B$:
\begin{equation}
B\omega = \frac{\epsilon}{2} - \frac{\epsilon}{2} = 0 \Rightarrow B = 0
\end{equation}

\textbf{Exact solution:}
\begin{equation}
\boxed{f_{\text{exact}}(t) = e^{\frac{\epsilon t}{2}}\cos\left(\frac{t}{2}\sqrt{4-\epsilon^2}\right)}
\end{equation}

\begin{itemize}[leftmargin=*]
\item \stage{STAGE X (Simple form):} The exact solution has a remarkably simple form because $B = 0$.

\item \stage{STAGE Y (Why $B = 0$):} The initial condition $f'(0) = \epsilon/2$ was chosen precisely so that the sine coefficient vanishes. This is what makes the comparison with the perturbation expansion particularly clean.

\item \stage{STAGE Z (Structure):} The solution is an exponentially growing cosine with a slightly modified frequency.
\end{itemize}

\section{Step 7: Expand the Exact Solution}

\subsection*{Expand Each Component}

\textbf{1. Exponential term:}
\begin{equation}
e^{\frac{\epsilon t}{2}} = 1 + \frac{\epsilon t}{2} + \frac{\epsilon^2 t^2}{8} + \mathcal{O}(\epsilon^3)
\end{equation}

\textbf{2. Frequency term:}
\begin{align}
\omega &= \frac{1}{2}\sqrt{4 - \epsilon^2} = \frac{1}{2} \cdot 2\sqrt{1 - \frac{\epsilon^2}{4}} \\
&= 1 \cdot \left(1 - \frac{\epsilon^2}{8} + \mathcal{O}(\epsilon^4)\right) = 1 - \frac{\epsilon^2}{8} + \mathcal{O}(\epsilon^4)
\end{align}

\begin{itemize}[leftmargin=*]
\item \stage{STAGE X (Binomial expansion):} Used $(1+x)^{1/2} = 1 + \frac{1}{2}x - \frac{1}{8}x^2 + \cdots$ with $x = -\epsilon^2/4$.

\item \stage{STAGE Y (Why to order $\epsilon^2$):} We need $\omega$ accurate to $\mathcal{O}(\epsilon^2)$ to get $f$ correct to $\mathcal{O}(\epsilon)$.

\item \stage{STAGE Z (Apply to trig functions):} The argument of cosine becomes $t(1 - \epsilon^2/8) = t - \epsilon^2 t/8$.
\end{itemize}

\textbf{3. Cosine term:}
\begin{align}
\cos(\omega t) &= \cos\left(t\left[1 - \frac{\epsilon^2}{8}\right]\right) = \cos\left(t - \frac{\epsilon^2 t}{8}\right) \\
&= \cos t \cos\left(\frac{\epsilon^2 t}{8}\right) + \sin t \sin\left(\frac{\epsilon^2 t}{8}\right) \\
&= \cos t \left[1 - \frac{1}{2}\left(\frac{\epsilon^2 t}{8}\right)^2 + \cdots\right] + \sin t \left[\frac{\epsilon^2 t}{8} + \cdots\right] \\
&= \cos t + \mathcal{O}(\epsilon^2)
\end{align}

\subsection*{Combine All Terms}

\begin{align}
f_{\text{exact}}(t) &= e^{\frac{\epsilon t}{2}} \cos(\omega t) \\
&= \left(1 + \frac{\epsilon t}{2} + \mathcal{O}(\epsilon^2)\right)\left(\cos t + \mathcal{O}(\epsilon^2)\right) \\
&= \cos t + \frac{\epsilon t}{2}\cos t + \mathcal{O}(\epsilon^2)
\end{align}

\textbf{Expanded exact solution:}
\begin{equation}
\boxed{f_{\text{exact}}(t) = \cos t + \frac{\epsilon t}{2}\cos t + \mathcal{O}(\epsilon^2)}
\end{equation}

\section{Step 8: Comparison and Verification}

\subsection*{Direct Comparison}

\textbf{Perturbation expansion result (from Step 5):}
\begin{equation}
f_{\text{pert}}(t) = \cos t + \frac{\epsilon t}{2}\cos t + \mathcal{O}(\epsilon^2)
\end{equation}

\textbf{Expanded exact solution (from Step 7):}
\begin{equation}
f_{\text{exact}}(t) = \cos t + \frac{\epsilon t}{2}\cos t + \mathcal{O}(\epsilon^2)
\end{equation}

\begin{center}
\fbox{\parbox{0.9\textwidth}{
\textbf{PERFECT AGREEMENT}: The two-term perturbation expansion matches the expansion of the exact solution identically to $\mathcal{O}(\epsilon)$.
}}
\end{center}

\begin{itemize}[leftmargin=*]
\item \stage{STAGE X (What we verified):} Both methods give identical results through order $\epsilon$.

\item \stage{STAGE Y (Why this confirms validity):} The perturbation method is correct for this regular problem. The secular term $\frac{\epsilon t}{2}\cos t$ is a genuine feature of the exact solution, not an artifact of the method.

\item \stage{STAGE Z (Physical interpretation):} The term $\epsilon f'$ on the right-hand side acts as anti-damping (energy input), causing:
\begin{enumerate}
\item Exponential amplitude growth captured by $e^{\epsilon t/2} \approx 1 + \epsilon t/2$
\item A slight frequency shift of order $\mathcal{O}(\epsilon^2)$, not visible at this order
\end{enumerate}
\end{itemize}

\subsection*{Key Observations}

\textbf{1. Secular Term Resolution:}

The secular term $\frac{\epsilon t}{2}\cos t$ appears unbounded as $t \to \infty$, but in the exact solution it's part of:
\begin{equation}
e^{\frac{\epsilon t}{2}}\cos(\omega t) \approx \left(1 + \frac{\epsilon t}{2}\right)\cos t
\end{equation}

The linear growth $1 + \epsilon t/2$ is actually the leading-order expansion of the bounded exponential $e^{\epsilon t/2}$, representing physical amplitude growth due to the anti-damping effect.

\textbf{2. Validity Domain:}

The expansion is uniformly valid for:
\begin{equation}
t \ll \frac{1}{\epsilon}
\end{equation}

For $t = \mathcal{O}(1/\epsilon)$, the $\mathcal{O}(\epsilon)$ term becomes $\mathcal{O}(1)$ and the expansion breaks down. The multiple scales method would be needed for longer times.

\textbf{3. Initial Condition Impact:}

The initial condition $f'(0) = \epsilon/2$ was specifically chosen so that:
\begin{itemize}
\item The leading order has zero initial velocity: $f_0'(0) = 0$
\item The first correction has initial velocity $f_1'(0) = 1/2$
\item This exactly cancels the contribution from the particular solution, giving $D = 0$
\item In the exact solution, this makes $B = 0$, yielding a pure growing cosine
\end{itemize}

\section{Verification Checklist}

\begin{enumerate}[leftmargin=*]
\item[$\checkmark$] \textbf{Problem classified:} Regular perturbation (highest derivative not multiplied by $\epsilon$)
\item[$\checkmark$] \textbf{Expansion ansatz:} $f = f_0 + \epsilon f_1 + \mathcal{O}(\epsilon^2)$
\item[$\checkmark$] \textbf{Initial conditions distributed:} $f_0(0) = 1$, $f_0'(0) = 0$; $f_1(0) = 0$, $f_1'(0) = 1/2$
\item[$\checkmark$] \textbf{Order $\mathcal{O}(1)$:} $f_0 = \cos t$
\item[$\checkmark$] \textbf{Order $\mathcal{O}(\epsilon)$:} $f_1 = \frac{t}{2}\cos t$ (secular term only)
\item[$\checkmark$] \textbf{Exact solution found:} $f = e^{\epsilon t/2}\cos(\omega t)$ with $B = 0$
\item[$\checkmark$] \textbf{Exact solution expanded:} Careful Taylor expansion to $\mathcal{O}(\epsilon)$
\item[$\checkmark$] \textbf{Perfect agreement:} Perturbation and exact expansion match
\item[$\checkmark$] \textbf{Secular term explained:} Genuine feature from exponential growth $e^{\epsilon t/2}$
\item[$\checkmark$] \textbf{Validity domain identified:} $t \ll 1/\epsilon$
\end{enumerate}

\vspace{1em}
\textit{This solution demonstrates that regular perturbation theory correctly captures the behavior of weakly anti-damped oscillators for moderate times, including secular terms that represent physical exponential growth.}

\end{document}
