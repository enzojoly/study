\documentclass[11pt,a4paper]{article}
\usepackage{inputenc}
\usepackage{amsmath,amssymb,amsthm}
\usepackage[margin=2.5cm]{geometry}
\usepackage{enumitem}
\usepackage{xcolor}

% Custom environments for pedagogical structure
\newtheoremstyle{problem}
  {10pt}{10pt}{\normalfont}{}{\bfseries}{.}{.5em}{}
\theoremstyle{problem}
\newtheorem{problem}{Problem}

\newenvironment{strategy}{\par\noindent\textbf{Strategy:}\itshape}{\par}
\newenvironment{justification}{\par\noindent\textbf{Justification:}\itshape}{\par}
\newenvironment{technique}{\par\noindent\textbf{Technique:}\itshape}{\par}
\newenvironment{reflection}{\par\noindent\textbf{Reflection:}\itshape}{\par}
\newenvironment{keyconcept}{\par\noindent\textbf{Key Concept:}\itshape}{\par}

\title{Asymptotics Problem 7.3: Complete Pedagogical Solution}
\author{Perturbed WKB: Multiple Scaling Regimes}
\date{}

\begin{document}

\maketitle

\begin{problem}
For the equation
\[
\varepsilon^2 y''(x) + [q(x) + \chi(\varepsilon)r(x)]y(x) = 0
\]
deduce the leading order solution as $\varepsilon \to 0$ for those cases where $\chi(\varepsilon)$ is $o(1)$ and yet not so small that the standard WKB solution is appropriate. Consider all relevant cases.
\end{problem}

\section*{Solution: Step-by-Step Atomic Breakdown}

\subsection*{Step 1: Understanding the Problem Structure}

\begin{strategy}
We have a perturbation of the standard WKB equation. The standard form $\varepsilon^2 y'' + q(x)y = 0$ now has an additional term $\chi(\varepsilon)r(x)y$. Our task is to determine:
\begin{enumerate}[leftmargin=*]
\item Under what conditions on $\chi(\varepsilon)$ does this term affect the leading order solution?
\item How to modify the WKB solution to account for this perturbation?
\item What are the different regimes based on the size of $\chi(\varepsilon)$?
\end{enumerate}
\end{strategy}

\begin{justification}
The problem states that $\chi(\varepsilon) = o(1)$, meaning $\chi(\varepsilon) \to 0$ as $\varepsilon \to 0$. However, we're told it's ``not so small that the standard WKB solution is appropriate.'' This suggests there's a critical threshold: if $\chi$ is too small, it has no effect; if it's large enough (but still $\to 0$), it modifies the solution. We need to identify this threshold.
\end{justification}

\subsection*{Step 2: Recalling the Standard WKB Solution}

\noindent\textbf{Standard case:} For $\varepsilon^2 y'' + q(x)y = 0$, the WKB solution (from Problem 2 and Lecture Notes \S6.3.2) is:
\[
y_\pm(x) \sim \frac{A_\pm}{q(x)^{1/4}}\exp\left(\pm\frac{i}{\varepsilon}\int^x\sqrt{q(s)}\,ds\right) \quad \text{as } \varepsilon \to 0.
\]

\begin{keyconcept}
The WKB solution has two key components:
\begin{itemize}
\item \textbf{Amplitude:} $A(x) = q(x)^{-1/4}$ (varies slowly with $x$)
\item \textbf{Phase:} $S(x) = \int^x\sqrt{q(s)}\,ds$ (oscillates rapidly with frequency $\sim 1/\varepsilon$)
\end{itemize}
The phase accumulates at rate $\sqrt{q(x)}$ per unit $x$, so the solution oscillates with local wavelength $\sim \varepsilon/\sqrt{q(x)}$.
\end{keyconcept}

\subsection*{Step 3: Setting Up the WKB Ansatz for the Perturbed Problem}

\noindent\textbf{What we do:} Apply the WKB ansatz to the perturbed equation.

\begin{technique}
The WKB method assumes solutions of the form:
\[
y(x) = A(x)\exp\left(\frac{i}{\varepsilon}S(x)\right),
\]
where $S(x)$ is the phase function and $A(x)$ is the slowly-varying amplitude. Both $S$ and $A$ may depend on $\varepsilon$ implicitly through the equation's structure.
\end{technique}

\subsection*{Step 4: Computing Derivatives of the WKB Ansatz}

\noindent\textbf{What we need:} $y'$ and $y''$ for $y = A\exp(iS/\varepsilon)$.

\subsubsection*{Step 4a: First Derivative}

\begin{technique}
Use the product rule and chain rule:
\begin{align*}
y' &= A' \exp\left(\frac{iS}{\varepsilon}\right) + A \cdot \frac{iS'}{\varepsilon}\exp\left(\frac{iS}{\varepsilon}\right)\\
&= \left(A' + \frac{iS'}{\varepsilon}A\right)\exp\left(\frac{iS}{\varepsilon}\right).
\end{align*}
\end{technique}

\subsubsection*{Step 4b: Second Derivative}

\noindent Differentiate $y' = (A' + iS'A/\varepsilon)e^{iS/\varepsilon}$:
\begin{align*}
y'' &= \left(A'' + \frac{iS''}{\varepsilon}A + \frac{iS'}{\varepsilon}A'\right)e^{iS/\varepsilon} + \left(A' + \frac{iS'}{\varepsilon}A\right) \cdot \frac{iS'}{\varepsilon}e^{iS/\varepsilon}\\
&= \left[A'' + \frac{2iS'}{\varepsilon}A' + \frac{iS''}{\varepsilon}A + \left(\frac{iS'}{\varepsilon}\right)^2A\right]e^{iS/\varepsilon}\\
&= \left[A'' + \frac{2iS'}{\varepsilon}A' + \frac{iS''}{\varepsilon}A - \frac{(S')^2}{\varepsilon^2}A\right]e^{iS/\varepsilon}.
\end{align*}

\subsection*{Step 5: Substituting into the ODE}

\noindent Substitute into $\varepsilon^2 y'' + [q(x) + \chi(\varepsilon)r(x)]y = 0$:
\begin{align*}
&\varepsilon^2\left[A'' + \frac{2iS'}{\varepsilon}A' + \frac{iS''}{\varepsilon}A - \frac{(S')^2}{\varepsilon^2}A\right]e^{iS/\varepsilon}\\
&\qquad + [q(x) + \chi(\varepsilon)r(x)]Ae^{iS/\varepsilon} = 0.
\end{align*}

\noindent Dividing by $e^{iS/\varepsilon}$ (which is never zero):
\[
\varepsilon^2 A'' + 2i\varepsilon S'A' + i\varepsilon S''A - (S')^2A + [q + \chi r]A = 0.
\]

\noindent Rearranging by powers of $\varepsilon$:
\[
-(S')^2A + [q + \chi r]A + \varepsilon[2iS'A' + iS''A] + \varepsilon^2 A'' = 0.
\]

\subsection*{Step 6: Identifying the Leading Order Balance}

\begin{strategy}
In the WKB method, we solve hierarchically by examining different orders in $\varepsilon$. The key question is: \emph{At what order does $\chi r$ appear?}

This depends on the relative size of $\chi(\varepsilon)$ compared to powers of $\varepsilon$.
\end{strategy}

\subsubsection*{Step 6a: The $O(\varepsilon^0) = O(1)$ Terms}

\noindent At leading order (collecting all terms not explicitly multiplied by $\varepsilon$ or $\varepsilon^2$):
\[
-(S')^2 + q + \chi r = 0.
\]

\begin{justification}
This is the \emph{eikonal equation} (or Hamilton-Jacobi equation) for the phase $S(x)$. It determines how rapidly the phase accumulates as we move through $x$. Notice that $\chi r$ appears at this leading order because it's not multiplied by any power of $\varepsilon$ in the original equation.
\end{justification}

\noindent Solving for $S'$:
\[
(S')^2 = q(x) + \chi(\varepsilon)r(x) \quad \Longrightarrow \quad S'(x) = \pm\sqrt{q(x) + \chi(\varepsilon)r(x)}.
\]

\subsection*{Step 7: Determining When $\chi r$ Affects the Solution}

\begin{keyconcept}
The phase function is:
\[
S(x) = \pm\int^x \sqrt{q(s) + \chi(\varepsilon)r(s)}\,ds.
\]
This appears in the solution as $\exp(iS/\varepsilon)$. The term $\chi r$ affects the \emph{observable} phase if:
\[
\frac{1}{\varepsilon}\left|\int^x \chi r(s)(\text{correction})\,ds\right| = O(1),
\]
i.e., if the integrated effect is comparable to or larger than $O(\varepsilon)$.
\end{keyconcept}

\subsection*{Step 8: Expanding for Small $\chi$}

\noindent\textbf{What we do:} Since $\chi = o(1)$, expand $\sqrt{q + \chi r}$ for small $\chi$.

\begin{technique}
Use the binomial expansion $(1+u)^{1/2} \approx 1 + \frac{u}{2} - \frac{u^2}{8} + \cdots$ for $|u| \ll 1$:
\begin{align*}
\sqrt{q + \chi r} &= \sqrt{q}\sqrt{1 + \frac{\chi r}{q}}\\
&= \sqrt{q}\left(1 + \frac{\chi r}{2q} - \frac{\chi^2 r^2}{8q^2} + O(\chi^3)\right).
\end{align*}
\end{technique}

\begin{justification}
This expansion is valid when $|\chi r/q| \ll 1$, which we assume holds. If $r$ and $q$ are of similar magnitude and $\chi \to 0$, this condition is satisfied.
\end{justification}

\noindent Therefore:
\[
S'(x) = \pm\sqrt{q(x)}\left[1 + \frac{\chi(\varepsilon)r(x)}{2q(x)} + O(\chi^2)\right].
\]

\noindent Integrating:
\[
S(x) = \pm\int^x\sqrt{q(s)}\,ds \pm \frac{\chi(\varepsilon)}{2}\int^x\frac{r(s)}{\sqrt{q(s)}}\,ds + O(\chi^2).
\]

\noindent Let's define:
\begin{align*}
S_0(x) &= \int^x\sqrt{q(s)}\,ds \quad \text{(standard WKB phase)}\\
S_1(x) &= \frac{1}{2}\int^x\frac{r(s)}{\sqrt{q(s)}}\,ds \quad \text{(first-order correction)}.
\end{align*}

\noindent Then:
\[
S(x) = \pm\left[S_0(x) + \chi(\varepsilon)S_1(x) + O(\chi^2)\right].
\]

\subsection*{Step 9: Analyzing Different Regimes of $\chi(\varepsilon)$}

\begin{strategy}
The correction to the phase is $\chi S_1$, which appears in the exponent as:
\[
\exp\left(\frac{iS}{\varepsilon}\right) = \exp\left(\frac{iS_0}{\varepsilon}\right)\exp\left(\frac{i\chi S_1}{\varepsilon}\right).
\]
The second exponential modifies the solution significantly if $\chi S_1/\varepsilon = O(1)$ or larger. This motivates us to compare $\chi$ with $\varepsilon$.
\end{strategy}

\subsubsection*{Case 1: $\chi(\varepsilon) = o(\varepsilon)$ (Very Small Perturbation)}

\noindent\textbf{Condition:} $\chi(\varepsilon) \ll \varepsilon$, e.g., $\chi = \varepsilon^{1+\delta}$ for some $\delta > 0$.

\begin{justification}
In this regime:
\[
\frac{\chi S_1}{\varepsilon} = \frac{\chi}{\varepsilon} \cdot S_1 \to 0 \quad \text{as } \varepsilon \to 0.
\]
The phase correction is asymptotically negligible. We can expand:
\[
\exp\left(\frac{i\chi S_1}{\varepsilon}\right) \approx 1 + \frac{i\chi S_1}{\varepsilon} + \cdots
\]
Since $\chi/\varepsilon \to 0$, this approaches $1$ and the standard WKB solution applies.
\end{justification}

\noindent\textbf{Leading order solution:}
\[
\boxed{y(x) \sim \frac{A_\pm}{q(x)^{1/4}}\exp\left(\pm\frac{i}{\varepsilon}\int^x\sqrt{q(s)}\,ds\right) \quad \text{(Standard WKB)}}
\]

\begin{reflection}
This makes sense: if $\chi$ is much smaller than $\varepsilon$, the perturbation $\chi r$ is negligible compared to the inherent approximation error of the WKB method (which is $O(\varepsilon)$).
\end{reflection}

\subsubsection*{Case 2: $\chi(\varepsilon) \sim \varepsilon$ (Critical Perturbation)}

\noindent\textbf{Condition:} $\chi(\varepsilon)/\varepsilon \to C$ for some nonzero constant $C$, e.g., $\chi = \alpha\varepsilon$ with $\alpha = O(1)$.

\begin{justification}
In this regime:
\[
\frac{\chi S_1}{\varepsilon} = \frac{\chi}{\varepsilon} \cdot S_1 \sim C \cdot S_1 = O(1).
\]
The phase correction is $O(1)$ and cannot be neglected. The perturbation genuinely modifies the leading order WKB solution.
\end{justification}

\noindent\textbf{Leading order solution:} We must include the correction term.
\[
S(x) = S_0(x) + \chi S_1(x) + O(\chi^2) = S_0(x) + \chi S_1(x) + O(\varepsilon^2).
\]

\noindent Since $O(\varepsilon^2) = o(\varepsilon) = o(\chi)$ in this regime, we have to leading order:
\[
\boxed{y(x) \sim \frac{A_\pm}{q(x)^{1/4}}\exp\left(\pm\frac{i}{\varepsilon}\left[\int^x\sqrt{q(s)}\,ds + \frac{\chi(\varepsilon)}{2}\int^x\frac{r(s)}{\sqrt{q(s)}}\,ds\right]\right)}
\]

\begin{reflection}
This is the interesting regime the problem asks about! When $\chi \sim \varepsilon$, the perturbation affects the phase but not the amplitude at leading order. The physical interpretation: the local frequency of oscillation is modified from $\sqrt{q}/\varepsilon$ to $\sqrt{q+\chi r}/\varepsilon \approx (\sqrt{q} + \chi r/(2\sqrt{q}))/\varepsilon$.
\end{reflection}

\subsubsection*{Case 3: $\varepsilon \ll \chi(\varepsilon) \ll 1$ (Large but Vanishing Perturbation)}

\noindent\textbf{Condition:} $\chi(\varepsilon) \gg \varepsilon$ but $\chi(\varepsilon) \to 0$, e.g., $\chi = \varepsilon^\beta$ for $0 < \beta < 1$.

\begin{justification}
In this regime:
\[
\frac{\chi S_1}{\varepsilon} = \frac{\chi}{\varepsilon} \cdot S_1 \to \infty \quad \text{as } \varepsilon \to 0.
\]
The phase correction dominates. The ratio $\chi/\varepsilon \to \infty$ means we should treat the full $\sqrt{q + \chi r}$ without expanding.
\end{justification}

\noindent\textbf{Leading order solution:} Use the exact phase function without Taylor expansion:
\[
\boxed{y(x) \sim \frac{A_\pm}{[q(x) + \chi(\varepsilon)r(x)]^{1/4}}\exp\left(\pm\frac{i}{\varepsilon}\int^x\sqrt{q(s) + \chi(\varepsilon)r(s)}\,ds\right)}
\]

\begin{justification}
Why do we also modify the amplitude? From the $O(\varepsilon)$ equation (transport equation), the amplitude satisfies:
\[
2S'A' + S''A = 0 \quad \Longrightarrow \quad A \propto (S')^{-1/2} = [q + \chi r]^{-1/4}.
\]
Since $\chi r$ appears at leading order in $S'$, it also affects the amplitude at leading order.
\end{justification}

\begin{reflection}
In this regime, the perturbed coefficient $q + \chi r$ should simply replace $q$ in the WKB formula. The distinction between $q$ and $q + \chi r$ becomes important, even though $\chi r/q \to 0$.
\end{reflection}

\subsection*{Step 10: Computing the Amplitude Correction}

\noindent\textbf{What remains:} Verify the amplitude formula from the $O(\varepsilon)$ equation.

\subsubsection*{Step 10a: The Transport Equation}

\noindent From our expansion in Step 5, the $O(\varepsilon)$ terms give:
\[
2iS'A' + iS''A = 0 \quad \Longrightarrow \quad 2S'A' = -S''A.
\]

\begin{technique}
Separate variables:
\[
\frac{A'}{A} = -\frac{S''}{2S'}.
\]
Integrate both sides:
\[
\ln|A| = -\frac{1}{2}\int\frac{S''}{S'}dx = -\frac{1}{2}\ln|S'| + \text{const.}
\]
Therefore:
\[
A(x) = \frac{C}{[S'(x)]^{1/2}}.
\]
\end{technique}

\subsubsection*{Step 10b: Expressing $A$ in Terms of $q$ and $\chi r$}

\noindent Since $S'(x) = \sqrt{q(x) + \chi(\varepsilon)r(x)}$:
\[
A(x) = \frac{C}{[q(x) + \chi(\varepsilon)r(x)]^{1/4}}.
\]

\noindent For small $\chi$, we can expand:
\[
[q + \chi r]^{-1/4} = q^{-1/4}(1 + \chi r/q)^{-1/4} \approx q^{-1/4}\left(1 - \frac{\chi r}{4q}\right).
\]

\begin{justification}
This confirms our amplitude formulas:
\begin{itemize}
\item \textbf{Case 1} ($\chi \ll \varepsilon$): $A \approx q^{-1/4}$ (standard WKB amplitude)
\item \textbf{Case 2} ($\chi \sim \varepsilon$): $A \approx q^{-1/4}(1 - \chi r/(4q)) = q^{-1/4} + O(\varepsilon)$ (first-order correction)
\item \textbf{Case 3} ($\chi \gg \varepsilon$): $A = [q+\chi r]^{-1/4}$ (full modified amplitude)
\end{itemize}
\end{justification}

\subsection*{Step 11: Summary Table of All Cases}

\begin{center}
\renewcommand{\arraystretch}{2}
\begin{tabular}{|c|c|p{8cm}|}
\hline
\textbf{Regime} & \textbf{Condition} & \textbf{Leading Order Solution} \\
\hline
\textbf{Case 1} & $\chi \ll \varepsilon$ &
$y \sim \dfrac{A_\pm}{q^{1/4}}\exp\left(\pm\dfrac{i}{\varepsilon}\displaystyle\int^x\sqrt{q}\,ds\right)$

Standard WKB applies \\
\hline
\textbf{Case 2} & $\chi \sim \varepsilon$ &
$y \sim \dfrac{A_\pm}{q^{1/4}}\exp\left(\pm\dfrac{i}{\varepsilon}\left[\displaystyle\int^x\sqrt{q}\,ds + \dfrac{\chi}{2}\displaystyle\int^x\dfrac{r}{\sqrt{q}}\,ds\right]\right)$

Phase correction essential \\
\hline
\textbf{Case 3} & $\varepsilon \ll \chi \ll 1$ &
$y \sim \dfrac{A_\pm}{(q+\chi r)^{1/4}}\exp\left(\pm\dfrac{i}{\varepsilon}\displaystyle\int^x\sqrt{q+\chi r}\,ds\right)$

Full modified WKB \\
\hline
\end{tabular}
\end{center}

\subsection*{Step 12: Physical Interpretation and Key Insights}

\begin{keyconcept}
The three regimes correspond to different physical scenarios:

\begin{enumerate}
\item \textbf{Negligible perturbation} ($\chi \ll \varepsilon$): The additional term $\chi r(x)y$ is so small that it has no observable effect on the leading order solution. It's smaller than the inherent WKB approximation error.

\item \textbf{Resonant perturbation} ($\chi \sim \varepsilon$): This is the \emph{critical case} the problem asks about. The perturbation is small enough to treat perturbatively (we can expand $\sqrt{q+\chi r}$), but large enough to produce an $O(1)$ phase shift over the integration domain. This is analogous to a resonant forcing in oscillator theory.

\item \textbf{Dominant perturbation} ($\chi \gg \varepsilon$): The term $\chi r$ is comparable to $q$ in determining the solution structure. We should use the full modified coefficient $q + \chi r$ rather than treating $\chi r$ as a perturbation.
\end{enumerate}
\end{keyconcept}

\subsection*{Step 13: Connection to Lecture Material}

\begin{reflection}
This problem illustrates several key asymptotic principles from the lectures:

\begin{itemize}
\item \textbf{Multiple scales} (\S7.1): The parameter $\chi/\varepsilon$ determines which scale is relevant. When $\chi \sim \varepsilon$, both the fast oscillation scale ($\sim \varepsilon$) and the perturbation correction scale ($\sim \chi$) are important.

\item \textbf{Dominant balance} (\S2.2.2): We must identify which terms balance at leading order. For $\chi \ll \varepsilon$, the term $\chi ry$ is subdominant; for $\chi \sim \varepsilon$ or larger, it enters the dominant balance.

\item \textbf{Asymptotic sequences} (\S2.5): The natural asymptotic sequence here is not just powers of $\varepsilon$, but involves both $\varepsilon$ and $\chi(\varepsilon)$. We must understand the relative ordering.

\item \textbf{Uniform validity} (\S7.1.1): The Case 1 solution is uniformly valid for $\chi \ll \varepsilon$, but fails when $\chi \sim \varepsilon$. The Case 2 solution corrects this and is uniformly valid when $\chi \sim \varepsilon$.
\end{itemize}
\end{reflection}

\subsection*{Step 14: Verification Example}

\noindent\textbf{Concrete example:} Let $\chi(\varepsilon) = \varepsilon^\alpha$ for $0 < \alpha < 2$.

\begin{itemize}
\item If $\alpha > 1$: Then $\chi/\varepsilon = \varepsilon^{\alpha-1} \to 0$, so $\chi \ll \varepsilon$ $\Rightarrow$ \textbf{Case 1}

\item If $\alpha = 1$: Then $\chi/\varepsilon = O(1)$, so $\chi \sim \varepsilon$ $\Rightarrow$ \textbf{Case 2}

\item If $0 < \alpha < 1$: Then $\chi/\varepsilon = \varepsilon^{\alpha-1} \to \infty$, so $\chi \gg \varepsilon$ $\Rightarrow$ \textbf{Case 3}
\end{itemize}

\begin{justification}
This confirms our regime classification. The critical exponent is $\alpha = 1$, corresponding to $\chi \sim \varepsilon$.
\end{justification}

\subsection*{Final Answer Summary}

\begin{center}
\fbox{\begin{minipage}{0.95\textwidth}
\textbf{Complete Solution:}

For the equation $\varepsilon^2 y'' + [q(x) + \chi(\varepsilon)r(x)]y = 0$ with $\chi(\varepsilon) = o(1)$:

\vspace{0.3cm}
\textbf{The critical comparison is between $\chi(\varepsilon)$ and $\varepsilon$:}

\begin{enumerate}
\item If $\chi(\varepsilon) = o(\varepsilon)$: Standard WKB applies
\[
y \sim \frac{A_\pm}{q^{1/4}}\exp\left(\pm\frac{i}{\varepsilon}\int^x\sqrt{q}\,ds\right)
\]

\item If $\chi(\varepsilon) \sim \varepsilon$: Phase correction required (the non-trivial case!)
\[
y \sim \frac{A_\pm}{q^{1/4}}\exp\left(\pm\frac{i}{\varepsilon}\left[\int^x\sqrt{q}\,ds + \frac{\chi}{2}\int^x\frac{r}{\sqrt{q}}\,ds\right]\right)
\]

\item If $\chi(\varepsilon) \gg \varepsilon$ (but still $\chi \to 0$): Use modified WKB
\[
y \sim \frac{A_\pm}{(q+\chi r)^{1/4}}\exp\left(\pm\frac{i}{\varepsilon}\int^x\sqrt{q+\chi r}\,ds\right)
\]
\end{enumerate}

The phrase ``not so small that the standard WKB solution is appropriate'' refers to \textbf{Cases 2 and 3}, where $\chi \geq O(\varepsilon)$.
\end{minipage}}
\end{center}

\end{document}
