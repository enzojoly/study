\documentclass[11pt,a4paper]{article}
\usepackage{inputenc}
\usepackage{amsmath,amssymb,amsthm}
\usepackage[margin=2.5cm]{geometry}
\usepackage{enumitem}
\usepackage{xcolor}

% Custom environments for pedagogical structure
\newtheoremstyle{problem}
  {10pt}{10pt}{\normalfont}{}{\bfseries}{.}{.5em}{}
\theoremstyle{problem}
\newtheorem{problem}{Problem}

\newenvironment{strategy}{\par\noindent\textbf{Strategy:}\itshape}{\par}
\newenvironment{justification}{\par\noindent\textbf{Justification:}\itshape}{\par}
\newenvironment{technique}{\par\noindent\textbf{Technique:}\itshape}{\par}

\title{Asymptotics Problem 7.1: Complete Pedagogical Solution}
\author{WKB Method --- Elimination of First Derivative}
\date{}

\begin{document}

\maketitle

\begin{problem}
Consider the ordinary differential equation
\[
u''(x) + p(x)u'(x) + q(x)u(x) = 0.
\]
Set $u(x) = f(x)y(x)$ and find a function $f(x)$ such that the resulting ODE for $y(x)$ has no term in $y'(x)$.
\end{problem}

\section*{Solution: Step-by-Step Atomic Breakdown}

\subsection*{Step 1: Understanding the Problem}

\begin{strategy}
We are given a second-order linear ODE with non-constant coefficients. Our goal is to perform a transformation that simplifies the equation by eliminating the first derivative term. This is a standard technique in asymptotic analysis and is particularly important for:
\begin{itemize}[leftmargin=*]
\item Connecting boundary layer problems to WKB-type equations (as noted in Lecture Notes \S6.3.1, equation (377))
\item Simplifying ODEs before applying the WKB approximation
\item Converting equations into a form amenable to asymptotic analysis
\end{itemize}
\end{strategy}

\begin{justification}
Why do we want to eliminate the first derivative term? From the lecture notes \S6.3, we learned that the WKB method works most naturally on equations of the form $\varepsilon^2 y'' + q(x)y = 0$. If we encounter an equation with a $y'$ term, we need to transform it away first. This transformation reveals the underlying oscillatory or exponential structure of solutions.
\end{justification}

\subsection*{Step 2: Setting Up the Transformation}

\noindent\textbf{What we do:} We substitute $u(x) = f(x)y(x)$ into the original ODE.

\begin{justification}
This substitution is called a \emph{dependent variable transformation}. We're expressing the unknown function $u(x)$ as a product of:
\begin{itemize}
\item A known (to-be-determined) function $f(x)$ that will absorb the unwanted $y'$ term
\item A new unknown function $y(x)$ that will satisfy a simpler ODE
\end{itemize}
This is analogous to an integrating factor in first-order ODEs, but here we seek to change the structure of a second-order equation.
\end{justification}

\subsection*{Step 3: Computing the Required Derivatives}

\noindent\textbf{What we need:} To substitute $u = fy$ into the ODE, we need $u'$ and $u''$.

\subsubsection*{Step 3a: First Derivative}

\begin{technique}
Apply the product rule: If $u = f \cdot y$, then
\[
u' = \frac{d}{dx}(f \cdot y) = f' y + f y'.
\]
\end{technique}

\noindent This is straightforward differentiation. We obtain:
\[
u'(x) = f'(x)y(x) + f(x)y'(x).
\]

\subsubsection*{Step 3b: Second Derivative}

\begin{technique}
We differentiate $u' = f'y + fy'$ using the product rule on each term:
\begin{align*}
u'' &= \frac{d}{dx}(f'y + fy')\\
&= \frac{d}{dx}(f'y) + \frac{d}{dx}(fy')\\
&= (f''y + f'y') + (f'y' + fy'')\\
&= f''y + 2f'y' + fy''.
\end{align*}
\end{technique}

\noindent Therefore:
\[
u''(x) = f''(x)y(x) + 2f'(x)y'(x) + f(x)y''(x).
\]

\begin{justification}
Notice the structure: The second derivative $u''$ produces three terms involving $y$, $y'$, and $y''$. The term $2f'y'$ is particularly important --- this is where we'll have leverage to eliminate the first derivative from the transformed equation.
\end{justification}

\subsection*{Step 4: Substituting into the Original ODE}

\noindent\textbf{What we do:} Substitute $u$, $u'$, and $u''$ into $u'' + pu' + qu = 0$.

\begin{align*}
u'' + p(x)u' + q(x)u &= 0\\
\bigl[f''y + 2f'y' + fy''\bigr] + p(x)\bigl[f'y + fy'\bigr] + q(x)\bigl[fy\bigr] &= 0.
\end{align*}

\noindent Expanding the middle term:
\[
f''y + 2f'y' + fy'' + pf'y + pfy' + qfy = 0.
\]

\subsection*{Step 5: Collecting Terms by Derivative Order}

\begin{technique}
Group terms according to which derivative of $y$ they contain: $y''$, $y'$, or $y$. This reveals the structure of the transformed ODE.
\end{technique}

\noindent\textbf{Grouping by $y$ and its derivatives:}
\begin{align*}
\text{Coefficient of } y'': &\quad f\\
\text{Coefficient of } y': &\quad 2f' + pf\\
\text{Coefficient of } y: &\quad f'' + pf' + qf
\end{align*}

\noindent Therefore, the equation becomes:
\[
f \cdot y'' + (2f' + pf) \cdot y' + (f'' + pf' + qf) \cdot y = 0.
\]

\subsection*{Step 6: Converting to Standard Form}

\noindent\textbf{What we do:} Divide through by $f(x)$ (assuming $f(x) \neq 0$) to obtain:
\[
y'' + \left(\frac{2f'}{f} + p\right) y' + \left(\frac{f''}{f} + p\frac{f'}{f} + q\right) y = 0.
\]

\begin{justification}
Dividing by $f$ gives us the standard form of a second-order ODE with the coefficient of $y''$ equal to 1. This makes it easier to see what condition on $f$ will eliminate the $y'$ term.
\end{justification}

\subsection*{Step 7: Imposing the Condition to Eliminate $y'$}

\noindent\textbf{The Key Step:} We want the coefficient of $y'$ to vanish.

\begin{strategy}
Set the coefficient of $y'$ equal to zero:
\[
\frac{2f'}{f} + p = 0.
\]
This is a first-order ODE for $f(x)$ that we can solve.
\end{strategy}

\begin{justification}
Why does this work? If the coefficient of $y'$ is zero, then the transformed ODE contains only $y''$ and $y$ terms, which is exactly what we want. This is a necessary and sufficient condition for eliminating the first derivative.
\end{justification}

\subsection*{Step 8: Solving for $f(x)$}

\noindent\textbf{What we need to solve:}
\[
\frac{2f'}{f} + p(x) = 0 \quad \Longrightarrow \quad \frac{2f'}{f} = -p(x).
\]

\begin{technique}
This is a separable first-order ODE. We can write:
\[
\frac{f'}{f} = -\frac{p(x)}{2}.
\]
Recall that $\frac{d}{dx}\ln|f| = \frac{f'}{f}$, so:
\[
\frac{d}{dx}\ln|f| = -\frac{p(x)}{2}.
\]
\end{technique}

\noindent Integrating both sides with respect to $x$:
\[
\ln|f(x)| = -\frac{1}{2}\int p(x)\,dx + C,
\]
where $C$ is a constant of integration.

\noindent Exponentiating both sides:
\[
|f(x)| = \exp\left(-\frac{1}{2}\int p(x)\,dx + C\right) = e^C \cdot \exp\left(-\frac{1}{2}\int p(x)\,dx\right).
\]

\noindent Since $e^C$ is just a positive constant, and we can absorb it (we only need one particular function $f$, not the general solution with arbitrary constants), we write:
\[
f(x) = \exp\left(-\frac{1}{2}\int p(x)\,dx\right).
\]

\begin{justification}
We drop the absolute value and the arbitrary constant because:
\begin{itemize}
\item We only need \emph{one} function $f(x)$ that eliminates the $y'$ term, not a family of solutions
\item The constant would factor out of the entire ODE and can be absorbed into $y(x)$
\item The exponential is always positive, so $|f| = f$
\end{itemize}
\end{justification}

\subsection*{Step 9: Verifying the Result}

\noindent\textbf{Check:} Let's verify that with $f(x) = \exp\left(-\frac{1}{2}\int p(x)\,dx\right)$, we have $2f'/f + p = 0$.

\begin{technique}
Differentiate $f$ using the chain rule:
\[
f'(x) = \exp\left(-\frac{1}{2}\int p(x)\,dx\right) \cdot \left(-\frac{1}{2}p(x)\right) = -\frac{p(x)}{2} \cdot f(x).
\]
\end{technique}

\noindent Therefore:
\[
\frac{f'}{f} = -\frac{p(x)}{2} \quad \Longrightarrow \quad \frac{2f'}{f} = -p(x) \quad \Longrightarrow \quad \frac{2f'}{f} + p(x) = 0. \quad \checkmark
\]

\subsection*{Step 10: The Transformed ODE for $y(x)$}

\noindent With the $y'$ term eliminated, the ODE for $y(x)$ becomes:
\[
y''(x) + \left(\frac{f''}{f} + p\frac{f'}{f} + q\right) y(x) = 0.
\]

\begin{technique}
Let's simplify the coefficient of $y$. We know:
\begin{itemize}
\item $\frac{f'}{f} = -\frac{p}{2}$
\item $\frac{f''}{f}$ can be computed by differentiating $f'/f$
\end{itemize}
\end{technique}

\noindent Computing $f''/f$: Differentiate $f' = -\frac{p}{2}f$:
\begin{align*}
f'' &= -\frac{p'}{2}f - \frac{p}{2}f'\\
&= -\frac{p'}{2}f - \frac{p}{2}\left(-\frac{p}{2}f\right)\\
&= -\frac{p'}{2}f + \frac{p^2}{4}f.
\end{align*}

\noindent Therefore:
\[
\frac{f''}{f} = -\frac{p'}{2} + \frac{p^2}{4}.
\]

\noindent The coefficient of $y$ becomes:
\begin{align*}
\frac{f''}{f} + p\frac{f'}{f} + q &= \left(-\frac{p'}{2} + \frac{p^2}{4}\right) + p\left(-\frac{p}{2}\right) + q\\
&= -\frac{p'}{2} + \frac{p^2}{4} - \frac{p^2}{2} + q\\
&= -\frac{p'}{2} - \frac{p^2}{4} + q.
\end{align*}

\subsection*{Final Answer}

\begin{center}
\fbox{\begin{minipage}{0.9\textwidth}
\textbf{The function $f(x)$ that eliminates the first derivative term is:}
\[
\boxed{f(x) = \exp\left(-\frac{1}{2}\int p(x)\,dx\right)}
\]

\textbf{The transformed ODE for $y(x)$ is:}
\[
\boxed{y''(x) + \left(q(x) - \frac{p'(x)}{2} - \frac{p(x)^2}{4}\right)y(x) = 0}
\]
\end{minipage}}
\end{center}

\subsection*{Connection to the Lecture Notes}

\begin{justification}
This result appears in the lecture notes \S6.3.1, equation (377). There, the transformation $y(x) = \exp\left(-\frac{1}{2\varepsilon}\int p(s)\,ds\right)z(x)$ is used to eliminate the first derivative from:
\[
\varepsilon y'' + p(x)y' + q(x)y = 0,
\]
yielding:
\[
\varepsilon^2 z'' + \left(-\frac{p^2}{4} - \frac{p'}{2} + q\right)z = 0.
\]
This confirms that our transformation is correct and shows its importance in connecting boundary layer problems to WKB-type equations.
\end{justification}

\subsection*{Physical Interpretation}

\begin{justification}
The transformation $u = fy$ with $f = \exp\left(-\frac{1}{2}\int p\,dx\right)$ has a physical meaning:
\begin{itemize}
\item The function $f(x)$ acts as an amplitude modulation factor
\item It absorbs the damping or growth encoded in the $p(x)u'$ term
\item What remains in $y(x)$ is the purely oscillatory or exponential behavior determined by the effective potential $q - p'/2 - p^2/4$
\item This is why the transformation is essential before applying WKB: it separates amplitude effects from phase effects
\end{itemize}
\end{justification}

\end{document}
