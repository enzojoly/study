\documentclass[11pt,a4paper]{article}
\usepackage{inputenc}
\usepackage{amsmath,amssymb,amsthm}
\usepackage[margin=2.5cm]{geometry}
\usepackage{enumitem}
\usepackage{xcolor}

% Custom environments for pedagogical structure
\newtheoremstyle{problem}
  {10pt}{10pt}{\normalfont}{}{\bfseries}{.}{.5em}{}
\theoremstyle{problem}
\newtheorem{problem}{Problem}

\newenvironment{strategy}{\par\noindent\textbf{Strategy:}\itshape}{\par}
\newenvironment{justification}{\par\noindent\textbf{Justification:}\itshape}{\par}
\newenvironment{technique}{\par\noindent\textbf{Technique:}\itshape}{\par}
\newenvironment{reflection}{\par\noindent\textbf{Reflection:}\itshape}{\par}

\title{Asymptotics Problem 7.2: Complete Pedagogical Solution}
\author{WKB Method via Double Transformation}
\date{}

\begin{document}

\maketitle

\begin{problem}
For the equation $\varepsilon^2 y''(x) + q(x)y(x) = 0$, look for transformations of both the dependent and independent variables, $z = \phi(x)$, $\nu(z) = \psi(x)y(x)$ so that with suitable choice for the functions $\varphi(x)$ and $\psi(x)$, the ODE becomes $\varepsilon^2\nu''(z) + \nu(z) = 0$ to the leading order as $\varepsilon \to 0$. Hence deduce the leading order solution and show it is equivalent to the WKB solution. (You may assume that $q(x)$ is a positive function.)
\end{problem}

\section*{Solution: Step-by-Step Atomic Breakdown}

\subsection*{Step 1: Understanding the Problem and Overall Strategy}

\begin{strategy}
We are performing a \emph{double transformation}:
\begin{itemize}[leftmargin=*]
\item \textbf{Independent variable transformation:} $z = \phi(x)$ (change of coordinate)
\item \textbf{Dependent variable transformation:} $\nu(z) = \psi(x)y(x)$ (amplitude rescaling)
\end{itemize}
The goal is to transform the ODE with variable coefficient $q(x)$ into the \emph{simplest possible form} with constant coefficients: $\varepsilon^2\nu'' + \nu = 0$.
\end{strategy}

\begin{justification}
Why do we want this particular form? Because $\varepsilon^2\nu'' + \nu = 0$ is exactly solvable:
\[
\nu(z) = A\cos\left(\frac{z}{\varepsilon}\right) + B\sin\left(\frac{z}{\varepsilon}\right),
\]
which exhibits the rapid oscillations characteristic of WKB solutions. By transforming our original equation into this form, we can:
\begin{enumerate}
\item Solve it exactly in the transformed variables
\item Transform back to obtain the WKB approximation
\item Understand geometrically what the WKB method is doing
\end{enumerate}
This approach appears in Lecture Notes \S7.2.1, equations (479)-(511), where it's used to derive WKB from the averaging method.
\end{justification}

\subsection*{Step 2: Setting Up the Transformation Framework}

\noindent\textbf{What we have:}
\begin{align*}
\text{Original equation:} &\quad \varepsilon^2 y''(x) + q(x)y(x) = 0\\
\text{New independent variable:} &\quad z = \phi(x)\\
\text{New dependent variable:} &\quad \nu(z) = \psi(x)y(x)\\
\text{Target equation:} &\quad \varepsilon^2\nu''(z) + \nu(z) = 0
\end{align*}

\begin{technique}
To substitute into the ODE, we need to express $y$ and its derivatives in terms of $\nu$ and its derivatives. This requires careful application of the chain rule for both transformations.
\end{technique}

\subsection*{Step 3: Relating $y$ to $\nu$}

\noindent From $\nu(z) = \psi(x)y(x)$, we can express $y$ in terms of $\nu$:
\[
y(x) = \frac{\nu(z)}{\psi(x)} = \frac{\nu(\phi(x))}{\psi(x)}.
\]

\begin{justification}
This inverse relationship is crucial: we're expressing the old unknown $y(x)$ in terms of the new unknown $\nu(z)$. Note that $\nu$ is a function of $z$, which itself depends on $x$ through $z = \phi(x)$.
\end{justification}

\subsection*{Step 4: Computing the First Derivative $y'(x)$}

\noindent\textbf{What we need:} $\frac{dy}{dx}$ using $y = \frac{\nu(z)}{\psi(x)}$ and $z = \phi(x)$.

\begin{technique}
Use the quotient rule combined with the chain rule. We have:
\[
y(x) = \frac{\nu(z(x))}{\psi(x)}.
\]
Applying the quotient rule:
\[
\frac{dy}{dx} = \frac{\frac{d\nu}{dx} \cdot \psi(x) - \nu(z) \cdot \frac{d\psi}{dx}}{\psi(x)^2}.
\]
\end{technique}

\noindent But $\nu$ depends on $x$ through $z$, so by the chain rule:
\[
\frac{d\nu}{dx} = \frac{d\nu}{dz} \cdot \frac{dz}{dx} = \nu'(z) \cdot \phi'(x).
\]

\noindent Therefore:
\[
y'(x) = \frac{\nu'(z)\phi'(x) \cdot \psi(x) - \nu(z) \cdot \psi'(x)}{\psi(x)^2}.
\]

\noindent Simplifying:
\[
y'(x) = \frac{\phi'(x)}{\psi(x)}\nu'(z) - \frac{\psi'(x)}{\psi(x)^2}\nu(z).
\]

\subsection*{Step 5: Computing the Second Derivative $y''(x)$}

\noindent\textbf{What we need:} Differentiate $y'(x)$ with respect to $x$.

\begin{technique}
We have two terms to differentiate:
\[
y'(x) = \frac{\phi'(x)}{\psi(x)}\nu'(z) - \frac{\psi'(x)}{\psi(x)^2}\nu(z).
\]
Each term requires the product rule and chain rule.
\end{technique}

\subsubsection*{Step 5a: Differentiating the First Term}

\noindent For the term $\frac{\phi'}{\psi}\nu'$, apply the product rule:
\begin{align*}
\frac{d}{dx}\left[\frac{\phi'(x)}{\psi(x)}\nu'(z)\right] &= \frac{d}{dx}\left[\frac{\varphi'}{\psi}\right] \cdot \nu' + \frac{\varphi'}{\psi} \cdot \frac{d\nu'}{dx}.
\end{align*}

\noindent For the first factor, use the quotient rule:
\[
\frac{d}{dx}\left[\frac{\phi'}{\psi}\right] = \frac{\varphi''\psi - \varphi'\psi'}{\psi^2}.
\]

\noindent For the second factor, use the chain rule:
\[
\frac{d\nu'}{dx} = \frac{d}{dx}\left[\frac{d\nu}{dz}\right] = \frac{d^2\nu}{dz^2} \cdot \frac{dz}{dx} = \nu''(z) \cdot \phi'(x).
\]

\noindent Therefore:
\[
\frac{d}{dx}\left[\frac{\phi'}{\psi}\nu'\right] = \frac{\varphi''\psi - \varphi'\psi'}{\psi^2}\nu' + \frac{\varphi'}{\psi}\nu''\varphi' = \frac{\varphi''\psi - \varphi'\psi'}{\psi^2}\nu' + \frac{(\varphi')^2}{\psi}\nu''.
\]

\subsubsection*{Step 5b: Differentiating the Second Term}

\noindent For the term $-\frac{\psi'}{\psi^2}\nu$, apply the product rule:
\begin{align*}
\frac{d}{dx}\left[-\frac{\psi'}{\psi^2}\nu\right] &= -\frac{d}{dx}\left[\frac{\psi'}{\psi^2}\right] \cdot \nu - \frac{\psi'}{\psi^2} \cdot \frac{d\nu}{dx}.
\end{align*}

\noindent For the first factor, use the quotient rule:
\[
\frac{d}{dx}\left[\frac{\psi'}{\psi^2}\right] = \frac{\psi''\psi^2 - \psi' \cdot 2\psi\psi'}{\psi^4} = \frac{\psi''\psi - 2(\psi')^2}{\psi^3}.
\]

\noindent For the second factor:
\[
\frac{d\nu}{dx} = \nu'(z)\phi'(x).
\]

\noindent Therefore:
\[
\frac{d}{dx}\left[-\frac{\psi'}{\psi^2}\nu\right] = -\frac{\psi''\psi - 2(\psi')^2}{\psi^3}\nu - \frac{\psi'}{\psi^2}\nu'\phi'.
\]

\subsubsection*{Step 5c: Combining Both Terms}

\noindent Adding the results from Steps 5a and 5b:
\begin{align*}
y''(x) &= \frac{\phi''\psi - \varphi'\psi'}{\psi^2}\nu' + \frac{(\varphi')^2}{\psi}\nu'' - \frac{\psi''\psi - 2(\psi')^2}{\psi^3}\nu - \frac{\psi'\varphi'}{\psi^2}\nu'.
\end{align*}

\noindent Collecting terms by derivative order of $\nu$:
\begin{align*}
y''(x) &= \frac{(\phi')^2}{\psi}\nu'' + \left[\frac{\varphi''\psi - \varphi'\psi'}{\psi^2} - \frac{\psi'\varphi'}{\psi^2}\right]\nu' - \frac{\psi''\psi - 2(\psi')^2}{\psi^3}\nu.
\end{align*}

\noindent Simplifying the coefficient of $\nu'$:
\[
\frac{\phi''\psi - \varphi'\psi' - \psi'\varphi'}{\psi^2} = \frac{\varphi''\psi - 2\varphi'\psi'}{\psi^2}.
\]

\noindent Therefore:
\[
y''(x) = \frac{(\phi')^2}{\psi}\nu'' + \frac{\varphi''\psi - 2\varphi'\psi'}{\psi^2}\nu' - \frac{\psi''\psi - 2(\psi')^2}{\psi^3}\nu.
\]

\subsection*{Step 6: Substituting into the Original ODE}

\noindent\textbf{What we do:} Substitute $y$ and $y''$ into $\varepsilon^2 y'' + qy = 0$.

\begin{align*}
\varepsilon^2 y'' + q(x)y &= 0\\
\varepsilon^2\left[\frac{(\phi')^2}{\psi}\nu'' + \frac{\varphi''\psi - 2\varphi'\psi'}{\psi^2}\nu' - \frac{\psi''\psi - 2(\psi')^2}{\psi^3}\nu\right] + q(x)\frac{\nu}{\psi} &= 0.
\end{align*}

\noindent Multiply through by $\psi$ to clear denominators:
\[
\varepsilon^2(\phi')^2\nu'' + \varepsilon^2\frac{\varphi''\psi - 2\varphi'\psi'}{\psi}\nu' - \varepsilon^2\frac{\psi''\psi - 2(\psi')^2}{\psi^2}\nu + q(x)\nu = 0.
\]

\subsection*{Step 7: Imposing Conditions for the Target Form}

\noindent\textbf{Our goal:} We want $\varepsilon^2\nu''(z) + \nu(z) = 0$, which means:
\begin{itemize}
\item Coefficient of $\nu''$: should be $\varepsilon^2$
\item Coefficient of $\nu'$: should be $0$
\item Coefficient of $\nu$: should be $1$
\end{itemize}

\begin{strategy}
We have three conditions and two unknown functions ($\phi$ and $\psi$). We'll impose the conditions strategically:
\begin{enumerate}
\item Make the coefficient of $\nu''$ equal to $\varepsilon^2$
\item Make the coefficient of $\nu'$ equal to $0$
\item Check what remains for the coefficient of $\nu$
\end{enumerate}
\end{strategy}

\subsubsection*{Step 7a: Condition from the $\nu''$ Term}

\noindent The coefficient of $\nu''$ is $\varepsilon^2(\phi')^2$. We require:
\[
\varepsilon^2(\phi')^2 = \varepsilon^2 \quad \Longrightarrow \quad (\varphi')^2 = 1 \quad \Longrightarrow \quad \varphi'(x) = \pm 1.
\]

\begin{justification}
We choose the positive sign (the negative would just reverse the direction of $z$):
\[
\phi'(x) = 1 \quad \Longrightarrow \quad \varphi(x) = x + \text{const.}
\]
Without loss of generality, we can set the constant to zero, so $\phi(x) = x$, which means $z = x$.
\end{justification}

\begin{reflection}
Wait! If $\phi(x) = x$, then we're not really changing the independent variable at all. This seems strange at first, but it makes sense: the essence of the WKB transformation is in the \emph{dependent variable} transformation $\psi(x)$, not the independent variable. However, let's not assume this yet and continue more generally.
\end{reflection}

\subsubsection*{Step 7b: General Case - Condition from the $\nu'$ Term}

\noindent Let's continue with general $\phi'(x)$ satisfying $(\varphi')^2 = 1$. The coefficient of $\nu'$ is:
\[
\varepsilon^2\frac{\phi''\psi - 2\varphi'\psi'}{\psi}.
\]

\noindent We require this to vanish:
\[
\phi''\psi - 2\varphi'\psi' = 0 \quad \Longrightarrow \quad \varphi''\psi = 2\varphi'\psi'.
\]

\noindent Rearranging:
\[
\frac{\phi''}{\varphi'} = 2\frac{\psi'}{\psi}.
\]

\begin{technique}
Recognize that $\frac{\phi''}{\varphi'} = \frac{d}{dx}\ln|\varphi'|$ and $\frac{\psi'}{\psi} = \frac{d}{dx}\ln|\psi|$. Therefore:
\[
\frac{d}{dx}\ln|\phi'| = 2\frac{d}{dx}\ln|\psi| = \frac{d}{dx}\ln|\psi^2|.
\]
\end{technique}

\noindent Integrating:
\[
\ln|\phi'| = \ln|\psi^2| + C \quad \Longrightarrow \quad |\varphi'| = K\psi^2,
\]
where $K = e^C$ is a positive constant.

\noindent Combined with $(\phi')^2 = 1$, we have $|\varphi'| = 1$, so:
\[
\psi^2 = \frac{1}{K}.
\]

\noindent Choosing $K = 1$ (absorbing the constant into $\nu$), we get:
\[
\psi(x) = 1.
\]

\begin{reflection}
This is surprising! The dependent variable transformation is trivial: $\psi(x) = 1$ means $\nu(z) = y(x)$. Combined with $\phi(x) = x$, we seem to have done nothing. But this can't be right - let's reconsider our approach.
\end{reflection}

\subsection*{Step 8: Reconsidering the Transformation Strategy}

\begin{strategy}
The issue is that we've been too restrictive. Let's instead require that the coefficient of $\nu''$ be $\varepsilon^2$ times \emph{something that could vary with $x$}, but which we'll handle at leading order as $\varepsilon \to 0$. The key insight is that we want:
\[
\varepsilon^2(\phi')^2 = \varepsilon^2 \quad \text{to leading order}.
\]
But actually, let's try a different approach: set $(\phi')^2 = q(x)$ instead!
\end{strategy}

\begin{justification}
Why this choice? Because if $(\phi')^2 = q(x)$, then:
\[
\phi'(x) = \sqrt{q(x)},
\]
and the coefficient of $\nu''$ becomes $\varepsilon^2 q(x)$. If we then divide the entire equation by $q(x)$, we can potentially get the desired form. Let's try this!
\end{justification}

\subsection*{Step 9: New Approach - Setting $\phi'(x) = \sqrt{q(x)}$}

\noindent\textbf{Choice:} Let $\phi'(x) = \sqrt{q(x)}$, so:
\[
z = \phi(x) = \int^x \sqrt{q(s)}\,ds.
\]

\begin{justification}
This is the natural choice because $\sqrt{q(x)}$ is the local "frequency" of oscillation in the WKB solution. By integrating it, we're creating a new coordinate $z$ that measures accumulated phase. This connects directly to the WKB solution formula (Lecture Notes \S6.3.2, equation (382)):
\[
y_\pm(x) \sim \frac{A_\pm}{|q(x)|^{1/4}}\exp\left(\pm\frac{i}{\varepsilon}\int^x\sqrt{q(s)}\,ds\right).
\]
\end{justification}

\noindent With this choice, the coefficient of $\nu''$ becomes:
\[
\varepsilon^2(\phi')^2 = \varepsilon^2 q(x).
\]

\subsection*{Step 10: Finding $\psi(x)$ with the New $\phi$}

\noindent Now we need to find $\psi(x)$ such that the $\nu'$ term vanishes. The coefficient of $\nu'$ is:
\[
\varepsilon^2\frac{\phi''\psi - 2\varphi'\psi'}{\psi}.
\]

\noindent Setting this to zero:
\[
\phi''\psi - 2\varphi'\psi' = 0 \quad \Longrightarrow \quad \frac{\varphi''}{\varphi'} = 2\frac{\psi'}{\psi}.
\]

\noindent With $\phi' = \sqrt{q}$:
\[
\phi'' = \frac{d}{dx}\sqrt{q} = \frac{q'}{2\sqrt{q}} = \frac{q'}{2\varphi'}.
\]

\noindent Therefore:
\[
\frac{\phi''}{\varphi'} = \frac{q'}{2q}.
\]

\noindent The condition becomes:
\[
\frac{q'}{2q} = 2\frac{\psi'}{\psi} \quad \Longrightarrow \quad \frac{\psi'}{\psi} = \frac{q'}{4q}.
\]

\begin{technique}
Integrate: $\ln|\psi| = \frac{1}{4}\ln|q| + C$, so:
\[
\psi(x) = Kq(x)^{1/4},
\]
where $K$ is a constant. Choosing $K = 1$:
\[
\psi(x) = q(x)^{1/4}.
\]
\end{technique}

\subsection*{Step 11: Verifying the Coefficient of $\nu$}

\noindent With $\phi' = \sqrt{q}$ and $\psi = q^{1/4}$, let's check the coefficient of $\nu$ in our transformed equation.

\noindent The original coefficient (after multiplying by $\psi$) was:
\[
-\varepsilon^2\frac{\psi''\psi - 2(\psi')^2}{\psi^2} + q(x).
\]

\begin{technique}
We need to compute $\psi'$ and $\psi''$:
\begin{align*}
\psi &= q^{1/4}\\
\psi' &= \frac{1}{4}q^{-3/4}q' = \frac{q'}{4q^{3/4}}\\
\psi'' &= \frac{d}{dx}\left[\frac{q'}{4q^{3/4}}\right] = \frac{1}{4}\left[\frac{q''q^{3/4} - q' \cdot \frac{3}{4}q^{-1/4}q'}{q^{3/2}}\right]\\
&= \frac{1}{4q^{3/4}}\left[q'' - \frac{3(q')^2}{4q}\right].
\end{align*}
\end{technique}

\noindent Computing $\psi''\psi$:
\begin{align*}
\psi''\psi &= \frac{1}{4q^{3/4}}\left[q'' - \frac{3(q')^2}{4q}\right] \cdot q^{1/4}\\
&= \frac{1}{4q^{1/2}}\left[q'' - \frac{3(q')^2}{4q}\right].
\end{align*}

\noindent Computing $(\psi')^2$:
\[
(\psi')^2 = \frac{(q')^2}{16q^{3/2}}.
\]

\noindent Therefore:
\begin{align*}
\psi''\psi - 2(\psi')^2 &= \frac{1}{4q^{1/2}}\left[q'' - \frac{3(q')^2}{4q}\right] - \frac{(q')^2}{8q^{3/2}}\\
&= \frac{1}{4q^{1/2}}q'' - \frac{3(q')^2}{16q^{3/2}} - \frac{2(q')^2}{16q^{3/2}}\\
&= \frac{1}{4q^{1/2}}q'' - \frac{5(q')^2}{16q^{3/2}}.
\end{align*}

\noindent And:
\[
\frac{\psi''\psi - 2(\psi')^2}{\psi^2} = \frac{\frac{1}{4q^{1/2}}q'' - \frac{5(q')^2}{16q^{3/2}}}{q^{1/2}} = \frac{1}{4q}q'' - \frac{5(q')^2}{16q^2}.
\]

\noindent The coefficient of $\nu$ is:
\[
-\varepsilon^2\left[\frac{1}{4q}q'' - \frac{5(q')^2}{16q^2}\right] + q = q\left[1 - \varepsilon^2\left(\frac{q''}{4q^2} - \frac{5(q')^2}{16q^3}\right)\right].
\]

\subsection*{Step 12: Taking the Leading Order as $\varepsilon \to 0$}

\begin{justification}
As $\varepsilon \to 0$, the term multiplied by $\varepsilon^2$ becomes negligible. To leading order:
\[
\text{Coefficient of } \nu \to q(x) \quad \text{as } \varepsilon \to 0.
\]
\end{justification}

\noindent So our transformed equation is (to leading order):
\[
\varepsilon^2 q(x)\nu''(z) + q(x)\nu(z) = 0.
\]

\noindent Dividing by $q(x)$:
\[
\varepsilon^2\nu''(z) + \nu(z) = 0.
\]

\begin{center}
\fbox{\begin{minipage}{0.9\textwidth}
\textbf{Success!} With the transformations:
\begin{align*}
z &= \phi(x) = \int^x \sqrt{q(s)}\,ds\\
\nu(z) &= \psi(x)y(x) = q(x)^{1/4}y(x)
\end{align*}
the ODE $\varepsilon^2 y'' + q(x)y = 0$ becomes $\varepsilon^2\nu'' + \nu = 0$ to leading order as $\varepsilon \to 0$.
\end{minipage}}
\end{center}

\subsection*{Step 13: Solving the Transformed Equation}

\noindent\textbf{The equation:} $\varepsilon^2\nu''(z) + \nu(z) = 0$ is a constant-coefficient ODE.

\begin{technique}
Try $\nu(z) = e^{\lambda z}$:
\[
\varepsilon^2\lambda^2 e^{\lambda z} + e^{\lambda z} = 0 \quad \Longrightarrow \quad \lambda^2 = -\frac{1}{\varepsilon^2} \quad \Longrightarrow \quad \lambda = \pm\frac{i}{\varepsilon}.
\]
\end{technique}

\noindent The general solution is:
\[
\nu(z) = A\exp\left(\frac{iz}{\varepsilon}\right) + B\exp\left(-\frac{iz}{\varepsilon}\right),
\]
or equivalently:
\[
\nu(z) = C\cos\left(\frac{z}{\varepsilon}\right) + D\sin\left(\frac{z}{\varepsilon}\right).
\]

\subsection*{Step 14: Transforming Back to $y(x)$}

\noindent From $\nu(z) = \psi(x)y(x) = q(x)^{1/4}y(x)$, we have:
\[
y(x) = \frac{\nu(z)}{q(x)^{1/4}} = \frac{\nu(\phi(x))}{q(x)^{1/4}}.
\]

\noindent Substituting $\nu(z) = A\exp(iz/\varepsilon) + B\exp(-iz/\varepsilon)$ and $z = \int^x\sqrt{q(s)}\,ds$:
\[
y(x) = \frac{1}{q(x)^{1/4}}\left[A\exp\left(\frac{i}{\varepsilon}\int^x\sqrt{q(s)}\,ds\right) + B\exp\left(-\frac{i}{\varepsilon}\int^x\sqrt{q(s)}\,ds\right)\right].
\]

\subsection*{Step 15: Comparing with the WKB Solution}

\begin{justification}
From Lecture Notes \S6.3.2, equation (382), the WKB solution for $\varepsilon^2 y'' + q(x)y = 0$ with $q(x) > 0$ is:
\[
y_\pm(x) \sim \frac{A_\pm}{|q(x)|^{1/4}}\exp\left(\pm\frac{i}{\varepsilon}\int^x\sqrt{q(s)}\,ds + O(\varepsilon)\right) \quad \text{as } \varepsilon \to 0.
\]
Since $q(x) > 0$, we have $|q(x)| = q(x)$, so this becomes:
\[
y_\pm(x) \sim \frac{A_\pm}{q(x)^{1/4}}\exp\left(\pm\frac{i}{\varepsilon}\int^x\sqrt{q(s)}\,ds\right).
\]
\end{justification}

\noindent Our solution from the transformation method is:
\[
y(x) = \frac{A}{q(x)^{1/4}}\exp\left(\frac{i}{\varepsilon}\int^x\sqrt{q(s)}\,ds\right) + \frac{B}{q(x)^{1/4}}\exp\left(-\frac{i}{\varepsilon}\int^x\sqrt{q(s)}\,ds\right).
\]

\begin{center}
\fbox{\begin{minipage}{0.9\textwidth}
\textbf{Conclusion:} The solution obtained by the double transformation method is \emph{exactly} the leading order WKB solution! This proves that the WKB approximation can be understood as a systematic coordinate and amplitude transformation that converts the variable-coefficient ODE into a constant-coefficient one.
\end{minipage}}
\end{center}

\subsection*{Final Summary and Physical Interpretation}

\begin{reflection}
What have we learned?
\begin{enumerate}
\item \textbf{The transformation $z = \int^x\sqrt{q(s)}\,ds$} converts the spatial variable $x$ into a ``phase variable'' $z$ that measures accumulated oscillations. Where $q(x)$ is large, oscillations are rapid and $dz/dx = \sqrt{q}$ is large.

\item \textbf{The transformation $\nu = q^{1/4}y$} rescales the amplitude. The factor $q^{1/4}$ appears because it's the unique power that eliminates the first derivative term after the coordinate transformation.

\item \textbf{Geometric insight:} In the $(z,\nu)$ coordinates, solutions are pure sinusoids with wavelength $\sim\varepsilon$. When transformed back to $(x,y)$ coordinates, these become:
\begin{itemize}
\item Amplitude-modulated by $q(x)^{-1/4}$
\item Phase-modulated by $\int\sqrt{q(s)}\,ds$
\end{itemize}

\item \textbf{Connection to \S7.2.1:} This is exactly the approach used in Lecture Notes equations (479)-(511) to derive WKB from the averaging method. The key equations there are:
\begin{itemize}
\item Equation (488): $\dot{h}(t) = \sqrt{q(\varepsilon t)}$ (same as our $\phi'$)
\item Equation (510): $R(T) = c_2/q(\tau)^{1/4}$ (same as our $\psi^{-1}$)
\item Equation (511): The final WKB form
\end{itemize}
\end{enumerate}
\end{reflection}

\end{document}
