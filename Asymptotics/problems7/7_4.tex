\documentclass[11pt,a4paper]{article}
\usepackage{amsmath,amssymb,amsthm}
\usepackage[margin=1in]{geometry}
\usepackage{inputenc}
\usepackage{enumitem}
\usepackage{xcolor}

\newcommand{\dd}{\mathrm{d}}
\newcommand{\eps}{\varepsilon}

\title{Problem 7, Question 4: When is the WKB Solution Exact?}
\author{Pedagogical Breakdown}
\date{}

\begin{document}
\maketitle

\section*{Question Statement}
For what choices of $q(x)$ in the equation
\begin{equation}
\eps^2 y'' + q(x)y = 0
\end{equation}
is the WKB solution exact?

\section*{Solution}

\subsection*{Step 1: Recall the Structure of the WKB Approximation}

\textbf{What are we doing?} We begin by recalling the form of the leading-order WKB approximation as developed in Section 6.3.2 of the lecture notes.

\textbf{Why?} Before we can determine when the WKB solution is \emph{exact}, we must understand what the WKB solution \emph{is}. The WKB approximation provides an asymptotic solution; exactness means this asymptotic form satisfies the ODE without any residual error.

\textbf{The WKB Solution:} For $q(x) > 0$, the leading-order WKB approximation (equations 382--383, page 69 of the lecture notes) gives:
\begin{equation}
y_\pm(x) = \frac{A_\pm}{[q(x)]^{1/4}} \exp\left(\pm \frac{i}{\eps}\int^x \sqrt{q(s)}\,\dd s\right)
\end{equation}

For $q(x) < 0$, the corresponding form is:
\begin{equation}
y_\pm(x) = \frac{B_\pm}{|q(x)|^{1/4}} \exp\left(\pm \frac{1}{\eps}\int^x \sqrt{|q(s)|}\,\dd s\right)
\end{equation}

\textbf{Key observation:} The WKB solution has two components:
\begin{itemize}
\item An \emph{amplitude factor}: $[q(x)]^{-1/4}$
\item A \emph{phase factor}: $\exp\left(\pm \frac{i}{\eps}\int^x \sqrt{q(s)}\,\dd s\right)$
\end{itemize}

\subsection*{Step 2: Criterion for Exactness -- Direct Substitution}

\textbf{What are we doing?} We substitute the WKB solution directly into the ODE $\eps^2 y'' + q(x)y = 0$ to determine when it is satisfied exactly.

\textbf{Why?} The WKB solution is derived as an asymptotic approximation. It is \emph{exact} if and only if substituting it into the ODE yields zero identically, not just to leading order in $\eps$.

\textbf{Setup:} Let us write the WKB solution as:
\begin{equation}
y(x) = A(x) e^{\pm i\phi(x)}
\end{equation}
where the amplitude is $A(x) = [q(x)]^{-1/4}$ and the phase is $\phi(x) = \frac{1}{\eps}\int^x \sqrt{q(s)}\,\dd s$.

\textbf{Computing derivatives:}
\begin{align}
y' &= \left(A' \pm i\phi' A\right) e^{\pm i\phi} \\
y'' &= \left(A'' \pm 2i\phi' A' \pm i\phi'' A - (\phi')^2 A\right) e^{\pm i\phi}
\end{align}

\textbf{Substituting into the ODE:}
\begin{equation}
\eps^2 y'' + qy = \eps^2\left(A'' \pm 2i\phi' A' \pm i\phi'' A - (\phi')^2 A\right) e^{\pm i\phi} + qA\, e^{\pm i\phi}
\end{equation}

\textbf{Using $\phi' = \frac{\sqrt{q}}{\eps}$:}
\begin{equation}
(\phi')^2 = \frac{q}{\eps^2}
\end{equation}

Therefore:
\begin{equation}
\eps^2 y'' + qy = \left(\eps^2 A'' \pm 2i\eps^2\phi' A' \pm i\eps^2\phi'' A - qA + qA\right) e^{\pm i\phi}
\end{equation}

The terms $-qA + qA$ cancel, leaving:
\begin{equation}
\eps^2 y'' + qy = \left(\eps^2 A'' \pm 2i\eps^2\phi' A' \pm i\eps^2\phi'' A\right) e^{\pm i\phi}
\end{equation}

\subsection*{Step 3: Condition for Exact Solution}

\textbf{What are we doing?} We determine when the residual from Step 2 vanishes identically.

\textbf{Why?} For the WKB solution to be exact, we need $\eps^2 y'' + qy = 0$, which requires the expression in parentheses to vanish.

\textbf{The residual:} For the WKB solution to be exact, we need:
\begin{equation}
\eps^2 A'' \pm 2i\eps^2\phi' A' \pm i\eps^2\phi'' A = 0
\end{equation}

\textbf{Separating real and imaginary parts:} Since this must hold for both $y_+$ and $y_-$, the real and imaginary parts must separately vanish:
\begin{align}
\text{Real part:} \quad & \eps^2 A'' = 0 \\
\text{Imaginary part:} \quad & 2\eps^2\phi' A' + \eps^2\phi'' A = 0
\end{align}

\textbf{The key condition:} Since $\eps \neq 0$, the real part condition gives:
\begin{equation}
A''(x) = 0
\end{equation}

But wait -- the imaginary part is automatically satisfied! To see this, note that:
\begin{equation}
2\phi' A' + \phi'' A = \frac{\dd}{\dd x}\left(\phi' A^2\right) \cdot \frac{1}{A}
\end{equation}

Since $\phi' = \frac{\sqrt{q}}{\eps}$ and $A = q^{-1/4}$, we have $\phi' A^2 = \frac{\sqrt{q}}{\eps} \cdot q^{-1/2} = \frac{1}{\eps}$, which is constant. Thus the imaginary condition is automatically satisfied.

\textbf{Therefore, the condition for exactness is simply:}
\begin{equation}
\boxed{A''(x) = 0 \quad \text{where} \quad A(x) = [q(x)]^{-1/4}}
\end{equation}

\subsection*{Step 4: Computing $A''(x)$}

\textbf{What are we doing?} We compute the second derivative of $A(x) = [q(x)]^{-1/4}$ explicitly.

\textbf{Why?} This will give us an explicit condition on $q(x)$.

\textbf{First derivative:}
\begin{equation}
A = q^{-1/4} \quad \Rightarrow \quad A' = -\frac{1}{4}q^{-5/4} \cdot q' = -\frac{q'}{4q^{5/4}}
\end{equation}

\textbf{Second derivative:}
\begin{align}
A'' &= -\frac{1}{4}\frac{\dd}{\dd x}\left(q' \cdot q^{-5/4}\right) \\
&= -\frac{1}{4}\left(q'' \cdot q^{-5/4} + q' \cdot \left(-\frac{5}{4}\right)q^{-9/4} \cdot q'\right) \\
&= -\frac{1}{4}\left(\frac{q''}{q^{5/4}} - \frac{5(q')^2}{4q^{9/4}}\right) \\
&= -\frac{q''}{4q^{5/4}} + \frac{5(q')^2}{16q^{9/4}}
\end{align}

\textbf{Factoring out $q^{-9/4}$:}
\begin{equation}
A'' = q^{-9/4}\left(-\frac{q'' q}{4} + \frac{5(q')^2}{16}\right) = \frac{1}{q^{9/4}}\left(\frac{5(q')^2}{16} - \frac{q'' q}{4}\right)
\end{equation}

\subsection*{Step 5: The Differential Equation for $q(x)$}

\textbf{What are we doing?} We set $A'' = 0$ and derive the resulting condition on $q(x)$.

\textbf{Why?} This gives us the explicit differential equation that $q(x)$ must satisfy for the WKB solution to be exact.

\textbf{Setting $A'' = 0$:} Since $q^{-9/4} \neq 0$ (assuming $q \neq 0$), we require:
\begin{equation}
\frac{5(q')^2}{16} - \frac{q'' q}{4} = 0
\end{equation}

\textbf{Rearranging:}
\begin{equation}
\frac{q''}{4q} = \frac{5(q')^2}{16q^2}
\end{equation}

\textbf{Simplifying:}
\begin{equation}
\frac{q''}{q'} = \frac{5q'}{4q}
\end{equation}

This can be written as:
\begin{equation}
\frac{\dd}{\dd x}\ln|q'| = \frac{5}{4} \cdot \frac{\dd}{\dd x}\ln|q|
\end{equation}

\subsection*{Step 6: First Integration}

\textbf{What are we doing?} We integrate the differential equation $\frac{q''}{q'} = \frac{5q'}{4q}$ once.

\textbf{Why?} This reduces the second-order ODE for $q(x)$ to a first-order ODE.

\textbf{Integrating both sides:}
\begin{equation}
\int \frac{q''}{q'}\,\dd x = \int \frac{5q'}{4q}\,\dd x
\end{equation}

\begin{equation}
\ln|q'| = \frac{5}{4}\ln|q| + C_1
\end{equation}

\textbf{Exponentiating:}
\begin{equation}
|q'| = e^{C_1} |q|^{5/4}
\end{equation}

\textbf{Writing with a constant $K$:}
\begin{equation}
q' = K q^{5/4}
\end{equation}
where $K = \pm e^{C_1}$ is an arbitrary nonzero constant.

\subsection*{Step 7: Second Integration -- Separation of Variables}

\textbf{What are we doing?} We solve the first-order ODE $q' = Kq^{5/4}$ by separation of variables.

\textbf{Why?} This gives us the explicit form of $q(x)$.

\textbf{Separating variables:}
\begin{equation}
\frac{\dd q}{q^{5/4}} = K\,\dd x
\end{equation}

\textbf{Integrating the left side:}
\begin{equation}
\int q^{-5/4}\,\dd q = \frac{q^{-5/4+1}}{-5/4+1} = \frac{q^{-1/4}}{-1/4} = -4q^{-1/4}
\end{equation}

\textbf{Integrating the right side:}
\begin{equation}
\int K\,\dd x = Kx + C_2
\end{equation}

\textbf{Combining:}
\begin{equation}
-4q^{-1/4} = Kx + C_2
\end{equation}

\subsection*{Step 8: Solving for $q(x)$}

\textbf{What are we doing?} We solve the equation $-4q^{-1/4} = Kx + C_2$ for $q(x)$.

\textbf{Why?} This gives us the final answer.

\textbf{Isolating $q^{-1/4}$:}
\begin{equation}
q^{-1/4} = -\frac{Kx + C_2}{4}
\end{equation}

\textbf{Relabeling constants:} Let $a = -K/4$ and $b = -C_2/4$. Then:
\begin{equation}
q^{-1/4} = ax + b
\end{equation}

\textbf{Raising both sides to the power $-4$:}
\begin{equation}
q = (ax + b)^{-4}
\end{equation}

\textbf{Equivalently:}
\begin{equation}
\boxed{q(x) = (ax + b)^{-4}}
\end{equation}
where $a$ and $b$ are arbitrary constants (with $a \neq 0$ for a non-constant solution).

\subsection*{Step 9: Verification}

\textbf{What are we doing?} We verify that $q(x) = (ax+b)^{-4}$ satisfies the condition $\frac{q''}{4q} = \frac{5(q')^2}{16q^2}$.

\textbf{Why?} It is good practice to check our answer by substitution.

\textbf{Computing derivatives:} Let $u = ax + b$. Then $q = u^{-4}$.
\begin{align}
q' &= -4u^{-5} \cdot a = -4a u^{-5} \\
q'' &= -4a \cdot (-5) u^{-6} \cdot a = 20a^2 u^{-6}
\end{align}

\textbf{Computing the left-hand side:}
\begin{equation}
\frac{q''}{4q} = \frac{20a^2 u^{-6}}{4 u^{-4}} = \frac{20a^2}{4} u^{-2} = 5a^2 u^{-2}
\end{equation}

\textbf{Computing the right-hand side:}
\begin{equation}
\frac{5(q')^2}{16q^2} = \frac{5 \cdot 16a^2 u^{-10}}{16 u^{-8}} = 5a^2 u^{-2}
\end{equation}

\textbf{Comparison:} Both sides equal $5a^2 u^{-2}$. \checkmark

\subsection*{Step 10: The Explicit Exact Solution}

\textbf{What are we doing?} We write out the exact WKB solution when $q(x) = (ax+b)^{-4}$.

\textbf{Why?} Having identified when the WKB approximation is exact, we should state the explicit form of this exact solution.

\textbf{For $q(x) = (ax+b)^{-4}$:}

The amplitude factor is:
\begin{equation}
[q(x)]^{-1/4} = \left[(ax+b)^{-4}\right]^{-1/4} = (ax+b)^1 = ax + b
\end{equation}

The phase integral is:
\begin{equation}
\int^x \sqrt{q(s)}\,\dd s = \int^x (as+b)^{-2}\,\dd s = -\frac{1}{a(ax+b)}
\end{equation}

\textbf{The exact solution:}
\begin{equation}
y_\pm(x) = A_\pm (ax+b) \exp\left(\mp \frac{i}{\eps \cdot a(ax+b)}\right)
\end{equation}

Or in real form:
\begin{align}
y_1(x) &= (ax+b)\cos\left(\frac{1}{\eps \cdot a(ax+b)}\right) \\
y_2(x) &= (ax+b)\sin\left(\frac{1}{\eps \cdot a(ax+b)}\right)
\end{align}

\textbf{Verification by direct substitution:} One can verify that these functions satisfy $\eps^2 y'' + (ax+b)^{-4} y = 0$ exactly.

\subsection*{Step 11: Special Case -- Constant $q$}

\textbf{What are we doing?} We check whether constant $q(x)$ is included in our answer.

\textbf{Why?} When $q$ is constant, the ODE $\eps^2 y'' + qy = 0$ has exact sinusoidal solutions. Is this consistent with our result?

\textbf{Analysis:} A constant $q = c$ corresponds to the limiting case $a \to 0$ in $(ax+b)^{-4}$. More precisely:
\begin{equation}
\lim_{a \to 0} (ax + b)^{-4} = b^{-4} = \text{constant}
\end{equation}

For constant $q = c$, the exact solutions are:
\begin{equation}
y(x) = A\cos\left(\frac{\sqrt{c}}{\eps}x\right) + B\sin\left(\frac{\sqrt{c}}{\eps}x\right)
\end{equation}

The WKB approximation for constant $q$ gives:
\begin{equation}
y_{\text{WKB}}(x) = \frac{A}{c^{1/4}}\exp\left(\pm \frac{i\sqrt{c}}{\eps}x\right)
\end{equation}

These are indeed exact solutions (up to the constant prefactor), confirming that constant $q$ is a special case.

\subsection*{Step 12: Summary and Final Answer}

\textbf{The WKB solution to $\eps^2 y'' + q(x)y = 0$ is exact if and only if:}
\begin{equation}
\boxed{q(x) = (ax + b)^{-4}}
\end{equation}
where $a$ and $b$ are arbitrary constants.

\textbf{Equivalent characterizations:}
\begin{enumerate}
\item \textbf{Differential equation form:} $\displaystyle \frac{q''}{4q} = \frac{5(q')^2}{16q^2}$, or equivalently, $q' = Kq^{5/4}$
\item \textbf{Amplitude condition:} The amplitude factor $A(x) = [q(x)]^{-1/4}$ satisfies $A''(x) = 0$, meaning $A(x)$ is linear in $x$
\item \textbf{Algebraic form:} $q(x) = (ax+b)^{-4}$ (inverse fourth power of a linear function)
\end{enumerate}

\textbf{Physical interpretation:} The inverse fourth power form $q(x) \propto (ax+b)^{-4}$ represents a very special ``potential'' in the corresponding Schr\"odinger-like equation. For this particular form, the WKB phase integral and amplitude modulation combine in precisely the right way to yield an exact solution.

\textbf{Note on signs and domains:} For $q(x) > 0$ (oscillatory case), we need $(ax+b)^{-4} > 0$, which is satisfied for all $x \neq -b/a$. The solution is valid on any interval not containing the singularity at $x = -b/a$.

\end{document}
