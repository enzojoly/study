\documentclass[11pt,a4paper]{article}
\usepackage{amsmath,amssymb,amsthm}
\usepackage[margin=1in]{geometry}
\usepackage{inputenc}
\usepackage{enumitem}
\usepackage{xcolor}

\newcommand{\dd}{\mathrm{d}}
\newcommand{\eps}{\varepsilon}

\title{Problem 7, Question 5: WKB Approximation for Eigenvalue Problem}
\author{Pedagogical Breakdown}
\date{}

\begin{document}
\maketitle

\section*{Question Statement}
Use the WKB approximation to estimate the large eigenvalues, $\lambda$, of the eigenvalue problem
\begin{equation}
y'' + \frac{\lambda^2}{x^2}y = 0, \quad y(1) = 0, \quad y(e) = 0.
\end{equation}
Find also the exact solutions and the exact eigenvalues. (Try $y(x) = x^\alpha$.) Consider the two sets of eigenvalues:
\begin{enumerate}[label=(\roman*)]
\item Are the discrepancies between them consistent with the approximation made? If so, explain briefly why.
\item Will more terms of the WKB approximation give a better result? If your answer is yes, determine the form of the next term in the approximation to $y(x)$ and show how this gives a better result for the eigenvalues.
\end{enumerate}

\section*{Part A: Identifying the Structure and Setting Up WKB}

\subsection*{Step 1: Recognize the Form of the ODE}

\textbf{What are we doing?} We begin by identifying that the given eigenvalue problem is of a form amenable to WKB analysis.

\textbf{Why?} The WKB method, as developed in Section 6.3 of the lecture notes, applies to equations of the form $\eps^2 y'' + q(x)y = 0$. We must first recognize how our equation fits this pattern, and what plays the role of the small parameter $\eps$.

\textbf{Rewriting the equation:} The given ODE is
\begin{equation}
y'' + \frac{\lambda^2}{x^2}y = 0.
\end{equation}

\textbf{Key observation:} We are told to consider \emph{large} eigenvalues $\lambda$. This suggests we should think of $1/\lambda$ as a small parameter. However, the standard WKB form has the small parameter multiplying $y''$, not inside $q(x)$.

\textbf{Identifying the WKB parameter:} Let us define
\begin{equation}
\eps := \frac{1}{\lambda}
\end{equation}
so that $\lambda = 1/\eps$ and $\lambda \to \infty$ corresponds to $\eps \to 0$.

\textbf{Does this give WKB form?} Rewriting:
\begin{equation}
y'' + \frac{1}{\eps^2 x^2}y = 0
\end{equation}

Multiplying through by $\eps^2$:
\begin{equation}
\eps^2 y'' + \frac{1}{x^2}y = 0
\end{equation}

\textbf{Identification:} This is precisely the WKB form $\eps^2 y'' + q(x)y = 0$ with
\begin{equation}
q(x) = \frac{1}{x^2}.
\end{equation}

\textbf{Domain and positivity:} Since $x \in [1, e]$, we have $x > 0$, hence $q(x) > 0$ throughout the domain. This means we are in the \emph{oscillatory regime} where WKB solutions involve trigonometric functions (Section 6.3.3, page 69).

\subsection*{Step 2: Verify That WKB is Appropriate for Large $\lambda$}

\textbf{What are we doing?} Before proceeding with the WKB calculation, we verify that the method is valid for our problem.

\textbf{Why?} The WKB approximation is an asymptotic method valid when $\eps \to 0$. We must check that the conditions stated in Section 6.3.3 (page 69) are satisfied.

\textbf{The WKB validity criterion:} From the lecture notes, the WKB approximation is good when
\begin{equation}
\frac{1}{\eps}S_0(x) \gg S_1(x) \gg \eps S_2(x) \quad \text{and} \quad \eps S_2(x) \ll 1
\end{equation}
in the interval considered.

\textbf{Order of magnitude estimates:} For $q(x) = 1/x^2$ on $[1,e]$:
\begin{align}
S_0 &\sim \int \frac{1}{x}\dd x \sim \log x \sim O(1) \\
S_1 &\sim \log(x^{-1/2}) \sim O(1) \\
S_2 &\sim O(1)
\end{align}

Thus:
\begin{equation}
\frac{1}{\eps}S_0 \sim \frac{1}{\eps} \gg 1 \gg \eps \quad \text{for } \eps \to 0
\end{equation}

\textbf{Conclusion:} The WKB approximation is valid for large $\lambda$ (small $\eps$) on the interval $[1,e]$.

\textbf{Note on turning points:} Since $q(x) = 1/x^2 > 0$ for all $x \in [1,e]$, there are no turning points in the domain. We are entirely in the oscillatory regime.

\section*{Part B: Applying the WKB Method}

\subsection*{Step 3: Recall the WKB Solution in the Oscillatory Regime}

\textbf{What are we doing?} We now write down the general form of the WKB solution for $q(x) > 0$.

\textbf{Why?} Before imposing boundary conditions to find eigenvalues, we need the general solution. From Section 6.3.2, equations (382) on page 69, the WKB solution for $q(x) > 0$ is:

\textbf{The WKB solution:}
\begin{equation}
y(x) = \frac{A}{[q(x)]^{1/4}}\exp\left(\frac{i}{\eps}\int^x \sqrt{q(s)}\,\dd s\right) + \frac{B}{[q(x)]^{1/4}}\exp\left(-\frac{i}{\eps}\int^x \sqrt{q(s)}\,\dd s\right)
\end{equation}

where $A$ and $B$ are constants to be determined by boundary conditions.

\textbf{Alternative form:} Using Euler's formula, this can be rewritten as:
\begin{equation}
y(x) = \frac{C}{[q(x)]^{1/4}}\cos\left(\frac{1}{\eps}\int^x \sqrt{q(s)}\,\dd s + \phi\right)
\end{equation}

where $C$ and $\phi$ are constants related to $A$ and $B$.

\subsection*{Step 4: Evaluate the WKB Phase Integral}

\textbf{What are we doing?} We compute the phase integral $\int^x \sqrt{q(s)}\,\dd s$ explicitly for $q(x) = 1/x^2$.

\textbf{Why?} This integral appears in the exponential/trigonometric arguments of the WKB solution. To apply boundary conditions and find eigenvalues, we need its explicit form.

\textbf{Computation:} For $q(x) = 1/x^2$:
\begin{equation}
\sqrt{q(x)} = \frac{1}{x}
\end{equation}

\textbf{Setting the integration limits:} We choose the lower limit as $x = 1$ (the left boundary) for convenience:
\begin{equation}
\int_1^x \sqrt{q(s)}\,\dd s = \int_1^x \frac{1}{s}\,\dd s = [\log s]_1^x = \log x - \log 1 = \log x
\end{equation}

\textbf{The amplitude factor:} The amplitude prefactor is:
\begin{equation}
[q(x)]^{-1/4} = \left(\frac{1}{x^2}\right)^{-1/4} = x^{1/2}
\end{equation}

\subsection*{Step 5: Write the Explicit WKB Solution}

\textbf{What are we doing?} We substitute our computed integrals into the general WKB form.

\textbf{Why?} This gives us the concrete form of the solution that we can apply boundary conditions to.

\textbf{The WKB solution becomes:}
\begin{equation}
y(x) = \sqrt{x}\left[A\exp\left(\frac{i\log x}{\eps}\right) + B\exp\left(-\frac{i\log x}{\eps}\right)\right]
\end{equation}

\textbf{Simplifying the exponentials:} Using the property $e^{i\log x} = e^{\log x^i} = x^i$:
\begin{equation}
\exp\left(\frac{i\log x}{\eps}\right) = x^{i/\eps}
\end{equation}

\textbf{Thus:}
\begin{equation}
y(x) = \sqrt{x}\left[Ax^{i/\eps} + Bx^{-i/\eps}\right]
\end{equation}

\textbf{Alternative trigonometric form:} Using $x^{i/\eps} = e^{i\log x/\eps} = \cos(\log x/\eps) + i\sin(\log x/\eps)$, we can write:
\begin{equation}
y(x) = \sqrt{x}\left[C\cos\left(\frac{\log x}{\eps}\right) + D\sin\left(\frac{\log x}{\eps}\right)\right]
\end{equation}

where $C$ and $D$ are real constants if we want a real solution.

\subsection*{Step 6: Apply the First Boundary Condition}

\textbf{What are we doing?} We impose the boundary condition $y(1) = 0$.

\textbf{Why?} Eigenvalue problems require the solution to satisfy both boundary conditions. Each condition constrains the constants in the general solution.

\textbf{Applying $y(1) = 0$:} Using the trigonometric form:
\begin{equation}
y(1) = \sqrt{1}\left[C\cos\left(\frac{\log 1}{\eps}\right) + D\sin\left(\frac{\log 1}{\eps}\right)\right] = 0
\end{equation}

\textbf{Simplification:} Since $\log 1 = 0$:
\begin{equation}
C\cos(0) + D\sin(0) = C \cdot 1 + D \cdot 0 = C = 0
\end{equation}

\textbf{Conclusion:} The first boundary condition forces $C = 0$, so:
\begin{equation}
y(x) = D\sqrt{x}\sin\left(\frac{\log x}{\eps}\right)
\end{equation}

\subsection*{Step 7: Apply the Second Boundary Condition to Find Eigenvalues}

\textbf{What are we doing?} We now impose $y(e) = 0$ to determine the allowed values of $\eps$ (and hence $\lambda$).

\textbf{Why?} The second boundary condition, combined with the requirement of non-trivial solutions ($D \neq 0$), yields the eigenvalue equation.

\textbf{Applying $y(e) = 0$:}
\begin{equation}
y(e) = D\sqrt{e}\sin\left(\frac{\log e}{\eps}\right) = 0
\end{equation}

\textbf{Non-triviality:} For a non-trivial solution, we need $D \neq 0$ and $\sqrt{e} \neq 0$. Therefore:
\begin{equation}
\sin\left(\frac{\log e}{\eps}\right) = 0
\end{equation}

\textbf{General solution of sine equation:} The sine function vanishes when its argument is an integer multiple of $\pi$:
\begin{equation}
\frac{\log e}{\eps} = n\pi, \quad n = 1, 2, 3, \ldots
\end{equation}

(We exclude $n = 0$ as it would give $y \equiv 0$, and negative $n$ give the same eigenvalues as positive $n$.)

\textbf{Recalling $\log e = 1$:}
\begin{equation}
\frac{1}{\eps} = n\pi
\end{equation}

\textbf{Substituting $\eps = 1/\lambda$:}
\begin{equation}
\lambda = n\pi
\end{equation}

\subsection*{Step 8: State the WKB Eigenvalue Prediction}

\textbf{What have we found?} The leading-order WKB approximation predicts the eigenvalues are:
\begin{equation}
\boxed{\lambda_n^{\text{WKB}} = n\pi, \quad n = 1, 2, 3, \ldots}
\end{equation}

\textbf{The corresponding eigenfunctions (WKB):}
\begin{equation}
y_n^{\text{WKB}}(x) = \sqrt{x}\sin(n\pi\log x)
\end{equation}

\textbf{Verification of boundary conditions:}
\begin{itemize}
\item At $x = 1$: $y_n(1) = \sqrt{1}\sin(n\pi \cdot 0) = \sin(0) = 0$ \checkmark
\item At $x = e$: $y_n(e) = \sqrt{e}\sin(n\pi \cdot 1) = \sqrt{e}\sin(n\pi) = 0$ \checkmark
\end{itemize}

\section*{Part C: Finding the Exact Solution}

\subsection*{Step 9: Use the Suggested Ansatz $y(x) = x^\alpha$}

\textbf{What are we doing?} Following the hint in the problem, we try a power-law solution $y(x) = x^\alpha$ for some constant $\alpha$ to be determined.

\textbf{Why this ansatz?} The original ODE has the form $y'' + (\lambda^2/x^2)y = 0$. Since the coefficient $\lambda^2/x^2$ is a power of $x$, and differentiation of power functions yields power functions, a power-law ansatz is natural.

\textbf{This is called an Euler equation or Cauchy-Euler equation:} Equations of the form
\begin{equation}
x^2 y'' + pxy' + qy = 0
\end{equation}
are known to have power-law solutions. Our equation is of this type.

\subsection*{Step 10: Compute Derivatives of the Ansatz}

\textbf{What are we doing?} We compute the derivatives of $y(x) = x^\alpha$ to substitute into the ODE.

\textbf{Why?} To determine $\alpha$, we must substitute our ansatz into the differential equation and find which values of $\alpha$ satisfy it.

\textbf{Computation:}
\begin{align}
y(x) &= x^\alpha \\
y'(x) &= \alpha x^{\alpha - 1} \\
y''(x) &= \alpha(\alpha - 1)x^{\alpha - 2}
\end{align}

\subsection*{Step 11: Substitute into the ODE and Derive the Characteristic Equation}

\textbf{What are we doing?} We substitute $y = x^\alpha$ and its derivatives into the ODE $y'' + (\lambda^2/x^2)y = 0$.

\textbf{Why?} This will give us an algebraic equation for $\alpha$, known as the characteristic equation or indicial equation.

\textbf{Substitution:}
\begin{equation}
\alpha(\alpha-1)x^{\alpha-2} + \frac{\lambda^2}{x^2}x^\alpha = 0
\end{equation}

\textbf{Simplification:}
\begin{equation}
\alpha(\alpha-1)x^{\alpha-2} + \lambda^2 x^{\alpha-2} = 0
\end{equation}

\textbf{Factoring out $x^{\alpha-2}$:}
\begin{equation}
x^{\alpha-2}\left[\alpha(\alpha-1) + \lambda^2\right] = 0
\end{equation}

\textbf{Key observation:} Since $x^{\alpha-2} \neq 0$ for $x \in [1,e]$, we must have:
\begin{equation}
\alpha(\alpha-1) + \lambda^2 = 0
\end{equation}

This is the \textbf{characteristic equation}.

\subsection*{Step 12: Solve the Characteristic Equation for $\alpha$}

\textbf{What are we doing?} We solve the quadratic equation $\alpha(\alpha-1) + \lambda^2 = 0$ for $\alpha$.

\textbf{Why?} This determines the exponents in the power-law solutions, giving us the two linearly independent solutions needed for a second-order ODE.

\textbf{Rearranging:}
\begin{equation}
\alpha^2 - \alpha + \lambda^2 = 0
\end{equation}

\textbf{Using the quadratic formula:}
\begin{equation}
\alpha = \frac{1 \pm \sqrt{1 - 4\lambda^2}}{2}
\end{equation}

\textbf{For large $\lambda$:} When $\lambda^2 > 1/4$, we have $1 - 4\lambda^2 < 0$, so:
\begin{equation}
\alpha = \frac{1 \pm \sqrt{-(4\lambda^2 - 1)}}{2} = \frac{1 \pm i\sqrt{4\lambda^2 - 1}}{2}
\end{equation}

\textbf{Defining $\mu$:} Let us define
\begin{equation}
\mu := \sqrt{\lambda^2 - \frac{1}{4}}
\end{equation}

Then $\sqrt{4\lambda^2 - 1} = 2\mu$, and:
\begin{equation}
\alpha = \frac{1 \pm 2i\mu}{2} = \frac{1}{2} \pm i\mu
\end{equation}

\textbf{The two roots:}
\begin{equation}
\alpha_1 = \frac{1}{2} + i\mu, \quad \alpha_2 = \frac{1}{2} - i\mu
\end{equation}

\textbf{Critical observation:} Note that $\mu = \sqrt{\lambda^2 - 1/4} \neq \lambda$. This distinction is crucial for finding the exact eigenvalues.

\subsection*{Step 13: Write the General Exact Solution}

\textbf{What are we doing?} We construct the general solution as a linear combination of the two linearly independent power-law solutions.

\textbf{Why?} A second-order linear ODE has a two-dimensional solution space. Any solution can be written as a linear combination of two independent solutions.

\textbf{The general solution:}
\begin{equation}
y(x) = Ax^{\alpha_1} + Bx^{\alpha_2} = Ax^{\frac{1}{2} + i\mu} + Bx^{\frac{1}{2} - i\mu}
\end{equation}

\textbf{Factoring out common power:}
\begin{equation}
y(x) = \sqrt{x}\left(Ax^{i\mu} + Bx^{-i\mu}\right)
\end{equation}

\textbf{Converting to trigonometric form:} Using $x^{i\mu} = e^{i\mu\log x}$ and Euler's formula:
\begin{align}
x^{i\mu} &= \cos(\mu\log x) + i\sin(\mu\log x) \\
x^{-i\mu} &= \cos(\mu\log x) - i\sin(\mu\log x)
\end{align}

For real solutions:
\begin{equation}
y(x) = \sqrt{x}\left[C\cos(\mu\log x) + D\sin(\mu\log x)\right]
\end{equation}

where $C$ and $D$ are real constants and $\mu = \sqrt{\lambda^2 - 1/4}$.

\textbf{Comparison with WKB:} The WKB solution has the form $\sqrt{x}\sin(\lambda\log x)$, while the exact solution has $\sqrt{x}\sin(\mu\log x)$ where $\mu = \sqrt{\lambda^2 - 1/4}$. These are \emph{not} the same!

\subsection*{Step 14: Apply Boundary Conditions to Find Exact Eigenvalues}

\textbf{What are we doing?} We impose the boundary conditions $y(1) = 0$ and $y(e) = 0$ on the exact solution.

\textbf{Why?} This determines the allowed values of $\lambda$ exactly.

\textbf{First boundary condition, $y(1) = 0$:}
\begin{equation}
y(1) = \sqrt{1}\left[C\cos(\mu\log 1) + D\sin(\mu\log 1)\right] = C\cos(0) + D\sin(0) = C = 0
\end{equation}

So $C = 0$ and:
\begin{equation}
y(x) = D\sqrt{x}\sin(\mu\log x)
\end{equation}

\textbf{Second boundary condition, $y(e) = 0$:}
\begin{equation}
y(e) = D\sqrt{e}\sin(\mu\log e) = D\sqrt{e}\sin(\mu) = 0
\end{equation}

For non-trivial solutions ($D \neq 0$):
\begin{equation}
\sin(\mu) = 0
\end{equation}

\textbf{Solution:}
\begin{equation}
\mu = n\pi, \quad n = 1, 2, 3, \ldots
\end{equation}

\subsection*{Step 15: Convert from $\mu$ to $\lambda$}

\textbf{What are we doing?} We use the relation $\mu = \sqrt{\lambda^2 - 1/4}$ to find the exact eigenvalues $\lambda_n$.

\textbf{Why?} The boundary condition determines $\mu_n = n\pi$, but the problem asks for $\lambda_n$.

\textbf{From $\mu_n = n\pi$:}
\begin{equation}
\sqrt{\lambda_n^2 - \frac{1}{4}} = n\pi
\end{equation}

\textbf{Squaring both sides:}
\begin{equation}
\lambda_n^2 - \frac{1}{4} = n^2\pi^2
\end{equation}

\textbf{Solving for $\lambda_n$:}
\begin{equation}
\lambda_n^2 = n^2\pi^2 + \frac{1}{4}
\end{equation}

\begin{equation}
\boxed{\lambda_n^{\text{exact}} = \sqrt{n^2\pi^2 + \frac{1}{4}}, \quad n = 1, 2, 3, \ldots}
\end{equation}

\subsection*{Step 16: Compare WKB and Exact Eigenvalues}

\textbf{What have we found?} Let us compare the two sets of eigenvalues:

\begin{align}
\lambda_n^{\text{WKB}} &= n\pi \\
\lambda_n^{\text{exact}} &= \sqrt{n^2\pi^2 + \frac{1}{4}}
\end{align}

\textbf{Key observation:} The WKB and exact eigenvalues are \emph{not} identical! There is a discrepancy.

\textbf{Expanding the exact eigenvalue for large $n$:}
\begin{align}
\lambda_n^{\text{exact}} &= \sqrt{n^2\pi^2 + \frac{1}{4}} = n\pi\sqrt{1 + \frac{1}{4n^2\pi^2}} \\
&= n\pi\left(1 + \frac{1}{8n^2\pi^2} - \frac{1}{128n^4\pi^4} + O(n^{-6})\right) \\
&= n\pi + \frac{1}{8n\pi} + O(n^{-3})
\end{align}

\textbf{The discrepancy:}
\begin{equation}
\boxed{\Delta\lambda_n = \lambda_n^{\text{exact}} - \lambda_n^{\text{WKB}} = \frac{1}{8n\pi} + O(n^{-3})}
\end{equation}

\textbf{This is an $O(1/\lambda)$ correction}, which is consistent with the WKB approximation being accurate to leading order but not exact.

\section*{Part D: Analysis of the Results}

\subsection*{Step 17: Answer Question (i) -- Are the Discrepancies Consistent?}

\textbf{What are we being asked?} Part (i) asks: ``Are the discrepancies between them consistent with the approximation made? If so, explain briefly why.''

\textbf{Answer: Yes, the discrepancies are consistent with the WKB approximation.}

\textbf{Explanation:}

\begin{enumerate}
\item \textbf{The WKB method is an asymptotic approximation for large $\lambda$.} The leading-order WKB solution captures the behavior as $\lambda \to \infty$ but includes errors of relative order $O(1/\lambda)$.

\item \textbf{The discrepancy $\Delta\lambda_n = 1/(8n\pi) + O(n^{-3})$ is indeed $O(1/\lambda)$.} Since $\lambda_n \sim n\pi$, we have:
\begin{equation}
\frac{\Delta\lambda_n}{\lambda_n} \sim \frac{1/(8n\pi)}{n\pi} = \frac{1}{8n^2\pi^2} = O(\lambda^{-2})
\end{equation}
This is a small relative error that vanishes as $\lambda \to \infty$, exactly as expected for a leading-order asymptotic approximation.

\item \textbf{The source of the discrepancy:} The characteristic equation gives $\mu = \sqrt{\lambda^2 - 1/4}$, not $\mu = \lambda$. The WKB method effectively approximates $\mu \approx \lambda$ for large $\lambda$, which introduces an error of order $O(1/\lambda)$.

\item \textbf{Physical interpretation:} The $1/4$ term in $\lambda^2 - 1/4$ arises from the amplitude modulation factor $q(x)^{-1/4} = x^{1/2}$ in the WKB solution. This amplitude factor contributes to the effective phase of the exact solution, but the leading-order WKB eigenvalue condition only accounts for the explicit phase integral $\int\sqrt{q}\,\dd x$.
\end{enumerate}

\textbf{Conclusion for part (i):} The discrepancy $\Delta\lambda_n = O(1/n)$ is fully consistent with the leading-order WKB approximation, which is accurate to $O(1)$ in $\lambda$ but not beyond.

\subsection*{Step 18: Answer Question (ii) -- Will More WKB Terms Help?}

\textbf{What are we being asked?} Part (ii) asks whether including higher-order WKB corrections will give a better result for the eigenvalues.

\textbf{Answer: Yes, including the next WKB correction gives a better result.}

\textbf{Let us demonstrate this explicitly.}

\subsection*{Step 19: The Higher-Order WKB Expansion}

\textbf{What are we doing?} We include the next term in the WKB expansion to improve the eigenvalue estimate.

\textbf{Why?} The leading-order WKB gave eigenvalues accurate to $O(1)$. The next correction should capture the $O(1/\lambda)$ term.

\textbf{Recall from Section 6.3.2:} The WKB expansion for $p = S'$ is:
\begin{equation}
p(x,\eps) = \frac{1}{\eps}p_0(x) + p_1(x) + \eps p_2(x) + \cdots
\end{equation}

where:
\begin{align}
p_0 &= \pm i\sqrt{q} = \pm \frac{i}{x} \\
p_1 &= -\frac{q'}{4q} = -\frac{-2x^{-3}}{4x^{-2}} = \frac{1}{2x}
\end{align}

\textbf{The next term $p_2$:} From the lecture notes (page 68), the recursion relation gives:
\begin{equation}
p_2 = -\frac{p_1' + p_1^2}{2p_0}
\end{equation}

\textbf{Computing $p_1'$ and $p_1^2$:}
\begin{align}
p_1 &= \frac{1}{2x} \\
p_1' &= -\frac{1}{2x^2} \\
p_1^2 &= \frac{1}{4x^2}
\end{align}

\textbf{Thus:}
\begin{equation}
p_1' + p_1^2 = -\frac{1}{2x^2} + \frac{1}{4x^2} = -\frac{1}{4x^2}
\end{equation}

\textbf{And:}
\begin{equation}
p_2 = -\frac{-1/(4x^2)}{2 \cdot (\pm i/x)} = \frac{1/(4x^2)}{\pm 2i/x} = \mp\frac{i}{8x}
\end{equation}

\subsection*{Step 20: The Improved WKB Phase}

\textbf{What are we doing?} We integrate to find the corrected phase function.

\textbf{Why?} The phase $S(x) = \int p\,\dd x$ determines the oscillatory behavior of the solution.

\textbf{The total phase:}
\begin{equation}
S(x) = \frac{1}{\eps}\int p_0\,\dd x + \int p_1\,\dd x + \eps\int p_2\,\dd x + O(\eps^2)
\end{equation}

\textbf{Computing each integral from $x=1$:}
\begin{align}
\int_1^x p_0\,\dd s &= \pm i\int_1^x \frac{1}{s}\,\dd s = \pm i\log x \\
\int_1^x p_1\,\dd s &= \int_1^x \frac{1}{2s}\,\dd s = \frac{1}{2}\log x \\
\int_1^x p_2\,\dd s &= \mp i\int_1^x \frac{1}{8s}\,\dd s = \mp\frac{i\log x}{8}
\end{align}

\textbf{The corrected WKB solution (taking the sine combination):}
\begin{equation}
y(x) = D\sqrt{x}\sin\left(\frac{\log x}{\eps} - \frac{\eps\log x}{8}\right)
\end{equation}

\textbf{Recalling $\eps = 1/\lambda$:}
\begin{equation}
y(x) = D\sqrt{x}\sin\left(\lambda\log x - \frac{\log x}{8\lambda}\right)
\end{equation}

\subsection*{Step 21: Apply Boundary Conditions with the Correction}

\textbf{What are we doing?} We apply the boundary conditions to the corrected WKB solution.

\textbf{Why?} This will give us improved eigenvalue estimates.

\textbf{The boundary condition $y(1) = 0$:} At $x = 1$, $\log 1 = 0$, so $y(1) = 0$ is automatically satisfied. \checkmark

\textbf{The boundary condition $y(e) = 0$:}
\begin{equation}
y(e) = D\sqrt{e}\sin\left(\lambda - \frac{1}{8\lambda}\right) = 0
\end{equation}

For non-trivial solutions:
\begin{equation}
\lambda - \frac{1}{8\lambda} = n\pi, \quad n = 1, 2, 3, \ldots
\end{equation}

\textbf{Solving for $\lambda$:} Multiplying by $\lambda$:
\begin{equation}
\lambda^2 - n\pi\lambda - \frac{1}{8} = 0
\end{equation}

Wait, this should be:
\begin{equation}
\lambda^2 - \frac{1}{8} = n\pi\lambda
\end{equation}

Actually, let me redo this. From $\lambda - \frac{1}{8\lambda} = n\pi$:
\begin{equation}
\lambda^2 - \frac{1}{8} = n\pi\lambda
\end{equation}

\begin{equation}
\lambda^2 - n\pi\lambda - \frac{1}{8} = 0
\end{equation}

Hmm, this doesn't match what we want. Let me reconsider the sign of the correction.

\textbf{Re-examining the correction:} The phase correction from $p_2$ should be reconsidered. Looking at the structure more carefully, the higher-order WKB correction to the eigenvalue condition can be written as:
\begin{equation}
\lambda - \frac{1}{8\lambda} = n\pi
\end{equation}

\textbf{For large $\lambda$:} We can solve this perturbatively. Write $\lambda = n\pi + \delta$ where $\delta \ll n\pi$:
\begin{equation}
n\pi + \delta - \frac{1}{8(n\pi + \delta)} = n\pi
\end{equation}

\begin{equation}
\delta = \frac{1}{8(n\pi + \delta)} \approx \frac{1}{8n\pi}
\end{equation}

\textbf{The improved WKB eigenvalue:}
\begin{equation}
\lambda_n^{\text{WKB (improved)}} = n\pi + \frac{1}{8n\pi} + O(n^{-3})
\end{equation}

\subsection*{Step 22: Compare with the Exact Result}

\textbf{What are we doing?} We compare the improved WKB eigenvalue with the exact eigenvalue.

\textbf{Recall the exact eigenvalue expansion:}
\begin{equation}
\lambda_n^{\text{exact}} = n\pi + \frac{1}{8n\pi} + O(n^{-3})
\end{equation}

\textbf{The improved WKB eigenvalue:}
\begin{equation}
\lambda_n^{\text{WKB (improved)}} = n\pi + \frac{1}{8n\pi} + O(n^{-3})
\end{equation}

\textbf{These agree to order $O(1/n)$!}

\begin{equation}
\boxed{\lambda_n^{\text{WKB (improved)}} = \lambda_n^{\text{exact}} + O(n^{-3})}
\end{equation}

\subsection*{Step 23: The Form of the Next Term in the Approximation to $y(x)$}

\textbf{What are we doing?} We state explicitly the form of the higher-order WKB correction to the eigenfunction.

\textbf{The improved WKB solution:}
\begin{equation}
y(x) = \sqrt{x}\sin\left(\lambda\log x - \frac{\log x}{8\lambda}\right)
\end{equation}

\textbf{Alternative form:} This can be written as:
\begin{equation}
y(x) = \sqrt{x}\sin\left[\left(\lambda - \frac{1}{8\lambda}\right)\log x\right]
\end{equation}

\textbf{Comparison with the exact solution:} The exact solution is:
\begin{equation}
y(x) = \sqrt{x}\sin(\mu\log x) = \sqrt{x}\sin\left(\sqrt{\lambda^2 - \frac{1}{4}}\log x\right)
\end{equation}

For large $\lambda$:
\begin{equation}
\mu = \sqrt{\lambda^2 - \frac{1}{4}} = \lambda\sqrt{1 - \frac{1}{4\lambda^2}} \approx \lambda - \frac{1}{8\lambda} + O(\lambda^{-3})
\end{equation}

So the improved WKB solution matches the exact solution to next order in $1/\lambda$.

\subsection*{Step 24: Final Answer to Part (ii)}

\textbf{Answer to part (ii):}

\emph{Yes, including more terms of the WKB approximation gives a better result for the eigenvalues.}

\emph{The form of the next term in the approximation to $y(x)$:}

The leading-order WKB solution is:
\begin{equation}
y^{(0)}(x) = \sqrt{x}\sin(\lambda\log x)
\end{equation}

Including the next-order correction, the improved solution is:
\begin{equation}
y^{(1)}(x) = \sqrt{x}\sin\left(\lambda\log x - \frac{\log x}{8\lambda}\right)
\end{equation}

\emph{How this gives a better result for the eigenvalues:}

\begin{enumerate}
\item The leading-order WKB eigenvalue condition $\sin(\lambda) = 0$ gives $\lambda_n^{(0)} = n\pi$.

\item The improved WKB eigenvalue condition $\sin\left(\lambda - \frac{1}{8\lambda}\right) = 0$ gives:
\begin{equation}
\lambda - \frac{1}{8\lambda} = n\pi
\end{equation}
Solving perturbatively: $\lambda_n^{(1)} = n\pi + \frac{1}{8n\pi} + O(n^{-3})$.

\item The exact eigenvalues are:
\begin{equation}
\lambda_n^{\text{exact}} = \sqrt{n^2\pi^2 + \frac{1}{4}} = n\pi + \frac{1}{8n\pi} + O(n^{-3})
\end{equation}

\item \textbf{Comparison:}
\begin{align}
|\lambda_n^{(0)} - \lambda_n^{\text{exact}}| &= \frac{1}{8n\pi} + O(n^{-3}) \\
|\lambda_n^{(1)} - \lambda_n^{\text{exact}}| &= O(n^{-3})
\end{align}

The improved WKB approximation reduces the error from $O(n^{-1})$ to $O(n^{-3})$.
\end{enumerate}

\section*{Summary and Conclusion}

\textbf{Complete answer to Question 5:}

\begin{enumerate}
\item \textbf{Leading-order WKB eigenvalues:} $\lambda_n^{\text{WKB}} = n\pi$, $n = 1,2,3,\ldots$

\item \textbf{Exact eigenvalues:} $\displaystyle\lambda_n^{\text{exact}} = \sqrt{n^2\pi^2 + \frac{1}{4}}$, $n = 1,2,3,\ldots$

\item \textbf{Discrepancy:} $\displaystyle\Delta\lambda_n = \lambda_n^{\text{exact}} - \lambda_n^{\text{WKB}} = \frac{1}{8n\pi} + O(n^{-3})$

\item \textbf{Part (i):} Yes, the discrepancies are consistent with the WKB approximation. The WKB method is an asymptotic approximation for large $\lambda$, and the $O(1/n)$ discrepancy is precisely the expected error for a leading-order asymptotic result.

\item \textbf{Part (ii):} Yes, including more WKB terms gives a better result. The next term in the WKB approximation to $y(x)$ is:
\begin{equation}
y(x) = \sqrt{x}\sin\left(\lambda\log x - \frac{\log x}{8\lambda}\right)
\end{equation}
This leads to the improved eigenvalue condition $\lambda - \frac{1}{8\lambda} = n\pi$, which gives eigenvalues accurate to $O(n^{-3})$ instead of $O(n^{-1})$.
\end{enumerate}

\textbf{Key lessons:}
\begin{itemize}
\item WKB is an asymptotic method that typically gives approximate, not exact, results.
\item For this problem, the WKB eigenvalues differ from exact eigenvalues by $O(1/\lambda)$.
\item The discrepancy arises because the exact characteristic exponent is $\mu = \sqrt{\lambda^2 - 1/4}$, not $\mu = \lambda$.
\item Higher-order WKB corrections systematically improve the approximation.
\item The next-order correction captures the $1/4$ term in $\lambda^2 - 1/4$, reducing the error from $O(1/\lambda)$ to $O(1/\lambda^3)$.
\end{itemize}

\end{document}
