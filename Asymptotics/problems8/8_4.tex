\documentclass[11pt,a4paper]{article}
\usepackage{inputenc}
\usepackage{amsmath,amssymb,amsthm}
\usepackage[margin=2.5cm]{geometry}
\usepackage{enumitem}
\usepackage{xcolor}

% Custom environments for pedagogical structure
\newtheoremstyle{problem}
  {10pt}{10pt}{\normalfont}{}{\bfseries}{.}{.5em}{}
\theoremstyle{problem}
\newtheorem{problem}{Problem}

\newenvironment{strategy}{\par\noindent\textbf{Strategy:}\itshape}{\par}
\newenvironment{justification}{\par\noindent\textbf{Justification:}\itshape}{\par}
\newenvironment{technique}{\par\noindent\textbf{Technique:}\itshape}{\par}
\newenvironment{reflection}{\par\noindent\textbf{Reflection:}\itshape}{\par}
\newenvironment{keyconcept}{\par\noindent\textbf{Key Concept:}\itshape}{\par}

\title{Asymptotics Problem 8.4: Complete Pedagogical Solution}
\author{Boundary Layer with Non-Standard Width}
\date{}

\begin{document}

\maketitle

\begin{problem}
Find a first-order uniform expansion as $\varepsilon \to 0$ for $y(x)$ satisfying
\[
\varepsilon y'' + x^2 y' - x^3 y = 0, \quad y(0) = \alpha, \quad y(1) = \beta.
\]
\end{problem}

\section*{Solution: Step-by-Step Atomic Breakdown}

\subsection*{Step 1: Problem Classification and Initial Analysis}

\begin{strategy}
This is a singularly perturbed second-order linear ODE of the general form
\[
\varepsilon y'' + p(x)y' + q(x)y = 0
\]
with $p(x) = x^2$ and $q(x) = -x^3$. Our systematic approach follows the workflow from Lecture Notes \S6.2.3:
\begin{enumerate}[leftmargin=*]
\item Identify candidate locations for boundary layers
\item Determine the boundary layer width $\delta(\varepsilon)$ via dominant balance
\item Compute outer and inner solutions
\item Match solutions using Prandtl's matching criterion
\item Construct the composite solution
\end{enumerate}
\end{strategy}

\begin{justification}
Why is this a boundary layer problem? The coefficient $\varepsilon$ multiplying the highest derivative $y''$ is small. As $\varepsilon \to 0$, the order of the ODE effectively reduces from 2 to 1, but we have \emph{two} boundary conditions. This is the hallmark of a singular perturbation problem where rapid variation (a boundary layer) must occur somewhere to accommodate both boundary conditions.
\end{justification}

\subsection*{Step 2: Identifying the Boundary Layer Location}

\noindent\textbf{What we examine:} The sign and zeros of the coefficient $p(x) = x^2$.

\begin{keyconcept}
From the general theory of boundary layers (Lecture Notes \S6.2.1, equations (340)--(354)), for the ODE $\varepsilon y'' + p(x)y' + q(x)y = 0$:
\begin{itemize}
\item If $p(x) > 0$ on $[0,1]$: boundary layer at $x = 0$
\item If $p(x) < 0$ on $[0,1]$: boundary layer at $x = 1$
\item If $p(x_0) = 0$ for some $x_0 \in [0,1]$: special treatment required
\end{itemize}
The physical intuition: the sign of $p(x)$ determines whether information ``propagates'' from left to right or right to left.
\end{keyconcept}

\noindent\textbf{Analysis of $p(x) = x^2$:}

\begin{itemize}[leftmargin=*]
\item For $x > 0$: $p(x) = x^2 > 0$ \quad $\checkmark$
\item At $x = 0$: $p(0) = 0$ \quad (vanishes at boundary!)
\item At $x = 1$: $p(1) = 1 > 0$ \quad $\checkmark$
\end{itemize}

\begin{justification}
Since $p(x) = x^2 > 0$ for all $x > 0$, there \emph{cannot} be a boundary layer in the interior or at $x = 1$. The only possible location is $x = 0$.

However, this case is special because $p(0) = 0$---the coefficient of $y'$ vanishes exactly at the boundary point where we expect the boundary layer! According to Lecture Notes \S6.2.2 (equation (356) and surrounding discussion): ``For a boundary point $x_0$ with $p(x_0) = 0$, we can have boundary layers of different width than $\sim \varepsilon$.''

This is precisely our situation. We must determine the actual boundary layer width through careful dominant balance analysis.
\end{justification}

\begin{center}
\fbox{\textbf{Conclusion:} Boundary layer at $x = 0$ with non-standard width $\delta \neq O(\varepsilon)$.}
\end{center}

\subsection*{Step 3: Computing the Outer Solution}

\noindent\textbf{What we do:} In the outer region (away from $x = 0$), the solution varies slowly, so we neglect the $\varepsilon y''$ term.

\begin{technique}
Setting $\varepsilon = 0$ in the original ODE gives the leading-order outer equation:
\[
x^2 y'_0 - x^3 y_0 = 0.
\]
\end{technique}

\noindent\textbf{Solving the outer equation:}

\noindent Divide by $x^2$ (valid for $x \neq 0$, which is the outer region):
\[
y'_0 - x y_0 = 0.
\]

\begin{technique}
This is a first-order linear homogeneous ODE. Separate variables:
\[
\frac{y'_0}{y_0} = x \quad \Longrightarrow \quad \frac{dy_0}{y_0} = x\,dx.
\]
Integrate both sides:
\[
\ln|y_0| = \frac{x^2}{2} + C'.
\]
Exponentiate:
\[
y_0(x) = C\exp\left(\frac{x^2}{2}\right),
\]
where $C$ is an arbitrary constant.
\end{technique}

\noindent\textbf{Applying the boundary condition at $x = 1$:}

\begin{justification}
Since the boundary layer is at $x = 0$, the outer solution is valid at $x = 1$. We can therefore apply the boundary condition $y(1) = \beta$ directly to the outer solution:
\[
y_0(1) = C\exp\left(\frac{1}{2}\right) = \beta \quad \Longrightarrow \quad C = \beta e^{-1/2} = \frac{\beta}{\sqrt{e}}.
\]
\end{justification}

\noindent\textbf{Final outer solution:}
\[
\boxed{y_0(x) = \beta \exp\left(\frac{x^2 - 1}{2}\right)}
\]

\begin{reflection}
Notice that $y_0(0) = \beta e^{-1/2} = \beta/\sqrt{e} \neq \alpha$ in general. This confirms that the outer solution cannot satisfy both boundary conditions---there must be a boundary layer at $x = 0$ to transition from $y(0) = \alpha$ to the outer solution.
\end{reflection}

\subsection*{Step 4: Setting Up the Inner Region}

\noindent\textbf{What we do:} Introduce stretched (inner) coordinates to resolve the rapid variation near $x = 0$.

\begin{technique}
Define the inner variable $X$ by:
\[
x = \delta X, \quad \text{where } \delta = \delta(\varepsilon) \to 0 \text{ as } \varepsilon \to 0.
\]
Define the inner solution $Y(X) = y(x) = y(\delta X)$.
\end{technique}

\begin{justification}
The scaling $x = \delta X$ ``zooms in'' on the boundary layer region near $x = 0$. When $x = O(\delta)$, we have $X = O(1)$, so the inner variable $X$ is order one within the boundary layer. The function $\delta(\varepsilon)$ represents the boundary layer \emph{width}---the region where rapid variation occurs.
\end{justification}

\noindent\textbf{Transforming derivatives:}

Using the chain rule:
\begin{align*}
\frac{dy}{dx} &= \frac{dY}{dX} \cdot \frac{dX}{dx} = \frac{1}{\delta} Y'(X),\\[6pt]
\frac{d^2y}{dx^2} &= \frac{d}{dx}\left(\frac{1}{\delta}Y'\right) = \frac{1}{\delta}\cdot \frac{dY'}{dX}\cdot \frac{dX}{dx} = \frac{1}{\delta^2}Y''(X).
\end{align*}

\noindent\textbf{Substituting into the original ODE:}

The original equation is:
\[
\varepsilon y'' + x^2 y' - x^3 y = 0.
\]

With $x = \delta X$ and $y = Y$:
\begin{align*}
\varepsilon \cdot \frac{1}{\delta^2} Y'' + (\delta X)^2 \cdot \frac{1}{\delta} Y' - (\delta X)^3 Y &= 0\\[6pt]
\frac{\varepsilon}{\delta^2} Y'' + \frac{\delta^2 X^2}{\delta} Y' - \delta^3 X^3 Y &= 0\\[6pt]
\frac{\varepsilon}{\delta^2} Y'' + \delta X^2 Y' - \delta^3 X^3 Y &= 0.
\end{align*}

\begin{center}
\fbox{Inner equation: $\displaystyle \frac{\varepsilon}{\delta^2} Y'' + \delta X^2 Y' - \delta^3 X^3 Y = 0$}
\end{center}

\subsection*{Step 5: Dominant Balance Analysis}

\begin{strategy}
To find the boundary layer width $\delta$, we require that the leading terms in the inner equation balance each other as $\varepsilon \to 0$. This is the \emph{dominant balance} principle (Lecture Notes \S2.2.2 and \S6.2.2).
\end{strategy}

\noindent\textbf{Examining the three terms:}

The inner equation has three terms with coefficients:
\begin{enumerate}
\item $Y''$ term: coefficient $\varepsilon/\delta^2$
\item $Y'$ term: coefficient $\delta$
\item $Y$ term: coefficient $\delta^3$
\end{enumerate}

\begin{justification}
We seek $\delta \to 0$ as $\varepsilon \to 0$. Compare the magnitudes:
\begin{itemize}
\item The $Y$ term has coefficient $\delta^3$.
\item The $Y'$ term has coefficient $\delta$.
\item Since $\delta \to 0$, we have $\delta^3 \ll \delta$, so the $Y$ term is \emph{smaller} than the $Y'$ term.
\end{itemize}
Therefore, the third term ($-\delta^3 X^3 Y$) is subdominant and can be neglected at leading order.
\end{justification}

\noindent\textbf{Balancing the remaining two terms:}

\begin{technique}
For a distinguished limit (where both remaining terms contribute at the same order), we require:
\[
\frac{\varepsilon}{\delta^2} \sim \delta.
\]
Solving for $\delta$:
\[
\varepsilon \sim \delta^3 \quad \Longrightarrow \quad \boxed{\delta = \varepsilon^{1/3}}
\]
\end{technique}

\begin{reflection}
This is a \textbf{non-standard boundary layer width}! In the typical case where $p(x_0) \neq 0$ at the boundary, we get $\delta = \varepsilon$. Here, because $p(0) = 0$, we obtain $\delta = \varepsilon^{1/3}$, which is \emph{larger} than $\varepsilon$ (since $\varepsilon^{1/3} \gg \varepsilon$ for small $\varepsilon$).

This confirms the statement from Lecture Notes \S6.2.2: when the coefficient of $y'$ vanishes at the boundary, ``we can have boundary layers of different width than $\sim \varepsilon$, here $\sim \sqrt{\varepsilon}$'' (in some cases) or $\sim \varepsilon^{1/3}$ (in our case).
\end{reflection}

\begin{keyconcept}
Verification of dominant balance: With $\delta = \varepsilon^{1/3}$:
\begin{itemize}
\item $\varepsilon/\delta^2 = \varepsilon/\varepsilon^{2/3} = \varepsilon^{1/3} = \delta$ \quad $\checkmark$ (these balance)
\item $\delta^3 = \varepsilon \ll \delta = \varepsilon^{1/3}$ \quad $\checkmark$ (third term is subdominant)
\end{itemize}
The analysis is self-consistent.
\end{keyconcept}

\subsection*{Step 6: Deriving the Leading-Order Inner Equation}

\noindent\textbf{What we do:} With $\delta = \varepsilon^{1/3}$, write the inner equation and identify the leading-order problem.

\noindent Substituting $\delta = \varepsilon^{1/3}$:
\[
\varepsilon^{1/3} Y'' + \varepsilon^{1/3} X^2 Y' - \varepsilon X^3 Y = 0.
\]

\noindent Divide through by $\varepsilon^{1/3}$:
\[
Y'' + X^2 Y' - \varepsilon^{2/3} X^3 Y = 0.
\]

\begin{technique}
As $\varepsilon \to 0$, the term $\varepsilon^{2/3} X^3 Y \to 0$. At leading order:
\[
\boxed{Y''_0 + X^2 Y'_0 = 0}
\]
This is the leading-order inner equation.
\end{technique}

\begin{justification}
Why can we neglect $\varepsilon^{2/3} X^3 Y$? In the inner region, $X = O(1)$, so $X^3 = O(1)$. The coefficient $\varepsilon^{2/3} \to 0$, making this term small compared to the $O(1)$ terms $Y''$ and $X^2 Y'$.
\end{justification}

\subsection*{Step 7: Solving the Inner Equation}

\noindent\textbf{What we solve:} $Y''_0 + X^2 Y'_0 = 0$.

\subsubsection*{Step 7a: First Integration}

\begin{technique}
This is a second-order ODE that can be reduced to first order by the substitution $P = Y'_0$:
\[
P' + X^2 P = 0.
\]
This is a separable first-order linear ODE.
\end{technique}

\noindent Separate variables:
\[
\frac{dP}{P} = -X^2\,dX.
\]

\noindent Integrate:
\[
\ln|P| = -\frac{X^3}{3} + C_1.
\]

\noindent Exponentiate:
\[
P = Y'_0 = A\exp\left(-\frac{X^3}{3}\right),
\]
where $A$ is an arbitrary constant.

\subsubsection*{Step 7b: Second Integration}

\begin{technique}
Integrate $Y'_0$ to find $Y_0$:
\[
Y_0(X) = A\int_0^X \exp\left(-\frac{s^3}{3}\right)ds + B,
\]
where $B$ is another arbitrary constant, and we choose the lower limit of integration as $0$ for convenience.
\end{technique}

\begin{justification}
The choice of lower limit $0$ is convenient because:
\begin{enumerate}
\item At $X = 0$, the integral vanishes, giving $Y_0(0) = B$ directly.
\item This form makes applying the boundary condition at $x = 0$ (i.e., $X = 0$) straightforward.
\end{enumerate}
\end{justification}

\noindent\textbf{General inner solution:}
\[
\boxed{Y_0(X) = A\int_0^X \exp\left(-\frac{s^3}{3}\right)ds + B}
\]

\subsection*{Step 8: Applying the Boundary Condition at $x = 0$}

\noindent\textbf{What we apply:} $y(0) = \alpha$, which in inner variables is $Y_0(0) = \alpha$.

\noindent From the inner solution:
\[
Y_0(0) = A\int_0^0 \exp\left(-\frac{s^3}{3}\right)ds + B = 0 + B = B.
\]

\noindent Therefore:
\[
\boxed{B = \alpha}
\]

\noindent\textbf{Inner solution with boundary condition:}
\[
Y_0(X) = A\int_0^X \exp\left(-\frac{s^3}{3}\right)ds + \alpha
\]

\begin{reflection}
We have one remaining unknown constant $A$, which will be determined by matching the inner and outer solutions in the overlap (intermediate) region.
\end{reflection}

\subsection*{Step 9: Asymptotic Matching}

\begin{strategy}
We use \textbf{Prandtl's matching rule} (Lecture Notes \S6.1.2): the outer limit of the inner solution must equal the inner limit of the outer solution. This is valid here because both limits approach constants, making the matching straightforward.
\end{strategy}

\subsubsection*{Step 9a: Inner Limit of the Outer Solution}

\begin{technique}
The ``inner limit'' means taking $x \to 0^+$ in the outer solution:
\[
\lim_{x \to 0^+} y_0(x) = \lim_{x \to 0^+} \beta\exp\left(\frac{x^2-1}{2}\right) = \beta\exp\left(\frac{0-1}{2}\right) = \beta e^{-1/2} = \frac{\beta}{\sqrt{e}}.
\]
\end{technique}

\subsubsection*{Step 9b: Outer Limit of the Inner Solution}

\begin{technique}
The ``outer limit'' means taking $X \to +\infty$ in the inner solution (since $X = x/\varepsilon^{1/3} \to \infty$ as we leave the boundary layer):
\[
\lim_{X \to \infty} Y_0(X) = \lim_{X \to \infty} \left[A\int_0^X \exp\left(-\frac{s^3}{3}\right)ds + \alpha\right] = A \cdot I + \alpha,
\]
where we define the integral:
\[
I = \int_0^\infty \exp\left(-\frac{s^3}{3}\right)ds.
\]
\end{technique}

\subsubsection*{Step 9c: Evaluating the Integral $I$}

\begin{technique}
This is a generalized Gaussian integral. Using the substitution $u = s^3/3$, so $s = (3u)^{1/3}$ and $ds = (3u)^{-2/3}du$:
\[
I = \int_0^\infty e^{-u} \cdot (3u)^{-2/3}\,du = 3^{-2/3}\int_0^\infty u^{-2/3}e^{-u}\,du = 3^{-2/3}\Gamma\left(\frac{1}{3}\right),
\]
where $\Gamma(z) = \int_0^\infty t^{z-1}e^{-t}dt$ is the gamma function (Lecture Notes \S2.6.1).
\end{technique}

\begin{justification}
We used the gamma function identity $\Gamma(z) = \int_0^\infty t^{z-1}e^{-t}dt$ with $z = 1/3$:
\[
\int_0^\infty u^{1/3 - 1}e^{-u}\,du = \int_0^\infty u^{-2/3}e^{-u}\,du = \Gamma\left(\frac{1}{3}\right).
\]
The numerical value is $\Gamma(1/3) \approx 2.679$.
\end{justification}

\noindent\textbf{Result:}
\[
I = \int_0^\infty \exp\left(-\frac{s^3}{3}\right)ds = 3^{-2/3}\Gamma\left(\frac{1}{3}\right)
\]

\subsubsection*{Step 9d: Applying Prandtl's Matching Condition}

\begin{technique}
Equate the inner limit of outer and outer limit of inner:
\[
\frac{\beta}{\sqrt{e}} = \alpha + A \cdot I.
\]
Solve for $A$:
\[
A = \frac{\beta e^{-1/2} - \alpha}{I} = \frac{\beta/\sqrt{e} - \alpha}{I}.
\]
\end{technique}

\noindent\textbf{Matched constant:}
\[
\boxed{A = \frac{\beta e^{-1/2} - \alpha}{I} = \frac{\beta e^{-1/2} - \alpha}{3^{-2/3}\Gamma(1/3)}}
\]

\begin{reflection}
The constant $A$ encodes the ``mismatch'' between the boundary value $\alpha$ and the limiting value $\beta e^{-1/2}$ of the outer solution as $x \to 0$. If $\alpha = \beta e^{-1/2}$, then $A = 0$ and the inner solution reduces to the constant $\alpha$---no boundary layer correction is needed!
\end{reflection}

\subsection*{Step 10: Constructing the Composite Solution}

\begin{strategy}
The composite solution is constructed by adding the outer and inner solutions and subtracting their common limit (to avoid double-counting in the overlap region). This follows Lecture Notes \S6.1.3 and equation (353):
\[
y_c(x) = y_{\text{outer}}(x) + Y_{\text{inner}}\left(\frac{x}{\delta}\right) - (\text{common limit}).
\]
\end{strategy}

\noindent\textbf{Components:}
\begin{align*}
\text{Outer solution:} \quad & y_0(x) = \beta\exp\left(\frac{x^2-1}{2}\right)\\[6pt]
\text{Inner solution:} \quad & Y_0(X) = \alpha + A\int_0^X \exp\left(-\frac{s^3}{3}\right)ds\\[6pt]
\text{Common limit:} \quad & \frac{\beta}{\sqrt{e}} = \beta e^{-1/2}
\end{align*}

\begin{technique}
The common limit is what both solutions approach in the intermediate (matching) region:
\begin{itemize}
\item As $x \to 0^+$: $y_0(x) \to \beta e^{-1/2}$
\item As $X \to \infty$: $Y_0(X) \to \alpha + AI = \beta e^{-1/2}$ (by matching)
\end{itemize}
\end{technique}

\noindent\textbf{Composite solution formula:}
\[
y_c(x) = y_0(x) + Y_0\left(\frac{x}{\varepsilon^{1/3}}\right) - \frac{\beta}{\sqrt{e}}
\]

\noindent Substituting:
\[
y_c(x) = \beta\exp\left(\frac{x^2-1}{2}\right) + \alpha + A\int_0^{x/\varepsilon^{1/3}} \exp\left(-\frac{s^3}{3}\right)ds - \frac{\beta}{\sqrt{e}}
\]

\noindent\textbf{Simplifying:}

\begin{technique}
Using $A = (\beta e^{-1/2} - \alpha)/I$ and the fact that $\int_0^\infty e^{-s^3/3}ds = I$:
\[
A\int_0^{X} e^{-s^3/3}ds = \frac{\beta e^{-1/2} - \alpha}{I}\int_0^{X} e^{-s^3/3}ds = \left(\frac{\beta}{\sqrt{e}} - \alpha\right)\frac{\int_0^{X} e^{-s^3/3}ds}{\int_0^{\infty} e^{-s^3/3}ds}.
\]
\end{technique}

\noindent Therefore, the composite solution can be written as:
\begin{align*}
y_c(x) &= \beta\exp\left(\frac{x^2-1}{2}\right) + \left(\frac{\beta}{\sqrt{e}} - \alpha\right)\left[\frac{\int_0^{x\varepsilon^{-1/3}} e^{-s^3/3}ds}{\int_0^{\infty} e^{-s^3/3}ds} - 1\right]
\end{align*}

\subsection*{Step 11: Final Answer in Standard Form}

\begin{center}
\fbox{\begin{minipage}{0.95\textwidth}
\textbf{First-Order Uniform Expansion:}
\[
y_c(x) = \beta\exp\left(\frac{x^2-1}{2}\right) + \left(\frac{\beta}{\sqrt{e}} - \alpha\right)\left[\frac{\displaystyle\int_0^{x\varepsilon^{-1/3}} e^{-s^3/3}\,ds}{\displaystyle\int_0^{\infty} e^{-s^3/3}\,ds} - 1\right]
\]

Equivalently, with explicit integral notation:
\[
y_c(x) = \beta\exp\left(\frac{x^2-1}{2}\right) + \left(\frac{\beta}{\sqrt{e}} - \alpha\right)\left[\frac{\displaystyle\int_0^{x/\varepsilon^{1/3}} \exp\left(-\frac{s^3}{3}\right)ds}{\displaystyle\int_0^{\infty} \exp\left(-\frac{s^3}{3}\right)ds} - 1\right]
\]

where $\displaystyle\int_0^{\infty} \exp\left(-\frac{s^3}{3}\right)ds = 3^{-2/3}\Gamma\left(\frac{1}{3}\right)$.
\end{minipage}}
\end{center}

\subsection*{Step 12: Verification of the Solution}

\subsubsection*{Verification 1: Boundary Condition at $x = 0$}

\begin{technique}
As $x \to 0^+$:
\begin{itemize}
\item $\beta\exp\left(\frac{x^2-1}{2}\right) \to \beta e^{-1/2}$
\item $x\varepsilon^{-1/3} \to 0$, so $\int_0^{x\varepsilon^{-1/3}} e^{-s^3/3}ds \to 0$
\end{itemize}
Therefore:
\[
y_c(0) = \frac{\beta}{\sqrt{e}} + \left(\frac{\beta}{\sqrt{e}} - \alpha\right)(0 - 1) = \frac{\beta}{\sqrt{e}} - \frac{\beta}{\sqrt{e}} + \alpha = \alpha \quad \checkmark
\]
\end{technique}

\subsubsection*{Verification 2: Boundary Condition at $x = 1$}

\begin{technique}
At $x = 1$, the inner variable $X = 1/\varepsilon^{1/3} \to \infty$ as $\varepsilon \to 0$:
\begin{itemize}
\item $\beta\exp\left(\frac{1-1}{2}\right) = \beta$
\item $\int_0^{\varepsilon^{-1/3}} e^{-s^3/3}ds \to \int_0^{\infty} e^{-s^3/3}ds = I$
\end{itemize}
Therefore:
\[
y_c(1) = \beta + \left(\frac{\beta}{\sqrt{e}} - \alpha\right)\left(\frac{I}{I} - 1\right) = \beta + \left(\frac{\beta}{\sqrt{e}} - \alpha\right)(0) = \beta \quad \checkmark
\]
\end{technique}

\subsubsection*{Verification 3: Behavior in Outer Region}

\begin{technique}
For fixed $x > 0$ as $\varepsilon \to 0$:
\[
x\varepsilon^{-1/3} \to \infty, \quad \text{so} \quad \frac{\int_0^{x\varepsilon^{-1/3}} e^{-s^3/3}ds}{I} \to 1.
\]
Therefore:
\[
y_c(x) \to \beta\exp\left(\frac{x^2-1}{2}\right) + \left(\frac{\beta}{\sqrt{e}} - \alpha\right)(1-1) = \beta\exp\left(\frac{x^2-1}{2}\right) = y_0(x) \quad \checkmark
\]
\end{technique}

\subsection*{Step 13: Physical Interpretation and Key Insights}

\begin{reflection}
This problem illustrates several important asymptotic concepts:

\begin{enumerate}
\item \textbf{Non-standard boundary layer width:} When the coefficient of $y'$ vanishes at the boundary ($p(0) = x^2|_{x=0} = 0$), the boundary layer width is determined by dominant balance as $\delta = \varepsilon^{1/3}$, not $\delta = \varepsilon$. This is because near $x = 0$, the $y'$ term ``turns on'' gradually (like $x^2$), requiring a wider region for the solution to transition.

\item \textbf{The integral $\int_0^\infty e^{-s^3/3}ds$:} This generalized Gaussian integral appears because the inner equation $Y'' + X^2 Y' = 0$ has solutions involving $e^{-X^3/3}$. The exponent $X^3/3$ comes from integrating the coefficient $X^2$.

\item \textbf{Matching via Prandtl's rule:} Both the inner and outer solutions approach constants as they enter the intermediate region, making Prandtl's matching (comparing limits) straightforward. Van Dyke matching would give the same result here.

\item \textbf{Composite solution structure:} The additive form $y_c = y_{\text{outer}} + Y_{\text{inner}} - (\text{common limit})$ ensures that:
\begin{itemize}
\item In the inner region: $y_{\text{outer}} \approx \text{common limit}$, so $y_c \approx Y_{\text{inner}}$
\item In the outer region: $Y_{\text{inner}} \approx \text{common limit}$, so $y_c \approx y_{\text{outer}}$
\end{itemize}
This is the essence of matched asymptotic expansions from Lecture Notes \S6.1.3.
\end{enumerate}
\end{reflection}

\subsection*{Summary Table}

\begin{center}
\renewcommand{\arraystretch}{1.8}
\begin{tabular}{|l|l|}
\hline
\textbf{Quantity} & \textbf{Value} \\
\hline
Boundary layer location & $x = 0$ \\
\hline
Boundary layer width & $\delta = \varepsilon^{1/3}$ \\
\hline
Inner variable & $X = x/\varepsilon^{1/3}$ \\
\hline
Outer solution & $y_0(x) = \beta\exp\left(\dfrac{x^2-1}{2}\right)$ \\
\hline
Inner equation & $Y''_0 + X^2 Y'_0 = 0$ \\
\hline
Inner solution & $Y_0(X) = \alpha + A\displaystyle\int_0^X e^{-s^3/3}\,ds$ \\
\hline
Matching constant & $A = \dfrac{\beta e^{-1/2} - \alpha}{3^{-2/3}\Gamma(1/3)}$ \\
\hline
Common limit & $\beta e^{-1/2} = \dfrac{\beta}{\sqrt{e}}$ \\
\hline
\end{tabular}
\end{center}

\subsection*{Connection to Lecture Material}

\begin{keyconcept}
This problem is a direct application of the boundary layer workflow from Lecture Notes \S6.2.3, with the special feature that $p(x_0) = 0$ at the boundary (case discussed in \S6.2.2, equation (356)). The key steps parallel Example 1 on page 61--62 of the lecture notes, which treats $\varepsilon y'' + x^2 y' - y = 0$---a closely related problem.

The dominant balance analysis follows \S2.2.2, the matching procedure follows \S6.1.2 (Prandtl's rule), and the composite solution construction follows \S6.1.3 (equation (353)).
\end{keyconcept}

\end{document}
