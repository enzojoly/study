\documentclass[11pt,a4paper]{article}
\usepackage{inputenc}
\usepackage{amsmath,amssymb,amsthm}
\usepackage[margin=2.5cm]{geometry}
\usepackage{enumitem}
\usepackage{xcolor}

% Custom environments for pedagogical structure
\newtheoremstyle{problem}
  {10pt}{10pt}{\normalfont}{}{\bfseries}{.}{.5em}{}
\theoremstyle{problem}
\newtheorem{problem}{Problem}

\newenvironment{strategy}{\par\noindent\textbf{Strategy:}\itshape}{\par}
\newenvironment{justification}{\par\noindent\textbf{Justification:}\itshape}{\par}
\newenvironment{technique}{\par\noindent\textbf{Technique:}\itshape}{\par}
\newenvironment{reflection}{\par\noindent\textbf{Reflection:}\itshape}{\par}
\newenvironment{keyconcept}{\par\noindent\textbf{Key Concept:}\itshape}{\par}

\title{Asymptotics Problem 8.7: Complete Pedagogical Solution}
\author{Interior Layer Analysis via Matched Asymptotic Expansions}
\date{}

\begin{document}

\maketitle

\begin{problem}
Find an asymptotic expansion to leading order for the solution $y(x)$ to
\[
\varepsilon y'' + xy' + xy = 0, \quad \text{in } -1 < x < 1 \text{ for } \varepsilon \to 0
\]
with $y(-1) = e$ and $y(1) = 2e^{-1}$, given that the solution has an `interior layer'.
\end{problem}

\section*{Solution: Step-by-Step Atomic Breakdown}

\subsection*{Step 1: Identifying Problem Type and Layer Location}

\begin{strategy}
We have a singularly perturbed second-order linear ODE of the form $\varepsilon y'' + p(x)y' + q(x)y = 0$. We must:
\begin{enumerate}[leftmargin=*]
\item Identify where layers can occur by analysing the coefficient $p(x)$
\item Construct outer solutions in regions away from the layer
\item Construct an inner solution valid near the layer
\item Match these solutions using Prandtl's matching rule
\item Form a composite solution
\end{enumerate}
\end{strategy}

\noindent\textbf{What we have:} The ODE is
\[
\varepsilon y'' + xy' + xy = 0,
\]
so we identify $p(x) = x$ and $q(x) = x$.

\begin{justification}
From the general theory of boundary layers (Lecture Notes \S6.2.1, equations (340)--(353)), the sign of $p(x)$ at the boundaries determines where boundary layers can occur:
\begin{itemize}
\item At $x = -1$: $p(-1) = -1 < 0$. This means if we tried a boundary layer at $x = -1$, the inner solution would grow exponentially as we move into the domain, preventing matching.
\item At $x = 1$: $p(1) = 1 > 0$. This means if we tried a boundary layer at $x = 1$, the inner solution would again grow exponentially into the domain.
\end{itemize}
Since neither boundary can support a boundary layer, yet the outer equation cannot satisfy both boundary conditions (as we shall verify), the layer must occur at an \textbf{interior point} where $p(x) = 0$.
\end{justification}

\noindent The coefficient $p(x) = x$ vanishes at $x = 0$. This is where the interior layer is located.

\begin{keyconcept}
An \textbf{interior layer} occurs at a point $x_0$ inside the domain where the coefficient $p(x)$ vanishes. Unlike boundary layers (which occur at domain endpoints), interior layers separate the domain into two regions, each requiring its own outer solution. The interior layer acts as a transition zone connecting these outer solutions. This is described in Lecture Notes \S6.2.2, equations (354)--(356).
\end{keyconcept}

\subsection*{Step 2: Constructing the Outer Solutions}

\noindent\textbf{Goal:} Find the leading-order outer solutions valid in $x < 0$ and $x > 0$, away from the interior layer at $x = 0$.

\subsubsection*{Step 2a: Deriving the Outer Equation}

\begin{technique}
To obtain the outer expansion, we neglect the $\varepsilon$-dependent term in the ODE. This is justified because in the outer region (away from rapid transitions), the solution varies on the $O(1)$ length scale, so $y'' = O(1)$ and thus $\varepsilon y'' = O(\varepsilon) \ll O(1)$.
\end{technique}

\noindent Setting $\varepsilon = 0$ in the original ODE gives the leading-order outer equation:
\[
xy'_0 + xy_0 = 0.
\]

\noindent For $x \neq 0$, we can divide by $x$ to obtain:
\[
y'_0 + y_0 = 0.
\]

\subsubsection*{Step 2b: Solving the Outer Equation}

\noindent This is a first-order linear ODE. Separating variables:
\[
\frac{dy_0}{y_0} = -dx.
\]

\noindent Integrating both sides:
\[
\ln|y_0| = -x + C'.
\]

\noindent Exponentiating:
\[
y_0(x) = ae^{-x},
\]
where $a$ is an arbitrary constant.

\begin{justification}
This general solution is valid for $x \neq 0$. Since the interior layer at $x = 0$ separates the domain into two regions, we need different constants for the left ($x < 0$) and right ($x > 0$) outer solutions. Each outer solution must satisfy its respective boundary condition.
\end{justification}

\subsubsection*{Step 2c: Left Outer Solution ($x < 0$)}

\noindent We denote the left outer solution by $y_{0,a}(x)$. Applying the boundary condition at $x = -1$:
\[
y_{0,a}(-1) = e.
\]

\noindent Substituting $y_{0,a}(x) = ae^{-x}$:
\[
ae^{-(-1)} = ae^{1} = e.
\]

\noindent Solving for $a$:
\[
a = 1.
\]

\noindent Therefore, the \textbf{left outer solution} is:
\[
\boxed{y_{0,a}(x) = e^{-x}, \quad x < 0}
\]

\subsubsection*{Step 2d: Right Outer Solution ($x > 0$)}

\noindent We denote the right outer solution by $y_{0,b}(x)$. Applying the boundary condition at $x = 1$:
\[
y_{0,b}(1) = 2e^{-1}.
\]

\noindent Substituting $y_{0,b}(x) = ae^{-x}$:
\[
ae^{-1} = 2e^{-1}.
\]

\noindent Solving for $a$:
\[
a = 2.
\]

\noindent Therefore, the \textbf{right outer solution} is:
\[
\boxed{y_{0,b}(x) = 2e^{-x}, \quad x > 0}
\]

\subsubsection*{Step 2e: Verification of Inconsistency}

\begin{justification}
Notice that the two outer solutions have different limits as $x \to 0$:
\begin{align*}
\lim_{x \to 0^-} y_{0,a}(x) &= e^{0} = 1,\\
\lim_{x \to 0^+} y_{0,b}(x) &= 2e^{0} = 2.
\end{align*}
These limits do not match! This discontinuity confirms that an interior layer at $x = 0$ is necessary to connect the two outer solutions. The inner solution must smoothly transition from $y \to 1$ on the left to $y \to 2$ on the right.
\end{justification}

\subsection*{Step 3: Setting Up the Inner Problem}

\noindent\textbf{Goal:} Derive the inner equation valid in a neighbourhood of $x = 0$ of width $O(\delta(\varepsilon))$, where $\delta \to 0$ as $\varepsilon \to 0$.

\subsubsection*{Step 3a: Introducing the Inner Variable}

\begin{technique}
For an interior layer at $x = 0$, we introduce a rescaled ``inner'' variable:
\[
X = \frac{x}{\delta},
\]
where $\delta = \delta(\varepsilon)$ is the boundary layer width to be determined. We also define the inner solution:
\[
Y(X) = y(x) = y(\delta X).
\]
\end{technique}

\subsubsection*{Step 3b: Transforming Derivatives}

\noindent We need to express the derivatives of $y$ in terms of derivatives of $Y$.

\noindent Since $x = \delta X$:
\[
\frac{dy}{dx} = \frac{dY}{dX} \cdot \frac{dX}{dx} = \frac{1}{\delta}\frac{dY}{dX} = \frac{1}{\delta}Y'.
\]

\noindent Similarly:
\[
\frac{d^2y}{dx^2} = \frac{d}{dx}\left(\frac{1}{\delta}Y'\right) = \frac{1}{\delta^2}Y''.
\]

\subsubsection*{Step 3c: Substituting into the ODE}

\noindent The original equation $\varepsilon y'' + xy' + xy = 0$ becomes:
\[
\varepsilon \cdot \frac{1}{\delta^2}Y'' + (\delta X) \cdot \frac{1}{\delta}Y' + (\delta X) \cdot Y = 0.
\]

\noindent Simplifying:
\[
\frac{\varepsilon}{\delta^2}Y'' + XY' + \delta XY = 0.
\]

\subsection*{Step 4: Dominant Balance to Determine Layer Width}

\noindent\textbf{Goal:} Determine the scaling $\delta(\varepsilon)$ by requiring the most important terms in the inner equation to balance.

\subsubsection*{Step 4a: Identifying Terms}

\noindent Our transformed equation is:
\[
\frac{\varepsilon}{\delta^2}Y'' + XY' + \delta XY = 0.
\]

\noindent The three terms have coefficients:
\begin{itemize}
\item Term 1 ($Y''$): coefficient $\varepsilon/\delta^2$
\item Term 2 ($XY'$): coefficient $1$
\item Term 3 ($XY$): coefficient $\delta$
\end{itemize}

\subsubsection*{Step 4b: Comparing Term Sizes}

\begin{justification}
We seek a \textit{distinguished limit} where the leading terms balance. Since $\delta \to 0$ as $\varepsilon \to 0$, we have:
\begin{itemize}
\item The third term $\delta XY$ is \textbf{smaller} than the second term $XY'$ (since $\delta \ll 1$).
\item For a distinguished limit, the first term $(\varepsilon/\delta^2)Y''$ must balance the second term $XY'$.
\end{itemize}
\end{justification}

\noindent Setting the coefficients of the first two terms equal:
\[
\frac{\varepsilon}{\delta^2} = 1.
\]

\noindent Solving for $\delta$:
\[
\delta^2 = \varepsilon \implies \boxed{\delta = \sqrt{\varepsilon}}
\]

\begin{keyconcept}
The interior layer has width $O(\sqrt{\varepsilon})$, not $O(\varepsilon)$ as in standard boundary layers. This is a consequence of the coefficient $p(x) = x$ vanishing linearly at $x = 0$. From Lecture Notes \S6.2.2, when $p(x_0) = 0$ and $p'(x_0) \neq 0$, the boundary layer width is $\delta = \sqrt{\varepsilon}$ rather than $\delta = \varepsilon$.
\end{keyconcept}

\subsubsection*{Step 4c: Verifying Neglected Term}

\noindent With $\delta = \sqrt{\varepsilon}$, check that the third term is indeed negligible:
\[
\text{Third term coefficient} = \delta = \sqrt{\varepsilon} \to 0 \text{ as } \varepsilon \to 0. \quad \checkmark
\]

\subsection*{Step 5: Solving the Leading-Order Inner Equation}

\noindent\textbf{Goal:} Solve for the leading-order inner solution $Y_0(X)$.

\subsubsection*{Step 5a: The Inner ODE}

\noindent With $\delta = \sqrt{\varepsilon}$, neglecting the $O(\delta)$ term, the leading-order inner equation is:
\[
Y''_0 + XY'_0 = 0.
\]

\subsubsection*{Step 5b: Reducing the Order}

\begin{technique}
This is a second-order ODE with no explicit $Y_0$ term. We can reduce the order by setting $P = Y'_0$:
\[
P' + XP = 0.
\]
\end{technique}

\subsubsection*{Step 5c: Solving for $P = Y'_0$}

\noindent Separating variables:
\[
\frac{dP}{P} = -X\,dX.
\]

\noindent Integrating:
\[
\ln|P| = -\frac{X^2}{2} + C_1.
\]

\noindent Exponentiating:
\[
P = Y'_0 = A\exp\left(-\frac{X^2}{2}\right),
\]
where $A$ is an arbitrary constant.

\subsubsection*{Step 5d: Integrating to Find $Y_0$}

\noindent Integrating $Y'_0$:
\[
Y_0(X) = A\int_0^X \exp\left(-\frac{s^2}{2}\right)ds + B,
\]
where $B$ is another integration constant, and we have chosen the lower limit of integration as $0$ for convenience.

\noindent Therefore, the \textbf{general inner solution} is:
\[
\boxed{Y_0(X) = A\int_0^X \exp\left(-\frac{s^2}{2}\right)ds + B}
\]

\subsection*{Step 6: Matching the Inner and Outer Solutions}

\noindent\textbf{Goal:} Determine the constants $A$ and $B$ by requiring that the inner solution matches the outer solutions as $X \to \pm\infty$.

\subsubsection*{Step 6a: Prandtl's Matching Rule}

\begin{technique}
Prandtl's matching rule states that the inner limit of the outer solution must equal the outer limit of the inner solution. For an interior layer, we have two matching conditions:
\begin{align*}
\text{Left matching:} \quad &\lim_{x \to 0^-} y_{0,a}(x) = \lim_{X \to -\infty} Y_0(X),\\
\text{Right matching:} \quad &\lim_{x \to 0^+} y_{0,b}(x) = \lim_{X \to +\infty} Y_0(X).
\end{align*}
\end{technique}

\subsubsection*{Step 6b: Computing the Outer Limits}

\noindent From Step 2:
\begin{align*}
\lim_{x \to 0^-} y_{0,a}(x) &= \lim_{x \to 0^-} e^{-x} = e^0 = 1,\\
\lim_{x \to 0^+} y_{0,b}(x) &= \lim_{x \to 0^+} 2e^{-x} = 2e^0 = 2.
\end{align*}

\subsubsection*{Step 6c: Computing the Inner Limits}

\noindent For the inner solution $Y_0(X) = A\int_0^X e^{-s^2/2}ds + B$, we need the limits as $X \to \pm\infty$.

\begin{technique}
Recall the Gaussian integral:
\[
\int_0^\infty e^{-s^2/2}ds = \sqrt{\frac{\pi}{2}}.
\]
By symmetry of the Gaussian:
\[
\int_0^{-\infty} e^{-s^2/2}ds = -\int_{-\infty}^0 e^{-s^2/2}ds = -\sqrt{\frac{\pi}{2}}.
\]
\end{technique}

\noindent Therefore:
\begin{align*}
\lim_{X \to -\infty} Y_0(X) &= A \cdot \left(-\sqrt{\frac{\pi}{2}}\right) + B = B - A\sqrt{\frac{\pi}{2}},\\
\lim_{X \to +\infty} Y_0(X) &= A \cdot \sqrt{\frac{\pi}{2}} + B = B + A\sqrt{\frac{\pi}{2}}.
\end{align*}

\subsubsection*{Step 6d: Applying the Matching Conditions}

\noindent\textbf{Left matching} ($X \to -\infty$, matching with left outer solution):
\[
B - A\sqrt{\frac{\pi}{2}} = 1. \tag{M1}
\]

\noindent\textbf{Right matching} ($X \to +\infty$, matching with right outer solution):
\[
B + A\sqrt{\frac{\pi}{2}} = 2. \tag{M2}
\]

\subsubsection*{Step 6e: Solving for $A$ and $B$}

\noindent Adding equations (M1) and (M2):
\[
2B = 1 + 2 = 3 \implies \boxed{B = \frac{3}{2}}
\]

\noindent Subtracting (M1) from (M2):
\[
2A\sqrt{\frac{\pi}{2}} = 2 - 1 = 1 \implies A = \frac{1}{2\sqrt{\pi/2}} = \frac{1}{\sqrt{2\pi}}.
\]

\[
\boxed{A = \frac{1}{\sqrt{2\pi}}}
\]

\subsubsection*{Step 6f: The Matched Inner Solution}

\noindent Substituting $A$ and $B$ into the inner solution:
\[
Y_0(X) = \frac{1}{\sqrt{2\pi}}\int_0^X \exp\left(-\frac{s^2}{2}\right)ds + \frac{3}{2}.
\]

\begin{technique}
This can be written in terms of the error function. Recall:
\[
\text{erf}(z) = \frac{2}{\sqrt{\pi}}\int_0^z e^{-t^2}dt.
\]
With the substitution $t = s/\sqrt{2}$, we have $dt = ds/\sqrt{2}$, so:
\[
\int_0^X e^{-s^2/2}ds = \sqrt{2}\int_0^{X/\sqrt{2}} e^{-t^2}dt = \sqrt{2} \cdot \frac{\sqrt{\pi}}{2}\text{erf}\left(\frac{X}{\sqrt{2}}\right) = \sqrt{\frac{\pi}{2}}\text{erf}\left(\frac{X}{\sqrt{2}}\right).
\]
\end{technique}

\noindent Therefore:
\[
Y_0(X) = \frac{1}{\sqrt{2\pi}} \cdot \sqrt{\frac{\pi}{2}}\text{erf}\left(\frac{X}{\sqrt{2}}\right) + \frac{3}{2} = \frac{1}{2}\text{erf}\left(\frac{X}{\sqrt{2}}\right) + \frac{3}{2}.
\]

\noindent The matched inner solution is:
\[
\boxed{Y_0(X) = \frac{3}{2} + \frac{1}{2}\text{erf}\left(\frac{X}{\sqrt{2}}\right)}
\]

\subsection*{Step 7: Constructing the Composite Solution}

\noindent\textbf{Goal:} Form a uniformly valid composite solution across the entire domain $-1 < x < 1$.

\subsubsection*{Step 7a: Standard Composite Formula}

\begin{keyconcept}
For problems with a single boundary or interior layer, the standard composite solution is:
\[
y_c(x) = y_{\text{outer}}(x) + Y_{\text{inner}}\left(\frac{x}{\delta}\right) - (\text{common limit}).
\]
However, for an interior layer with \textit{two} outer solutions, the situation is more subtle because neither outer solution vanishes in the region where the other is valid.
\end{keyconcept}

\subsubsection*{Step 7b: Special Form for Interior Layers}

\begin{justification}
From the solution to Problem 8.7 and as noted in the solutions, when we have two outer solutions that match to a single inner solution, the composite solution takes a special form. The key observation is that both outer solutions have the form $ae^{-x}$ for different values of $a$. The inner solution smoothly transitions between the left limit ($a=1$) and right limit ($a=2$).

We can combine these by noting:
\[
y_c(x) = \left(\frac{3}{2} + \frac{1}{2}\text{erf}\left(\frac{x}{\sqrt{2\varepsilon}}\right)\right)e^{-x}.
\]
This works because:
\begin{itemize}
\item As $x \to -1$ (with $\varepsilon$ small): $\text{erf}(x/\sqrt{2\varepsilon}) \to -1$, so $y_c \to (3/2 - 1/2)e^{-x} = e^{-x} = y_{0,a}(x)$. \checkmark
\item As $x \to +1$ (with $\varepsilon$ small): $\text{erf}(x/\sqrt{2\varepsilon}) \to +1$, so $y_c \to (3/2 + 1/2)e^{-x} = 2e^{-x} = y_{0,b}(x)$. \checkmark
\item Near $x = 0$: The solution transitions smoothly via the error function.
\end{itemize}
\end{justification}

\subsubsection*{Step 7c: Final Composite Solution}

\noindent The uniformly valid \textbf{composite solution} to leading order is:
\[
\boxed{y_c(x) = \left(\frac{3}{2} + \frac{1}{2}\text{erf}\left(\frac{x}{\sqrt{2\varepsilon}}\right)\right)e^{-x}}
\]

\subsection*{Step 8: Verification and Interpretation}

\subsubsection*{Step 8a: Checking Boundary Conditions}

\noindent\textbf{At $x = -1$:} For small $\varepsilon$, we have $x/\sqrt{2\varepsilon} = -1/\sqrt{2\varepsilon} \to -\infty$, so $\text{erf}(x/\sqrt{2\varepsilon}) \to -1$:
\[
y_c(-1) \approx \left(\frac{3}{2} - \frac{1}{2}\right)e^{-(-1)} = 1 \cdot e = e. \quad \checkmark
\]

\noindent\textbf{At $x = 1$:} For small $\varepsilon$, we have $x/\sqrt{2\varepsilon} = 1/\sqrt{2\varepsilon} \to +\infty$, so $\text{erf}(x/\sqrt{2\varepsilon}) \to +1$:
\[
y_c(1) \approx \left(\frac{3}{2} + \frac{1}{2}\right)e^{-1} = 2e^{-1}. \quad \checkmark
\]

\subsubsection*{Step 8b: Physical Interpretation}

\begin{reflection}
The solution exhibits the following behaviour:
\begin{enumerate}
\item \textbf{In the region $x < 0$ (away from $x = 0$):} The solution follows $y \approx e^{-x}$, an exponentially decaying function as $x$ increases.

\item \textbf{In the region $x > 0$ (away from $x = 0$):} The solution follows $y \approx 2e^{-x}$, which has the same exponential decay but with twice the amplitude.

\item \textbf{Near $x = 0$ (the interior layer):} The solution rapidly transitions from amplitude $1$ to amplitude $2$ over a narrow region of width $O(\sqrt{\varepsilon})$. This transition is mediated by the error function, which provides a smooth ``sigmoid-like'' interpolation.

\item \textbf{The layer width $\delta = \sqrt{\varepsilon}$:} This is larger than the $O(\varepsilon)$ width of standard boundary layers because the coefficient $p(x) = x$ vanishes at the layer location. The linear vanishing of $p(x)$ leads to the square-root scaling, as predicted by dominant balance.
\end{enumerate}
\end{reflection}

\subsection*{Final Summary}

\begin{center}
\fbox{\begin{minipage}{0.95\textwidth}
\textbf{Complete Solution for Problem 8.7:}

\vspace{0.3cm}
\textbf{Given:} $\varepsilon y'' + xy' + xy = 0$ on $(-1,1)$ with $y(-1) = e$, $y(1) = 2e^{-1}$.

\vspace{0.2cm}
\textbf{Layer location:} Interior layer at $x = 0$ (where $p(x) = x$ vanishes).

\vspace{0.2cm}
\textbf{Layer width:} $\delta = \sqrt{\varepsilon}$ (determined by dominant balance).

\vspace{0.2cm}
\textbf{Left outer solution} ($x < 0$): $y_{0,a}(x) = e^{-x}$

\vspace{0.2cm}
\textbf{Right outer solution} ($x > 0$): $y_{0,b}(x) = 2e^{-x}$

\vspace{0.2cm}
\textbf{Inner solution:} $Y_0(X) = \dfrac{3}{2} + \dfrac{1}{2}\text{erf}\left(\dfrac{X}{\sqrt{2}}\right)$

\vspace{0.2cm}
\textbf{Composite solution:}
\[
y_c(x) = \left(\frac{3}{2} + \frac{1}{2}\text{erf}\left(\frac{x}{\sqrt{2\varepsilon}}\right)\right)e^{-x}
\]
\end{minipage}}
\end{center}

\subsection*{Connection to Lecture Notes}

\begin{reflection}
This problem illustrates several key concepts from the lecture notes:
\begin{itemize}
\item \textbf{\S6.2.2 (Singular points inside the domain):} The theory of interior layers when $p(x_0) = 0$ for some $x_0 \in (0,1)$.

\item \textbf{\S6.2.1 (Matching problem for singular points):} The use of Prandtl's matching rule to connect inner and outer solutions.

\item \textbf{Dominant balance (\S2.2.2):} The determination of layer width $\delta = \sqrt{\varepsilon}$ from requiring leading terms to balance.

\item \textbf{Example 3 in \S6.2.3:} The lecture notes present a similar interior layer problem $\varepsilon y'' + \sin(x)y' + \sin(x)y = 0$ with $p(x) = \sin(x)$ vanishing at $x = 0$, leading to analogous analysis.
\end{itemize}
\end{reflection}

\end{document}
