\documentclass[11pt,a4paper]{article}
\usepackage{inputenc}
\usepackage{amsmath,amssymb,amsthm}
\usepackage[margin=2.5cm]{geometry}
\usepackage{enumitem}
\usepackage{xcolor}

% Custom environments for pedagogical structure
\newtheoremstyle{problem}
  {10pt}{10pt}{\normalfont}{}{\bfseries}{.}{.5em}{}
\theoremstyle{problem}
\newtheorem{problem}{Problem}

\newenvironment{strategy}{\par\noindent\textbf{Strategy:}\itshape}{\par}
\newenvironment{justification}{\par\noindent\textbf{Justification:}\itshape}{\par}
\newenvironment{technique}{\par\noindent\textbf{Technique:}\itshape}{\par}
\newenvironment{reflection}{\par\noindent\textbf{Reflection:}\itshape}{\par}
\newenvironment{keyconcept}{\par\noindent\textbf{Key Concept:}\itshape}{\par}

\title{Asymptotics Problem 9.2: Complete Pedagogical Solution}
\author{Multiple Scales Method for the Lightly Damped Oscillator to Second Order}
\date{}

\begin{document}

\maketitle

\begin{problem}
For the lightly damped linear oscillator
\[
\ddot{x} + 2\varepsilon\dot{x} + \omega^2 x = 0
\]
\begin{enumerate}[(a)]
\item Use multiple-scale analysis to find the solution in second order, i.e.\ go one order beyond what was done in the lecture.
\item Compare your result with the exact solution.
\end{enumerate}
Initial conditions: $x(0) = 1$, $\dot{x}(0) = 0$.
\end{problem}

\section*{Solution: Step-by-Step Atomic Breakdown}

\subsection*{Step 1: Understanding the Physical Problem}

\begin{strategy}
This is the canonical example of the multiple scales method from Lecture Notes \S7.1.2, equations (394)--(406). The damped harmonic oscillator exhibits two distinct time scales:
\begin{itemize}[leftmargin=*]
\item \textbf{Fast oscillations} with period $\sim 2\pi/\omega$
\item \textbf{Slow amplitude decay} with time scale $\sim 1/\varepsilon$
\end{itemize}
A regular perturbation expansion fails because it produces secular terms that grow like $t$, invalidating the expansion for $t = O(1/\varepsilon)$.
\end{strategy}

\begin{justification}
Why does regular perturbation fail? If we try $x = x_0 + \varepsilon x_1 + \cdots$ with $x_0 = \cos(\omega t)$, the $O(\varepsilon)$ equation becomes $\ddot{x}_1 + \omega^2 x_1 = -2\dot{x}_0 = 2\omega\sin(\omega t)$. The right-hand side $\sin(\omega t)$ is a solution of the homogeneous equation, producing a secular term $t\cos(\omega t)$ in $x_1$. This is precisely the resonance phenomenon described in Lecture Notes \S7.1.1, equation (393).
\end{justification}

\subsection*{Step 2: The Exact Solution (For Later Comparison)}

\noindent Before applying multiple scales, let us record the exact solution for comparison in part (b).

\begin{technique}
For the ODE $\ddot{x} + 2\varepsilon\dot{x} + \omega^2 x = 0$, try $x = e^{mt}$:
\[
m^2 + 2\varepsilon m + \omega^2 = 0.
\]
Using the quadratic formula:
\[
m = -\varepsilon \pm \sqrt{\varepsilon^2 - \omega^2} = -\varepsilon \pm i\sqrt{\omega^2 - \varepsilon^2}.
\]
\end{technique}

\noindent The general solution is:
\[
x(t) = e^{-\varepsilon t}\left[A\cos\left(t\sqrt{\omega^2 - \varepsilon^2}\right) + B\sin\left(t\sqrt{\omega^2 - \varepsilon^2}\right)\right].
\]

\noindent Applying initial conditions $x(0) = 1$ and $\dot{x}(0) = 0$:

\noindent From $x(0) = 1$: $A = 1$.

\noindent From $\dot{x}(0) = 0$: Differentiating and setting $t = 0$ gives $B = \frac{\varepsilon}{\sqrt{\omega^2 - \varepsilon^2}}$.

\begin{center}
\fbox{\begin{minipage}{0.9\textwidth}
\textbf{Exact Solution:}
\[
x(t) = e^{-\varepsilon t}\cos\left(t\sqrt{\omega^2 - \varepsilon^2}\right) + \frac{\varepsilon}{\sqrt{\omega^2 - \varepsilon^2}}e^{-\varepsilon t}\sin\left(t\sqrt{\omega^2 - \varepsilon^2}\right)
\]
\end{minipage}}
\end{center}

\subsection*{Step 3: Setting Up the Multiple Scales Framework}

\subsubsection*{Step 3a: Defining the Time Scales}

\noindent We introduce two time variables:
\begin{align*}
t_0 &= t \quad \text{(fast time scale --- captures oscillations)}\\
t_1 &= \varepsilon t \quad \text{(slow time scale --- captures amplitude decay)}
\end{align*}

\begin{justification}
The oscillations occur on the $O(1)$ time scale with frequency $\omega$, so $t_0 = t$ is the natural fast variable. The damping causes the amplitude to decay on the time scale $1/\varepsilon$; when $t = O(1/\varepsilon)$, we have $t_1 = \varepsilon t = O(1)$. This is the standard two-scale setup from Lecture Notes \S7.1.2.
\end{justification}

\subsubsection*{Step 3b: Transforming Derivatives}

\noindent Treating $x = x(t_0, t_1)$ with both variables independent:
\[
\frac{d}{dt} = \frac{\partial}{\partial t_0} + \varepsilon\frac{\partial}{\partial t_1}.
\]

\noindent For the second derivative, we apply this operator twice:
\begin{align*}
\frac{d^2}{dt^2} &= \left(\frac{\partial}{\partial t_0} + \varepsilon\frac{\partial}{\partial t_1}\right)^2\\
&= \frac{\partial^2}{\partial t_0^2} + 2\varepsilon\frac{\partial^2}{\partial t_0\partial t_1} + \varepsilon^2\frac{\partial^2}{\partial t_1^2}.
\end{align*}

\begin{center}
\fbox{\begin{minipage}{0.85\textwidth}
\textbf{Derivative Transformations:}
\begin{align*}
\frac{d}{dt} &= \frac{\partial}{\partial t_0} + \varepsilon\frac{\partial}{\partial t_1}\\[6pt]
\frac{d^2}{dt^2} &= \frac{\partial^2}{\partial t_0^2} + 2\varepsilon\frac{\partial^2}{\partial t_0\partial t_1} + \varepsilon^2\frac{\partial^2}{\partial t_1^2}
\end{align*}
\end{minipage}}
\end{center}

\subsection*{Step 4: Transforming the ODE}

\noindent\textbf{Original ODE:} $\ddot{x} + 2\varepsilon\dot{x} + \omega^2 x = 0$.

\noindent Substituting the derivative transformations:
\begin{align*}
&\left(\frac{\partial^2}{\partial t_0^2} + 2\varepsilon\frac{\partial^2}{\partial t_0\partial t_1} + \varepsilon^2\frac{\partial^2}{\partial t_1^2}\right)x + 2\varepsilon\left(\frac{\partial}{\partial t_0} + \varepsilon\frac{\partial}{\partial t_1}\right)x + \omega^2 x = 0.
\end{align*}

\noindent Expanding and collecting by powers of $\varepsilon$:
\[
\frac{\partial^2 x}{\partial t_0^2} + \omega^2 x + 2\varepsilon\left(\frac{\partial^2 x}{\partial t_0\partial t_1} + \frac{\partial x}{\partial t_0}\right) + \varepsilon^2\left(\frac{\partial^2 x}{\partial t_1^2} + 2\frac{\partial x}{\partial t_1}\right) = 0.
\]

\subsection*{Step 5: Expanding $x$ in Powers of $\varepsilon$}

\noindent We expand:
\[
x(t_0, t_1) = x_0(t_0, t_1) + \varepsilon x_1(t_0, t_1) + \varepsilon^2 x_2(t_0, t_1) + O(\varepsilon^3).
\]

\noindent Each $x_n$ depends on both time scales.

\subsection*{Step 6: Transforming Initial Conditions}

\noindent At $t = 0$: $t_0 = 0$ and $t_1 = 0$.

\noindent\textbf{Condition $x(0) = 1$:}
\[
x_0(0,0) + \varepsilon x_1(0,0) + \varepsilon^2 x_2(0,0) + \cdots = 1.
\]
Therefore:
\begin{align*}
x_0(0,0) &= 1,\\
x_n(0,0) &= 0 \quad \text{for } n \geq 1.
\end{align*}

\noindent\textbf{Condition $\dot{x}(0) = 0$:}

\noindent Using $\frac{dx}{dt} = \frac{\partial x}{\partial t_0} + \varepsilon\frac{\partial x}{\partial t_1}$:
\[
\frac{\partial x_0}{\partial t_0}(0,0) + \varepsilon\left(\frac{\partial x_1}{\partial t_0}(0,0) + \frac{\partial x_0}{\partial t_1}(0,0)\right) + O(\varepsilon^2) = 0.
\]
Therefore:
\begin{align*}
\frac{\partial x_0}{\partial t_0}(0,0) &= 0,\\
\frac{\partial x_1}{\partial t_0}(0,0) &= -\frac{\partial x_0}{\partial t_1}(0,0).
\end{align*}

\subsection*{Step 7: The $O(1)$ Problem}

\noindent At order $\varepsilon^0$:
\[
\frac{\partial^2 x_0}{\partial t_0^2} + \omega^2 x_0 = 0.
\]

\noindent With initial conditions: $x_0(0,0) = 1$ and $\frac{\partial x_0}{\partial t_0}(0,0) = 0$.

\subsubsection*{Step 7a: Solving the Leading-Order Equation}

\begin{technique}
This is the simple harmonic oscillator equation in $t_0$, with $t_1$ as a parameter. The general solution is:
\[
x_0(t_0, t_1) = A(t_1)\cos(\omega t_0) + B(t_1)\sin(\omega t_0),
\]
where $A(t_1)$ and $B(t_1)$ are undetermined functions of the slow time.
\end{technique}

\subsubsection*{Step 7b: Applying Initial Conditions}

\noindent From $x_0(0,0) = 1$:
\[
A(0)\cos(0) + B(0)\sin(0) = A(0) = 1.
\]

\noindent From $\frac{\partial x_0}{\partial t_0}(0,0) = 0$:
\[
\frac{\partial x_0}{\partial t_0} = -\omega A(t_1)\sin(\omega t_0) + \omega B(t_1)\cos(\omega t_0).
\]
At $(0,0)$: $\omega B(0) = 0$, so $B(0) = 0$.

\begin{center}
\fbox{\begin{minipage}{0.85\textwidth}
\textbf{Leading-Order Solution:}
\[
x_0(t_0, t_1) = A(t_1)\cos(\omega t_0) + B(t_1)\sin(\omega t_0)
\]
with $A(0) = 1$ and $B(0) = 0$.
\end{minipage}}
\end{center}

\subsection*{Step 8: The $O(\varepsilon)$ Problem}

\noindent At order $\varepsilon^1$:
\[
\frac{\partial^2 x_1}{\partial t_0^2} + \omega^2 x_1 = -2\frac{\partial^2 x_0}{\partial t_0\partial t_1} - 2\frac{\partial x_0}{\partial t_0}.
\]

\subsubsection*{Step 8a: Computing the Right-Hand Side}

\noindent\textbf{First term:} $\frac{\partial x_0}{\partial t_0}$
\[
\frac{\partial x_0}{\partial t_0} = -\omega A\sin(\omega t_0) + \omega B\cos(\omega t_0).
\]

\noindent\textbf{Second term:} $\frac{\partial^2 x_0}{\partial t_0\partial t_1}$
\[
\frac{\partial^2 x_0}{\partial t_0\partial t_1} = \frac{\partial}{\partial t_1}\left(\frac{\partial x_0}{\partial t_0}\right) = -\omega\frac{dA}{dt_1}\sin(\omega t_0) + \omega\frac{dB}{dt_1}\cos(\omega t_0).
\]

\noindent\textbf{The complete RHS:}
\begin{align*}
\text{RHS} &= -2\left(-\omega\frac{dA}{dt_1}\sin(\omega t_0) + \omega\frac{dB}{dt_1}\cos(\omega t_0)\right) - 2\left(-\omega A\sin(\omega t_0) + \omega B\cos(\omega t_0)\right)\\
&= 2\omega\sin(\omega t_0)\left(\frac{dA}{dt_1} + A\right) - 2\omega\cos(\omega t_0)\left(\frac{dB}{dt_1} + B\right).
\end{align*}

\subsubsection*{Step 8b: Identifying Secular Terms}

\begin{keyconcept}
\textbf{Secularity Condition for Oscillators} (Lecture Notes \S7.1.2): For the equation $\frac{\partial^2 x_1}{\partial t_0^2} + \omega^2 x_1 = f(t_0, t_1)$, secular terms arise from any component of $f$ that is proportional to $\sin(\omega t_0)$ or $\cos(\omega t_0)$---the homogeneous solutions. To eliminate secular growth, we must set the coefficients of these resonant terms to zero.
\end{keyconcept}

\noindent The RHS contains:
\begin{itemize}
\item Coefficient of $\sin(\omega t_0)$: $2\omega\left(\frac{dA}{dt_1} + A\right)$
\item Coefficient of $\cos(\omega t_0)$: $-2\omega\left(\frac{dB}{dt_1} + B\right)$
\end{itemize}

\noindent Both are resonant with the homogeneous equation.

\subsubsection*{Step 8c: Eliminating Secular Terms}

\noindent Setting both coefficients to zero:
\begin{align}
\frac{dA}{dt_1} + A &= 0, \label{eq:A}\\
\frac{dB}{dt_1} + B &= 0. \label{eq:B}
\end{align}

\subsubsection*{Step 8d: Solving for $A(t_1)$ and $B(t_1)$}

\noindent\textbf{Equation for $A$:} $\frac{dA}{dt_1} = -A$ with $A(0) = 1$.
\[
A(t_1) = e^{-t_1}.
\]

\noindent\textbf{Equation for $B$:} $\frac{dB}{dt_1} = -B$ with $B(0) = 0$.
\[
B(t_1) = 0.
\]

\begin{center}
\fbox{\begin{minipage}{0.7\textwidth}
\textbf{Slow-Time Functions (First Order):}
\[
A(t_1) = e^{-t_1}, \quad B(t_1) = 0
\]
\end{minipage}}
\end{center}

\subsubsection*{Step 8e: Solving for $x_1$}

\noindent With secular terms eliminated, the RHS is zero:
\[
\frac{\partial^2 x_1}{\partial t_0^2} + \omega^2 x_1 = 0.
\]

\noindent The general solution is:
\[
x_1(t_0, t_1) = C(t_1)\cos(\omega t_0) + D(t_1)\sin(\omega t_0),
\]
where $C(t_1)$ and $D(t_1)$ are new undetermined functions.

\subsubsection*{Step 8f: Initial Conditions for $x_1$}

\noindent From $x_1(0,0) = 0$:
\[
C(0) = 0.
\]

\noindent From $\frac{\partial x_1}{\partial t_0}(0,0) = -\frac{\partial x_0}{\partial t_1}(0,0)$:

\noindent We need $\frac{\partial x_0}{\partial t_1}$:
\[
\frac{\partial x_0}{\partial t_1} = \frac{dA}{dt_1}\cos(\omega t_0) + \frac{dB}{dt_1}\sin(\omega t_0).
\]
At $(0,0)$: $\frac{\partial x_0}{\partial t_1}(0,0) = \frac{dA}{dt_1}(0) = -A(0) = -1$.

\noindent And $\frac{\partial x_1}{\partial t_0} = -\omega C\sin(\omega t_0) + \omega D\cos(\omega t_0)$.

\noindent At $(0,0)$: $\omega D(0) = -(-1) = 1$, so $D(0) = \frac{1}{\omega}$.

\begin{center}
\fbox{\begin{minipage}{0.7\textwidth}
\textbf{Initial Conditions for $x_1$:}
\[
C(0) = 0, \quad D(0) = \frac{1}{\omega}
\]
\end{minipage}}
\end{center}

\subsection*{Step 9: The $O(\varepsilon^2)$ Problem}

\noindent At order $\varepsilon^2$:
\[
\frac{\partial^2 x_2}{\partial t_0^2} + \omega^2 x_2 = -2\frac{\partial^2 x_1}{\partial t_0\partial t_1} - 2\frac{\partial x_1}{\partial t_0} - \frac{\partial^2 x_0}{\partial t_1^2} - 2\frac{\partial x_0}{\partial t_1}.
\]

\subsubsection*{Step 9a: Computing Terms Involving $x_0$}

\noindent With $A(t_1) = e^{-t_1}$ and $B(t_1) = 0$, we have $x_0 = e^{-t_1}\cos(\omega t_0)$.

\noindent\textbf{Term:} $\frac{\partial x_0}{\partial t_1}$
\[
\frac{\partial x_0}{\partial t_1} = -e^{-t_1}\cos(\omega t_0).
\]

\noindent\textbf{Term:} $\frac{\partial^2 x_0}{\partial t_1^2}$
\[
\frac{\partial^2 x_0}{\partial t_1^2} = e^{-t_1}\cos(\omega t_0).
\]

\noindent\textbf{Contribution from $x_0$ terms:}
\[
-\frac{\partial^2 x_0}{\partial t_1^2} - 2\frac{\partial x_0}{\partial t_1} = -e^{-t_1}\cos(\omega t_0) - 2(-e^{-t_1}\cos(\omega t_0)) = e^{-t_1}\cos(\omega t_0).
\]

\subsubsection*{Step 9b: Computing Terms Involving $x_1$}

\noindent With $x_1 = C(t_1)\cos(\omega t_0) + D(t_1)\sin(\omega t_0)$:

\noindent\textbf{Term:} $\frac{\partial x_1}{\partial t_0}$
\[
\frac{\partial x_1}{\partial t_0} = -\omega C\sin(\omega t_0) + \omega D\cos(\omega t_0).
\]

\noindent\textbf{Term:} $\frac{\partial^2 x_1}{\partial t_0\partial t_1}$
\[
\frac{\partial^2 x_1}{\partial t_0\partial t_1} = -\omega\frac{dC}{dt_1}\sin(\omega t_0) + \omega\frac{dD}{dt_1}\cos(\omega t_0).
\]

\noindent\textbf{Contribution from $x_1$ terms:}
\begin{align*}
-2\frac{\partial^2 x_1}{\partial t_0\partial t_1} - 2\frac{\partial x_1}{\partial t_0} &= -2\left(-\omega\frac{dC}{dt_1}\sin(\omega t_0) + \omega\frac{dD}{dt_1}\cos(\omega t_0)\right)\\
&\quad - 2\left(-\omega C\sin(\omega t_0) + \omega D\cos(\omega t_0)\right)\\
&= 2\omega\sin(\omega t_0)\left(\frac{dC}{dt_1} + C\right) - 2\omega\cos(\omega t_0)\left(\frac{dD}{dt_1} + D\right).
\end{align*}

\subsubsection*{Step 9c: The Complete RHS of the $O(\varepsilon^2)$ Equation}

\begin{align*}
\text{RHS} &= 2\omega\sin(\omega t_0)\left(\frac{dC}{dt_1} + C\right) - 2\omega\cos(\omega t_0)\left(\frac{dD}{dt_1} + D\right) + e^{-t_1}\cos(\omega t_0).
\end{align*}

\noindent Grouping by trigonometric functions:
\begin{align*}
\text{RHS} &= 2\omega\left(\frac{dC}{dt_1} + C\right)\sin(\omega t_0) + \left(e^{-t_1} - 2\omega\left(\frac{dD}{dt_1} + D\right)\right)\cos(\omega t_0).
\end{align*}

\subsubsection*{Step 9d: Eliminating Secular Terms at $O(\varepsilon^2)$}

\noindent To avoid secular terms, both coefficients must vanish:

\noindent\textbf{Coefficient of $\sin(\omega t_0)$:}
\[
\frac{dC}{dt_1} + C = 0, \quad C(0) = 0.
\]
Solution: $C(t_1) = 0$.

\noindent\textbf{Coefficient of $\cos(\omega t_0)$:}
\[
e^{-t_1} - 2\omega\left(\frac{dD}{dt_1} + D\right) = 0.
\]
\[
\frac{dD}{dt_1} + D = \frac{e^{-t_1}}{2\omega}, \quad D(0) = \frac{1}{\omega}.
\]

\subsubsection*{Step 9e: Solving for $D(t_1)$}

\begin{technique}
This is a first-order linear ODE: $\frac{dD}{dt_1} + D = \frac{e^{-t_1}}{2\omega}$.

The integrating factor is $\mu(t_1) = e^{t_1}$.

Multiply through: $e^{t_1}\frac{dD}{dt_1} + e^{t_1}D = \frac{1}{2\omega}$.

The LHS is $\frac{d}{dt_1}(e^{t_1}D)$.

Integrate: $e^{t_1}D = \frac{t_1}{2\omega} + K$.

Therefore: $D(t_1) = \frac{t_1}{2\omega}e^{-t_1} + Ke^{-t_1}$.
\end{technique}

\noindent Applying $D(0) = \frac{1}{\omega}$:
\[
D(0) = 0 + K = \frac{1}{\omega} \quad \Longrightarrow \quad K = \frac{1}{\omega}.
\]

\noindent Therefore:
\[
D(t_1) = \frac{1}{\omega}e^{-t_1} + \frac{t_1}{2\omega}e^{-t_1} = \frac{e^{-t_1}}{\omega}\left(1 + \frac{t_1}{2}\right).
\]

\begin{center}
\fbox{\begin{minipage}{0.75\textwidth}
\textbf{Slow-Time Functions (Second Order):}
\[
C(t_1) = 0, \quad D(t_1) = \frac{1}{\omega}e^{-t_1} + \frac{t_1}{2\omega}e^{-t_1}
\]
\end{minipage}}
\end{center}

\subsection*{Step 10: Assembling the Complete Solution}

\subsubsection*{Step 10a: The Solution in Terms of $t_0$ and $t_1$}

\noindent We have:
\begin{align*}
x_0(t_0, t_1) &= e^{-t_1}\cos(\omega t_0),\\
x_1(t_0, t_1) &= D(t_1)\sin(\omega t_0) = \left(\frac{1}{\omega}e^{-t_1} + \frac{t_1}{2\omega}e^{-t_1}\right)\sin(\omega t_0).
\end{align*}

\noindent The approximation through $O(\varepsilon)$ is:
\[
x(t_0, t_1) = x_0 + \varepsilon x_1 + O(\varepsilon^2).
\]

\subsubsection*{Step 10b: Converting to the Original Variable $t$}

\noindent Substituting $t_0 = t$ and $t_1 = \varepsilon t$:
\begin{align*}
x(t) &= e^{-\varepsilon t}\cos(\omega t) + \varepsilon\left(\frac{1}{\omega}e^{-\varepsilon t} + \frac{\varepsilon t}{2\omega}e^{-\varepsilon t}\right)\sin(\omega t) + O(\varepsilon^2)\\
&= e^{-\varepsilon t}\cos(\omega t) + \frac{\varepsilon}{\omega}e^{-\varepsilon t}\sin(\omega t) + \frac{\varepsilon^2 t}{2\omega}e^{-\varepsilon t}\sin(\omega t) + O(\varepsilon^2).
\end{align*}

\begin{center}
\fbox{\begin{minipage}{0.9\textwidth}
\textbf{Multiple Scales Approximation (Second Order):}
\[
x(t) = e^{-\varepsilon t}\cos(\omega t) + \left(\frac{\varepsilon}{\omega} + \frac{\varepsilon^2 t}{2\omega}\right)e^{-\varepsilon t}\sin(\omega t) + O(\varepsilon^2)
\]
\end{minipage}}
\end{center}

\subsection*{Step 11: Comparison with the Exact Solution}

\subsubsection*{Step 11a: Expanding the Exact Solution}

\noindent The exact solution is:
\[
x_{\text{exact}}(t) = e^{-\varepsilon t}\cos\left(t\sqrt{\omega^2 - \varepsilon^2}\right) + \frac{\varepsilon}{\sqrt{\omega^2 - \varepsilon^2}}e^{-\varepsilon t}\sin\left(t\sqrt{\omega^2 - \varepsilon^2}\right).
\]

\begin{technique}
We need to expand $\sqrt{\omega^2 - \varepsilon^2}$ and the trigonometric functions for small $\varepsilon$.

First, expand $\sqrt{\omega^2 - \varepsilon^2}$:
\[
\sqrt{\omega^2 - \varepsilon^2} = \omega\sqrt{1 - \frac{\varepsilon^2}{\omega^2}} = \omega\left(1 - \frac{\varepsilon^2}{2\omega^2} + O(\varepsilon^4)\right) = \omega - \frac{\varepsilon^2}{2\omega} + O(\varepsilon^4).
\]
\end{technique}

\subsubsection*{Step 11b: Expanding the Coefficient}

\noindent The coefficient of the sine term:
\[
\frac{\varepsilon}{\sqrt{\omega^2 - \varepsilon^2}} = \frac{\varepsilon}{\omega - \frac{\varepsilon^2}{2\omega} + O(\varepsilon^4)} = \frac{\varepsilon}{\omega}\left(1 - \frac{\varepsilon^2}{2\omega^2}\right)^{-1} = \frac{\varepsilon}{\omega}\left(1 + \frac{\varepsilon^2}{2\omega^2} + O(\varepsilon^4)\right).
\]
To order $\varepsilon^2$:
\[
\frac{\varepsilon}{\sqrt{\omega^2 - \varepsilon^2}} = \frac{\varepsilon}{\omega} + O(\varepsilon^3).
\]

\subsubsection*{Step 11c: Expanding the Trigonometric Arguments}

\noindent The argument of the trig functions is:
\[
t\sqrt{\omega^2 - \varepsilon^2} = t\left(\omega - \frac{\varepsilon^2}{2\omega}\right) + O(\varepsilon^4) = \omega t - \frac{\varepsilon^2 t}{2\omega} + O(\varepsilon^4).
\]

\noindent Using Taylor expansions around $\omega t$:
\begin{align*}
\cos\left(\omega t - \frac{\varepsilon^2 t}{2\omega}\right) &= \cos(\omega t)\cos\left(\frac{\varepsilon^2 t}{2\omega}\right) + \sin(\omega t)\sin\left(\frac{\varepsilon^2 t}{2\omega}\right)\\
&= \cos(\omega t)\left(1 + O(\varepsilon^4)\right) + \sin(\omega t)\left(\frac{\varepsilon^2 t}{2\omega} + O(\varepsilon^4)\right)\\
&= \cos(\omega t) + \frac{\varepsilon^2 t}{2\omega}\sin(\omega t) + O(\varepsilon^4).
\end{align*}

\begin{align*}
\sin\left(\omega t - \frac{\varepsilon^2 t}{2\omega}\right) &= \sin(\omega t)\cos\left(\frac{\varepsilon^2 t}{2\omega}\right) - \cos(\omega t)\sin\left(\frac{\varepsilon^2 t}{2\omega}\right)\\
&= \sin(\omega t) - \frac{\varepsilon^2 t}{2\omega}\cos(\omega t) + O(\varepsilon^4).
\end{align*}

\subsubsection*{Step 11d: Assembling the Expanded Exact Solution}

\begin{align*}
x_{\text{exact}}(t) &= e^{-\varepsilon t}\left[\cos(\omega t) + \frac{\varepsilon^2 t}{2\omega}\sin(\omega t)\right]\\
&\quad + \frac{\varepsilon}{\omega}e^{-\varepsilon t}\left[\sin(\omega t) - \frac{\varepsilon^2 t}{2\omega}\cos(\omega t)\right] + O(\varepsilon^3)\\
&= e^{-\varepsilon t}\cos(\omega t) + \frac{\varepsilon}{\omega}e^{-\varepsilon t}\sin(\omega t) + \frac{\varepsilon^2 t}{2\omega}e^{-\varepsilon t}\sin(\omega t) + O(\varepsilon^3).
\end{align*}

\subsubsection*{Step 11e: Comparison}

\noindent\textbf{Multiple scales result:}
\[
x_{\text{MS}}(t) = e^{-\varepsilon t}\cos(\omega t) + \frac{\varepsilon}{\omega}e^{-\varepsilon t}\sin(\omega t) + \frac{\varepsilon^2 t}{2\omega}e^{-\varepsilon t}\sin(\omega t) + O(\varepsilon^2).
\]

\noindent\textbf{Expanded exact solution:}
\[
x_{\text{exact}}(t) = e^{-\varepsilon t}\cos(\omega t) + \frac{\varepsilon}{\omega}e^{-\varepsilon t}\sin(\omega t) + \frac{\varepsilon^2 t}{2\omega}e^{-\varepsilon t}\sin(\omega t) + O(\varepsilon^3).
\]

\begin{center}
\fbox{\begin{minipage}{0.9\textwidth}
\textbf{Conclusion:} The multiple scales approximation through second order \textbf{exactly matches} the expansion of the exact solution through order $\varepsilon^2$. The methods agree term by term!
\end{minipage}}
\end{center}

\subsection*{Step 12: Discussion of the Secular Term}

\begin{reflection}
The term $\frac{\varepsilon^2 t}{2\omega}e^{-\varepsilon t}\sin(\omega t)$ in our solution appears to be a secular term because of the factor of $t$. However, this is \textbf{not} a problematic secular term because:

\begin{enumerate}
\item The factor $e^{-\varepsilon t}$ dominates for large $t$, ensuring the entire term decays.

\item This term arises from the expansion of the exact frequency $\sqrt{\omega^2 - \varepsilon^2} \approx \omega - \frac{\varepsilon^2}{2\omega}$. The exact solution oscillates at a slightly different frequency, and expanding this frequency shift produces apparent secular terms.

\item From Lecture Notes \S7.1.2, the solution notes that to avoid this secular term, we would need to introduce a third time scale $t_2 = \varepsilon^2 t$ or modify the fast time scale to $t_0 = \delta(\varepsilon)t$ where $\delta$ captures the frequency correction.
\end{enumerate}

The appearance of this secular term at $O(\varepsilon^2)$ indicates the limitation of the two-scale expansion. For longer times or higher accuracy, additional slow time scales would be needed.
\end{reflection}

\subsection*{Final Summary}

\begin{center}
\fbox{\begin{minipage}{0.95\textwidth}
\textbf{Complete Solution for Problem 9.2:}

\vspace{0.3cm}
\textbf{Given:} $\ddot{x} + 2\varepsilon\dot{x} + \omega^2 x = 0$ with $x(0) = 1$, $\dot{x}(0) = 0$.

\vspace{0.2cm}
\textbf{Time scales:} $t_0 = t$ (fast), $t_1 = \varepsilon t$ (slow).

\vspace{0.2cm}
\textbf{Leading order ($O(1)$):}
\[
x_0 = A(t_1)\cos(\omega t_0) + B(t_1)\sin(\omega t_0)
\]

\textbf{Secularity conditions at $O(\varepsilon)$:}
\[
A(t_1) = e^{-t_1}, \quad B(t_1) = 0
\]

\textbf{First correction ($O(\varepsilon)$):}
\[
x_1 = C(t_1)\cos(\omega t_0) + D(t_1)\sin(\omega t_0)
\]

\textbf{Secularity conditions at $O(\varepsilon^2)$:}
\[
C(t_1) = 0, \quad D(t_1) = \frac{1}{\omega}e^{-t_1} + \frac{t_1}{2\omega}e^{-t_1}
\]

\vspace{0.2cm}
\textbf{Second-order approximation:}
\[
x(t) = e^{-\varepsilon t}\cos(\omega t) + \left(\frac{\varepsilon}{\omega} + \frac{\varepsilon^2 t}{2\omega}\right)e^{-\varepsilon t}\sin(\omega t)
\]

\vspace{0.2cm}
\textbf{Exact solution:}
\[
x(t) = e^{-\varepsilon t}\cos(t\sqrt{\omega^2-\varepsilon^2}) + \frac{\varepsilon}{\sqrt{\omega^2-\varepsilon^2}}e^{-\varepsilon t}\sin(t\sqrt{\omega^2-\varepsilon^2})
\]

\vspace{0.2cm}
\textbf{Verification:} Expanding the exact solution to $O(\varepsilon^2)$ reproduces the multiple scales result exactly.
\end{minipage}}
\end{center}

\subsection*{Connection to Lecture Notes}

\begin{reflection}
This problem extends the analysis in Lecture Notes \S7.1.2, equations (394)--(406), which covers the damped oscillator to first order. The key additional insights from going to second order are:

\begin{enumerate}
\item \textbf{Secularity conditions propagate:} At each order, new undetermined functions appear, and secularity conditions at the next order determine them.

\item \textbf{Frequency corrections appear at $O(\varepsilon^2)$:} The exact frequency $\sqrt{\omega^2 - \varepsilon^2}$ differs from $\omega$ by $O(\varepsilon^2)$. This correction manifests as the secular-looking term $\varepsilon^2 t$ in our expansion.

\item \textbf{Limitations of two-scale expansions:} As noted in the solutions, the secular term at $O(\varepsilon^2)$ could be removed by introducing a modified fast time scale that incorporates the frequency correction. This would require $t_0 = \delta(\varepsilon)t$ with $\delta = \omega - \frac{\varepsilon^2}{2\omega} + O(\varepsilon^4)$.

\item \textbf{Physical interpretation:} The $e^{-\varepsilon t}$ decay captures amplitude attenuation, while the modified frequency $\sqrt{\omega^2 - \varepsilon^2}$ shows that damping also slightly reduces the oscillation frequency.
\end{enumerate}
\end{reflection}

\end{document}
