\documentclass[11pt,a4paper]{article}
\usepackage{amsmath,amssymb,amsthm}
\usepackage{geometry}
\geometry{margin=1in}
\usepackage{enumitem}
\usepackage{xcolor}

\newtheorem{theorem}{Theorem}
\newtheorem{lemma}[theorem]{Lemma}
\newtheorem{definition}[theorem]{Definition}

\newcommand{\why}[1]{\textcolor{blue}{\textbf{Why:} #1}}
\newcommand{\what}[1]{\textcolor{red}{\textbf{What we see:} #1}}
\newcommand{\how}[1]{\textcolor{purple}{\textbf{How we know:} #1}}

\title{Asymptotics 2025/2026 Sheet 1\\Problem 4: Detailed Solution}
\author{}
\date{}

\begin{document}

\maketitle

\section*{Problem 4}

\textbf{Problem Statement:} Explain why the sequence $\{\phi_n(x) = x^{-n}\cos(nx)\}$, $n = 0, 1, \ldots$, is \textbf{not} an asymptotic sequence as $x \to \infty$.

\section{Stage 1: Understanding What We Need to Prove}

\subsection{What is the Question Asking?}

\what{We are given a sequence of functions:}
\begin{align}
\phi_0(x) &= x^0 \cos(0 \cdot x) = 1 \cdot 1 = 1\\
\phi_1(x) &= x^{-1} \cos(x) = \frac{\cos(x)}{x}\\
\phi_2(x) &= x^{-2} \cos(2x) = \frac{\cos(2x)}{x^2}\\
\phi_3(x) &= x^{-3} \cos(3x) = \frac{\cos(3x)}{x^3}\\
&\vdots \nonumber
\end{align}

\why{We list these explicitly because we need to see the \emph{pattern} of how the sequence behaves. Each successive term has:
\begin{enumerate}
\item An \textbf{algebraic part}: $x^{-n}$ that gets smaller as $n$ increases (for fixed large $x$)
\item An \textbf{oscillatory part}: $\cos(nx)$ that oscillates with increasing frequency as $n$ increases
\end{enumerate}
The question is asking us to determine whether this sequence satisfies the formal definition of an ``asymptotic sequence.''}

\subsection{Recalling the Definition from the Lecture Notes}

\begin{definition}[Asymptotic Sequence]
From Lecture Notes Section 2.5, page 9: A sequence of functions $\{\phi_n(x)\}$, $n = 0, 1, 2, \ldots$ is an \textbf{asymptotic sequence} as $x \to x_0$ if, for all $n$,
\begin{equation}
\phi_{n+1}(x) = o(\phi_n(x)) \quad \text{as } x \to x_0.
\end{equation}
\end{definition}

\why{This definition is the \emph{precise mathematical criterion} we must check. The notation $\phi_{n+1}(x) = o(\phi_n(x))$ means that each successive function in the sequence is ``asymptotically smaller'' than the previous one.}

\subsection{Understanding the ``Little-oh'' Notation}

\begin{definition}[Little-oh, from Lecture Notes Section 2.4.1, page 6]
We write $f(x) = o(g(x))$ as $x \to x_0$ if
\begin{equation}
\lim_{x \to x_0} \frac{f(x)}{g(x)} = 0.
\end{equation}
Alternatively, we write $f(x) \ll g(x)$ as $x \to x_0$.
\end{definition}

\why{The little-oh notation captures the idea that $f$ is ``negligible compared to $g$'' near $x_0$. When we take the ratio and it goes to zero, it means $f$ decays \emph{faster} than $g$ (or grows slower than $g$).}

\how{We translate the asymptotic sequence condition into a concrete limit we can evaluate:
\begin{equation}
\text{For } \{\phi_n(x)\} \text{ to be asymptotic, we need: } \lim_{x \to \infty} \frac{\phi_{n+1}(x)}{\phi_n(x)} = 0 \text{ for every } n.
\end{equation}}

\section{Stage 2: Setting Up the Test}

\subsection{What We Must Verify}

\what{To determine if $\{\phi_n(x) = x^{-n}\cos(nx)\}$ is an asymptotic sequence as $x \to \infty$, we must check whether:}
\begin{equation}
\lim_{x \to \infty} \frac{\phi_{n+1}(x)}{\phi_n(x)} = 0 \quad \text{for every } n = 0, 1, 2, \ldots
\end{equation}

\why{If this limit equals zero for \emph{all} $n$, then the sequence is asymptotic. If we can find even \emph{one} value of $n$ for which this limit either:
\begin{enumerate}
\item Does not exist, or
\item Exists but does not equal zero,
\end{enumerate}
then the sequence \textbf{fails} to be asymptotic.}

\subsection{Computing the Ratio}

\what{Let us compute the ratio explicitly:}
\begin{align}
\frac{\phi_{n+1}(x)}{\phi_n(x)} &= \frac{x^{-(n+1)} \cos((n+1)x)}{x^{-n} \cos(nx)}\\
&= \frac{x^{-n-1}}{x^{-n}} \cdot \frac{\cos((n+1)x)}{\cos(nx)}\\
&= x^{-n-1-(-n)} \cdot \frac{\cos((n+1)x)}{\cos(nx)}\\
&= x^{-1} \cdot \frac{\cos((n+1)x)}{\cos(nx)}\\
&= \frac{\cos((n+1)x)}{x \cos(nx)}.
\end{align}

\why{We separate the ratio into two parts:
\begin{enumerate}
\item \textbf{The algebraic part}: $x^{-1} = \frac{1}{x}$, which \emph{does} go to zero as $x \to \infty$
\item \textbf{The trigonometric part}: $\frac{\cos((n+1)x)}{\cos(nx)}$, which we must analyze carefully
\end{enumerate}
The key question is: does the product of these two parts have a well-defined limit as $x \to \infty$?}

\section{Stage 3: Analyzing the Limit}

\subsection{The Behavior of the Trigonometric Ratio}

\what{We need to understand:}
\begin{equation}
\lim_{x \to \infty} \frac{\cos((n+1)x)}{x \cos(nx)} = \lim_{x \to \infty} \frac{1}{x} \cdot \frac{\cos((n+1)x)}{\cos(nx)}.
\end{equation}

\why{The factor $\frac{1}{x} \to 0$ as $x \to \infty$, which is \emph{good} for our purposes (we want the limit to be zero). However, the behavior of $\frac{\cos((n+1)x)}{\cos(nx)}$ as $x \to \infty$ is \textbf{problematic}.}

\subsection{Why the Trigonometric Ratio is Problematic}

\subsubsection{Observation 1: Oscillatory Behavior}

\what{Both $\cos((n+1)x)$ and $\cos(nx)$ are \textbf{periodic functions} that oscillate between $-1$ and $+1$ as $x$ increases.}

\begin{itemize}
\item $\cos(nx)$ oscillates with period $T_n = \frac{2\pi}{n}$
\item $\cos((n+1)x)$ oscillates with period $T_{n+1} = \frac{2\pi}{n+1}$
\end{itemize}

\why{As $x \to \infty$, both functions continue to oscillate \emph{indefinitely}. They do not settle down to any particular value. This means:}
\begin{equation}
\lim_{x \to \infty} \cos(nx) \text{ \textbf{does not exist}}, \quad \lim_{x \to \infty} \cos((n+1)x) \text{ \textbf{does not exist}}.
\end{equation}

\how{We know this from basic analysis: a function that oscillates between two values indefinitely cannot have a limit. For example, $\lim_{x \to \infty} \sin(x)$ does not exist because $\sin(x)$ takes all values in $[-1, 1]$ infinitely often as $x \to \infty$.}

\subsubsection{Observation 2: The Denominator Can Vanish}

\what{The denominator $\cos(nx)$ equals zero whenever:}
\begin{equation}
nx = \frac{\pi}{2} + k\pi \quad \text{for } k \in \mathbb{Z},
\end{equation}
which occurs at:
\begin{equation}
x = \frac{\pi(2k+1)}{2n} \quad \text{for } k = 0, 1, 2, \ldots
\end{equation}

\why{This is a \textbf{critical issue}. As $x \to \infty$, there are \emph{infinitely many} values of $x$ where $\cos(nx) = 0$. At these points:
\begin{equation}
\frac{\cos((n+1)x)}{\cos(nx)} = \frac{\cos((n+1)x)}{0},
\end{equation}
which is \textbf{undefined} (if $\cos((n+1)x) \neq 0$) or an indeterminate form $\frac{0}{0}$ (if both vanish simultaneously).}

\how{We can verify this by substituting. For example, take $n = 1$ and $x = \frac{\pi}{2}$:
\begin{align}
\cos(x) &= \cos\left(\frac{\pi}{2}\right) = 0\\
\cos(2x) &= \cos(\pi) = -1\\
\frac{\cos(2x)}{\cos(x)} &= \frac{-1}{0} = \text{undefined}.
\end{align}}

\subsection{Constructing a Sequence to Show Non-Existence}

\what{Let us construct a specific sequence of $x$-values to demonstrate that the limit does not exist.}

\subsubsection{Case 1: When $\cos(nx)$ is Near Zero}

\textbf{Choose:} $x_k = \frac{\pi(2k+1)}{2n}$ for large integers $k$.

\why{At these values, $\cos(nx_k) = 0$ exactly.}

\what{Then:}
\begin{equation}
\frac{\cos((n+1)x_k)}{x_k \cos(nx_k)} = \frac{\cos\left((n+1) \cdot \frac{\pi(2k+1)}{2n}\right)}{x_k \cdot 0}.
\end{equation}

\why{If $\cos((n+1)x_k) \neq 0$, this ratio is \textbf{unbounded} (either $+\infty$ or $-\infty$).}

\how{We can verify that $\cos((n+1)x_k) \neq 0$ for \emph{most} values of $k$ (in general, the numerator and denominator zeros do not coincide). Therefore, along this sequence $\{x_k\}$:}
\begin{equation}
\left|\frac{\cos((n+1)x_k)}{x_k \cos(nx_k)}\right| \to \infty.
\end{equation}

\subsubsection{Case 2: When Both Functions Are Non-Zero}

\textbf{Choose:} $x_j = \frac{2\pi j}{n}$ for large integers $j$.

\why{At these values, $\cos(nx_j) = \cos(2\pi j) = 1$.}

\what{Then:}
\begin{align}
\frac{\cos((n+1)x_j)}{x_j \cos(nx_j)} &= \frac{\cos\left((n+1) \cdot \frac{2\pi j}{n}\right)}{x_j \cdot 1}\\
&= \frac{\cos\left(2\pi j + \frac{2\pi j}{n}\right)}{x_j}\\
&= \frac{\cos\left(\frac{2\pi j}{n}\right)}{x_j}.
\end{align}

\why{Using the periodicity of cosine: $\cos(2\pi j + \theta) = \cos(\theta)$.}

\what{Now, as $j \to \infty$ (and hence $x_j \to \infty$):}
\begin{equation}
\frac{\cos\left(\frac{2\pi j}{n}\right)}{x_j} = \frac{\cos\left(\frac{2\pi j}{n}\right)}{\frac{2\pi j}{n}} \to 0 \text{ as } j \to \infty,
\end{equation}
\textbf{provided} $\cos\left(\frac{2\pi j}{n}\right)$ remains bounded (which it does, oscillating between $-1$ and $1$).

\why{Along \emph{this} sequence, the ratio does approach zero because the denominator grows like $x_j$ while the numerator is bounded.}

\subsection{The Contradiction: Limit Does Not Exist}

\what{We have shown that:}
\begin{enumerate}
\item Along the sequence $x_k = \frac{\pi(2k+1)}{2n}$, the ratio $\to \infty$ (unbounded)
\item Along the sequence $x_j = \frac{2\pi j}{n}$, the ratio $\to 0$
\end{enumerate}

\why{For a limit $\lim_{x \to \infty} f(x)$ to exist, $f(x)$ must approach the \textbf{same value} along \emph{every} sequence $\{x_n\}$ with $x_n \to \infty$.}

\how{This is the \textbf{sequential criterion for limits}:
\begin{equation}
\lim_{x \to x_0} f(x) = L \iff \lim_{n \to \infty} f(x_n) = L \text{ for every sequence } x_n \to x_0.
\end{equation}
Since we have found two sequences giving different limiting behaviors (one unbounded, one zero), the limit:}
\begin{equation}
\lim_{x \to \infty} \frac{\cos((n+1)x)}{x\cos(nx)} \quad \textbf{does not exist}.
\end{equation}

\section{Stage 4: Conclusion}

\subsection{Applying the Definition}

\what{We needed to verify:}
\begin{equation}
\lim_{x \to \infty} \frac{\phi_{n+1}(x)}{\phi_n(x)} = 0 \quad \text{for all } n.
\end{equation}

\what{We have shown that for \textbf{any} $n \geq 1$:}
\begin{equation}
\lim_{x \to \infty} \frac{\phi_{n+1}(x)}{\phi_n(x)} = \lim_{x \to \infty} \frac{\cos((n+1)x)}{x\cos(nx)} \quad \textbf{does not exist}.
\end{equation}

\why{Since the limit does not exist, we cannot have $\phi_{n+1}(x) = o(\phi_n(x))$.}

\how{By the definition of an asymptotic sequence from the lecture notes (Section 2.5), if even \emph{one} of the ratios fails the little-oh condition, the entire sequence is \textbf{not} asymptotic.}

\subsection{The Fundamental Reason}

\begin{center}
\fbox{\begin{minipage}{0.9\textwidth}
\textbf{Why the sequence fails:} The \textbf{oscillatory factors} $\cos(nx)$ and $\cos((n+1)x)$ do not decay as $x \to \infty$. Instead, they oscillate indefinitely. When we form the ratio $\frac{\cos((n+1)x)}{\cos(nx)}$, the denominator passes through zero infinitely often, causing the ratio to become unbounded along certain subsequences. This prevents the overall limit from existing, violating the requirement that $\phi_{n+1} = o(\phi_n)$.
\end{minipage}}
\end{center}

\subsection{Contrasting with a True Asymptotic Sequence}

\what{Compare with the standard asymptotic sequence $\{x^{-n}\}$ (without the cosine factors):}
\begin{equation}
\lim_{x \to \infty} \frac{x^{-(n+1)}}{x^{-n}} = \lim_{x \to \infty} \frac{1}{x} = 0 \quad \checkmark
\end{equation}

\why{Here, there are \textbf{no oscillatory factors}, just pure decay. The ratio has a well-defined limit of zero, so $\{x^{-n}\}$ \textbf{is} an asymptotic sequence as $x \to \infty$.}

\section{Final Answer}

\begin{center}
\fbox{\begin{minipage}{0.9\textwidth}
The sequence $\{\phi_n(x) = x^{-n}\cos(nx)\}$ is \textbf{not} an asymptotic sequence as $x \to \infty$ because:

\begin{enumerate}
\item The ratio $\frac{\phi_{n+1}(x)}{\phi_n(x)} = \frac{\cos((n+1)x)}{x\cos(nx)}$ does not have a well-defined limit as $x \to \infty$.

\item The denominator $\cos(nx)$ vanishes at infinitely many points as $x \to \infty$, causing the ratio to become unbounded along certain sequences.

\item Along other sequences where $\cos(nx) \neq 0$, the ratio may approach zero, but the limit must be the same along \emph{all} sequences for the limit to exist.

\item Since the limit does not exist, we cannot have $\phi_{n+1}(x) = o(\phi_n(x))$, violating the definition of an asymptotic sequence (Lecture Notes, Section 2.5).
\end{enumerate}
\end{minipage}}
\end{center}

\end{document}
