\documentclass[11pt,a4paper]{article}
\usepackage{amsmath,amssymb,amsthm}
\usepackage{geometry}
\geometry{margin=1in}
\usepackage{enumitem}
\usepackage{xcolor}

\newtheorem{theorem}{Theorem}
\newtheorem{lemma}[theorem]{Lemma}
\newtheorem{definition}[theorem]{Definition}

\title{Asymptotics 2025/2026 Sheet 1\\Problem 5: Detailed Solution}
\author{}
\date{}

\begin{document}

\maketitle

\section*{Problem 5}

\textbf{Problem Statement:} Prove that $\displaystyle\sum_{n=1}^{\infty} \frac{1}{z^n}$ is an asymptotic expansion of $\displaystyle\frac{1}{z-1}$ as $z \to \infty$.

\section{Understanding What We Must Prove}

\subsection{What Do We Have?}

We are given:
\begin{itemize}
\item A function: $f(z) = \displaystyle\frac{1}{z-1}$
\item A formal infinite series: $\displaystyle\sum_{n=1}^{\infty} \frac{1}{z^n}$
\item A limiting process: $z \to \infty$
\end{itemize}

\subsection{What Must We Show?}

We must prove that this series is an \textbf{asymptotic expansion} of $f(z)$. But what does this mean precisely?

\subsubsection{Why We Need the Formal Definition}

The phrase ``asymptotic expansion'' has a precise mathematical meaning that we cannot prove without invoking. We must consult the lecture notes Section 2.5, which provides:

\begin{definition}[Asymptotic Series Expansion, from Lecture Notes p.~9]
Given an asymptotic sequence $\{\phi_n(x)\}$, the formal series $\sum_{n=0}^{\infty} a_n\phi_n(x)$ is an asymptotic expansion of $f(x)$ if, for every $N$,
\begin{equation}
\lim_{x \to x_0} \frac{f(x) - \sum_{n=0}^{N} a_n\phi_n(x)}{\phi_N(x)} = 0.
\end{equation}
\end{definition}

\subsubsection{Why This Definition?}

This definition captures the idea that:
\begin{enumerate}
\item The partial sums $\sum_{n=0}^{N} a_n\phi_n(x)$ approximate $f(x)$
\item The error $f(x) - \sum_{n=0}^{N} a_n\phi_n(x)$ is ``small compared to the last term kept''
\item ``Small compared to'' means the ratio goes to zero
\item This must hold for \textit{every} $N$, not just one particular truncation
\end{enumerate}

The key insight is that the remainder $R_N = f(x) - \sum_{n=0}^{N} a_n\phi_n(x)$ must satisfy:
\[
R_N = o(\phi_N(x)) \quad \text{as } x \to x_0.
\]

This is equation (60) in the lecture notes.

\section{Identifying the Components}

\subsection{What is the Asymptotic Sequence?}

From our series $\sum_{n=1}^{\infty} \frac{1}{z^n}$, we can identify:
\[
\phi_n(z) = \frac{1}{z^n} = z^{-n}, \quad n = 1, 2, 3, \ldots
\]

\subsubsection{Why Must We Verify This is an Asymptotic Sequence?}

The definition of asymptotic expansion \textit{requires} that $\{\phi_n(z)\}$ be an asymptotic sequence. From the lecture notes (p.~9):

\begin{definition}[Asymptotic Sequence]
A sequence of functions $\{\phi_n(x)\}$, $n = 1, 2, \ldots$ is an asymptotic sequence as $x \to x_0$ if, for all $n$,
\[
\phi_{n+1}(x) = o(\phi_n(x)) \quad \text{as } x \to x_0,
\]
i.e., $\displaystyle\lim_{x \to x_0} \frac{\phi_{n+1}(x)}{\phi_n(x)} = 0$.
\end{definition}

\subsection{Verification That $\{z^{-n}\}$ is an Asymptotic Sequence}

\textbf{What we must check:} For all $n$, does $\phi_{n+1}(z) = o(\phi_n(z))$ as $z \to \infty$?

\textbf{Computation:}
\[
\frac{\phi_{n+1}(z)}{\phi_n(z)} = \frac{z^{-(n+1)}}{z^{-n}} = \frac{z^{-n-1}}{z^{-n}} = \frac{z^{-n} \cdot z^{-1}}{z^{-n}} = z^{-1} = \frac{1}{z}.
\]

\textbf{Why this computation?} We divided numerator by denominator using properties of exponents.

\textbf{Taking the limit:}
\[
\lim_{z \to \infty} \frac{\phi_{n+1}(z)}{\phi_n(z)} = \lim_{z \to \infty} \frac{1}{z} = 0.
\]

\textbf{Why does this limit equal zero?} As $z$ grows without bound, $1/z$ shrinks toward zero. This is a fundamental limit.

\textbf{Conclusion:} Since the limit is zero for all $n$, we have $\phi_{n+1}(z) = o(\phi_n(z))$ for all $n$, confirming that $\{z^{-n}\}$ is indeed an asymptotic sequence as $z \to \infty$. \checkmark

\subsection{What are the Coefficients?}

From the series $\sum_{n=1}^{\infty} \frac{1}{z^n}$, we have:
\[
a_n = 1 \quad \text{for all } n \geq 1.
\]

\textbf{Why?} Each term is simply $1 \cdot z^{-n}$, so the coefficient multiplying $\phi_n(z) = z^{-n}$ is $a_n = 1$.

\section{Strategy for the Proof}

\subsection{What Must We Prove According to the Definition?}

We must show that for every positive integer $N$:
\[
\lim_{z \to \infty} \frac{f(z) - \sum_{n=1}^{N} a_n \phi_n(z)}{\phi_N(z)} = 0.
\]

Substituting our specific functions:
\[
\lim_{z \to \infty} \frac{\frac{1}{z-1} - \sum_{n=1}^{N} \frac{1}{z^n}}{\frac{1}{z^N}} = 0.
\]

\subsection{Why This Approach?}

This is not merely \textit{one way} to prove the result—it is \textit{the definition} we must verify. We have no choice but to:
\begin{enumerate}
\item Compute the partial sum $\sum_{n=1}^{N} \frac{1}{z^n}$
\item Compute the remainder $R_N = f(z) - \sum_{n=1}^{N} \frac{1}{z^n}$
\item Show that $R_N/\phi_N(z) \to 0$ as $z \to \infty$
\end{enumerate}

\section{Computing the Partial Sum}

\subsection{What is the Partial Sum?}

We need to evaluate:
\[
S_N = \sum_{n=1}^{N} \frac{1}{z^n} = \frac{1}{z} + \frac{1}{z^2} + \frac{1}{z^3} + \cdots + \frac{1}{z^N}.
\]

\subsection{Why is This a Geometric Series?}

\textbf{Observation:} Each term is obtained from the previous by multiplying by the common ratio $r = \frac{1}{z}$.

Indeed:
\[
\frac{1}{z^{n+1}} = \frac{1}{z^n} \cdot \frac{1}{z}.
\]

This is the defining property of a geometric series.

\subsection{Geometric Series Formula}

From elementary calculus/analysis, a geometric series with first term $a$ and common ratio $r$ has partial sum:
\[
\sum_{k=0}^{N-1} ar^k = a \cdot \frac{1 - r^N}{1 - r}, \quad r \neq 1.
\]

\subsection{Applying the Formula}

In our case:
\begin{itemize}
\item First term: $a = \frac{1}{z}$
\item Common ratio: $r = \frac{1}{z}$
\item Number of terms: $N$ (from $n=1$ to $n=N$)
\end{itemize}

\textbf{Therefore:}
\[
\sum_{n=1}^{N} \frac{1}{z^n} = \frac{1}{z} \cdot \frac{1 - \left(\frac{1}{z}\right)^N}{1 - \frac{1}{z}}.
\]

\subsection{Simplifying the Expression}

\textbf{Simplify the numerator:}
\[
\frac{1}{z} \cdot \left(1 - \frac{1}{z^N}\right) = \frac{1}{z} - \frac{1}{z^{N+1}}.
\]

\textbf{Simplify the denominator:}
\[
1 - \frac{1}{z} = \frac{z - 1}{z}.
\]

\textbf{Why these simplifications?} We're expressing everything with a common denominator and combining fractions—standard algebraic manipulations.

\textbf{Complete the division:}
\[
S_N = \frac{\frac{1}{z} - \frac{1}{z^{N+1}}}{\frac{z-1}{z}} = \frac{\frac{1}{z} - \frac{1}{z^{N+1}}}{\frac{z-1}{z}} = \left(\frac{1}{z} - \frac{1}{z^{N+1}}\right) \cdot \frac{z}{z-1}.
\]

\textbf{Distribute:}
\[
S_N = \frac{1}{z-1} - \frac{1}{z^N(z-1)}.
\]

\textbf{Alternative form:} Factor out from numerator:
\[
S_N = \frac{1}{z-1} \left(1 - \frac{1}{z^N}\right).
\]

\subsection{Why This Form is Useful}

This form explicitly shows:
\begin{itemize}
\item The leading behavior: $\frac{1}{z-1}$ (which is our target function $f(z)$!)
\item The correction term: $-\frac{1}{z^N(z-1)}$
\end{itemize}

This structure will make computing the remainder trivial.

\section{Computing the Remainder}

\subsection{Definition of Remainder}

The remainder after $N$ terms is:
\[
R_N = f(z) - S_N = \frac{1}{z-1} - \sum_{n=1}^{N} \frac{1}{z^n}.
\]

\subsection{Substituting Our Result}

We found $S_N = \frac{1}{z-1} - \frac{1}{z^N(z-1)}$, therefore:
\[
R_N = \frac{1}{z-1} - \left[\frac{1}{z-1} - \frac{1}{z^N(z-1)}\right].
\]

\subsection{Simplification}

\textbf{Distribute the negative sign:}
\[
R_N = \frac{1}{z-1} - \frac{1}{z-1} + \frac{1}{z^N(z-1)}.
\]

\textbf{Cancel like terms:}
\[
R_N = \frac{1}{z^N(z-1)}.
\]

\textbf{Why does this make sense?} The partial sum $S_N$ was designed to approximate $f(z) = \frac{1}{z-1}$, and we explicitly computed it as $\frac{1}{z-1}$ minus a correction. The remainder is precisely that correction term.

\subsection{Alternative Expression}

We can also write:
\[
R_N = \frac{1}{z^N} \cdot \frac{1}{z-1}.
\]

\textbf{Why factor this way?} Because $\phi_N(z) = \frac{1}{z^N}$ appears explicitly, which will be essential for computing the ratio $R_N/\phi_N(z)$.

\section{Verifying the Asymptotic Condition}

\subsection{What Must We Show?}

According to the definition, we must prove:
\[
\lim_{z \to \infty} \frac{R_N}{\phi_N(z)} = 0.
\]

\subsection{Computing the Ratio}

\textbf{Substitute our expressions:}
\[
\frac{R_N}{\phi_N(z)} = \frac{\frac{1}{z^N(z-1)}}{\frac{1}{z^N}}.
\]

\textbf{Why this substitution?} We use $R_N = \frac{1}{z^N(z-1)}$ and $\phi_N(z) = \frac{1}{z^N}$ directly from our previous work.

\subsection{Simplifying the Complex Fraction}

\textbf{Method 1: Multiply numerator and denominator by $z^N$}

Multiply both numerator and denominator by $z^N$:
\[
\frac{R_N}{\phi_N(z)} = \frac{\frac{1}{z^N(z-1)} \cdot z^N}{\frac{1}{z^N} \cdot z^N} = \frac{\frac{z^N}{z^N(z-1)}}{1} = \frac{1}{z-1}.
\]

\textbf{Why multiply by $z^N$?} This is the standard technique for simplifying complex fractions: multiply by the LCD of all denominators. Here, $z^N$ appears in denominators of both the numerator and denominator of our complex fraction.

\textbf{Method 2: Division of fractions}

Recall that $\frac{a/b}{c/d} = \frac{a}{b} \cdot \frac{d}{c}$:
\[
\frac{R_N}{\phi_N(z)} = \frac{1}{z^N(z-1)} \cdot \frac{z^N}{1} = \frac{z^N}{z^N(z-1)} = \frac{1}{z-1}.
\]

\textbf{Why does this work?} The $z^N$ factors cancel.

\textbf{Both methods yield:}
\[
\frac{R_N}{\phi_N(z)} = \frac{1}{z-1}.
\]

\subsection{Taking the Limit}

\textbf{We must evaluate:}
\[
\lim_{z \to \infty} \frac{1}{z-1}.
\]

\textbf{Why can we take this limit?} The function $\frac{1}{z-1}$ is defined for all $z \neq 1$, and we're considering $z \to \infty$, far from the point $z = 1$.

\subsection{Evaluating the Limit}

\textbf{Intuition:} As $z$ grows large, $z - 1$ also grows large (approximately like $z$), so $\frac{1}{z-1}$ shrinks toward zero.

\textbf{Formal argument:}

Method 1 (Direct):
\[
\lim_{z \to \infty} \frac{1}{z-1} = 0,
\]
because the denominator grows without bound while the numerator remains constant.

Method 2 (Using limit laws):
\[
\lim_{z \to \infty} \frac{1}{z-1} = \frac{\lim_{z \to \infty} 1}{\lim_{z \to \infty} (z-1)} = \frac{1}{\infty} = 0.
\]

Method 3 (Rigorous $\epsilon$-$\delta$):
For any $\epsilon > 0$, we need $z$ large enough that $\left|\frac{1}{z-1}\right| < \epsilon$.

This requires: $\frac{1}{|z-1|} < \epsilon$, i.e., $|z-1| > \frac{1}{\epsilon}$.

For $z > \frac{1}{\epsilon} + 1$, this holds. So choosing $M = \frac{1}{\epsilon} + 1$, we have that $z > M$ implies $\left|\frac{1}{z-1}\right| < \epsilon$.

\textbf{Conclusion:}
\[
\lim_{z \to \infty} \frac{R_N}{\phi_N(z)} = \lim_{z \to \infty} \frac{1}{z-1} = 0. \quad \checkmark
\]

\section{Final Verification and Conclusion}

\subsection{What Have We Proven?}

We have shown that for \textit{every} positive integer $N$:
\[
\lim_{z \to \infty} \frac{f(z) - \sum_{n=1}^{N} \frac{1}{z^n}}{\frac{1}{z^N}} = 0.
\]

\subsection{Why Does This Complete the Proof?}

This is \textit{precisely} the definition of an asymptotic expansion from the lecture notes (Definition on p.~9, Equation 58). We have verified:

\begin{enumerate}
\item $\{z^{-n}\}$ is an asymptotic sequence as $z \to \infty$ (\checkmark, Section 2.2)
\item For every $N$, the remainder $R_N = f(z) - \sum_{n=1}^{N} a_n z^{-n}$ satisfies $R_N = o(z^{-N})$ as $z \to \infty$ (\checkmark, Section 6)
\end{enumerate}

These are the two requirements in the definition.

\subsection{Why Did This Work?}

The key insight is that our function $f(z) = \frac{1}{z-1}$ can be written exactly as:
\[
\frac{1}{z-1} = \sum_{n=1}^{N} \frac{1}{z^n} + \frac{1}{z^N(z-1)},
\]
where the remainder $\frac{1}{z^N(z-1)}$ is smaller than $\frac{1}{z^N}$ by a factor of $\frac{1}{z-1} \to 0$.

\subsection{Connection to Convergence}

\textbf{Important Note:} The series $\sum_{n=1}^{\infty} \frac{1}{z^n}$ actually \textit{converges} for $|z| > 1$:
\[
\sum_{n=1}^{\infty} \frac{1}{z^n} = \frac{1/z}{1 - 1/z} = \frac{1}{z-1}.
\]

So in this case, the asymptotic expansion is also a convergent series equal to $f(z)$.

\textbf{Why mention this?} To emphasize that being an asymptotic expansion is a \textit{weaker} condition than convergence. As noted in the lecture notes (p.~9): ``a convergent power series is asymptotic, while a power series can be asymptotic without being convergent.''

Our proof of the asymptotic property did not require proving convergence—it only required showing the remainder is $o(\phi_N)$.

\section{Summary}

\subsection{What We Proved}

We have rigorously established:

\begin{theorem}
The series $\displaystyle\sum_{n=1}^{\infty} \frac{1}{z^n}$ is an asymptotic expansion of $\displaystyle\frac{1}{z-1}$ as $z \to \infty$.
\end{theorem}

\subsection{Method Used}

Following the methodology from Lecture Notes Section 2.5:
\begin{enumerate}
\item Identified the asymptotic sequence $\{\phi_n(z)\} = \{z^{-n}\}$
\item Verified it satisfies the asymptotic sequence condition
\item Computed the partial sum $S_N = \sum_{n=1}^{N} \frac{1}{z^n}$ using geometric series formula
\item Computed the remainder $R_N = f(z) - S_N$
\item Showed $R_N/\phi_N(z) \to 0$ as $z \to \infty$
\item Concluded this holds for all $N$, completing the proof
\end{enumerate}

\subsection{Why Each Step Was Necessary}

\begin{itemize}
\item \textbf{Step 1-2:} The definition requires an asymptotic sequence
\item \textbf{Step 3:} We need the explicit form of the partial sum
\item \textbf{Step 4:} The remainder is what we must control
\item \textbf{Step 5:} This is the heart of the definition—proving the limit is zero
\item \textbf{Step 6:} The definition requires this for \textit{every} $N$
\end{itemize}

\begin{center}
\fbox{\parbox{0.9\textwidth}{
\textbf{Conclusion:} We have proven that $\displaystyle\sum_{n=1}^{\infty} \frac{1}{z^n}$ is an asymptotic expansion of $\displaystyle\frac{1}{z-1}$ as $z \to \infty$ by verifying the defining property: for every $N$, the remainder after $N$ terms is asymptotically smaller than the $N$-th term. \quad $\blacksquare$
}}
\end{center}

\end{document}
