\documentclass[11pt,a4paper]{article}
\usepackage[utf8]{inputenc}
\usepackage{amsmath,amssymb,amsthm}
\usepackage[margin=2.5cm]{geometry}
\usepackage{enumitem}
\usepackage{xcolor}

\newtheorem{theorem}{Theorem}
\newtheorem{lemma}{Lemma}

\title{Asymptotics Problem Sheet 2\\Local Approximation to Linear ODEs}
\author{Solution to Problem 1}
\date{Academic Year 2025/2026}

\begin{document}

\maketitle

\section*{Problem 1: Leading Behaviours as \texorpdfstring{$x \to 0^+$}{x → 0+}}

\noindent\textbf{Context and Methodology:} We seek the leading order behaviour of solutions to ODEs near $x = 0$. From Section 3.2 of the lecture notes, we know that $x = 0$ is an \textbf{irregular singular point} for these equations. The standard Frobenius series method fails at irregular singular points, so we employ the \textbf{controlling factor ansatz} combined with \textbf{dominant balance analysis}.

\vspace{1em}
\noindent\textbf{Core Strategy:}
\begin{enumerate}[leftmargin=*]
\item Use the ansatz $y(x) = e^{S(x)}$ where $S(x)$ contains the most rapidly changing behavior
\item Perform dominant balance analysis to find $S(x) \sim Cx^\beta$ as $x \to 0^+$
\item Determine $\beta$ and $C$ by balancing dominant terms in the ODE
\item The leading behavior is $y(x) \sim e^{S(x)}$ as $x \to 0^+$
\end{enumerate}

\subsection*{Problem 1(a): \texorpdfstring{$x^4 y''' = y$}{x⁴y''' = y}}

\noindent\textbf{Step 1: Identify the nature of $x = 0$.}

\noindent\textit{Why do we do this?} We must determine if $x = 0$ is an ordinary point, regular singular point, or irregular singular point to know which solution method to apply.

Rewrite the equation in standard form:
\begin{equation}
y''' = \frac{1}{x^4} y
\end{equation}

\noindent\textit{What do we observe?} The coefficient of $y$ has the form $\frac{1}{x^4}$, which becomes unbounded as $x \to 0$.

\noindent\textit{Why does this matter?} According to the classification in Section 3.1 of the lecture notes, for a third-order ODE written as
\begin{equation}
y''' + p_2(x)y'' + p_1(x)y' + p_0(x)y = 0,
\end{equation}
the point $x = 0$ is a regular singular point only if $x^3 p_0(x)$, $x^2 p_1(x)$, and $x p_2(x)$ are all analytic at $x = 0$.

\noindent\textit{What is our conclusion?} Here $p_0(x) = -\frac{1}{x^4}$, so $x^3 p_0(x) = -\frac{1}{x}$ is \textbf{not} analytic at $x = 0$. Therefore, $x = 0$ is an \textbf{irregular singular point}.

\vspace{1em}
\noindent\textbf{Step 2: Apply the controlling factor ansatz.}

\noindent\textit{Why this ansatz?} From Section 3.2.1 of the lecture notes, near irregular singular points, the solution exhibits exponential behavior. We therefore use:
\begin{equation}
y(x) = e^{S(x)}
\end{equation}

\noindent\textit{What are the derivatives?} We need to compute $y'''$. First:
\begin{align}
y' &= S' e^{S(x)} = S' y \\
y'' &= S'' e^{S(x)} + (S')^2 e^{S(x)} = (S'' + (S')^2) y \\
y''' &= S''' e^{S(x)} + 3S' S'' e^{S(x)} + (S')^3 e^{S(x)}
\end{align}

\noindent\textit{Why these forms?} Each derivative of $y = e^{S(x)}$ produces a polynomial in derivatives of $S(x)$ multiplied by $y$ itself.

Thus:
\begin{equation}
y''' = (S''' + 3S' S'' + (S')^3) y
\end{equation}

\vspace{1em}
\noindent\textbf{Step 3: Substitute into the ODE.}

Substituting $y''' = (S''' + 3S' S'' + (S')^3) y$ into $x^4 y''' = y$:
\begin{equation}
x^4 (S''' + 3S' S'' + (S')^3) y = y
\end{equation}

\noindent\textit{Why can we cancel $y$?} Since we seek non-trivial solutions, $y \neq 0$, and we can divide both sides by $y$:
\begin{equation}
x^4 S''' + 3x^4 S' S'' + x^4 (S')^3 = 1
\end{equation}

\vspace{1em}
\noindent\textbf{Step 4: Perform dominant balance analysis.}

\noindent\textit{What is dominant balance?} From Section 3.2.2, we assume $S(x) \sim Cx^\beta$ as $x \to 0^+$ and determine which terms dominate.

Let $S(x) \sim Cx^\beta$. Then:
\begin{align}
S'(x) &\sim C\beta x^{\beta-1} \\
S''(x) &\sim C\beta(\beta-1) x^{\beta-2} \\
S'''(x) &\sim C\beta(\beta-1)(\beta-2) x^{\beta-3}
\end{align}

\noindent\textit{What are the orders of each term?} Substituting into equation (7):
\begin{align}
x^4 S''' &\sim C\beta(\beta-1)(\beta-2) x^{\beta+1} \\
3x^4 S' S'' &\sim 3C^2\beta^2(\beta-1) x^{2\beta+2} \\
x^4(S')^3 &\sim C^3\beta^3 x^{3\beta+1} \\
1 &\sim x^0
\end{align}

\vspace{1em}
\noindent\textbf{Step 5: Determine which terms balance.}

\noindent\textit{Strategy:} We systematically check which pairs of terms can balance (have the same order) as $x \to 0^+$.

\noindent\textbf{Case 1:} Assume $x^4 S''' \sim 1$, i.e., $x^{\beta+1} \sim x^0$.

\noindent\textit{What does this give?} This implies $\beta + 1 = 0$, so $\beta = -1$.

\noindent\textit{Is this consistent?} With $\beta = -1$:
\begin{align}
x^4 S''' &\sim x^{-1+1} = x^0 \sim 1 \quad \checkmark \\
3x^4 S' S'' &\sim x^{-2+2} = x^0 \sim 1 \quad \text{(also order 1!)} \\
x^4(S')^3 &\sim x^{-3+1} = x^{-2} \to \infty \quad \text{(dominant!)}
\end{align}

\noindent\textit{Why is this inconsistent?} The term $x^4(S')^3$ becomes unbounded as $x \to 0^+$, dominating both other terms. We cannot neglect it. This violates our assumption.

\vspace{0.5em}
\noindent\textbf{Case 2:} Assume $3x^4 S' S'' \sim 1$, i.e., $x^{2\beta+2} \sim x^0$.

\noindent\textit{What does this give?} This implies $2\beta + 2 = 0$, so $\beta = -1$.

\noindent\textit{Is this consistent?} This is the same $\beta$ as Case 1, leading to the same inconsistency.

\vspace{0.5em}
\noindent\textbf{Case 3:} Assume $x^4(S')^3 \sim 1$, i.e., $x^{3\beta+1} \sim x^0$.

\noindent\textit{What does this give?} This implies $3\beta + 1 = 0$, so:
\begin{equation}
\beta = -\frac{1}{3}
\end{equation}

\noindent\textit{Is this consistent?} With $\beta = -\frac{1}{3}$:
\begin{align}
x^4 S''' &\sim x^{-1/3+1} = x^{2/3} \to 0 \quad \text{as } x \to 0^+ \\
3x^4 S' S'' &\sim x^{-2/3+2} = x^{4/3} \to 0 \quad \text{as } x \to 0^+ \\
x^4(S')^3 &\sim x^{-1+1} = x^0 \sim 1 \quad \checkmark
\end{align}

\noindent\textit{Why is this consistent?} Both $x^4 S'''$ and $3x^4 S' S''$ vanish as $x \to 0^+$, while $x^4(S')^3 \sim 1$. The dominant balance equation becomes:
\begin{equation}
x^4(S')^3 \sim 1 \quad \text{as } x \to 0^+
\end{equation}

\vspace{1em}
\noindent\textbf{Step 6: Solve for $C$.}

From $S(x) \sim Cx^{-1/3}$, we have $S'(x) \sim -\frac{C}{3} x^{-4/3}$. Therefore:
\begin{equation}
x^4 (S')^3 \sim x^4 \left(-\frac{C}{3}\right)^3 x^{-4} = -\frac{C^3}{27} \sim 1
\end{equation}

\noindent\textit{What does this tell us?} Solving for $C$:
\begin{equation}
C^3 = -27 \quad \Rightarrow \quad C = -3
\end{equation}

\noindent\textit{Why the choice of sign?} We take the real cube root. Thus:
\begin{equation}
S(x) \sim -3x^{-1/3} \quad \text{as } x \to 0^+
\end{equation}

\vspace{1em}
\noindent\textbf{Step 7: Write the leading behavior.}

\noindent\textit{Final answer:} The leading behavior as $x \to 0^+$ is:
\begin{equation}
\boxed{y(x) \sim e^{-3x^{-1/3}} \quad \text{as } x \to 0^+}
\end{equation}

\noindent\textit{Interpretation:} This solution exhibits extremely rapid decay as $x \to 0^+$ (the exponent $-3x^{-1/3} \to -\infty$), characteristic of solutions near irregular singular points.

\subsection*{Problem 1(b): \texorpdfstring{$y'' = (\cot x)^4 y$}{y'' = (cot x)⁴y}}

\noindent\textbf{Given hint:} $\cot x \sim \frac{1}{x} - \frac{x}{3} + \ldots$ as $x \to 0$

\vspace{0.5em}
\noindent\textbf{Step 1: Determine the leading behavior of the coefficient.}

\noindent\textit{Why do we need this?} To classify the singularity at $x = 0$ and perform dominant balance, we need the asymptotic form of $(\cot x)^4$ as $x \to 0$.

Using the given expansion:
\begin{equation}
\cot x \sim \frac{1}{x} - \frac{x}{3} + \ldots \sim \frac{1}{x}\left(1 - \frac{x^2}{3} + \ldots\right) \quad \text{as } x \to 0
\end{equation}

\noindent\textit{What is the leading term?} As $x \to 0$:
\begin{equation}
(\cot x)^4 \sim \frac{1}{x^4}
\end{equation}

\vspace{0.5em}
\noindent\textbf{Step 2: Identify the nature of $x = 0$.}

The ODE becomes:
\begin{equation}
y'' \sim \frac{1}{x^4} y \quad \text{as } x \to 0
\end{equation}

\noindent\textit{Why is this an irregular singular point?} For a second-order ODE $y'' + p_1(x)y' + p_0(x)y = 0$, the point $x = 0$ is regular singular only if $x^2 p_0(x)$ is analytic at $x = 0$. Here $p_0(x) = -\frac{1}{x^4}$, so:
\begin{equation}
x^2 p_0(x) = -\frac{1}{x^2}
\end{equation}
which is \textbf{not analytic} at $x = 0$. Thus $x = 0$ is an \textbf{irregular singular point}.

\vspace{1em}
\noindent\textbf{Step 3: Apply the controlling factor ansatz.}

Let $y(x) = e^{S(x)}$. Then:
\begin{align}
y' &= S' y \\
y'' &= (S'' + (S')^2) y
\end{align}

Substituting into $y'' = (\cot x)^4 y$:
\begin{equation}
(S'' + (S')^2) y = (\cot x)^4 y
\end{equation}

Canceling $y$ (for non-trivial solutions):
\begin{equation}
S'' + (S')^2 = (\cot x)^4
\end{equation}

\vspace{1em}
\noindent\textbf{Step 4: Apply the standard assumption.}

\noindent\textit{From the lecture notes (Section 3.2.2):} Near irregular singular points, we often have $S'' = o((S')^2)$ as $x \to 0^+$, meaning $S''$ is subdominant to $(S')^2$.

\noindent\textit{Why make this assumption?} This is a heuristic that works in many cases. We'll verify consistency after finding $S(x)$.

Assuming $S'' \ll (S')^2$, the dominant balance equation is:
\begin{equation}
(S')^2 \sim (\cot x)^4 \sim \frac{1}{x^4} \quad \text{as } x \to 0
\end{equation}

\vspace{0.5em}
\noindent\textbf{Step 5: Solve for $S'$.}

Taking square roots:
\begin{equation}
S' \sim \pm \frac{1}{x^2} \quad \text{as } x \to 0
\end{equation}

\noindent\textit{What does integration give?} Integrating:
\begin{equation}
S(x) \sim \pm \int \frac{1}{x^2} dx = \mp \frac{1}{x} + \text{const.}
\end{equation}

\noindent\textit{Why does the constant not matter?} Constants in $S(x)$ only contribute multiplicative constants to $y = e^{S(x)}$, which are absorbed into the general solution.

\vspace{0.5em}
\noindent\textbf{Step 6: Verify the assumption $S'' \ll (S')^2$.}

With $S(x) \sim \mp \frac{1}{x}$:
\begin{align}
S'(x) &\sim \pm \frac{1}{x^2} \\
S''(x) &\sim \mp \frac{2}{x^3} \\
(S')^2 &\sim \frac{1}{x^4}
\end{align}

\noindent\textit{Is the assumption valid?} As $x \to 0^+$:
\begin{equation}
\frac{|S''|}{|(S')^2|} \sim \frac{2/x^3}{1/x^4} = 2x \to 0
\end{equation}

\noindent\textit{Conclusion:} Yes! We have $S'' = o((S')^2)$ as $x \to 0^+$, confirming our assumption.

\vspace{1em}
\noindent\textbf{Step 7: Write the leading behavior.}

\noindent\textit{Final answer:} The leading behaviors as $x \to 0^+$ are:
\begin{equation}
\boxed{y(x) \sim e^{\pm 1/x} \quad \text{as } x \to 0^+}
\end{equation}

More explicitly, the general solution has the form:
\begin{equation}
y(x) \sim A e^{1/x} + B e^{-1/x} \quad \text{as } x \to 0^+
\end{equation}

\noindent\textit{Interpretation:} One solution grows extremely rapidly ($e^{1/x} \to \infty$) while the other decays extremely rapidly ($e^{-1/x} \to 0$) as $x \to 0^+$.

\subsection*{Problem 1(c): \texorpdfstring{$x^4 y''' - 3x^2 y' + 2y = 0$}{x⁴y''' - 3x²y' + 2y = 0}}

\noindent\textbf{Step 1: Check if $x = 0$ is an irregular singular point.}

Rewrite in standard form:
\begin{equation}
y''' = \frac{3x^2 y' - 2y}{x^4} = \frac{3}{x^2} y' - \frac{2}{x^4} y
\end{equation}

\noindent\textit{Classification:} For $p_2(x) = 0$, $p_1(x) = -\frac{3}{x^2}$, $p_0(x) = \frac{2}{x^4}$:
\begin{align}
x \cdot p_2(x) &= 0 \quad \text{(analytic)} \\
x^2 \cdot p_1(x) &= -3 \quad \text{(analytic)} \\
x^3 \cdot p_0(x) &= \frac{2}{x} \quad \text{(NOT analytic)}
\end{align}

Therefore $x = 0$ is an \textbf{irregular singular point}.

\vspace{1em}
\noindent\textbf{Step 2: Try a power law solution first.}

\noindent\textit{Why try this?} Before using the exponential ansatz, we check if a simple power law $y = x^\alpha$ works, as this is sometimes sufficient for certain irregular singular points.

Let $y = x^\alpha$. Then:
\begin{align}
y' &= \alpha x^{\alpha-1} \\
y''' &= \alpha(\alpha-1)(\alpha-2) x^{\alpha-3}
\end{align}

Substituting into the ODE:
\begin{equation}
x^4 \cdot \alpha(\alpha-1)(\alpha-2) x^{\alpha-3} - 3x^2 \cdot \alpha x^{\alpha-1} + 2x^\alpha = 0
\end{equation}

Simplifying:
\begin{equation}
\alpha(\alpha-1)(\alpha-2) x^{\alpha+1} - 3\alpha x^{\alpha+1} + 2x^\alpha = 0
\end{equation}

\noindent\textit{What must hold?} Factor out $x^\alpha$:
\begin{equation}
x^\alpha \left[\alpha(\alpha-1)(\alpha-2) x - 3\alpha x + 2\right] = 0
\end{equation}

\noindent\textit{Can this vanish for all $x$ near 0?} For this to hold as $x \to 0^+$, we need the coefficient of $x$ and the constant term to vanish separately:
\begin{align}
\alpha(\alpha-1)(\alpha-2) - 3\alpha &= 0 \\
2 &= 0 \quad \text{(Contradiction!)}
\end{align}

\noindent\textit{Conclusion:} A pure power law solution does not work.

\vspace{1em}
\noindent\textbf{Step 3: Apply the controlling factor ansatz.}

Let $y = e^{S(x)}$:
\begin{align}
y' &= S' y \\
y''' &= (S''' + 3S'S'' + (S')^3) y
\end{align}

Substituting:
\begin{equation}
x^4(S''' + 3S'S'' + (S')^3) y - 3x^2 S' y + 2y = 0
\end{equation}

Dividing by $y$:
\begin{equation}
x^4 S''' + 3x^4 S' S'' + x^4(S')^3 - 3x^2 S' + 2 = 0
\end{equation}

\vspace{1em}
\noindent\textbf{Step 4: Dominant balance analysis.}

Assume $S(x) \sim Cx^\beta$. The orders of each term as $x \to 0^+$ are:
\begin{align}
x^4 S''' &\sim x^{\beta+1} \\
3x^4 S' S'' &\sim x^{2\beta+2} \\
x^4(S')^3 &\sim x^{3\beta+1} \\
-3x^2 S' &\sim x^{\beta-1} \\
2 &\sim x^0
\end{align}

\noindent\textbf{Step 5: Find which terms balance.}

\noindent\textit{Strategy:} We need at least two terms of the same order. Let's systematically check.

\noindent\textbf{Try:} $x^4(S')^3 \sim 2$, i.e., $3\beta + 1 = 0 \Rightarrow \beta = -\frac{1}{3}$.

With $\beta = -\frac{1}{3}$:
\begin{align}
x^4 S''' &\sim x^{2/3} \to 0 \\
3x^4 S' S'' &\sim x^{4/3} \to 0 \\
x^4(S')^3 &\sim x^0 \sim 1 \\
-3x^2 S' &\sim x^{-4/3} \to \infty \quad \text{(Dominant!)}
\end{align}

\noindent\textit{Problem:} The term $-3x^2 S'$ dominates, contradicting our assumption.

\vspace{0.5em}
\noindent\textbf{Try:} $-3x^2 S' \sim 2$, i.e., $\beta - 1 = 0 \Rightarrow \beta = 1$.

With $\beta = 1$:
\begin{align}
x^4 S''' &\sim x^{2} \to 0 \\
3x^4 S' S'' &\sim x^{4} \to 0 \\
x^4(S')^3 &\sim x^{4} \to 0 \\
-3x^2 S' &\sim x^{0} \sim 1 \quad \checkmark
\end{align}

\noindent\textit{Is this consistent?} Yes! All other terms vanish as $x \to 0^+$. The dominant balance is:
\begin{equation}
-3x^2 S' + 2 \sim 0 \quad \text{as } x \to 0^+
\end{equation}

\vspace{0.5em}
\noindent\textbf{Step 6: Solve for $C$.}

With $S(x) \sim Cx$, we have $S'(x) \sim C$. Therefore:
\begin{equation}
-3x^2 \cdot C + 2 \sim 0 \quad \text{as } x \to 0^+
\end{equation}

\noindent\textit{Wait, this is problematic!} The term $-3Cx^2 \to 0$ as $x \to 0^+$, but the constant $2$ does not. They cannot balance.

\vspace{0.5em}
\noindent\textit{Re-examination:} Let's reconsider. Perhaps the solution has a different structure. Try $y = x^\alpha e^{S(x)}$ where $S(x)$ contains the exponential behavior.

Actually, let me try $y = x^\alpha$ more carefully. From equation (44):
\begin{equation}
\alpha(\alpha-1)(\alpha-2) x^{\alpha+1} - 3\alpha x^{\alpha+1} + 2x^\alpha = 0
\end{equation}

Factor out $x^\alpha$:
\begin{equation}
x^\alpha \left[x(\alpha(\alpha-1)(\alpha-2) - 3\alpha) + 2\right] = 0
\end{equation}

For this to be satisfied as $x \to 0^+$, we need:
\begin{equation}
2 = 0 \quad \text{or} \quad x \to 0 \text{ and coefficient of } x \text{ diverges}
\end{equation}

\noindent\textit{Alternative approach:} Perhaps there's a boundary layer. Or we need the logarithmic term. Let me try:
\begin{equation}
y = x^\alpha (\text{polynomial or log terms})
\end{equation}

Actually, looking at the structure, let me try $y \sim x^\alpha$ where we determine $\alpha$ from the constant term dominance:
\begin{equation}
2x^\alpha \sim \text{leading term}
\end{equation}

From the ODE, if $y \sim x^\alpha$, the terms scale as:
\begin{align}
x^4 y''' &\sim \alpha^3 x^{\alpha+1} \\
-3x^2 y' &\sim -3\alpha x^{\alpha+1} \\
2y &\sim 2x^\alpha
\end{align}

For the constant term $2y$ to be the leading term, we need:
\begin{equation}
\alpha < \alpha + 1
\end{equation}
which is always true. So $2y$ is indeed the leading term if $\alpha < \alpha + 1$.

But we need at least two terms to balance. Let's balance $2y$ with $-3x^2 y'$:
\begin{equation}
2x^\alpha \sim -3\alpha x^{\alpha+1}
\end{equation}

This gives $\alpha = \alpha + 1$, which is impossible.

Let's balance $2y$ with $x^4 y'''$:
\begin{equation}
2x^\alpha \sim \alpha(\alpha-1)(\alpha-2) x^{\alpha+1}
\end{equation}
Again, $\alpha = \alpha + 1$ is impossible.

\vspace{0.5em}
\noindent\textit{Conclusion:} The dominant balance suggests that as $x \to 0^+$, the constant term $2y$ must balance with itself being small, which means:
\begin{equation}
y(x) \to 0 \quad \text{as } x \to 0^+
\end{equation}

A more careful analysis (beyond basic dominant balance) would use modified Frobenius or study the full series structure. For our purposes:

\noindent\textit{Final answer:}
\begin{equation}
\boxed{y(x) = O(1) \text{ or decays as } x \to 0^+}
\end{equation}

More precisely, the solution likely has the form $y \sim x^\alpha$ with $\alpha > 0$, giving:
\begin{equation}
\boxed{y(x) \to 0 \quad \text{as } x \to 0^+}
\end{equation}

\subsection*{Problem 1(d): \texorpdfstring{$y'' = \sqrt{x} \, y$}{y'' = √x y}}

\noindent\textbf{Step 1: Identify the singularity.}

Rewrite as:
\begin{equation}
y'' - \sqrt{x} \, y = 0
\end{equation}

\noindent\textit{What is $p_0(x)$?} Here $p_0(x) = -\sqrt{x}$. Check $x^2 p_0(x) = -x^{2} \sqrt{x} = -x^{5/2}$.

\noindent\textit{Is this analytic at $x = 0$?} Yes! The function $x^{5/2}$ is analytic at $x = 0$ (though only infinitely differentiable from the right).

\noindent\textit{Wait, is $x = 0$ regular or irregular?} Actually, for real $x$, $x^{5/2}$ is well-defined and smooth for $x > 0$. But $\sqrt{x}$ is not analytic in the complex sense at $x = 0$ (it's a branch point). For the purpose of this problem, we treat $x = 0$ as a \textbf{regular singular point} or at worst, a weak singularity.

\vspace{1em}
\noindent\textbf{Step 2: Try a Frobenius ansatz.}

\noindent\textit{Why Frobenius?} Since the singularity is mild, a modified Frobenius series might work:
\begin{equation}
y(x) = x^\alpha \sum_{n=0}^\infty a_n x^n
\end{equation}

However, given the $\sqrt{x}$ coefficient, let's try:
\begin{equation}
y(x) = e^{S(x)}
\end{equation}

\vspace{0.5em}
\noindent\textbf{Step 3: Apply controlling factor ansatz.}

With $y = e^{S(x)}$:
\begin{equation}
y'' = (S'' + (S')^2) e^{S(x)} = (S'' + (S')^2) y
\end{equation}

The ODE becomes:
\begin{equation}
S'' + (S')^2 = \sqrt{x}
\end{equation}

\vspace{0.5em}
\noindent\textbf{Step 4: Dominant balance.}

Assume $S(x) \sim Cx^\beta$ as $x \to 0^+$:
\begin{align}
S' &\sim C\beta x^{\beta-1} \\
S'' &\sim C\beta(\beta-1) x^{\beta-2} \\
(S')^2 &\sim C^2\beta^2 x^{2\beta-2}
\end{align}

The RHS has order $x^{1/2}$.

\noindent\textbf{Try:} $(S')^2 \sim x^{1/2}$, i.e., $2\beta - 2 = \frac{1}{2} \Rightarrow \beta = \frac{5}{4}$.

With $\beta = \frac{5}{4}$:
\begin{align}
S'' &\sim x^{5/4-2} = x^{-3/4} \to \infty \\
(S')^2 &\sim x^{5/2-2} = x^{1/2}
\end{align}

\noindent\textit{Problem:} $S''$ diverges, dominating $(S')^2$ as $x \to 0^+$. This is inconsistent.

\noindent\textbf{Try:} $S'' \sim x^{1/2}$, i.e., $\beta - 2 = \frac{1}{2} \Rightarrow \beta = \frac{5}{2}$.

With $\beta = \frac{5}{2}$:
\begin{align}
S'' &\sim x^{1/2} \\
(S')^2 &\sim x^{3}
\end{align}

\noindent\textit{Is this consistent?} Yes! As $x \to 0^+$, $(S')^2 \to 0$ while $S'' \sim x^{1/2}$. The dominant balance is:
\begin{equation}
S'' \sim \sqrt{x} \quad \text{as } x \to 0^+
\end{equation}

\vspace{0.5em}
\noindent\textbf{Step 5: Solve for $S(x)$.}

With $S(x) \sim Cx^{5/2}$, we have:
\begin{equation}
S''(x) \sim C \cdot \frac{5}{2} \cdot \frac{3}{2} x^{1/2} = \frac{15C}{4} x^{1/2}
\end{equation}

Matching with $S'' \sim x^{1/2}$:
\begin{equation}
\frac{15C}{4} = 1 \quad \Rightarrow \quad C = \frac{4}{15}
\end{equation}

Therefore:
\begin{equation}
S(x) \sim \frac{4}{15} x^{5/2} \quad \text{as } x \to 0^+
\end{equation}

\vspace{0.5em}
\noindent\textbf{Step 6: Write the leading behavior.}

\noindent\textit{Final answer:}
\begin{equation}
\boxed{y(x) \sim e^{\pm \frac{4}{15} x^{5/2}} \sim 1 + O(x^{5/2}) \quad \text{as } x \to 0^+}
\end{equation}

\noindent\textit{Interpretation:} Near $x = 0$, the exponential factor is close to 1, so the solution is approximately constant (or slightly varying) as $x \to 0^+$.

\subsection*{Problem 1(e): \texorpdfstring{$x^5 y''' - 2xy' + y = 0$}{x⁵y''' - 2xy' + y = 0}}

\noindent\textbf{Step 1: Classify the singularity.}

In standard form:
\begin{equation}
y''' = \frac{2xy' - y}{x^5}
\end{equation}

\noindent\textit{Is $x = 0$ irregular?} We have $p_0(x) = -\frac{1}{x^5}$, so:
\begin{equation}
x^3 p_0(x) = -\frac{1}{x^2}
\end{equation}
which is not analytic at $x = 0$. Thus $x = 0$ is an \textbf{irregular singular point}.

\vspace{1em}
\noindent\textbf{Step 2: Try a power law first.}

Let $y = x^\alpha$:
\begin{align}
y' &= \alpha x^{\alpha-1} \\
y''' &= \alpha(\alpha-1)(\alpha-2) x^{\alpha-3}
\end{align}

Substituting:
\begin{equation}
x^5 \alpha(\alpha-1)(\alpha-2) x^{\alpha-3} - 2x \cdot \alpha x^{\alpha-1} + x^\alpha = 0
\end{equation}

Simplifying:
\begin{equation}
\alpha(\alpha-1)(\alpha-2) x^{\alpha+2} - 2\alpha x^\alpha + x^\alpha = 0
\end{equation}

Factor out $x^\alpha$:
\begin{equation}
x^\alpha \left[\alpha(\alpha-1)(\alpha-2) x^2 - 2\alpha + 1\right] = 0
\end{equation}

\noindent\textit{For what $\alpha$ does this hold as $x \to 0^+$?} The constant terms must vanish:
\begin{equation}
-2\alpha + 1 = 0 \quad \Rightarrow \quad \alpha = \frac{1}{2}
\end{equation}

\noindent\textit{Check:} With $\alpha = \frac{1}{2}$:
\begin{align}
\alpha(\alpha-1)(\alpha-2) &= \frac{1}{2} \cdot \left(-\frac{1}{2}\right) \cdot \left(-\frac{3}{2}\right) = \frac{3}{8} \\
-2\alpha + 1 &= -1 + 1 = 0 \quad \checkmark
\end{align}

The equation becomes:
\begin{equation}
x^{1/2} \left[\frac{3}{8} x^2\right] = 0
\end{equation}
which is satisfied as $x \to 0^+$.

\vspace{1em}
\noindent\textbf{Step 3: Verify there are other solutions.}

\noindent\textit{Why check?} A third-order ODE should have three linearly independent solutions. We've found one power law solution; there may be others.

The controlling factor ansatz $y = e^{S(x)}$ would give:
\begin{equation}
x^5(S''' + 3S'S'' + (S')^3) - 2xS' + 1 = 0
\end{equation}

\noindent\textbf{Dominant balance:} Assume $S \sim Cx^\beta$:
\begin{align}
x^5(S')^3 &\sim x^{3\beta+2} \\
-2xS' &\sim x^{\beta-1} \\
1 &\sim x^0
\end{align}

\noindent\textbf{Try:} $-2xS' \sim 1$, i.e., $\beta - 1 = 0 \Rightarrow \beta = 1$.

With $\beta = 1$:
\begin{align}
x^5(S')^3 &\sim x^{5} \to 0 \\
-2xS' &\sim x^{0} \sim 1
\end{align}

This gives $-2xC + 1 \sim 0$, which means $C \sim \frac{1}{2x}$, not a constant. This suggests $S \sim \log x$, but that's not of the form $Cx^\beta$.

\noindent\textit{Conclusion:} The dominant behavior is given by the power law solution.

\vspace{0.5em}
\noindent\textbf{Final answer:}
\begin{equation}
\boxed{y(x) \sim x^{1/2} \quad \text{as } x \to 0^+}
\end{equation}

\noindent\textit{Additional solutions:} There are likely other solutions involving $x^{1/2} \log x$ or exponential factors, but the leading power law behavior is $y \sim \sqrt{x}$.

\end{document}
