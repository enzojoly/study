\documentclass[11pt,a4paper]{article}
\usepackage{amsmath,amssymb,amsthm}
\usepackage{geometry}
\geometry{margin=1in}
\usepackage{enumitem}
\usepackage{xcolor}

\newtheorem{theorem}{Theorem}
\newtheorem{lemma}{Lemma}

\title{Asymptotics 2025/2026 -- Problem Sheet 3\\
\large Question 3: Watson's Lemma Application}
\author{Solution}
\date{}

\begin{document}

\maketitle

\section*{Question 3}

\textbf{Problem:} Use Watson's lemma to find an infinite asymptotic expansion of
\[
I(X) = \int_1^\infty e^{-X(t^2+1)} \, dt.
\]

\subsection*{Solution Strategy}

\textbf{Why this approach?} Watson's lemma (Section 4.2.2 of lecture notes) applies to integrals of the form
\[
\int_0^b f(t) e^{-Xt} \, dt
\]
where $f(t)$ has an asymptotic expansion near $t = 0$. Our integral has limits $[1,\infty)$ and exponent $-X(t^2+1)$, so we must transform it into Watson's lemma standard form.

\subsection*{Step 1: Factor the Exponential}

\textbf{What we have:}
\[
I(X) = \int_1^\infty e^{-X(t^2+1)} \, dt
\]

\textbf{What we do:} Factor $e^{-X(t^2+1)} = e^{-X} \cdot e^{-Xt^2}$:
\[
I(X) = e^{-X} \int_1^\infty e^{-Xt^2} \, dt
\]

\textbf{Why?} The factor $e^{-X}$ is independent of the integration variable $t$. Factoring it out isolates the $t$-dependent exponential $e^{-Xt^2}$, which we'll address with a substitution. This prepares us to transform the integral into Watson's standard form.

\subsection*{Step 2: Change of Variable}

\textbf{What we have:}
\[
I(X) = e^{-X} \int_1^\infty e^{-Xt^2} \, dt
\]

\textbf{What we do:} Substitute $s = t^2 - 1$, so $t^2 = s + 1$.

\textbf{Why this substitution?} We need the integral to start at $0$ (Watson's lemma requirement) and the exponent to be linear in the new variable. Since the lower limit is $t=1$, setting $s = t^2 - 1$ gives $s=0$ when $t=1$.

\textbf{Computing the differential:}
\begin{align*}
s &= t^2 - 1 \\
ds &= 2t \, dt \\
dt &= \frac{ds}{2t} = \frac{ds}{2\sqrt{s+1}}
\end{align*}

\textbf{Why this form?} From $s = t^2 - 1$, we have $t = \sqrt{s+1}$ (taking positive root since $t \geq 1$). Thus $dt = ds/(2\sqrt{s+1})$.

\textbf{Transforming limits:}
\begin{itemize}
\item When $t = 1$: $s = 1^2 - 1 = 0$
\item When $t \to \infty$: $s \to \infty$
\end{itemize}

\textbf{Why check limits?} Watson's lemma requires integration from $0$ to some positive limit. We verify the transformation achieves this.

\subsection*{Step 3: Rewrite the Integral}

\textbf{What we have after substitution:}
\[
I(X) = e^{-X} \int_0^\infty e^{-X(s+1)} \cdot \frac{1}{2\sqrt{s+1}} \, ds
\]

\textbf{What we do:} Factor the exponential:
\[
I(X) = e^{-X} \cdot e^{-X} \int_0^\infty e^{-Xs} \cdot \frac{1}{2\sqrt{s+1}} \, ds
\]
\[
I(X) = e^{-2X} \int_0^\infty e^{-Xs} \cdot \frac{1}{2\sqrt{s+1}} \, ds
\]

\textbf{Why?} We factor $e^{-X(s+1)} = e^{-Xs} \cdot e^{-X}$ to isolate $e^{-Xs}$, which is the required exponential form for Watson's lemma. The factor $e^{-2X}$ (combining both $e^{-X}$ terms) is pulled outside the integral.

\textbf{Identifying Watson's lemma components:}
\[
I(X) = e^{-2X} \int_0^\infty f(s) e^{-Xs} \, ds
\]
where
\[
f(s) = \frac{1}{2\sqrt{s+1}} = \frac{1}{2}(1+s)^{-1/2}
\]

\textbf{Why identify $f(s)$?} Watson's lemma requires $f(s)$ to have an asymptotic expansion as $s \to 0^+$. We've now achieved the standard form.

\subsection*{Step 4: Asymptotic Expansion of $f(s)$}

\textbf{What we need:} An expansion of $f(s) = \frac{1}{2}(1+s)^{-1/2}$ as $s \to 0^+$.

\textbf{What we do:} Use the binomial series:
\[
(1+s)^{-1/2} = \sum_{n=0}^\infty \binom{-1/2}{n} s^n
\]

\textbf{Why the binomial series?} For $|s| < 1$, the generalized binomial theorem gives:
\[
(1+s)^\alpha = \sum_{n=0}^\infty \binom{\alpha}{n} s^n
\]
where $\binom{\alpha}{n} = \frac{\alpha(\alpha-1)\cdots(\alpha-n+1)}{n!}$.

\textbf{Computing binomial coefficients:}
\[
\binom{-1/2}{n} = \frac{(-1/2)(-1/2-1)(-1/2-2)\cdots(-1/2-n+1)}{n!}
\]
\[
= \frac{(-1/2)(-3/2)(-5/2)\cdots(-(2n-1)/2)}{n!}
\]
\[
= \frac{(-1)^n \cdot 1 \cdot 3 \cdot 5 \cdots (2n-1)}{2^n \cdot n!}
\]

\textbf{Why this form?} Each factor in the numerator is $(-1/2 - k) = -(2k+1)/2$ for $k=0,1,\ldots,n-1$, giving us odd numbers with alternating sign.

\textbf{Expressing with Gamma functions:}
\[
\binom{-1/2}{n} = \frac{(-1)^n}{2^n} \cdot \frac{1 \cdot 3 \cdot 5 \cdots (2n-1)}{n!}
\]

The product of odd numbers can be written as:
\[
1 \cdot 3 \cdot 5 \cdots (2n-1) = \frac{(2n)!}{2^n \cdot n!}
\]

\textbf{Why?} Because $(2n)! = 1 \cdot 2 \cdot 3 \cdots (2n) = [1 \cdot 3 \cdot 5 \cdots (2n-1)] \cdot [2 \cdot 4 \cdot 6 \cdots (2n)]$ and $2 \cdot 4 \cdots (2n) = 2^n \cdot n!$.

Therefore:
\[
\binom{-1/2}{n} = \frac{(-1)^n}{2^n} \cdot \frac{(2n)!}{2^n \cdot n! \cdot n!} = \frac{(-1)^n (2n)!}{2^{2n} (n!)^2}
\]

\textbf{Using Gamma function notation:}
\[
\binom{-1/2}{n} = \frac{(-1)^n}{\sqrt{\pi}} \cdot \frac{\Gamma(n+1/2)}{n!}
\]

\textbf{Why Gamma functions?} The identity $\Gamma(n+1/2) = \frac{(2n)!\sqrt{\pi}}{2^{2n} n!}$ (from lecture notes example 2.6.2) simplifies our coefficients.

\subsection*{Step 5: Apply Watson's Lemma}

\textbf{What we have:}
\[
f(s) = \frac{1}{2}\sum_{n=0}^\infty \frac{(-1)^n}{\sqrt{\pi}} \cdot \frac{\Gamma(n+1/2)}{n!} s^n
\]

\textbf{Identifying Watson's lemma parameters:}
\begin{itemize}
\item $\alpha = 0$ (no leading power of $s$)
\item $\beta = 1$ (integer powers)
\item $a_n = \frac{1}{2} \cdot \frac{(-1)^n}{\sqrt{\pi}} \cdot \frac{\Gamma(n+1/2)}{n!}$
\end{itemize}

\textbf{Why these values?} Watson's lemma (Eq. 177 in notes) assumes $f(t) \sim t^\alpha \sum_{n=0}^\infty a_n t^{n\beta}$. Our expansion has no leading power ($\alpha=0$) and integer increments ($\beta=1$).

\textbf{Applying Watson's lemma formula (Eq. 177):}
\[
\int_0^b f(s) e^{-Xs} ds \sim \sum_{n=0}^\infty a_n \frac{\Gamma(\alpha + n\beta + 1)}{X^{\alpha + n\beta + 1}}
\]

With our values ($\alpha=0$, $\beta=1$):
\[
\int_0^\infty f(s) e^{-Xs} ds \sim \sum_{n=0}^\infty a_n \frac{\Gamma(n+1)}{X^{n+1}} = \sum_{n=0}^\infty a_n \frac{n!}{X^{n+1}}
\]

\textbf{Why $\Gamma(n+1) = n!$?} This is a fundamental property of the Gamma function (Eq. 69 in notes).

\subsection*{Step 6: Substitute Coefficients}

\textbf{What we do:}
\[
\int_0^\infty f(s) e^{-Xs} ds \sim \sum_{n=0}^\infty \frac{1}{2\sqrt{\pi}} \cdot \frac{(-1)^n \Gamma(n+1/2)}{n!} \cdot \frac{n!}{X^{n+1}}
\]

\textbf{Simplifying:}
\[
= \frac{1}{2\sqrt{\pi}} \sum_{n=0}^\infty \frac{(-1)^n \Gamma(n+1/2)}{X^{n+1}}
\]

\textbf{Why does $n!$ cancel?} The $n!$ in the denominator of $a_n$ cancels with the $n! = \Gamma(n+1)$ from Watson's lemma formula.

\subsection*{Step 7: Final Result}

\textbf{Recalling the full expression:}
\[
I(X) = e^{-2X} \int_0^\infty f(s) e^{-Xs} ds
\]

\textbf{Therefore:}
\[
\boxed{I(X) \sim \frac{e^{-2X}}{2\sqrt{\pi}} \sum_{n=0}^\infty \frac{(-1)^n \Gamma(n+1/2)}{X^{n+1}} \quad \text{as } X \to \infty}
\]

\textbf{Alternative form with explicit first terms:}

Computing $\Gamma(n+1/2)$ for first few terms:
\begin{align*}
\Gamma(1/2) &= \sqrt{\pi} \\
\Gamma(3/2) &= \frac{1}{2}\sqrt{\pi} \\
\Gamma(5/2) &= \frac{3}{4}\sqrt{\pi} \\
\Gamma(7/2) &= \frac{15}{8}\sqrt{\pi}
\end{align*}

\textbf{Why these values?} Using $\Gamma(z+1) = z\Gamma(z)$ recursively: $\Gamma(3/2) = \frac{1}{2}\Gamma(1/2) = \frac{\sqrt{\pi}}{2}$, etc.

\[
\boxed{I(X) \sim \frac{e^{-2X}}{2\sqrt{\pi}} \left[ \frac{\sqrt{\pi}}{X} - \frac{\sqrt{\pi}/2}{X^2} + \frac{3\sqrt{\pi}/4}{X^3} - \frac{15\sqrt{\pi}/8}{X^4} + \cdots \right]}
\]

\[
\boxed{I(X) \sim \frac{e^{-2X}}{2} \left[ \frac{1}{X} - \frac{1}{2X^2} + \frac{3}{4X^3} - \frac{15}{8X^4} + \cdots \right] \quad \text{as } X \to \infty}
\]

\subsection*{Verification of Conditions}

\textbf{Why is Watson's lemma applicable?}

\begin{enumerate}
\item \textbf{Convergence at origin:} $f(s) = \frac{1}{2\sqrt{1+s}}$ behaves as $s^{-1/2} \cdot s^{1/2} = O(1)$ near $s=0$, so the integral converges (since $\alpha = 0 > -1$).

\item \textbf{Convergence at infinity:} The integrand $e^{-Xs}/\sqrt{1+s}$ decays exponentially as $s \to \infty$ for $X > 0$, ensuring convergence.

\item \textbf{Asymptotic sequence:} $\{X^{-(n+1)}\}$ is an asymptotic sequence as $X \to \infty$ since $X^{-(n+2)}/X^{-(n+1)} = 1/X \to 0$.
\end{enumerate}

\end{document}
