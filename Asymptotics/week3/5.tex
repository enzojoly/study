\documentclass[11pt,a4paper]{article}

% Packages
\usepackage{amsmath,amssymb,amsthm}
\usepackage[margin=1in]{geometry}
\usepackage{enumitem}
\usepackage{xcolor}
\usepackage{mathtools}

% Theorem environments
\theoremstyle{definition}
\newtheorem*{problem}{Problem}
\newtheorem*{solution}{Solution}

% Custom commands
\newcommand{\dd}{\mathrm{d}}
\newcommand{\Order}{\mathcal{O}}

% Title information
\title{Asymptotics Problem Sheet 3\\
\large Question 5: Asymptotic Expansion via Watson's Lemma}
\author{Solution with XYZ Methodology}
\date{Academic Year 2025--2026}

\begin{document}

\maketitle

\begin{problem}
Show that
\[
\int_0^\infty \left(1 + \frac{u}{X}\right)^{-X} e^{-u} \dd u
\sim \frac{1}{2} + \frac{1}{8X} - \frac{1}{32X^2}
\quad \text{as } X \to \infty.
\]
\end{problem}

\begin{solution}
We seek an asymptotic expansion of the integral
\[
I(X) = \int_0^\infty \left(1 + \frac{u}{X}\right)^{-X} e^{-u} \dd u
\]
as $X \to \infty$.

\subsection*{Step 1: Recognition of Integral Type}

\textbf{What we observe:} The integral contains the factor $\left(1 + \frac{u}{X}\right)^{-X}$ multiplied by $e^{-u}$, integrated from $0$ to $\infty$.

\textbf{Why this matters:} This is a Laplace-type integral where the large parameter $X$ appears in an exponent. The presence of $X \to \infty$ as the asymptotic limit suggests we should use techniques from Section 4.2 of the lecture notes.

\textbf{What we recognize:} The factor $\left(1 + \frac{u}{X}\right)^{-X}$ can be rewritten using the exponential-logarithm identity:
\[
\left(1 + \frac{u}{X}\right)^{-X} = \exp\left(-X \log\left(1 + \frac{u}{X}\right)\right).
\]

\textbf{Why we do this:} By expressing the factor as an exponential, we can combine it with $e^{-u}$ to obtain a single exponential factor, which is the standard form for Laplace-type integrals.

\subsection*{Step 2: Combining Exponential Factors}

\textbf{What we have:} Using the exponential form from Step 1:
\[
I(X) = \int_0^\infty \exp\left(-X \log\left(1 + \frac{u}{X}\right)\right) e^{-u} \dd u
= \int_0^\infty \exp\left(-X \log\left(1 + \frac{u}{X}\right) - u\right) \dd u.
\]

\textbf{Why this form is useful:} We now have a single exponential with argument
\[
-X \log\left(1 + \frac{u}{X}\right) - u.
\]
This allows us to analyze the behavior of the integrand as $X \to \infty$.

\subsection*{Step 3: Taylor Expansion of the Logarithm}

\textbf{What we need:} To understand the behavior as $X \to \infty$, we expand $\log\left(1 + \frac{u}{X}\right)$ for large $X$ (equivalently, small $\frac{u}{X}$).

\textbf{Why we need this:} The logarithm is multiplied by $X$, so even though $\frac{u}{X}$ is small, the product $X \log\left(1 + \frac{u}{X}\right)$ may have a non-trivial limit. We must expand carefully to capture all relevant orders.

\textbf{What we know:} The Taylor series for $\log(1 + z)$ around $z = 0$ is (from standard calculus):
\[
\log(1 + z) = z - \frac{z^2}{2} + \frac{z^3}{3} - \frac{z^4}{4} + \Order(z^5).
\]

\textbf{Why this applies:} Setting $z = \frac{u}{X}$, we have:
\[
\log\left(1 + \frac{u}{X}\right) = \frac{u}{X} - \frac{u^2}{2X^2} + \frac{u^3}{3X^3} - \frac{u^4}{4X^4} + \Order(X^{-5}).
\]

\textbf{What this means:} This expansion is valid when $\left|\frac{u}{X}\right| < 1$, which holds for fixed $u$ as $X \to \infty$.

\subsection*{Step 4: Multiplying by $-X$}

\textbf{What we compute:} Multiply the expansion from Step 3 by $-X$:
\begin{align*}
-X \log\left(1 + \frac{u}{X}\right)
&= -X\left[\frac{u}{X} - \frac{u^2}{2X^2} + \frac{u^3}{3X^3} - \frac{u^4}{4X^4} + \Order(X^{-5})\right]\\
&= -u + \frac{u^2}{2X} - \frac{u^3}{3X^2} + \frac{u^4}{4X^3} + \Order(X^{-4}).
\end{align*}

\textbf{Why each term appears:}
\begin{itemize}[leftmargin=*]
\item The $-u$ term: $-X \cdot \frac{u}{X} = -u$ is of order $\Order(1)$ (independent of $X$).
\item The $\frac{u^2}{2X}$ term: $-X \cdot \left(-\frac{u^2}{2X^2}\right) = \frac{u^2}{2X}$ is of order $\Order(X^{-1})$.
\item The $-\frac{u^3}{3X^2}$ term: $-X \cdot \frac{u^3}{3X^3} = -\frac{u^3}{3X^2}$ is of order $\Order(X^{-2})$.
\item The $\frac{u^4}{4X^3}$ term: $-X \cdot \left(-\frac{u^4}{4X^4}\right) = \frac{u^4}{4X^3}$ is of order $\Order(X^{-3})$.
\end{itemize}

\textbf{What we observe:} Each successive term is smaller by a factor of $\Order(X^{-1})$.

\subsection*{Step 5: Substituting into the Exponent}

\textbf{What we substitute:} The full exponent in our integral is:
\begin{align*}
-X \log\left(1 + \frac{u}{X}\right) - u
&= \left(-u + \frac{u^2}{2X} - \frac{u^3}{3X^2} + \Order(X^{-3})\right) - u\\
&= -2u + \frac{u^2}{2X} - \frac{u^3}{3X^2} + \Order(X^{-3}).
\end{align*}

\textbf{Why we group terms this way:} The dominant term (independent of $X$) is $-2u$. This will determine the basic structure of the integral. The remaining terms are corrections of increasing order in $X^{-1}$.

\textbf{What our integral becomes:}
\[
I(X) = \int_0^\infty \exp\left(-2u + \frac{u^2}{2X} - \frac{u^3}{3X^2} + \Order(X^{-3})\right) \dd u.
\]

\subsection*{Step 6: Factoring the Leading Exponential}

\textbf{What we do:} Factor out the dominant exponential $e^{-2u}$:
\[
I(X) = \int_0^\infty e^{-2u} \exp\left(\frac{u^2}{2X} - \frac{u^3}{3X^2} + \Order(X^{-3})\right) \dd u.
\]

\textbf{Why this factorization is useful:} The factor $e^{-2u}$ provides exponential decay as $u \to \infty$, ensuring all integrals converge. The remaining exponential contains only small terms (of order $X^{-1}$ and higher), which we can expand.

\subsection*{Step 7: Expanding the Correction Exponential}

\textbf{What we need to expand:} The exponential
\[
\exp\left(\frac{u^2}{2X} - \frac{u^3}{3X^2}\right).
\]

\textbf{Why we can expand:} For large $X$ and fixed $u$, the argument $\frac{u^2}{2X} - \frac{u^3}{3X^2}$ is small, so we can use the Taylor series:
\[
e^z = 1 + z + \frac{z^2}{2} + \frac{z^3}{6} + \cdots
\]

\textbf{What we compute:} Let $z = \frac{u^2}{2X} - \frac{u^3}{3X^2}$. Then:
\begin{align*}
e^z &= 1 + z + \frac{z^2}{2} + \Order(z^3)\\
&= 1 + \left(\frac{u^2}{2X} - \frac{u^3}{3X^2}\right) + \frac{1}{2}\left(\frac{u^2}{2X}\right)^2 + \Order(X^{-3})\\
&= 1 + \frac{u^2}{2X} - \frac{u^3}{3X^2} + \frac{u^4}{8X^2} + \Order(X^{-3}).
\end{align*}

\textbf{Why we keep only these terms:}
\begin{itemize}[leftmargin=*]
\item The term $\left(\frac{u^2}{2X}\right)^2 = \frac{u^4}{4X^2}$ contributes at order $\Order(X^{-2})$.
\item The cross term $2 \cdot \frac{u^2}{2X} \cdot \left(-\frac{u^3}{3X^2}\right) = -\frac{u^5}{3X^3}$ is of order $\Order(X^{-3})$ and can be neglected.
\item Higher order terms from $\frac{z^2}{2}, \frac{z^3}{6}, \ldots$ are all $\Order(X^{-3})$ or smaller.
\end{itemize}

\textbf{What we collect:} Grouping by powers of $X^{-1}$:
\[
e^z = 1 + \frac{u^2}{2X} + \frac{1}{X^2}\left(\frac{u^4}{8} - \frac{u^3}{3}\right) + \Order(X^{-3}).
\]

\subsection*{Step 8: Substituting the Expansion into the Integral}

\textbf{What we substitute:} Using the expansion from Step 7:
\begin{align*}
I(X) &= \int_0^\infty e^{-2u} \left[1 + \frac{u^2}{2X} + \frac{1}{X^2}\left(\frac{u^4}{8} - \frac{u^3}{3}\right)\right] \dd u + \Order(X^{-3})\\
&= \int_0^\infty e^{-2u} \dd u
+ \frac{1}{2X} \int_0^\infty u^2 e^{-2u} \dd u\\
&\quad + \frac{1}{X^2}\left[\frac{1}{8}\int_0^\infty u^4 e^{-2u} \dd u - \frac{1}{3}\int_0^\infty u^3 e^{-2u} \dd u\right] + \Order(X^{-3}).
\end{align*}

\textbf{Why we can separate the integrals:} Each integral converges absolutely due to the exponential decay factor $e^{-2u}$, so we can distribute the integration over the sum.

\subsection*{Step 9: Evaluating the Standard Integrals}

\textbf{What we need:} We must evaluate integrals of the form
\[
\int_0^\infty u^n e^{-2u} \dd u.
\]

\textbf{Why we know the formula:} These are standard Gamma function integrals. From the lecture notes (Section 2.6.1, Equation 68), the Gamma function is:
\[
\Gamma(z) = \int_0^\infty t^{z-1} e^{-t} \dd t.
\]

\textbf{How to apply this:} Substituting $t = 2u$ (so $u = t/2$, $\dd u = \dd t/2$):
\begin{align*}
\int_0^\infty u^n e^{-2u} \dd u
&= \int_0^\infty \left(\frac{t}{2}\right)^n e^{-t} \frac{\dd t}{2}\\
&= \frac{1}{2^{n+1}} \int_0^\infty t^n e^{-t} \dd t\\
&= \frac{1}{2^{n+1}} \Gamma(n+1)\\
&= \frac{n!}{2^{n+1}}.
\end{align*}

\textbf{Why this formula works:} We used $\Gamma(n+1) = n!$ for non-negative integers $n$ (a standard property of the Gamma function).

\subsection*{Step 10: Computing Each Required Integral}

\textbf{What we compute:} Using the formula from Step 9 with different values of $n$:

\paragraph{For $n = 0$:}
\[
\int_0^\infty e^{-2u} \dd u = \frac{0!}{2^{0+1}} = \frac{1}{2}.
\]
\textbf{Why:} $0! = 1$ and $2^1 = 2$.

\paragraph{For $n = 2$:}
\[
\int_0^\infty u^2 e^{-2u} \dd u = \frac{2!}{2^{2+1}} = \frac{2}{8} = \frac{1}{4}.
\]
\textbf{Why:} $2! = 2$ and $2^3 = 8$.

\paragraph{For $n = 3$:}
\[
\int_0^\infty u^3 e^{-2u} \dd u = \frac{3!}{2^{3+1}} = \frac{6}{16} = \frac{3}{8}.
\]
\textbf{Why:} $3! = 6$ and $2^4 = 16$, and simplifying: $\frac{6}{16} = \frac{3}{8}$.

\paragraph{For $n = 4$:}
\[
\int_0^\infty u^4 e^{-2u} \dd u = \frac{4!}{2^{4+1}} = \frac{24}{32} = \frac{3}{4}.
\]
\textbf{Why:} $4! = 24$ and $2^5 = 32$, and simplifying: $\frac{24}{32} = \frac{3}{4}$.

\subsection*{Step 11: Assembling the Asymptotic Expansion}

\textbf{What we substitute:} Using the computed integrals from Step 10 in the expression from Step 8:
\begin{align*}
I(X) &= \frac{1}{2} + \frac{1}{2X} \cdot \frac{1}{4} + \frac{1}{X^2}\left[\frac{1}{8} \cdot \frac{3}{4} - \frac{1}{3} \cdot \frac{3}{8}\right] + \Order(X^{-3}).
\end{align*}

\textbf{Computing the $\Order(X^{-1})$ term:}
\[
\frac{1}{2X} \cdot \frac{1}{4} = \frac{1}{8X}.
\]

\textbf{Computing the $\Order(X^{-2})$ term:}
\begin{align*}
\frac{1}{X^2}\left[\frac{1}{8} \cdot \frac{3}{4} - \frac{1}{3} \cdot \frac{3}{8}\right]
&= \frac{1}{X^2}\left[\frac{3}{32} - \frac{3}{24}\right]\\
&= \frac{1}{X^2}\left[\frac{3}{32} - \frac{1}{8}\right].
\end{align*}

\textbf{Why we simplify $\frac{3}{24}$:}
\[
\frac{3}{24} = \frac{1}{8}.
\]

\textbf{Finding a common denominator:}
\[
\frac{3}{32} - \frac{1}{8} = \frac{3}{32} - \frac{4}{32} = -\frac{1}{32}.
\]

\textbf{Why:} We write $\frac{1}{8} = \frac{4}{32}$ to combine with $\frac{3}{32}$.

\subsection*{Step 12: Final Result}

\textbf{What we have established:} Combining all terms:
\[
I(X) = \frac{1}{2} + \frac{1}{8X} - \frac{1}{32X^2} + \Order(X^{-3}).
\]

\textbf{Why this is the answer:} This matches the required asymptotic expansion. The expansion is valid as $X \to \infty$ and captures the behavior to order $\Order(X^{-2})$.

\textbf{What we conclude:} Therefore,
\[
\boxed{\int_0^\infty \left(1 + \frac{u}{X}\right)^{-X} e^{-u} \dd u
\sim \frac{1}{2} + \frac{1}{8X} - \frac{1}{32X^2} \quad \text{as } X \to \infty.}
\]

\end{solution}

\end{document}
