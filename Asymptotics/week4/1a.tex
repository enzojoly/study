\documentclass[11pt,a4paper]{article}
\usepackage[margin=1in]{geometry}
\usepackage{amsmath,amssymb,amsthm}
\usepackage{mathtools}
\usepackage{enumitem}
\usepackage{xcolor}

\newcommand{\stage}[1]{\textbf{\textcolor{blue}{#1}}}

\title{Question 1(a): Asymptotic Analysis\\
Leading Order Behavior of $\int_X^\infty e^{-t^3} dt$}
\author{Asymptotics Course — Sheet 4}
\date{}

\begin{document}

\maketitle

\section*{Problem Statement}
Find the leading order asymptotic behavior as $X \to \infty$ of:
\[
I(X) = \int_X^\infty e^{-t^3} dt
\]

\section{Step 1: Problem Classification and Strategy}

\subsection*{Form Recognition}

The integral has the form:
\[
I(X) = \int_X^\infty e^{-t^3} dt
\]

\begin{itemize}[leftmargin=*]
\item \stage{STAGE X (What we have):} An integral with exponential decay $e^{-t^3}$ over $[X, \infty)$ where the decay rate accelerates rapidly as $t$ increases.

\item \stage{STAGE Y (Why this approach):} As $X \to \infty$, the exponential $e^{-t^3}$ decays extremely rapidly. The main contribution to the integral comes from values of $t$ near the lower limit $X$. This suggests using \textbf{integration by parts} (Section 4.1 of lecture notes).

\item \stage{STAGE Z (What this means):} The integral is exponentially small, and we seek its leading order behavior by extracting contributions from the boundary at $t = X$.
\end{itemize}

\section{Step 2: Variable Transformation}

To apply standard techniques, we first transform the integral using substitution.

\subsection*{Substitution}

Set:
\[
u = t^3 \quad \Rightarrow \quad du = 3t^2 dt \quad \Rightarrow \quad dt = \frac{du}{3t^2}
\]

Since $t = u^{1/3}$, we have $t^2 = u^{2/3}$, therefore:
\[
dt = \frac{du}{3u^{2/3}}
\]

\subsection*{Transformed Limits}
\begin{align*}
t = X &\quad \Rightarrow \quad u = X^3 \\
t = \infty &\quad \Rightarrow \quad u = \infty
\end{align*}

\subsection*{Transformed Integral}

\[
I(X) = \int_{X^3}^\infty e^{-u} \cdot \frac{1}{3u^{2/3}} \, du
\]

\begin{itemize}[leftmargin=*]
\item \stage{STAGE X (What we did):} Transformed to standard exponential form with $e^{-u}$ and identified the prefactor $f(u) = \frac{1}{3u^{2/3}}$.

\item \stage{STAGE Y (Why this helps):} The integral now has the form $\int_{a}^\infty f(u)e^{-u}du$ which is amenable to integration by parts as described in Section 4.1, Equation (165).

\item \stage{STAGE Z (Next step):} Apply integration by parts to extract the leading order contribution from the lower limit.
\end{itemize}

\section{Step 3: Integration by Parts}

\subsection*{General Formula (Section 4.1)}

For an integral of the form $\int_a^\infty f(u)e^{-u}du$, we write:
\[
e^{-u} = \frac{d}{du}(-e^{-u})
\]

Integration by parts gives:
\begin{align*}
\int_a^\infty f(u)e^{-u}du &= \left[-f(u)e^{-u}\right]_a^\infty + \int_a^\infty f'(u)e^{-u}du \\
&= f(a)e^{-a} + \int_a^\infty f'(u)e^{-u}du
\end{align*}

\subsection*{Application to Our Integral}

With $f(u) = \frac{1}{3u^{2/3}}$ and $a = X^3$:

\[
I(X) = \frac{1}{3(X^3)^{2/3}} e^{-X^3} + \int_{X^3}^\infty \frac{d}{du}\left(\frac{1}{3u^{2/3}}\right) e^{-u} du
\]

Simplify the first term:
\[
\frac{1}{3(X^3)^{2/3}} = \frac{1}{3X^2}
\]

Therefore:
\[
I(X) = \frac{e^{-X^3}}{3X^2} + \int_{X^3}^\infty \frac{d}{du}\left(\frac{1}{3u^{2/3}}\right) e^{-u} du
\]

\section{Step 4: Compute the Derivative}

\subsection*{Derivative Calculation}

\[
\frac{d}{du}\left(\frac{1}{3u^{2/3}}\right) = \frac{1}{3} \cdot \left(-\frac{2}{3}\right) u^{-5/3} = -\frac{2}{9}u^{-5/3}
\]

\subsection*{Updated Expression}

\[
I(X) = \frac{e^{-X^3}}{3X^2} - \frac{2}{9}\int_{X^3}^\infty u^{-5/3}e^{-u} du
\]

\section{Step 5: Asymptotic Order Analysis}

\subsection*{Order of the Remainder Term}

The remainder integral satisfies:
\[
\left|\int_{X^3}^\infty u^{-5/3}e^{-u} du\right| < (X^3)^{-5/3} \int_{X^3}^\infty e^{-u} du = X^{-5} e^{-X^3}
\]

\subsection*{Comparison of Terms}

\begin{itemize}[leftmargin=*]
\item Leading term: $\frac{e^{-X^3}}{3X^2} = O(X^{-2}e^{-X^3})$ as $X \to \infty$

\item Remainder term: $O(X^{-5}e^{-X^3})$ as $X \to \infty$
\end{itemize}

\subsection*{Dominance Relation}

\[
\frac{X^{-5}e^{-X^3}}{X^{-2}e^{-X^3}} = X^{-3} \to 0 \text{ as } X \to \infty
\]

\begin{itemize}[leftmargin=*]
\item \stage{STAGE X (What we found):} The remainder term is of order $X^{-5}e^{-X^3}$, which is three powers of $X$ smaller than the leading term.

\item \stage{STAGE Y (Why this matters):} The ratio $X^{-3} \to 0$ confirms the remainder is asymptotically negligible compared to the first term. The leading order behavior is completely determined by the boundary contribution at $u = X^3$ (equivalently $t = X$).

\item \stage{STAGE Z (Conclusion):} We can neglect the remainder term in the asymptotic limit.
\end{itemize}

\section{Step 6: Final Answer}

Using the proper asymptotic notation (Definition, page 8 of notes):

\begin{center}
\fbox{%
\parbox{0.85\textwidth}{%
\textbf{Leading Order Asymptotic:}
\[
I(X) = \int_X^\infty e^{-t^3} dt \sim \frac{e^{-X^3}}{3X^2} \quad \text{as } X \to \infty
\]
}}
\end{center}

\subsection*{Alternative Notation}

This can also be written as:
\[
I(X) = \frac{e^{-X^3}}{3X^2}\left[1 + O\left(\frac{1}{X^3}\right)\right] \quad \text{as } X \to \infty
\]

\section{Verification Checklist}

\textit{Following the standards of Section 4.1-4.2:}

\begin{enumerate}[leftmargin=*]
\item[$\checkmark$] \textbf{Method identified:} Integration by parts (Section 4.1)
\item[$\checkmark$] \textbf{Transformation applied:} Substitution $u = t^3$ to standard form
\item[$\checkmark$] \textbf{Leading term extracted:} Boundary contribution at lower limit
\item[$\checkmark$] \textbf{Remainder estimated:} Shown to be $O(X^{-5}e^{-X^3})$
\item[$\checkmark$] \textbf{Dominance verified:} Ratio $X^{-3} \to 0$ confirms asymptotic ordering
\item[$\checkmark$] \textbf{Proper notation:} Used $\sim$ for asymptotic equivalence
\end{enumerate}

\subsection*{Physical Interpretation}

\begin{itemize}[leftmargin=*]
\item \stage{STAGE X (Behavior):} The integral decays faster than any polynomial as $X \to \infty$ due to the $e^{-X^3}$ factor.

\item \stage{STAGE Y (Mechanism):} The rapid exponential decay means only a thin layer of width $\sim O(X^{-2})$ near $t = X$ contributes significantly to the integral.

\item \stage{STAGE Z (Result):} The leading order is determined entirely by the local behavior at the lower integration limit, with the asymptotic form $\frac{e^{-X^3}}{3X^2}$.
\end{itemize}

\end{document}
