\documentclass[11pt,a4paper]{article}
\usepackage[margin=2.5cm]{geometry}
\usepackage{amsmath,amssymb,amsthm}
\usepackage{mathtools}
\usepackage{enumitem}

\newcommand{\dd}{\mathrm{d}}

\title{Asymptotics 2025/2026 \\ Solution to Problem Sheet 5, Question 1(b)}
\author{}
\date{}

\begin{document}
\maketitle

\section*{Question 1(b)}
Use the method of stationary phase to obtain the leading asymptotic behaviour of
\[
I(X) = \int_{1/2}^{2} (1+t) \, e^{iX(-t+t^3/3)} \dd t \quad \text{as } X \to \infty.
\]

\section*{Solution}

We have a Fourier-type integral of the form
\[
I(X) = \int_a^b f(t) \, e^{iX\phi(t)} \dd t
\]
where $f(t) = 1 + t$, $\phi(t) = -t + \frac{t^3}{3}$, and the integration interval is $[a,b] = [1/2, 2]$.

\subsection*{Step 1: Identify stationary points}

Following the method of stationary phase (Section 4.3 of the lecture notes), we look for points where $\phi'(t) = 0$ within the integration interval.

Compute the derivative:
\[
\phi'(t) = -1 + t^2.
\]

Setting $\phi'(t) = 0$:
\[
-1 + t^2 = 0 \implies t^2 = 1 \implies t = \pm 1.
\]

Since the integration interval is $[1/2, 2]$, we have:
\begin{itemize}
\item $t = 1$ lies in $[1/2, 2]$ $\checkmark$
\item $t = -1$ lies outside $[1/2, 2]$ $\times$
\end{itemize}

Therefore, there is exactly one stationary point at $t = 1$ inside the integration interval.

\subsection*{Step 2: Classify the stationary point}

Compute the second derivative:
\[
\phi''(t) = 2t.
\]

At $t = 1$:
\[
\phi''(1) = 2 \neq 0.
\]

Since $\phi''(1) \neq 0$, this is a non-degenerate stationary point (i.e., $n = 2$ in the terminology of the notes).

\subsection*{Step 3: Evaluate required quantities at the stationary point}

At $t = 1$:
\begin{align*}
f(1) &= 1 + 1 = 2, \\
\phi(1) &= -1 + \frac{1^3}{3} = -1 + \frac{1}{3} = -\frac{2}{3}, \\
\phi''(1) &= 2.
\end{align*}

\subsection*{Step 4: Check contributions from endpoints}

According to the notes (Section 4.3.1), contributions from endpoints are algebraically smaller than contributions from interior stationary points. Specifically, endpoint contributions are $O(X^{-1})$ while stationary point contributions are $O(X^{-1/2})$.

Since $X^{-1} = o(X^{-1/2})$ as $X \to \infty$, the endpoint contributions are subdominant and can be neglected in the leading order asymptotic expansion.

\subsection*{Step 5: Apply the method of stationary phase formula}

From the lecture notes, Equation (235), the asymptotic expansion for a Fourier-type integral with a non-degenerate stationary point at $t = c$ is:
\[
I(X) \sim \sqrt{\frac{2\pi i}{X\phi''(c)}} f(c) e^{iX\phi(c)} \quad \text{as } X \to \infty.
\]

We need to evaluate $\sqrt{\frac{2\pi i}{X\phi''(c)}}$. Since $\phi''(1) = 2 > 0$, we use the relation:
\[
\sqrt{i} = e^{i\pi/4} = \frac{1}{\sqrt{2}} + \frac{i}{\sqrt{2}}.
\]

Therefore:
\begin{align*}
\sqrt{\frac{2\pi i}{X\phi''(1)}} &= \sqrt{\frac{2\pi i}{2X}} \\
&= \sqrt{\frac{\pi i}{X}} \\
&= \sqrt{\frac{\pi}{X}} \cdot \sqrt{i} \\
&= \sqrt{\frac{\pi}{X}} \cdot e^{i\pi/4}.
\end{align*}

\subsection*{Step 6: Compute the leading asymptotic behaviour}

Substituting all values into the formula:
\begin{align*}
I(X) &\sim \sqrt{\frac{\pi}{X}} e^{i\pi/4} \cdot 2 \cdot e^{iX(-2/3)} \\
&= 2\sqrt{\frac{\pi}{X}} e^{i\pi/4} e^{-2iX/3} \\
&= 2\sqrt{\frac{\pi}{X}} \exp\left(i\left(\frac{\pi}{4} - \frac{2X}{3}\right)\right).
\end{align*}

\subsection*{Final Answer}

\[
\boxed{I(X) \sim 2\sqrt{\frac{\pi}{X}} \exp\left(i\left(\frac{\pi}{4} - \frac{2X}{3}\right)\right) \quad \text{as } X \to \infty}
\]

Alternatively, this can be written as:
\[
I(X) \sim 2\sqrt{\frac{\pi}{X}} e^{i\pi/4} e^{-2iX/3} \quad \text{as } X \to \infty,
\]
or in terms of trigonometric functions:
\[
I(X) \sim 2\sqrt{\frac{\pi}{X}} \left[\cos\left(\frac{\pi}{4} - \frac{2X}{3}\right) + i\sin\left(\frac{\pi}{4} - \frac{2X}{3}\right)\right] \quad \text{as } X \to \infty.
\]

The asymptotic order is $O(X^{-1/2})$.

\end{document}
