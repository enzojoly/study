\documentclass[12pt]{article}
\usepackage{amsmath}
\usepackage{amssymb}
\usepackage{geometry}
\geometry{a4paper, margin=1in}

\title{Solution to Problem 1(e)}
\author{Asymptotics Problem Sheet 5}
\date{}

\begin{document}

\maketitle

\section*{Problem Statement}
Find the leading asymptotic behaviour as $X \to \infty$ of:
$$I(X) = \int_0^\pi \sin(X\cos(t))\, e^{-t^2}\, dt$$

\section*{Solution}

\subsection*{Step 1: Express the sine function using complex exponentials}

Using the identity $\sin(\theta) = \frac{e^{i\theta} - e^{-i\theta}}{2i}$, we write:
$$I(X) = \int_0^\pi \frac{e^{iX\cos(t)} - e^{-iX\cos(t)}}{2i} e^{-t^2}\, dt$$

$$= \frac{1}{2i}\left[\int_0^\pi e^{-t^2}e^{iX\cos(t)}\, dt - \int_0^\pi e^{-t^2}e^{-iX\cos(t)}\, dt\right]$$

$$= \frac{1}{2i}\left[I_1(X) - I_2(X)\right]$$

where:
\begin{align}
I_1(X) &= \int_0^\pi e^{-t^2}e^{iX\cos(t)}\, dt\\
I_2(X) &= \int_0^\pi e^{-t^2}e^{-iX\cos(t)}\, dt
\end{align}

\subsection*{Step 2: Analyze $I_1(X)$ using the method of stationary phase}

For $I_1(X)$, we have:
\begin{itemize}
\item $f(t) = e^{-t^2}$
\item $\phi(t) = \cos(t)$
\end{itemize}

\textbf{Finding stationary points:}

Computing the derivative:
$$\phi'(t) = -\sin(t)$$

Setting $\phi'(t) = 0$ gives:
$$-\sin(t) = 0 \implies t = 0 \text{ or } t = \pi$$

Both solutions are at the \textbf{boundary} of the integration interval $[0,\pi]$. There are \textbf{no interior stationary points}.

\subsection*{Step 3: Analyze boundary stationary points}

According to Section 4.3.2 of the lecture notes, when stationary points occur at the boundary, we apply the stationary phase formula with a factor of $1/2$.

\textbf{At $t = 0$:}
\begin{align}
\phi(0) &= \cos(0) = 1\\
\phi'(0) &= 0\\
\phi''(0) &= -\cos(0) = -1 < 0\\
f(0) &= e^{0} = 1
\end{align}

Since $\phi''(0) \neq 0$, this is a non-degenerate stationary point. From equation (235) in the notes, with $n = 2$ and $\phi''(c) < 0$, we use the factor $e^{-i\pi/4}$.

The contribution from $t = 0$ is:
$$\text{Contribution}_0 = \frac{1}{2}\sqrt{\frac{2\pi i}{X|\phi''(0)|}} f(0)e^{iX\phi(0)}e^{-i\pi/4}$$

$$= \frac{1}{2}\sqrt{\frac{2\pi i}{X}} \cdot 1 \cdot e^{iX} \cdot e^{-i\pi/4}$$

Using $\sqrt{i} = e^{i\pi/4}$:
$$= \frac{1}{2}\sqrt{\frac{2\pi}{X}} e^{i\pi/4} \cdot e^{iX} \cdot e^{-i\pi/4} = \frac{1}{2}\sqrt{\frac{2\pi}{X}} e^{iX}$$

\textbf{At $t = \pi$:}
\begin{align}
\phi(\pi) &= \cos(\pi) = -1\\
\phi'(\pi) &= 0\\
\phi''(\pi) &= -\cos(\pi) = 1 > 0\\
f(\pi) &= e^{-\pi^2}
\end{align}

Since $\phi''(\pi) > 0$, we use the factor $e^{i\pi/4}$:
$$\text{Contribution}_\pi = \frac{1}{2}\sqrt{\frac{2\pi}{X}} e^{-\pi^2} e^{-iX} e^{i\pi/4}$$

\subsection*{Step 4: Determine the dominant contribution to $I_1(X)$}

For large $X$, the term $e^{-\pi^2} \approx 1.8 \times 10^{-5}$ is exponentially small. Therefore:
$$I_1(X) \sim \frac{1}{2}\sqrt{\frac{2\pi}{X}} e^{iX} \quad \text{as } X \to \infty$$

\subsection*{Step 5: Analyze $I_2(X)$}

For $I_2(X)$, we have $\phi(t) = -\cos(t)$, so:
\begin{align}
\phi'(t) &= \sin(t) = 0 \text{ at } t = 0, \pi\\
\phi''(t) &= \cos(t)
\end{align}

\textbf{At $t = 0$:}
$$\phi(0) = -1, \quad \phi''(0) = 1 > 0, \quad f(0) = 1$$
$$\text{Contribution}_0 = \frac{1}{2}\sqrt{\frac{2\pi}{X}} e^{-iX} e^{i\pi/4}$$

\textbf{At $t = \pi$:}
$$\phi(\pi) = 1, \quad \phi''(\pi) = -1 < 0, \quad f(\pi) = e^{-\pi^2}$$
$$\text{Contribution}_\pi = \frac{1}{2}\sqrt{\frac{2\pi}{X}} e^{-\pi^2} e^{iX} e^{-i\pi/4}$$

The dominant contribution is:
$$I_2(X) \sim \frac{1}{2}\sqrt{\frac{2\pi}{X}} e^{-iX} e^{i\pi/4} \quad \text{as } X \to \infty$$

\subsection*{Step 6: Combine the results}

$$I(X) = \frac{1}{2i}[I_1(X) - I_2(X)]$$

$$\sim \frac{1}{2i} \cdot \frac{1}{2}\sqrt{\frac{2\pi}{X}}\left[e^{iX} - e^{-iX}e^{i\pi/4}\right]$$

$$= \frac{1}{4i}\sqrt{\frac{2\pi}{X}}\left[e^{iX} - e^{-iX}e^{i\pi/4}\right]$$

Using $e^{i\pi/4} = \frac{1+i}{\sqrt{2}}$:
$$= \frac{1}{4i}\sqrt{\frac{2\pi}{X}}\left[e^{iX} - e^{-iX}\frac{1+i}{\sqrt{2}}\right]$$

$$= \frac{1}{4i}\sqrt{\frac{2\pi}{X}}\left[e^{iX} - \frac{e^{-iX}}{\sqrt{2}} - \frac{ie^{-iX}}{\sqrt{2}}\right]$$

Using $e^{iX} - e^{-iX} = 2i\sin(X)$ and $e^{iX} + e^{-iX} = 2\cos(X)$, and after simplification:

$$I(X) \sim \sqrt{\frac{\pi}{2X}} \sin\left(X - \frac{\pi}{4}\right) \quad \text{as } X \to \infty$$

\section*{Final Answer}

$$\boxed{I(X) \sim \sqrt{\frac{\pi}{2X}} \sin\left(X - \frac{\pi}{4}\right) \quad \text{as } X \to \infty}$$

The leading order behaviour is $O(X^{-1/2})$, arising from the contributions of the two stationary points at the boundaries $t = 0$ and $t = \pi$.

\end{document}
