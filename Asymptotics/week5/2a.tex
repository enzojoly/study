\documentclass[11pt,a4paper]{article}
\usepackage[margin=2.5cm]{geometry}
\usepackage{amsmath,amssymb,amsthm}
\usepackage{mathtools}
\usepackage{graphicx}
\usepackage{enumitem}

% Theorem environments
\newtheorem{theorem}{Theorem}
\newtheorem{lemma}{Lemma}

% Custom commands
\newcommand{\dd}{\mathrm{d}}
\newcommand{\RR}{\mathbb{R}}
\newcommand{\CC}{\mathbb{C}}
\newcommand{\ii}{\mathrm{i}}

\title{Asymptotics 2025/2026 -- Problem Sheet 5\\
\Large Solution to Question 2(a)}
\author{Method of Steepest Descent}
\date{}

\begin{document}

\maketitle

\section*{Problem Statement}
Use the method of steepest descent to find the leading asymptotic behaviour as $X \to \infty$ of:
\begin{equation}
I(X) = \int_{-1}^{\infty} e^{X(t+\ii t-t^2/2)} \dd t
\end{equation}

\section*{Solution}

\subsection*{Step 1: Identify the Complex Function}

The integral has the form
\begin{equation}
I(X) = \int_{-1}^{\infty} e^{X\phi(t)} \dd t
\end{equation}
where the complex function $\phi(t)$ is given by:
\begin{equation}
\phi(t) = t + \ii t - \frac{t^2}{2} = t(1+\ii) - \frac{t^2}{2}
\end{equation}

This is a steepest descent problem with $\phi(t) \in \CC$ and integration parameter $X \to \infty$.

\subsection*{Step 2: Find Critical Points (Saddle Points)}

Following the methodology from Section 4.4.2 of the lecture notes, we locate saddle points by finding where $\phi'(t) = 0$:
\begin{equation}
\phi'(t) = 1 + \ii - t = 0
\end{equation}

Thus, the unique saddle point is:
\begin{equation}
t_0 = 1 + \ii
\end{equation}

The second derivative at this point is:
\begin{equation}
\phi''(t) = -1 \quad \text{for all } t
\end{equation}

In particular, $\phi''(t_0) = -1 = e^{\ii\pi}$, so $|\phi''(t_0)| = 1$ and $\alpha = \pi$.

\subsection*{Step 3: Evaluate $\phi$ at the Saddle Point}

We compute:
\begin{align}
\phi(t_0) &= (1+\ii)(1+\ii) - \frac{(1+\ii)^2}{2}\\
&= (1+\ii)^2 - \frac{(1+\ii)^2}{2}\\
&= \frac{(1+\ii)^2}{2}\\
&= \frac{1 + 2\ii - 1}{2}\\
&= \ii
\end{align}

\subsection*{Step 4: Determine Steepest Descent Contours}

To understand the geometry of the steepest descent paths, we decompose $\phi(t)$ into real and imaginary parts. Setting $t = x + \ii y$:
\begin{align}
\phi(x+\ii y) &= (x+\ii y)(1+\ii) - \frac{(x+\ii y)^2}{2}\\
&= x + \ii x + \ii y - y - \frac{x^2 - y^2 + 2\ii xy}{2}\\
&= \left(x - y - \frac{x^2 - y^2}{2}\right) + \ii\left(x + y - xy\right)
\end{align}

Therefore:
\begin{equation}
u(x,y) = x - y - \frac{x^2 - y^2}{2}, \qquad v(x,y) = x + y - xy
\end{equation}

At the saddle point $(x_0, y_0) = (1,1)$:
\begin{equation}
v(1,1) = 1 + 1 - 1 = 1
\end{equation}

The constant phase contour through the saddle point satisfies $v(x,y) = 1$.

\textbf{Local analysis near the saddle point:}

Near $t_0 = 1+\ii$, set $t = t_0 + se^{\ii\theta}$ for small $s \geq 0$. Then:
\begin{equation}
\phi(t) \approx \phi(t_0) + \frac{1}{2}\phi''(t_0)(t-t_0)^2 = \ii - \frac{1}{2}s^2e^{2\ii\theta}
\end{equation}

The real part is:
\begin{equation}
\text{Re}[\phi(t)] \approx -\frac{s^2}{2}\cos(2\theta)
\end{equation}

Using the general formula from the lecture notes (page 72), for $\phi''(t_0) = ae^{\ii\alpha}$ with $a = 1$ and $\alpha = \pi$, the directions of steepest descent are:
\begin{equation}
\theta_{\text{descent}} = -\frac{\alpha}{2} + (2p+1)\frac{\pi}{2}, \quad p = 0,1
\end{equation}

For $p=0$: $\theta = -\pi/2 + \pi/2 = 0$

For $p=1$: $\theta = -\pi/2 + 3\pi/2 = \pi$

Thus, the steepest descent directions are along $\theta = 0$ and $\theta = \pi$, i.e., along the horizontal direction (parallel to the real axis) passing through $t_0 = 1+\ii$.

\textbf{Parameterization of steepest descent path:}

The steepest descent path through $t_0$ is the horizontal line at height $y=1$:
\begin{equation}
t = x + \ii, \quad x \in (-\infty, \infty)
\end{equation}

Along this path:
\begin{align}
\phi(x+\ii) &= (x+\ii)(1+\ii) - \frac{(x+\ii)^2}{2}\\
&= x - 1 + \ii(x+1) - \frac{x^2 - 1 + 2\ii x}{2}\\
&= x - 1 - \frac{x^2-1}{2} + \ii\left(x + 1 - x\right)\\
&= -\frac{x^2}{2} + x - \frac{1}{2} + \ii\\
&= -\frac{(x-1)^2}{2} + \ii
\end{align}

The real part is:
\begin{equation}
u(x,1) = -\frac{(x-1)^2}{2}
\end{equation}

This has a maximum at $x=1$ (the saddle point) and decreases to $-\infty$ as $|x| \to \infty$, confirming exponential decay along the steepest descent path.

\subsection*{Step 5: Deform the Integration Contour}

The original integration path runs from $-1$ to $\infty$ along the real axis (at $y=0$). The saddle point $t_0 = 1+\ii$ lies off this path.

By Cauchy's integral theorem (Section 4.4, page 68), since $e^{X\phi(t)}$ is entire (analytic everywhere in $\CC$), we can deform the contour without changing the integral value, provided boundary contributions vanish.

We deform the contour to consist of:
\begin{enumerate}[label=(\roman*)]
\item A vertical segment from $-1$ to $-1+\ii$
\item The horizontal steepest descent path from $-1+\ii$ to $+\infty+\ii$ (passing through the saddle point $t_0 = 1+\ii$)
\end{enumerate}

\textbf{Justification for neglecting the vertical segment:}

Along the vertical segment $t = -1 + \ii s$ for $s \in [0,1]$:
\begin{align}
\text{Re}[\phi(-1+\ii s)] &= -1 - s - \frac{1 - s^2}{2}\\
&= -\frac{3}{2} - s + \frac{s^2}{2}\\
&\leq -\frac{3}{2} + \frac{1}{2} = -1
\end{align}

Therefore:
\begin{equation}
\left|\int_{0}^{1} e^{X\phi(-1+\ii s)} \ii \dd s\right| \leq \int_{0}^{1} e^{-X} \dd s = e^{-X}
\end{equation}

As $X \to \infty$, this contribution is exponentially small compared to the saddle point contribution and can be neglected: $e^{-X} = o(e^{0}) = o(1)$.

\subsection*{Step 6: Evaluate the Integral Along the Steepest Descent Path}

Along the horizontal line $y=1$, we have $t = x + \ii$ with $x \in (-\infty, \infty)$ and $\dd t = \dd x$. Using our result from Step 4:
\begin{equation}
\phi(x+\ii) = \ii - \frac{(x-1)^2}{2}
\end{equation}

Therefore:
\begin{align}
I(X) &\sim \int_{-\infty}^{\infty} e^{X[\ii - (x-1)^2/2]} \dd x\\
&= e^{\ii X} \int_{-\infty}^{\infty} e^{-X(x-1)^2/2} \dd x
\end{align}

\textbf{Change of variables:}

Let $u = x - 1$, so $\dd u = \dd x$:
\begin{equation}
I(X) = e^{\ii X} \int_{-\infty}^{\infty} e^{-Xu^2/2} \dd u
\end{equation}

\textbf{Evaluate the Gaussian integral:}

Using the standard result $\int_{-\infty}^{\infty} e^{-au^2} \dd u = \sqrt{\pi/a}$ for $a > 0$:
\begin{equation}
\int_{-\infty}^{\infty} e^{-Xu^2/2} \dd u = \sqrt{\frac{2\pi}{X}}
\end{equation}

\subsection*{Step 7: Final Result}

Combining the results, we obtain the leading asymptotic behaviour:

\begin{equation}
\boxed{I(X) \sim \sqrt{\frac{2\pi}{X}} \, e^{\ii X} \quad \text{as } X \to \infty}
\end{equation}

This can also be written as:
\begin{equation}
I(X) \sim \sqrt{\frac{2\pi}{X}} \left[\cos(X) + \ii\sin(X)\right] \quad \text{as } X \to \infty
\end{equation}

\subsection*{Verification}

The result is consistent with the general steepest descent formula from the lecture notes. For an integral of the form $\int f(z)e^{\lambda\phi(z)} \dd z$ with a saddle point at $z_0$ where $\phi'(z_0) = 0$ and $\phi''(z_0) \neq 0$, the leading asymptotic contribution is:
\begin{equation}
I(\lambda) \sim \sqrt{\frac{2\pi}{\lambda|\phi''(z_0)|}} \, f(z_0) \, e^{\lambda\phi(z_0)} \, e^{\ii\beta}
\end{equation}
where $\beta$ accounts for the phase factor depending on the direction of the steepest descent path.

In our case:
\begin{itemize}
\item $\lambda = X$
\item $f(t) = 1$, so $f(t_0) = 1$
\item $\phi(t_0) = \ii$
\item $|\phi''(t_0)| = |-1| = 1$
\item The steepest descent path is horizontal, contributing no additional phase
\end{itemize}

This yields:
\begin{equation}
I(X) \sim \sqrt{\frac{2\pi}{X}} \, e^{\ii X}
\end{equation}

confirming our result. \qed

\end{document}
