\documentclass[11pt,a4paper]{article}
\usepackage{amsmath,amssymb,amsthm}
\usepackage{geometry}
\geometry{margin=2.5cm}

\title{Asymptotics Problem Sheet 5 - Question 2(c)}
\author{}
\date{}

\begin{document}
\maketitle

\section*{Problem Statement}
Use the method of steepest descent to find the leading asymptotic behaviour as $X \to \infty$ of:
$$I(X) = \int_0^{\infty} e^{iX(t^4/4 + t^3/3)} e^{-t}\, dt$$

\section*{Solution}

\subsection*{Step 1: Identify the integral form and extend to complex plane}

We write the integral in the standard form for steepest descent:
$$I(X) = \int_0^{\infty} f(t)\, e^{X\phi(t)}\, dt$$
where:
\begin{align*}
f(t) &= e^{-t} \\
\phi(t) &= i\left(\frac{t^4}{4} + \frac{t^3}{3}\right)
\end{align*}

The original contour $C_0$ is along the positive real axis from $0$ to $\infty$. We extend $\phi(t)$ to the complex plane as:
$$\phi(z) = i\left(\frac{z^4}{4} + \frac{z^3}{3}\right)$$

\subsection*{Step 2: Find saddle points}

Following Section 4.4 of the notes, saddle points occur where $\phi'(z) = 0$:
$$\phi'(z) = i\left(z^3 + z^2\right) = iz^2(z + 1) = 0$$

This gives saddle points at:
\begin{itemize}
\item $z_0 = 0$ (on the original contour)
\item $z_1 = -1$ (off the original contour)
\end{itemize}

\subsection*{Step 3: Analyze the saddle point at $z = 0$}

We examine the order of the saddle point at $z = 0$ by computing derivatives:
\begin{align*}
\phi'(0) &= 0 \\
\phi''(0) &= i(3z^2 + 2z)\big|_{z=0} = 0 \\
\phi'''(0) &= i(6z + 2)\big|_{z=0} = 2i
\end{align*}

Since $\phi'(0) = \phi''(0) = 0$ but $\phi'''(0) \neq 0$, this is a \textbf{third-order saddle point} ($n = 3$).

\subsection*{Step 4: Determine steepest descent directions}

Near $z = 0$, the dominant behavior of $\phi(z)$ is:
$$\phi(z) \approx \frac{iz^3}{3}$$

For $z = re^{i\theta}$:
$$\phi(z) \approx \frac{ir^3e^{i3\theta}}{3} = \frac{r^3}{3}e^{i(3\theta + \pi/2)} = \frac{r^3}{3}\left[\cos(3\theta + \pi/2) + i\sin(3\theta + \pi/2)\right]$$

Constant phase contours (steepest descent/ascent paths) satisfy $\text{Im}[\phi(z)] = \text{const}$:
$$\sin(3\theta + \pi/2) = 0 \implies 3\theta + \frac{\pi}{2} = n\pi \implies \theta = \frac{(2n-1)\pi}{6}$$

For $n = 0, 1, 2$:
\begin{itemize}
\item $n = 0$: $\theta = -\pi/6$
\item $n = 1$: $\theta = \pi/6$
\item $n = 2$: $\theta = 5\pi/6$
\end{itemize}

The real part along these paths is:
$$\text{Re}[\phi(z)] \approx \frac{r^3}{3}\cos(3\theta + \pi/2) = -\frac{r^3}{3}\sin(3\theta)$$

\begin{itemize}
\item For $\theta = \pi/6$: $\sin(\pi/2) = 1 \implies \text{Re}[\phi] = -r^3/3 < 0$ (descending - steepest descent)
\item For $\theta = -\pi/6$: $\sin(-\pi/2) = -1 \implies \text{Re}[\phi] = r^3/3 > 0$ (ascending - steepest ascent)
\item For $\theta = 5\pi/6$: $\sin(5\pi/2) = 1 \implies \text{Re}[\phi] = -r^3/3 < 0$ (descending - steepest descent)
\end{itemize}

\subsection*{Step 5: Deform the contour}

We deform the original contour along the positive real axis ($\theta = 0$) to the steepest descent path at $\theta = \pi/6$. By Cauchy's theorem (Section 4.4), since there are no other singularities encountered, the integral value is preserved.

\subsection*{Step 6: Evaluate along the steepest descent path}

Along the path $z = se^{i\pi/6}$ with $s \geq 0$, we have $dz = e^{i\pi/6}\, ds$ and:
$$\phi(z) \approx \frac{is^3e^{i\pi/2}}{3} = -\frac{s^3}{3}$$

The integral becomes:
$$I(X) = e^{i\pi/6} \int_0^{\infty} e^{-se^{i\pi/6}} e^{-Xs^3/3}\, ds$$

For large $X$, the integrand is dominated by small values of $s$ (near the saddle point). We can expand:
$$e^{-se^{i\pi/6}} = e^{-s(\sqrt{3}/2 + i/2)} \approx 1 + O(s)$$

\subsection*{Step 7: Apply Watson's lemma}

To leading order:
$$I(X) \sim e^{i\pi/6} \int_0^{\infty} e^{-Xs^3/3}\, ds$$

We evaluate this integral using the substitution $u = Xs^3/3$:
$$s = \left(\frac{3u}{X}\right)^{1/3}, \quad ds = \left(\frac{3}{X}\right)^{1/3} \frac{1}{3u^{2/3}}\, du$$

Therefore:
\begin{align*}
\int_0^{\infty} e^{-Xs^3/3}\, ds &= \int_0^{\infty} e^{-u} \left(\frac{3}{X}\right)^{1/3} \frac{1}{3u^{2/3}}\, du \\
&= \frac{1}{3^{2/3} X^{1/3}} \int_0^{\infty} u^{-2/3} e^{-u}\, du \\
&= \frac{1}{3^{2/3} X^{1/3}} \Gamma(1/3)
\end{align*}

where we used $\Gamma(1/3) = \int_0^{\infty} u^{-2/3} e^{-u}\, du$ from Eq. (68) in the notes.

\subsection*{Step 8: Final result}

Combining all factors:
$$I(X) \sim e^{i\pi/6} \cdot \frac{\Gamma(1/3)}{3^{2/3} X^{1/3}}$$

\section*{Answer}

$$\boxed{I(X) \sim \frac{\Gamma(1/3)}{3^{2/3}} X^{-1/3} e^{i\pi/6} \quad \text{as } X \to \infty}$$

Alternatively, using $e^{i\pi/6} = \frac{\sqrt{3}}{2} + \frac{i}{2}$:
$$\boxed{I(X) \sim \frac{\Gamma(1/3)}{3^{2/3}} X^{-1/3} \left(\frac{\sqrt{3}}{2} + \frac{i}{2}\right) \quad \text{as } X \to \infty}$$

\textbf{Note:} The dominant contribution comes from the third-order saddle point at $z = 0$, with the asymptotic order $O(X^{-1/3})$.

\end{document}
