\documentclass[11pt]{article}
\usepackage{amsmath}
\usepackage{amssymb}
\usepackage{geometry}
\geometry{a4paper, margin=1in}

\title{Asymptotics Problem 2(e)}
\author{}
\date{}

\begin{document}
\maketitle

\section*{Problem 2(e)}
Find the leading asymptotic behaviour as $X \to \infty$ of:
$$I(X) = \int_{-1}^{\infty} \sqrt{1+t}\, \cos(Xt^2)\, e^{X(t-t^3/3)}\, dt$$

\section*{Solution}

\subsection*{Step 1: Identify the integral structure}

We have a complex integral that can be written as:
$$I(X) = \int_{-1}^{\infty} f(t)\, e^{X\phi(t)}\, dt$$

where the effective exponent combines both real and oscillatory parts. Note that:
\begin{itemize}
    \item $f(t) = \sqrt{1+t}\, \cos(Xt^2)$
    \item The real part of the exponent: $u(t) = t - \frac{t^3}{3}$
    \item Oscillatory factor: $\cos(Xt^2)$
\end{itemize}

\subsection*{Step 2: Locate the critical point}

Since $\cos(Xt^2)$ remains bounded and oscillatory, the dominant contribution comes from the maximum of the real part $u(t) = t - \frac{t^3}{3}$ on the integration domain $[-1, \infty)$.

Find stationary points:
$$u'(t) = 1 - t^2 = 0 \implies t = \pm 1$$

Evaluate $u(t)$ at critical points and boundaries:
\begin{align*}
u(-1) &= -1 - \frac{(-1)^3}{3} = -1 + \frac{1}{3} = -\frac{2}{3}\\
u(1) &= 1 - \frac{1}{3} = \frac{2}{3} \quad \text{(maximum)}\\
\lim_{t \to \infty} u(t) &= -\infty
\end{align*}

Check second derivative at $t = 1$:
$$u''(t) = -2t \implies u''(1) = -2 < 0$$

Therefore, $t = 1$ is a **maximum** of $u(t)$ and will give the dominant contribution as $X \to \infty$.

\subsection*{Step 3: Apply Laplace's method near the maximum}

Near $t = 1$, expand $u(t)$:
\begin{align*}
u(t) &= u(1) + u'(1)(t-1) + \frac{1}{2}u''(1)(t-1)^2 + O((t-1)^3)\\
&= \frac{2}{3} + 0 + \frac{1}{2}(-2)(t-1)^2 + O((t-1)^3)\\
&= \frac{2}{3} - (t-1)^2 + O((t-1)^3)
\end{align*}

Evaluate other functions at $t = 1$:
\begin{align*}
\sqrt{1+t}\big|_{t=1} &= \sqrt{2}\\
\cos(Xt^2)\big|_{t=1} &= \cos(X)
\end{align*}

For the oscillatory term, near $t = 1$:
$$t^2 = 1 + 2(t-1) + (t-1)^2 \implies \cos(Xt^2) \approx \cos(X + 2X(t-1) + X(t-1)^2)$$

To leading order as $X \to \infty$, we can approximate $\cos(Xt^2) \approx \cos(X)$ near the maximum.

\subsection*{Step 4: Set up the Laplace approximation}

Substitute $s = t - 1$, so $t = s + 1$ and $dt = ds$. The integration limits become $s \in [-2, \infty)$.

$$I(X) \sim \int_{-2}^{\infty} \sqrt{2}\, \cos(X)\, e^{X(2/3 - s^2)}\, ds$$

Factor out constants:
$$I(X) \sim \sqrt{2}\, e^{2X/3}\, \cos(X) \int_{-2}^{\infty} e^{-Xs^2}\, ds$$

\subsection*{Step 5: Evaluate the Gaussian integral}

For large $X$, the integrand $e^{-Xs^2}$ decays rapidly away from $s = 0$. The contribution from $s < -2$ is exponentially small, so we can extend the lower limit to $-\infty$:

$$\int_{-2}^{\infty} e^{-Xs^2}\, ds \sim \int_{-\infty}^{\infty} e^{-Xs^2}\, ds = \sqrt{\frac{\pi}{X}}$$

\subsection*{Step 6: Final result}

Combining all terms:
$$I(X) \sim \sqrt{2}\, e^{2X/3}\, \cos(X) \cdot \sqrt{\frac{\pi}{X}}$$

\begin{center}
\boxed{
I(X) \sim \sqrt{\frac{2\pi}{X}}\, e^{2X/3}\, \cos(X) \quad \text{as } X \to \infty
}
\end{center}

The leading asymptotic behaviour has:
\begin{itemize}
    \item Exponential growth: $e^{2X/3}$ (from the maximum of $u(t)$ at $t=1$)
    \item Algebraic decay: $X^{-1/2}$ (from the Gaussian integral near the maximum)
    \item Oscillation: $\cos(X)$ (from the $\cos(Xt^2)$ factor evaluated at $t=1$)
\end{itemize}

\end{document}
